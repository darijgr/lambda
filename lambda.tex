\documentclass[12pt,final,notitlepage,onecolumn,german]{article}%
\usepackage[all,cmtip]{xy}
\usepackage{lscape}
\usepackage{amsfonts}
\usepackage{amssymb}
\usepackage{graphicx}
\usepackage{amsmath}%
\setcounter{MaxMatrixCols}{30}
%TCIDATA{OutputFilter=latex2.dll}
%TCIDATA{Version=5.00.0.2570}
%TCIDATA{LastRevised=Saturday, September 26, 2009 00:49:35}
%TCIDATA{<META NAME="GraphicsSave" CONTENT="32">}
%TCIDATA{<META NAME="SaveForMode" CONTENT="1">}
\voffset=-2.5cm
\hoffset=-2.5cm
\setlength\textheight{24cm}
\setlength\textwidth{15.5cm}
\begin{document}

\begin{center}
\fbox{$\lambda$\textbf{-rings: Definitions and basic properties}}

Darij Grinberg

\textit{Version 0.0.3, last revised 25 Sep 2009}
\end{center}

This is a \textbf{BETA VERSION} and has never been proofread. \textbf{Please
notify me of any mistakes, typos and hard-to-understand arguments you
find!\footnote{my email address is \texttt{A@B.com}, where
\texttt{A=darijgrinberg} and \texttt{B=gmail}}}

At the moment, section 1 is missing a proof (namely, that the representation
ring is a \textit{special} $\lambda$-ring; I actually don't know this proof)
and sections 10-... are more or less missing.

\bigskip

\fbox{\textbf{What is this?}}

These notes were written for a seminar talk. They try to cover some of the
most important properties of $\lambda$-rings with proofs. At the moment, most
of what is written here is also in Knutson's book [2], albeit sometimes with a
different proof. In future I plan to add some results from Fulton/Lang [1]
with better proofs.

\begin{center}
\fbox{\textbf{0. Notation}}
\end{center}

Some notations that we will use later on:

\begin{itemize}
\item In the following, $\mathbb{N}$ will denote the set $\left\{
0,1,2,...\right\}  $.

\item When we say "ring", we will always mean "commutative ring with unity". A
"ring homomorphism" is always supposed to send $1$ to $1$.

\item Let $R$ be a ring. An \textit{extension ring} of $R$ will mean a ring
$S$ along with a ring monomorphism $R\rightarrow S$. (We will often sloppily
identify $R$ with a subring of $S$ if $S$ is an extension ring of $R$; we will
then also identify the polynomial ring $R\left[  T\right]  $ with a subring of
the polynomial ring $S\left[  T\right]  $, and so on.) An extension ring $S$
of $R$ is called \textit{finite-free} if and only if the $R$-module $S$ is
finite-free (i. e., a free $R$-module with a finite basis).

\item We will use multisets. If $I$ is a set, and $u_{i}$ is an object for
every $i\in I$, then we let $\left[  u_{i}\mid i\in I\right]  $ denote the
multiset formed by all the $u_{i}$ where $i$ ranges over $I$ (this multiset
will contain each object $o$ as often as it appears as an $u_{i}$ for some
$i\in I$).\newline If $I=\left\{  1,2,...,n\right\}  $ for some $n\in
\mathbb{N}$, then we also denote the multiset $\left[  u_{i}\mid i\in
I\right]  $ by $\left[  u_{1},u_{2},...,u_{n}\right]  $.

\item We have not defined $\lambda$-rings yet, but it is important to mention
some discrepancy in notation between different sources. Namely, some of the
literature (including [2], [5] and [6]) denotes as \textit{pre-}$\lambda
$\textit{-rings} what we call $\lambda$-rings and denotes as $\lambda
$\textit{-rings} what we call special $\lambda$-rings. Even worse, the
notations in [1] are totally inconsistent\footnote{Often, "$\lambda$-ring" in
[1] means "$\lambda$-ring with a positive structure" (such $\lambda$-rings are
automatically special), but sometimes it simply means "$\lambda$-ring".}.
\end{itemize}

\begin{center}
\fbox{\textbf{1. Motivation}}
\end{center}

What is the point of $\lambda$-rings?

Fulton/Lang [1] motivate $\lambda$-rings through vector bundles. Here we are
going for a more elementary motivation, namely through representation rings in
group representation theory:

Consider a finite group $G$ and a field $k$. In representation theory, one
define the so-called \textit{representation ring} of the group $G$ over the
field $k$. This ring can be constructed as follows:

We consider only finite-dimensional representations of $G$.

Let $\operatorname*{Rep}_{k}G$ be the set of all representations of the group
$G$ over the field $k$. (We disregard the set-theoretic problematics stemming
from the notion of such a big set. If you wish, you can call it a class or a
SET instead of a set, or restrict yourself to a smaller subset containing
every representation up to isomorphism.)

Let $\operatorname*{FRep}_{k}G$ be the free abelian group over
$\operatorname*{Rep}_{k}G$. Let $I$ be the subgroup%
\[
I=\left\langle
\begin{array}
[c]{c}%
U-V\ \mid\ U\text{ and }V\text{ are two isomorphic representations of }G;\\
U\oplus V-U-V\ \mid\ U\text{ and }V\text{ are two representations of }G
\end{array}
\right\rangle
\]
of the free abelian group $\operatorname*{FRep}_{k}G$. Then,
$\operatorname*{FRep}_{k}G\diagup I$ is an abelian group. Whenever $U$ is a
representation of $G,$ we should denote the equivalence class of
$U\in\operatorname*{FRep}_{k}G$ modulo the ideal $I$ by $\overline{U};$
however, since we are always going to work in $\operatorname*{FRep}%
_{k}G\diagup I$ (because there is nothing interesting to do in
$\operatorname*{FRep}_{k}G$ itself), we will simply write $U$ for this
equivalence class. This means that whenever $U$ and $V$ are two isomorphic
representations of $G$, we will simply write $U=V$, and whenever $U$ and $V$
are two representations of $G$, we will simply write $U+V=U\oplus V$.

Denote by $1$ the equivalence class of the trivial representation of $G$ on
$k$ (with every element of $G$ acting as identity) modulo $I$. We now define a
ring structure on $\operatorname*{FRep}_{k}G\diagup I$ by letting $1$ be the
one of this ring, and defining the product of two representations of $G$ as
their tensor product (over $k$). This is indeed a ring structure because we
have isomorphisms%
\begin{align*}
U\otimes\left(  V\otimes W\right)   &  \cong\left(  U\otimes V\right)  \otimes
W,\\
\left(  U\oplus V\right)  \otimes W  &  \cong\left(  U\otimes W\right)
\oplus\left(  V\otimes W\right)  ,\\
U\otimes V  &  \cong V\otimes U,\\
1\otimes U  &  \cong U\otimes1\cong U
\end{align*}
for any representations $U,$ $V$ and $W$ (and probably some others I have forgotten).

The ring $\operatorname*{FRep}_{k}G\diagup I$ is called the
\textit{representation ring of the group }$G$\textit{ over the field }$k$. The
elements of $\operatorname*{FRep}_{k}G\diagup I$ are called \textit{virtual
representations}.

This ring $\operatorname*{FRep}_{k}G\diagup I$ is helpful in working with
representations. However, its ring structure does not yet reflect everything
we can do with representations. In fact, we can build direct sums of
representations (this is addition in $\operatorname*{FRep}_{k}G\diagup I$) and
we can build tensor products (this is multiplication in $\operatorname*{FRep}%
_{k}G\diagup I$), but we can also build exterior powers of representations,
and we have no idea yet what operation on $\operatorname*{FRep}_{k}G\diagup I$
this entails. So we see that the abstract notion of a ring is not enough to
understand all of representation theory. We need a notion of a ring together
with some operations that "behave like" taking exterior powers. What axioms
should these operations satisfy?

Every representation $V$ of a group $G$ satisfies $\wedge^{0}V\cong1$ and
$\wedge^{1}V\cong V$. Besides, for any two representations $V$ and $W$ of $G$
and every $k\in\mathbb{N}$, there exists an isomorphism%
\begin{equation}
\wedge^{k}\left(  V\oplus W\right)  \cong\bigoplus_{i=0}^{k}\wedge^{i}%
V\otimes\wedge^{k-i}W \label{RepThV+W}%
\end{equation}
(see Exercise 1.1). In the representation ring, this means%
\[
\wedge^{k}\left(  V+W\right)  =\sum_{i=0}^{k}\left(  \wedge^{i}V\right)
\cdot\left(  \wedge^{k-i}W\right)  .
\]
This already gives us three axioms for the operations that we want to
introduce. If we extend these three axioms to arbitrary elements of
$\operatorname*{FRep}_{k}G\diagup I$ (and not just actual representations), we
can compute $\wedge^{k}$ of virtual representations (and it turns out that it
is well-defined), and we obtain the notion of a $\lambda$\textit{-ring}.

We can still wonder whether these axioms are all that we can say about group
representations. The answer is no: In addition to the formula (\ref{RepThV+W}%
), there exist relations of the form
\begin{align}
&  \wedge^{k}\left(  V\otimes W\right)  =P_{k}\left(  \wedge^{1}V,\wedge
^{2}V,...,\wedge^{k}V,\wedge^{1}W,\wedge^{2}W,...,\wedge^{k}W\right)
\nonumber\\
&  \ \ \ \ \ \ \ \ \ \ \text{for every }k\in\mathbb{N}\text{ and any two
representations }V\text{ and }W\text{ of }G \label{RepThVW}%
\end{align}
and%
\begin{align}
&  \wedge^{k}\left(  \wedge^{j}\left(  V\right)  \right)  =P_{k,j}\left(
\wedge^{1}V,\wedge^{2}V,...,\wedge^{kj}V\right) \nonumber\\
&  \ \ \ \ \ \ \ \ \ \ \text{for every }k\in\mathbb{N},\text{ }j\in
\mathbb{N}\text{ and any representation }V\text{ of }G, \label{RepThLL}%
\end{align}
where $P_{k}\in\mathbb{Z}\left[  \alpha_{1},\alpha_{2},...,\alpha_{k}%
,\beta_{1},\beta_{2},...,\beta_{k}\right]  $ and $P_{k,j}\in\mathbb{Z}\left[
\alpha_{1},\alpha_{2},...,\alpha_{kj}\right]  $ are "universal" polynomials
(i. e., polynomials only depending on $k$ resp. on $k$ and $j$, but not on
$V$, $W$ or $G$). These polynomials are rather hard to write down explicitely,
so it will need some theoretical preparation to define them.\footnote{Note
that these polynomials can have negative coefficients, so that the equality
(\ref{RepThVW}) does not necessarily mean an isomorphism of the kind%
\[
\wedge^{k}\left(  V\otimes W\right)  \cong\text{direct sum of some tensor
products of some }\wedge^{i}V\text{ and }\wedge^{j}W,
\]
but generally means an isomorphism of the kind%
\begin{align*}
&  \wedge^{k}\left(  V\otimes W\right)  \oplus\text{direct sum of some tensor
products of some }\wedge^{i}V\text{ and }\wedge^{j}W\\
&  =\text{(another) direct sum of some tensor products of some }\wedge
^{i}V\text{ and }\wedge^{j}W,
\end{align*}
and similarly (\ref{RepThLL}) has to be understood.}

These relations (\ref{RepThV+W}) and (\ref{RepThLL}), generalized to arbitrary
virtual representations, abstract to the notion of a \textit{special }%
$\lambda$\textit{-ring}. So$\ \operatorname*{FRep}_{k}G\diagup I$ is not just
a $\lambda$-ring; it is a special $\lambda$-ring. However, it has even more
structure than that: It is an \textit{augmented }$\lambda$\textit{-ring with
positive structure}. "Augmented" means the existence of a ring homomorphism
$\varepsilon:\operatorname*{FRep}_{k}G\diagup I\rightarrow\mathbb{Z}$ (a
so-called \textit{augmentation}) with certain properties; we will list these
properties later, but let us now notice that for our representation ring
$\operatorname*{FRep}_{k}G\diagup I,$ the obvious natural choice of
$\varepsilon$ is the homomorphism which maps every representation $V$ of $G$
to $\dim V\in\mathbb{Z}$. A "positive structure" is a subset of $K$ closed
under addition and multiplication and containing $1$, and satisfying other
properties; in our case, the best choice for a positive structure on
$\operatorname*{FRep}_{k}G\diagup I$ is the subset%
\[
\left\{  \overline{V}\mid V\text{ is a representation of }G\right\}
\setminus0\subseteq\operatorname*{FRep}\nolimits_{k}G\diagup I.
\]


Vector bundles over a given compact Hausdorff space are similar to
representations of a given group in several ways: They are somehow "enriched"
vector space structures (a vector bundle is, roughly speaking, a family of
vector spaces with additional topological structure; a representation of a
group is a vector space with a group action on it), so one can form direct
sums, tensor products and exterior powers of both of these. Hence, it is not
surprising that we can define a $\lambda$-ring structure on a kind of "ring of
vector bundles over a space" similarly to the $\lambda$-ring structure on the
representation ring of a group. However, we must be more careful with vector
bundles, because this "ring of vector bundles over a space" actually does not
consist of vector bundles, but of equivalence classes, and sometimes,
different vector bundles can lie in one and the same equivalence class (as
opposed to representations of a group, which are uniquely determined by their
equivalence class in the representation ring). This "ring of vector bundles"
is denoted by $K\left(  X\right)  $, where $X$ is the base space, and is the
first fundamental object of study in K-theory. We will not delve into K-theory
here; we will only provide some of its backbone, namely the abstract algebraic
theory of $\lambda$-rings (which appear not only in K-theory, but also in
representation theory and elsewhere).

\begin{quotation}
\textit{Exercise 1.1.} Let $G$ be a group, and let $V$ and $W$ be two
representations of $G$. Let $k\in\mathbb{N}$. Let $\iota_{V}:V\rightarrow
V\oplus W$ and $\iota_{W}:W\rightarrow V\oplus W$ be the canonical injections.
Prove that the vector space homomorphism%
\[
\bigoplus_{i=0}^{k}\wedge^{i}V\otimes\wedge^{k-i}W\rightarrow\wedge^{k}\left(
V\oplus W\right)
\]
composed of the homomorphisms%
\begin{align*}
\wedge^{i}V\otimes\wedge^{k-i}W  &  \rightarrow\wedge^{k}\left(  V\oplus
W\right)  ,\\
\left(  v_{1}\wedge v_{2}\wedge...\wedge v_{i}\right)  \otimes\left(
w_{1}\wedge w_{2}\wedge...\wedge w_{k-i}\right)   &  \mapsto\iota_{V}\left(
v_{1}\right)  \wedge\iota_{V}\left(  v_{2}\right)  \wedge...\wedge\iota
_{V}\left(  v_{i}\right)  \wedge\iota_{W}\left(  w_{1}\right)  \wedge\iota
_{W}\left(  w_{2}\right)  \wedge...\wedge\iota_{W}\left(  w_{k-i}\right)
\end{align*}
for all $i\in\left\{  0,1,...,k\right\}  $ is a canonical isomorphism of representations.
\end{quotation}

\begin{center}
\fbox{\textbf{2. }$\lambda$\textbf{-rings}}
\end{center}

\begin{quote}
\textbf{Definition.} \textbf{1)} Let $K$ be a ring. Let $\lambda
^{i}:K\rightarrow K$ be a mapping\footnote{Here, "mapping" actually means
"mapping" and not "group homomorphism" of "ring homomorphism".} for every
$i\in\mathbb{N}$ such that $\lambda^{0}\left(  x\right)  =1$ and $\lambda
^{1}\left(  x\right)  =x$ for every $x\in K$. Assume that%
\begin{equation}
\lambda^{k}\left(  x+y\right)  =\sum_{i=0}^{k}\lambda^{i}\left(  x\right)
\lambda^{k-i}\left(  y\right)  \ \ \ \ \ \ \ \ \ \ \text{for every }%
k\in\mathbb{N},\text{ }x\in K\text{ and }y\in K. \label{lambda1}%
\end{equation}
Then, we call $\left(  K,\left(  \lambda^{i}\right)  _{i\in\mathbb{N}}\right)
$ a $\lambda$\textit{-ring}. We will also call $K$ itself a $\lambda$-ring if
there is an obvious (from the context) choice of the sequence of mappings
$\left(  \lambda^{i}\right)  _{i\in\mathbb{N}}$ which makes $\left(  K,\left(
\lambda^{i}\right)  _{i\in\mathbb{N}}\right)  $ a $\lambda$-ring.

\textbf{2)} Let $\left(  K,\left(  \lambda^{i}\right)  _{i\in\mathbb{N}%
}\right)  $ and $\left(  L,\left(  \mu^{i}\right)  _{i\in\mathbb{N}}\right)  $
be two $\lambda$-rings. Let $f:K\rightarrow L$ be a map. Then, $f$ is called a
$\lambda$\textit{-ring homomorphism} (or \textit{homomorphism of }$\lambda
$\textit{-rings}) if and only if $f$ is a ring homomorphism and satisfies
$\mu^{i}\circ f=f\circ\lambda^{i}$ for every $i\in\mathbb{N}$.

\textbf{3)} Let $\left(  K,\left(  \lambda^{i}\right)  _{i\in\mathbb{N}%
}\right)  $ be a $\lambda$-ring. Let $L$ be a subring of $K$. Then, $L$ is
said to be a \textit{sub-}$\lambda$\textit{-ring} of $\left(  K,\left(
\lambda^{i}\right)  _{i\in\mathbb{N}}\right)  $ if and only if $\lambda
^{i}\left(  L\right)  \subseteq L$ for every $i\in\mathbb{N}$. Obviously, if
$L$ is a sub-$\lambda$-ring of $\left(  K,\left(  \lambda^{i}\right)
_{i\in\mathbb{N}}\right)  ,$ then $\left(  L,\left(  \lambda^{i}\mid
_{L}\right)  _{i\in\mathbb{N}}\right)  $ is a $\lambda$-ring, and the
canonical inclusion $L\rightarrow K$ is a $\lambda$-ring homomorphism.
\end{quote}

First, we give an alternative characterization of $\lambda$-rings:

\begin{quote}
\textbf{Theorem 2.1.} Let $K$ be a ring. Let $\lambda^{i}:K\rightarrow K$ be a
mapping\footnote{Here, "mapping" actually means "mapping" and not "group
homomorphism" of "ring homomorphism".} for every $i\in\mathbb{N}$ such that
$\lambda^{0}\left(  x\right)  =1$ and $\lambda^{1}\left(  x\right)  =x$ for
every $x\in K$. Consider the ring $K\left[  \left[  T\right]  \right]  $ of
formal series in the indeterminate $T$ over the ring $K.$ Define a map
$\lambda_{T}:K\rightarrow K\left[  \left[  T\right]  \right]  $ by
$\lambda_{T}\left(  x\right)  =\sum\limits_{i\in\mathbb{N}}\lambda^{i}\left(
x\right)  T^{i}$ for every $x\in K$. Note that the power series $\lambda
_{T}\left(  x\right)  =\sum\limits_{i\in\mathbb{N}}\lambda^{i}\left(
x\right)  T^{i}$ has the coefficient $\lambda^{0}\left(  x\right)  =1$ before
$T^{0};$ thus, it is invertible in $K\left[  \left[  T\right]  \right]  $.

\textbf{(a)} Then, $\lambda_{T}\left(  x\right)  \cdot\lambda_{T}\left(
y\right)  =\lambda_{T}\left(  x+y\right)  $ for every $x\in K$ and every $y\in
K$ if and only if $\left(  K,\left(  \lambda^{i}\right)  _{i\in\mathbb{N}%
}\right)  $ is a $\lambda$-ring.

\textbf{(b)} Let $\left(  K,\left(  \lambda^{i}\right)  _{i\in\mathbb{N}%
}\right)  $ be a $\lambda$-ring. Then,%
\begin{align*}
\lambda_{T}\left(  0\right)   &  =1;\\
\lambda_{T}\left(  -x\right)   &  =\left(  \lambda_{T}\left(  x\right)
\right)  ^{-1}\ \ \ \ \ \ \ \ \ \ \text{for every }x\in K;\\
\lambda_{T}\left(  x\right)  \cdot\lambda_{T}\left(  y\right)   &
=\lambda_{T}\left(  x+y\right)  \ \ \ \ \ \ \ \ \ \ \text{for every }x\in
K\text{ and every }y\in K.
\end{align*}


\textbf{(c)} Let $K$ and $L$ be two $\lambda$-rings. Let $f:K\rightarrow L$ be
a ring homomorphism. Consider the rings $K\left[  \left[  T\right]  \right]  $
and $L\left[  \left[  T\right]  \right]  $. Obviously, the homomorphism $f$
induces a homomorphism $f\left[  \left[  T\right]  \right]  :K\left[  \left[
T\right]  \right]  \rightarrow L\left[  \left[  T\right]  \right]  $ (defined
by $\left(  f\left[  \left[  T\right]  \right]  \right)  \left(
\sum\limits_{i\in\mathbb{N}}a_{i}T^{i}\right)  =\sum\limits_{i\in\mathbb{N}%
}f\left(  a_{i}\right)  T^{i}$ for every $\sum\limits_{i\in\mathbb{N}}%
a_{i}T^{i}\in K\left[  \left[  T\right]  \right]  $ with $a_{i}\in K$).

Then, $f$ is a $\lambda$-ring homomorphism if and only if $\mu_{T}\circ
f=f\left[  \left[  T\right]  \right]  \circ\lambda_{T}.$
\end{quote}

\textit{Proof of Theorem 2.1.} \textbf{(a)} Every $x\in K$ and every $y\in K$
satisfy%
\[
\lambda_{T}\left(  x\right)  \cdot\lambda_{T}\left(  y\right)  =\sum
\limits_{i\in\mathbb{N}}\lambda^{i}\left(  x\right)  T^{i}\cdot\sum
\limits_{i\in\mathbb{N}}\lambda^{i}\left(  y\right)  T^{i}=\sum\limits_{i\in
\mathbb{N}}\sum\limits_{j\in\mathbb{N}}\lambda^{i}\left(  x\right)
\lambda^{j}\left(  y\right)  T^{i+j}=\sum_{k\in\mathbb{N}}\sum_{i=0}%
^{k}\lambda^{i}\left(  x\right)  \lambda^{k-i}\left(  y\right)  \cdot T^{k}%
\]
and%
\[
\lambda_{T}\left(  x+y\right)  =\sum_{i\in\mathbb{N}}\lambda^{i}\left(
x+y\right)  T^{i}=\sum_{k\in\mathbb{N}}\lambda^{k}\left(  x+y\right)  T^{k}.
\]
Hence, $\lambda_{T}\left(  x\right)  \cdot\lambda_{T}\left(  y\right)
=\lambda_{T}\left(  x+y\right)  $ is equivalent to $\sum\limits_{k\in
\mathbb{N}}\sum\limits_{i=0}^{k}\lambda^{i}\left(  x\right)  \lambda
^{k-i}\left(  y\right)  \cdot T^{k}=\sum\limits_{k\in\mathbb{N}}\lambda
^{k}\left(  x+y\right)  T^{k}$, which, in turn, means that every
$k\in\mathbb{N}$ satisfies $\sum\limits_{i=0}^{k}\lambda^{i}\left(  x\right)
\lambda^{k-i}\left(  y\right)  =\lambda^{k}\left(  x+y\right)  $, and this is
exactly the property (\ref{lambda1}) from the definition of a $\lambda$-ring.
Thus, $\lambda_{T}\left(  x\right)  \cdot\lambda_{T}\left(  y\right)
=\lambda_{T}\left(  x+y\right)  $ for every $x\in K$ and every $y\in K$ if and
only if $\left(  K,\left(  \lambda^{i}\right)  _{i\in\mathbb{N}}\right)  $ is
a $\lambda$-ring. This proves Theorem 2.1 \textbf{(a)}.

\textbf{(b)} Part \textbf{(a)} tells us that $\lambda_{T}\left(  x\right)
\cdot\lambda_{T}\left(  y\right)  =\lambda_{T}\left(  x+y\right)  $ for every
$x\in K$ and every $y\in K$. Applied to $x=y=0,$ this takes the form
$\lambda_{T}\left(  0\right)  \cdot\lambda_{T}\left(  0\right)  =\lambda
_{T}\left(  0\right)  ,$ what yields $\lambda_{T}\left(  0\right)  =1$ (since
$\lambda_{T}\left(  0\right)  $ is invertible in $K\left[  \left[  T\right]
\right]  $). Now, $\lambda_{T}\left(  x\right)  \cdot\lambda_{T}\left(
y\right)  =\lambda_{T}\left(  x+y\right)  $, applied to $y=-x$, yields
$\lambda_{T}\left(  x\right)  \cdot\lambda_{T}\left(  -x\right)
=\underbrace{\lambda_{T}\left(  0\right)  }_{=1}$, hence $\lambda_{T}\left(
-x\right)  =\left(  \lambda_{T}\left(  x\right)  \right)  ^{-1}$, and Theorem
2.1 \textbf{(b)} is proven.

\textbf{(c)} We have $\left(  \mu_{T}\circ f\right)  \left(  x\right)
=\sum\limits_{i\in\mathbb{N}}\mu^{i}\left(  f\left(  x\right)  \right)  T^{i}$
and $\left(  f\left[  \left[  T\right]  \right]  \circ\lambda_{T}\right)
\left(  x\right)  =\left(  f\left[  \left[  T\right]  \right]  \right)
\left(  \sum\limits_{i\in\mathbb{N}}\lambda^{i}\left(  x\right)  T^{i}\right)
=\sum\limits_{i\in\mathbb{N}}f\left(  \lambda^{i}\left(  x\right)  \right)
T^{i}$ for every $x\in K$. Hence, $\mu_{T}\circ f=f\left[  \left[  T\right]
\right]  \circ\lambda_{T}$ is equivalent to $\sum\limits_{i\in\mathbb{N}}%
\mu^{i}\left(  f\left(  x\right)  \right)  T^{i}=\sum\limits_{i\in\mathbb{N}%
}f\left(  \lambda^{i}\left(  x\right)  \right)  T^{i}$ for every $x\in K,$
which in turn is equivalent to $\mu^{i}\left(  f\left(  x\right)  \right)
=f\left(  \lambda^{i}\left(  x\right)  \right)  $ for every $x\in K$ and every
$i\in\mathbb{N}$, which in turn means that $\mu^{i}\circ f=f\circ\lambda^{i}$
for every $i\in\mathbb{N}$. This proves Theorem 2.1 \textbf{(c)}.

\begin{quotation}
\textit{Exercise 2.1.} Let $\left(  K,\left(  \lambda^{i}\right)
_{i\in\mathbb{N}}\right)  $ and $\left(  L,\left(  \mu^{i}\right)
_{i\in\mathbb{N}}\right)  $ be two $\lambda$-rings. Let $f:K\rightarrow L$ be
a ring homomorphism. Let $E$ be a generating set of the $\mathbb{Z}$-module
$K$.

\textbf{(a)} Prove that $f$ is a $\lambda$-ring homomorphism if and only if
every $e\in E$ satisfies $\left(  \mu_{T}\circ f\right)  \left(  e\right)
=\left(  f\left[  \left[  T\right]  \right]  \circ\lambda_{T}\right)  \left(
e\right)  $.

\textbf{(b)} Prove that $f$ is a $\lambda$-ring homomorphism if and only if
every $e\in E$ satisfies $\left(  \mu^{i}\circ f\right)  \left(  e\right)
=\left(  f\circ\lambda^{i}\right)  \left(  e\right)  $ for every
$i\in\mathbb{N}$.

\textit{Exercise 2.2.} Let $\left(  K,\left(  \lambda^{i}\right)
_{i\in\mathbb{N}}\right)  $ be a $\lambda$-ring. Let $L$ be a subset of $K$
which is closed under addition, multiplication and the maps $\lambda^{i}$.
Assume that $0\in L$ and $1\in L$. Then, the subset $L-L$ of $K$ (this subset
$L-L$ is defined by $L-L=\left\{  \ell-\ell^{\prime}\mid\ell\in L,\ \ell
^{\prime}\in L\right\}  $) is a sub-$\lambda$-ring of $K$.
\end{quotation}

\begin{center}
\fbox{\textbf{3. Examples of }$\lambda$\textbf{-rings}}
\end{center}

Before we go deeper into the theory, it is time for some examples.

Obviously, the trivial ring $0$ (the ring satisfying $0=1$) along with the
trivial maps $\lambda^{i}:0\rightarrow0$ is a $\lambda$-ring. Let us move on
to more surprising examples:

\begin{quote}
\textbf{Theorem 3.1.} For every $i\in\mathbb{N}$, define a map $\lambda
^{i}:\mathbb{Z}\rightarrow\mathbb{Z}$ by $\lambda^{i}\left(  x\right)
=\dbinom{x}{i}$ for every $x\in\mathbb{Z}$.\ \ \ \ \footnote{Note that
$\dbinom{x}{i}$ is defined to be $\dfrac{x\cdot\left(  x-1\right)
\cdot...\cdot\left(  x-i+1\right)  }{i!}$ for every $x\in\mathbb{R}$ and
$i\in\mathbb{N}$.} Then, $\left(  \mathbb{Z},\left(  \lambda^{i}\right)
_{i\in\mathbb{N}}\right)  $ is a $\lambda$-ring.
\end{quote}

\textit{Proof of Theorem 3.1.} Trivially, $\lambda^{0}\left(  x\right)  =1$
and $\lambda^{1}\left(  x\right)  =x$ for every $x\in\mathbb{Z}$. The only
challenge, if there is a challenge in this proof, is to verify the identity
(\ref{lambda1}) for $K=\mathbb{Z}$. In other words, we have to prove that%
\begin{equation}
\dbinom{x+y}{k}=\sum_{i=0}^{k}\dbinom{x}{i}\dbinom{y}{k-i} \label{vandermonde}%
\end{equation}
for every $k\in\mathbb{N},$ $x\in\mathbb{Z}$ and $y\in\mathbb{Z}.$ This is the
so-called \textit{Vandermonde convolution identity}, and probably its shortest
proof is to fix $k\in\mathbb{N}$, then notice that it is a polynomial identity
in both $x$ and $y$, so it is enough to prove it for all natural $x$ and $y$
(because a polynomial identity holding for all natural variables must hold
everywhere). But for $x$ and $y$ natural, we have%
\[
\sum_{k=0}^{x+y}\sum_{i=0}^{k}\dbinom{x}{i}\dbinom{y}{k-i}T^{k}=\underbrace
{\sum_{i=0}^{x}\dbinom{x}{i}T^{i}}_{\substack{=\left(  1+T\right)  ^{x}\text{
by the}\\\text{binomial formula}}}\cdot\underbrace{\sum_{j=0}^{y}\dbinom{y}%
{j}T^{j}}_{\substack{=\left(  1+T\right)  ^{y}\text{ by the}\\\text{binomial
formula}}}=\left(  1+T\right)  ^{x+y}=\sum_{k=0}^{x+y}\dbinom{x+y}{k}T^{k}%
\]
in the polynomial ring $\mathbb{Z}\left[  T\right]  $, and thus
(\ref{vandermonde}) follows by comparison of
coefficients.\footnote{\textit{Remark.} It is tempting to apply this argument
to the general case (where $x$ and $y$ are not required to be natural),
because the binomial formula holds for negative exponents as well (of course,
this requires working in the formal series ring $\mathbb{Z}\left[  \left[
T\right]  \right]  $ rather than in the polynomial ring $\mathbb{Z}\left[
T\right]  $), but I am not sure whether this argument is free of circular
reasoning because it is not at all obvious that $\left(  1+T\right)
^{x}\left(  1+T\right)  ^{y}=\left(  1+T\right)  ^{x+y}$ in $\mathbb{Z}\left[
\left[  T\right]  \right]  $ for negative $x$ and $y$, and I even fear that
this is usually proven using (\ref{vandermonde}).} This proves Theorem 3.1.

Our next example is a generalization of Theorem 3.1:

\begin{quote}
\textbf{Definition.} Let $K$ be a ring. We call $K$ a \textit{binomial ring}
if and only if none of the elements $1,$ $2,$ $3,$ $...$ is a zero-divisor in
$K$, and $n!\mid x\cdot\left(  x-1\right)  \cdot...\cdot\left(  x-i+1\right)
$ for every $x\in K$ and every $n\in\mathbb{N}$.

\textbf{Theorem 3.2.} Let $K$ be a binomial ring. For every $i\in\mathbb{N}$,
define a map $\lambda^{i}:K\rightarrow K$ by $\lambda^{i}\left(  x\right)
=\dbinom{x}{i}$ for every $x\in K$ (where, again, $\dbinom{x}{i}$ is defined
to be $\dfrac{x\cdot\left(  x-1\right)  \cdot...\cdot\left(  x-i+1\right)
}{i!}$). Then, $\left(  K,\left(  \lambda^{i}\right)  _{i\in\mathbb{N}%
}\right)  $ is a $\lambda$-ring.
\end{quote}

Such $\lambda$-rings $K$ are called \textit{binomial }$\lambda$\textit{-rings}.

\textit{Proof of Theorem 3.2.} Obviously, $\lambda^{0}\left(  x\right)  =1$
and $\lambda^{1}\left(  x\right)  =x$ for every $x\in K$, so it only remains
to show that (\ref{lambda1}) is satisfied. This means proving
(\ref{vandermonde}) for every $k\in\mathbb{N},$ $x\in K$ and $y\in K$. But
this is easy now: Fix $k\in\mathbb{N}$. Then, (\ref{vandermonde}) is a
polynomial identity in both $x$ and $y$, and since we know that it holds for
every $x\in\mathbb{Z}$ and every $y\in\mathbb{Z}$ (as we have seen in the
proof of Theorem 3.1), it follows that it holds for every $x\in K$ and every
$y\in K$ (since a polynomial identity holding for all integer variables must
hold everywhere). This completes the proof of Theorem 3.2.

Obviously, the $\lambda$-ring $\left(  \mathbb{Z},\left(  \lambda^{i}\right)
_{i\in\mathbb{N}}\right)  $ defined in Theorem 3.1 is a binomial $\lambda
$-ring. For other examples of binomial $\lambda$-rings, see Exercise 3.1. Of
course, every $\mathbb{Q}$-algebra is a binomial ring as well.

Binomial $\lambda$-rings are not the main examples of $\lambda$-rings. We will
see an important example of $\lambda$-rings in Theorem 5.1 and Exercise 6.1.
Another simple way to construct new examples from known ones is the following one:

\begin{quote}
\textbf{Definition.} Let $K$ be a ring. Let $L$ be a $K$-algebra. Consider the
ring $K\left[  \left[  T\right]  \right]  $ of formal power series in the
indeterminate $T$ over the ring $K$, and the ring $L\left[  \left[  T\right]
\right]  $ of formal power series in the indeterminate $T$ over the ring $L$.
For every $\mu\in L$, we can define a $K$-algebra homomorphism
$\operatorname*{ev}_{\mu T}:K\left[  \left[  T\right]  \right]  \rightarrow
L\left[  \left[  T\right]  \right]  $ by setting $\operatorname*{ev}_{\mu
T}\left(  \sum\limits_{i\in\mathbb{N}}a_{i}T^{i}\right)  =\sum\limits_{i\in
\mathbb{N}}a_{i}\mu^{i}T^{i}$ for every power series $\sum\limits_{i\in
\mathbb{N}}a_{i}T^{i}\in K\left[  \left[  T\right]  \right]  $ (which
satisfies $a_{i}\in K$ for every $i\in\mathbb{N}$). (In other words,
$\operatorname*{ev}_{\mu T}$ is the map that takes any power series in $T$ and
replaces every $T$ in this power series by $\mu T$.)

\textbf{Theorem 3.3.} Let $\left(  K,\left(  \lambda^{i}\right)
_{i\in\mathbb{N}}\right)  $ be a $\lambda$-ring. Consider the polynomial ring
$K\left[  S\right]  $. For every $i\in\mathbb{N}$, define a map $\overline
{\lambda}^{i}:K\left[  S\right]  \rightarrow K\left[  S\right]  $ as follows:
For every $\sum\limits_{j\in\mathbb{N}}a_{j}S^{j}\in K\left[  S\right]  $
(with $a_{j}\in K$ for every $j\in\mathbb{N}$), let $\overline{\lambda}%
^{i}\left(  \sum\limits_{j\in\mathbb{N}}a_{j}S^{j}\right)  $ be the
coefficient of the power series $\prod\limits_{j\in\mathbb{N}}\lambda_{S^{j}%
T}\left(  a_{j}\right)  \in\left(  K\left[  S\right]  \right)  \left[  \left[
T\right]  \right]  $ before $T^{i}$, where the power series $\lambda_{S^{j}%
T}\left(  a_{j}\right)  \in\left(  K\left[  S\right]  \right)  \left[  \left[
T\right]  \right]  $ is defined as $\operatorname*{ev}_{S^{j}T}\left(
\lambda_{T}\left(  a_{j}\right)  \right)  $.

\textbf{(a)} Then, $\left(  K\left[  S\right]  ,\left(  \overline{\lambda}%
^{i}\right)  _{i\in\mathbb{N}}\right)  $ is a $\lambda$-ring. The ring $K$ is
a sub-$\lambda$-ring of $\left(  K\left[  S\right]  ,\left(  \overline
{\lambda}^{i}\right)  _{i\in\mathbb{N}}\right)  $. Besides, $\overline
{\lambda}^{i}\left(  S\right)  =0$ for every $i>0$.

\textbf{(b)} For every $a\in K$ and $\alpha\in\mathbb{N}$, we have
$\overline{\lambda}^{i}\left(  aS^{\alpha}\right)  =\lambda^{i}\left(
a\right)  S^{\alpha i}$ for every $i\in\mathbb{N}$.
\end{quote}

\textit{Proof of Theorem 3.3.} \textbf{(a)} Define a map $\overline{\lambda
}_{T}:K\left[  S\right]  \rightarrow\left(  K\left[  S\right]  \right)
\left[  \left[  T\right]  \right]  $ by $\overline{\lambda}_{T}\left(
u\right)  =\sum\limits_{i\in\mathbb{N}}\overline{\lambda}^{i}\left(  u\right)
T^{j}$ for every $u\in K\left[  S\right]  $. Then, according to the definition
of the maps $\overline{\lambda}^{i}$, we have $\overline{\lambda}_{T}\left(
\sum\limits_{j\in\mathbb{N}}a_{j}S^{j}\right)  =\prod\limits_{j\in\mathbb{N}%
}\lambda_{S^{j}T}\left(  a_{j}\right)  \in\left(  K\left[  S\right]  \right)
\left[  \left[  T\right]  \right]  $ for every $\sum\limits_{j\in\mathbb{N}%
}a_{j}S^{j}\in K\left[  S\right]  $ (with $a_{j}\in K$ for every
$j\in\mathbb{N}$).

But by Theorem 2.1 \textbf{(a)}, proving that $\left(  K\left[  S\right]
,\left(  \overline{\lambda}^{i}\right)  _{i\in\mathbb{N}}\right)  $ is a
$\lambda$-ring boils down to verifying that%
\[
\overline{\lambda}_{T}\left(  \sum\limits_{j\in\mathbb{N}}a_{j}S^{j}%
+\sum\limits_{j\in\mathbb{N}}b_{j}S^{j}\right)  =\overline{\lambda}_{T}\left(
\sum\limits_{j\in\mathbb{N}}a_{j}S^{j}\right)  \cdot\overline{\lambda}%
_{T}\left(  \sum\limits_{j\in\mathbb{N}}b_{j}S^{j}\right)
\]
for every $\sum\limits_{j\in\mathbb{N}}a_{j}S^{j}\in K\left[  S\right]  $ and
$\sum\limits_{j\in\mathbb{N}}b_{j}S^{j}\in K\left[  S\right]  $. This is
clear, because%
\begin{align*}
\overline{\lambda}_{T}\left(  \sum\limits_{j\in\mathbb{N}}a_{j}S^{j}%
+\sum\limits_{j\in\mathbb{N}}b_{j}S^{j}\right)   &  =\overline{\lambda}%
_{T}\left(  \sum\limits_{j\in\mathbb{N}}\left(  a_{j}+b_{j}\right)
S^{j}\right)  =\prod\limits_{j\in\mathbb{N}}\underbrace{\lambda_{S^{j}%
T}\left(  a_{j}+b_{j}\right)  }_{\substack{=\lambda_{S^{j}T}\left(
a_{j}\right)  \cdot\lambda_{S^{j}T}\left(  b_{j}\right)  ,\\\text{since
}\lambda_{T}\left(  a_{j}+b_{j}\right)  =\lambda_{T}\left(  a_{j}\right)
\cdot\lambda_{T}\left(  b_{j}\right)  \\\text{by Theorem 2.1 \textbf{(a)}}}}\\
&  =\prod\limits_{j\in\mathbb{N}}\lambda_{S^{j}T}\left(  a_{j}\right)
\cdot\prod\limits_{j\in\mathbb{N}}\lambda_{S^{j}T}\left(  b_{j}\right)
=\overline{\lambda}_{T}\left(  \sum\limits_{j\in\mathbb{N}}a_{j}S^{j}\right)
\cdot\overline{\lambda}_{T}\left(  \sum\limits_{j\in\mathbb{N}}b_{j}%
S^{j}\right)  .
\end{align*}
Thus, $\left(  K\left[  S\right]  ,\left(  \overline{\lambda}^{i}\right)
_{i\in\mathbb{N}}\right)  $ is a $\lambda$-ring. The rest of Theorem 3.3
\textbf{(a)} is yet easier to verify.

\textbf{(b)} We have $\overline{\lambda}_{T}\left(  aS^{\alpha}\right)
=\lambda_{S^{\alpha}T}\left(  a\right)  $ as a particular case of
$\overline{\lambda}_{T}\left(  \sum\limits_{j\in\mathbb{N}}a_{j}S^{j}\right)
=\prod\limits_{j\in\mathbb{N}}\lambda_{S^{j}T}\left(  a_{j}\right)  $. The
equation $\overline{\lambda}^{i}\left(  aS^{\alpha}\right)  =\lambda
^{i}\left(  a\right)  S^{\alpha i}$ for every $i\in\mathbb{N}$ follows by
comparing coefficients in $\overline{\lambda}_{T}\left(  aS^{\alpha}\right)
=\lambda_{S^{\alpha}T}\left(  a\right)  $. Thus, Theorem 3.3 \textbf{(b)} is proven.

The exercises below give some more examples.

\begin{quotation}
\textit{Exercise 3.1.} Let $p\in\mathbb{N}$ be a prime. Prove that the
localization $\left\{  1,p,p^{2},...\right\}  ^{-1}\mathbb{Z}$ of the ring
$\mathbb{Z}$ at the multiplicative subset $\left\{  1,p,p^{2},...\right\}  $
is a binomial ring.

{\small [One of the reasons I include this exercise here that I am looking for
a nice solution.]}

\textit{Exercise 3.2.} Let $K$ be a ring where none of the elements $1,$ $2,$
$3,$ $...$ is a zero-divisor. Let $E$ be a subset of $K$ that generates $K$ as
a ring. Assume that $n!\mid x\cdot\left(  x-1\right)  \cdot...\cdot\left(
x-i+1\right)  $ for every $x\in E$ and every $n\in\mathbb{N}$. Prove that $K$
is a binomial ring.

{\small [Again, I don't really like the solution that I have.]}

\textit{Exercise 3.3.} \textbf{(a)} Let $K$ be a binomial ring. Let $p\in
K\left[  \left[  T\right]  \right]  $ be a formal power series with
coefficient $1$ before $T^{0}$ (we will later denote the set of such power
series by $1+K\left[  \left[  T\right]  \right]  ^{+}$). For every
$i\in\mathbb{N}$, define a map $\lambda^{i}:K\rightarrow K$ as follows: For
every $x\in K$, let $\lambda^{i}\left(  x\right)  $ be the coefficient of the
formal power series $\left(  1+pT\right)  ^{x}$ (which is defined as
$\sum\limits_{k\in\mathbb{N}}\dbinom{x}{k}\left(  pT\right)  ^{k}%
\ \ \ \ $\footnote{If $K$ is a $\mathbb{Q}$-algebra, then this power series
also equals $\exp\left(  x\log\left(  1+pT\right)  \right)  $, where
$\log\left(  1+T\right)  $ is the power series $\log\left(  1+T\right)
=\sum\limits_{i\in\mathbb{N}\setminus\left\{  0\right\}  }\dfrac{\left(
-1\right)  ^{i-1}}{i}T^{i}$.}) before $T^{i}$. Prove that $\left(  K,\left(
\lambda^{i}\right)  _{i\in\mathbb{N}}\right)  $ is a $\lambda$-ring.

\textbf{(b)} If $p=1$, prove that this $\lambda$-ring is the one defined in
Theorem 3.2.

\textit{Exercise 3.4.} Let $M$ be a commutative monoid with neutral element,
written multiplicatively (this means, in particular, that we denote the
neutral element of $M$ as $1$). Define a $\mathbb{Z}$-algebra $\mathbb{Z}%
\left[  M\right]  $ as follows:

As a $\mathbb{Z}$-module, let $\mathbb{Z}\left[  M\right]  $ be the free
$\mathbb{Z}$-module with the basis $M$. Let the multiplication on
$\mathbb{Z}\left[  M\right]  $ be the $\mathbb{Z}$-linear extension of the
multiplication on the monoid $M$.

For every $i\in\mathbb{N}$, define a map $\lambda^{i}:\mathbb{Z}\left[
M\right]  \rightarrow\mathbb{Z}\left[  M\right]  $ as follows: For every
$\sum\limits_{m\in M}\alpha_{m}m\in\mathbb{Z}\left[  M\right]  $ (with
$\alpha_{m}\in\mathbb{Z}$ for every $m\in M$), let $\lambda^{i}\left(
\sum\limits_{m\in M}\alpha_{m}m\right)  $ be the coefficient of the power
series $\prod\limits_{m\in M}\left(  1+mT\right)  ^{\alpha_{m}}\in\left(
\mathbb{Z}\left[  M\right]  \right)  \left[  \left[  T\right]  \right]  $
before $T^{i}$.

Prove that $\left(  \mathbb{Z}\left[  M\right]  ,\left(  \lambda^{i}\right)
_{i\in\mathbb{N}}\right)  $ is a $\lambda$-ring.

\textit{Exercise 3.5.} \textbf{(a)} Let $M$ be a commutative monoid with
neutral element, written multiplicatively (this means, in particular, that we
denote the neutral element of $M$ as $1$). Let $\left(  K,\left(  \lambda
^{i}\right)  _{i\in\mathbb{N}}\right)  $ be a $\lambda$-ring.

Define a $K$-algebra $K\left[  M\right]  $ as follows:

As a $K$-module, let $K\left[  M\right]  $ be the free $K$-module with the
basis $M$. Let the multiplication on $K\left[  M\right]  $ be the $K$-linear
extension of the multiplication on the monoid $M$.

For every $i\in\mathbb{N}$, define a map $\overline{\lambda}^{i}:K\left[
M\right]  \rightarrow K\left[  M\right]  $ as follows: For every
$\sum\limits_{m\in M}\alpha_{m}m\in K\left[  M\right]  $ (with $\alpha_{m}\in
K$ for every $m\in M$), let $\overline{\lambda}^{i}\left(  \sum\limits_{m\in
M}\alpha_{m}m\right)  $ be the coefficient of the power series $\prod
\limits_{m\in M}\lambda_{mT}\left(  \alpha_{m}\right)  \in\left(  K\left[
M\right]  \right)  \left[  \left[  T\right]  \right]  $ before $T^{i}$, where
the power series $\lambda_{mT}\left(  \alpha_{m}\right)  \in\left(  K\left[
M\right]  \right)  \left[  \left[  T\right]  \right]  $ is defined as
$\operatorname*{ev}_{mT}\left(  \lambda_{T}\left(  \alpha_{m}\right)  \right)
$.

Prove that $\left(  K\left[  M\right]  ,\left(  \overline{\lambda}^{i}\right)
_{i\in\mathbb{N}}\right)  $ is a $\lambda$-ring. The ring $K$ is a
sub-$\lambda$-ring of $\left(  K\left[  M\right]  ,\left(  \overline{\lambda
}^{i}\right)  _{i\in\mathbb{N}}\right)  $. Besides, $\overline{\lambda}%
^{i}\left(  m\right)  =0$ for every $m\in M$ and $i>0$.

\textbf{(b)} Show that Exercise 3.4 is a particular case of \textbf{(a)} for
$K=\mathbb{Z}$, and that Theorem 3.3 \textbf{(a)} is a particular case of
\textbf{(a)} for $M\cong\mathbb{N}$ (where $\mathbb{N}$ denotes the
\textit{additive} monoid $\mathbb{N}$).
\end{quotation}

\begin{center}
\fbox{\textbf{4. Intermezzo: Symmetric polynomials}}
\end{center}

Our next plan is to introduce a rather general example of $\lambda$-rings that
we will use as a prototype to the notion of \textit{special }$\lambda
$\textit{-rings}. Before we do this, we need some rather clumsy theory of
symmetric polynomials. In case you can take the proofs for granted, you don't
need to read much of this paragraph - you only need to know Theorems 4.3 and
4.4 and the preceeding definitions (only the goals of the definitions; not the
actual constructions of the polynomials $P_{k}$ and $P_{k,j}$).

\begin{quote}
\textbf{Theorem 4.1 (characterization of symmetric polynomials).} Let $K$ be a
ring. Let $m\in\mathbb{N}$. Consider the ring $K\left[  U_{1},U_{2}%
,...,U_{m}\right]  $ (the polynomial ring in $m$ indeterminates $U_{1},$
$U_{2},$ $...,$ $U_{m}$ over the ring $K$). For every $i\in\mathbb{N}$, let
$X_{i}=\sum\limits_{\substack{S\subseteq\left\{  1,2,...,m\right\}
;\\\left\vert S\right\vert =i}}\prod\limits_{k\in S}U_{k}$ be the so-called
$i$\textit{-th elementary symmetric polynomial} in the variables $U_{1},$
$U_{2},$ $...,$ $U_{m}$. (In particular, $X_{0}=1$ and $X_{i}=0$ for every
$i>m$.)

A polynomial $P\in K\left[  U_{1},U_{2},...,U_{m}\right]  $ is called
\textit{symmetric} if it satisfies $P\left(  U_{1},U_{2},...,U_{m}\right)
=P\left(  U_{\pi\left(  1\right)  },U_{\pi\left(  2\right)  },...,U_{\pi
\left(  m\right)  }\right)  $ for every permutation $\pi$ of the set $\left\{
1,2,...,m\right\}  $.

\textbf{(a)} Let $P\in K\left[  U_{1},U_{2},...,U_{m}\right]  $ be a symmetric
polynomial. Then, there exists one and only one polynomial $Q\in
\underbrace{K\left[  \alpha_{1},\alpha_{2},...,\alpha_{m}\right]
}_{\text{polynomial ring}}$ such that $P\left(  U_{1},U_{2},...,U_{m}\right)
=Q\left(  X_{1},X_{2},...,X_{m}\right)  $.\ \ \ \ \footnote{In other words,
the subring%
\[
\left\{  P\in K\left[  U_{1},U_{2},...,U_{m}\right]  \ \mid\ P\text{ is
symmetric}\right\}
\]
of the polynomial ring $K\left[  U_{1},U_{2},...,U_{m}\right]  $ is generated
by the elements $X_{1},$ $X_{2},$ $...,$ $X_{m}$. Moreover, these elements
$X_{1},$ $X_{2},$ $...,$ $X_{m}$ are algebraically independent, so that the
map%
\[
\underbrace{K\left[  \alpha_{1},\alpha_{2},...,\alpha_{m}\right]
}_{\text{polynomial ring}}\rightarrow\left\{  P\in K\left[  U_{1}%
,U_{2},...,U_{m}\right]  \ \mid\ P\text{ is symmetric}\right\}
\]
which maps every $\alpha_{i}$ to $U_{i}$ is a bijection.}

\textbf{(b)} Let $\ell\in\mathbb{N}$. Assume, moreover, that $P\in K\left[
U_{1},U_{2},...,U_{m}\right]  $ is a symmetric polynomial of total degree
$\leq\ell$ in the variables $U_{1},$ $U_{2},$ $...,$ $U_{m}$. Then, the
variables $\alpha_{i}$ for $i>\ell$ do not appear in the polynomial $Q$.

There is a canonical homomorphism $K\left[  \alpha_{1},\alpha_{2}%
,...,\alpha_{m}\right]  \rightarrow K\left[  \alpha_{1},\alpha_{2}%
,...,\alpha_{\ell}\right]  $ (which maps every $\alpha_{i}$ to $\left\{
\begin{array}
[c]{c}%
\alpha_{i},\text{ if }i\leq\ell;\\
0,\text{ if }i>\ell
\end{array}
\right.  $)\ \ \ \ \footnote{This homomorphism is a surjection if $\ell\leq m$
and an injection if $\ell\geq m$.}. If we denote by $Q_{\ell}$ the image of
$Q\in K\left[  \alpha_{1},\alpha_{2},...,\alpha_{m}\right]  $ under this
homomorphism, then, $P\left(  U_{1},U_{2},...,U_{m}\right)  =Q\left(
X_{1},X_{2},...,X_{m}\right)  =Q_{\ell}\left(  X_{1},X_{2},...,X_{\ell
}\right)  $.
\end{quote}

We are not going to prove Theorem 4.1 here, since pretty much every good
algebra book does it.\footnote{Only one remark about Theorem 4.1 \textbf{(b)}:
We have $Q\left(  X_{1},X_{2},...,X_{m}\right)  =Q_{\ell}\left(  X_{1}%
,X_{2},...,X_{\ell}\right)  $, because the variables $\alpha_{i}$ for $i>\ell$
do not appear in the polynomial $Q$.} But we are going to extend it to two
sets of indeterminates:

\begin{quote}
\textbf{Theorem 4.2 (characterization of UV-symmetric polynomials).} Let $K$
be a ring. Let $m\in\mathbb{N}$ and $n\in\mathbb{N}$. Consider the ring
$K\left[  U_{1},U_{2},...,U_{m},V_{1},V_{2},...,V_{n}\right]  $ (the
polynomial ring in $m+n$ indeterminates $U_{1},$ $U_{2},$ $...,$ $U_{m},$
$V_{1},$ $V_{2},$ $...,$ $V_{n}$ over the ring $K$). For every $i\in
\mathbb{N}$, let $X_{i}=\sum\limits_{\substack{S\subseteq\left\{
1,2,...,m\right\}  ;\\\left\vert S\right\vert =i}}\prod\limits_{k\in S}U_{k}$
be the $i$-th elementary symmetric polynomial in the variables $U_{1},$
$U_{2},$ $...,$ $U_{m}$. For every $j\in\mathbb{N}$, let $Y_{j}=\sum
\limits_{\substack{S\subseteq\left\{  1,2,...,n\right\}  ;\\\left\vert
S\right\vert =j}}\prod\limits_{k\in S}V_{k}$ be the $j$-th elementary
symmetric polynomial in the variables $V_{1},$ $V_{2},$ $...,$ $V_{n}$.

A polynomial $P\in K\left[  U_{1},U_{2},...,U_{m},V_{1},V_{2},...,V_{n}%
\right]  $ is called \textit{UV-symmetric} if it satisfies%
\[
P\left(  U_{1},U_{2},...,U_{m},V_{1},V_{2},...,V_{n}\right)  =P\left(
U_{\pi\left(  1\right)  },U_{\pi\left(  2\right)  },...,U_{\pi\left(
m\right)  },V_{\sigma\left(  1\right)  },V_{\sigma\left(  2\right)
},...,V_{\sigma\left(  n\right)  }\right)
\]
for every permutation $\pi$ of the set $\left\{  1,2,...,m\right\}  $ and
every permutation $\sigma$ of the set $\left\{  1,2,...,n\right\}  $.

\textbf{(a)} Let $P\in K\left[  U_{1},U_{2},...,U_{m},V_{1},V_{2}%
,...,V_{n}\right]  $ be a UV-symmetric polynomial. Then, there exists one and
only one polynomial $Q\in K\left[  \alpha_{1},\alpha_{2},...,\alpha_{m}%
,\beta_{1},\beta_{2},...,\beta_{n}\right]  $ such that $P\left(  U_{1}%
,U_{2},...,U_{m},V_{1},V_{2},...,V_{n}\right)  =Q\left(  X_{1},X_{2}%
,...,X_{m},Y_{1},Y_{2},...,Y_{m}\right)  $.\ \ \ \ \footnote{In other words,
the subring%
\[
\left\{  P\in K\left[  U_{1},U_{2},...,U_{m},V_{1},V_{2},...,V_{n}\right]
\ \mid\ P\text{ is UV-symmetric}\right\}
\]
of the polynomial ring $K\left[  U_{1},U_{2},...,U_{m},V_{1},V_{2}%
,...,V_{n}\right]  $ is generated by the elements $X_{1},$ $X_{2},$ $...,$
$X_{m},$ $Y_{1},$ $Y_{2},$ $...,$ $Y_{n}$. Moreover, these elements $X_{1},$
$X_{2},$ $...,$ $X_{m},$ $Y_{1},$ $Y_{2},$ $...,$ $Y_{n}$ are algebraically
independent, so that the map%
\[
\underbrace{K\left[  \alpha_{1},\alpha_{2},...,\alpha_{m},\beta_{1},\beta
_{2},...,\beta_{n}\right]  }_{\text{polynomial ring}}\rightarrow\left\{  P\in
K\left[  U_{1},U_{2},...,U_{m},V_{1},V_{2},...,V_{n}\right]  \ \mid\ P\text{
is UV-symmetric}\right\}
\]
which maps every $\alpha_{i}$ to $X_{i}$ and every $\beta_{j}$ to $Y_{j}$ is a
bijection.}

\textbf{(b)} Let $\ell\in\mathbb{N}$ and $k\in\mathbb{N}$. Assume, moreover,
that $P\in K\left[  U_{1},U_{2},...,U_{m},V_{1},V_{2},...,V_{n}\right]  $ is a
UV-symmetric polynomial of total degree $\leq\ell$ in the variables $U_{1},$
$U_{2},$ $...,$ $U_{m}$ and of total degree $\leq k$ in the variables $V_{1},$
$V_{2},$ $...,$ $V_{n}.$ Then, neither the variables $\alpha_{i}$ for $i>\ell$
nor the variables $\beta_{j}$ for $j>k$ ever appear in the polynomial $Q$.

There is a canonical homomorphism $K\left[  \alpha_{1},\alpha_{2}%
,...,\alpha_{m},\beta_{1},\beta_{2},...,\beta_{n}\right]  \rightarrow K\left[
\alpha_{1},\alpha_{2},...,\alpha_{\ell},\beta_{1},\beta_{2},...,\beta
_{k}\right]  $ (which maps every $\alpha_{i}$ to $\left\{
\begin{array}
[c]{c}%
\alpha_{i},\text{ if }i\leq\ell;\\
0,\text{ if }i>\ell
\end{array}
\right.  $ and every $\beta_{j}$ to $\left\{
\begin{array}
[c]{c}%
\beta_{j},\text{ if }j\leq k;\\
0,\text{ if }j>k
\end{array}
\right.  $). If we denote by $Q_{\ell,k}$ the image of $Q\in K\left[
\alpha_{1},\alpha_{2},...,\alpha_{m},\beta_{1},\beta_{2},...,\beta_{n}\right]
$ under this homomorphism, then $P\left(  U_{1},U_{2},...,U_{m},V_{1}%
,V_{2},...,V_{n}\right)  =Q_{\ell,k}\left(  X_{1},X_{2},...,X_{\ell}%
,Y_{1},Y_{2},...,Y_{k}\right)  $.
\end{quote}

\textit{Proof of Theorem 4.2 (\textbf{very} roughly sketched).} \textbf{(a)}
Consider $P$ as a polynomial in the indeterminates $V_{1},$ $V_{2},$ $...,$
$V_{n}$ over the ring $K\left[  U_{1},U_{2},...,U_{m}\right]  $. Then, $P$ is
a symmetric polynomial in these indeterminates $V_{1},$ $V_{2},$ $...,$
$V_{n}$, so Theorem 4.1 \textbf{(a)} yields the existence of one and only one
polynomial $\widehat{Q}\in\underbrace{\left(  K\left[  U_{1},U_{2}%
,...,U_{m}\right]  \right)  \left[  \beta_{1},\beta_{2},...,\beta_{n}\right]
}_{\text{polynomial ring}}$ such that $P\left(  U_{1},U_{2},...,U_{m}%
,V_{1},V_{2},...,V_{n}\right)  =\widehat{Q}\left(  Y_{1},Y_{2},...,Y_{n}%
\right)  $. Now, for every $n$-tuple $\left(  \lambda_{1},\lambda
_{2},...,\lambda_{n}\right)  \in\mathbb{N}^{n}$, the coefficient of this
polynomial $\widehat{Q}$ before $\beta_{1}^{\lambda_{1}}\beta_{2}^{\lambda
_{2}}...\beta_{n}^{\lambda_{n}}$ is a symmetric polynomial in the variables
$U_{1},$ $U_{2},$ $...,$ $U_{m}\ \ \ \ $\footnote{Okay, the "symmetric" part
may need an explanation.
\par
{}In fact, let $\sigma$ be a permutation of the set $\left\{
1,2,...,m\right\}  .$
\par
Let $\rho$ be the canonical $K$-algebra isomorphism $\left(  K\left[
U_{1},U_{2},...,U_{m}\right]  \right)  \left[  \beta_{1},\beta_{2}%
,...,\beta_{n}\right]  \rightarrow\left(  K\left[  \beta_{1},\beta
_{2},...,\beta_{n}\right]  \right)  \left[  U_{1},U_{2},...,U_{m}\right]  $.
\par
Let $\widehat{Q}_{\sigma}\in\left(  K\left[  U_{1},U_{2},...,U_{m}\right]
\right)  \left[  \beta_{1},\beta_{2},...,\beta_{n}\right]  $ be the polynomial
defined by $\widehat{Q}_{\sigma}=\rho^{-1}\left(  \left(  \rho\left(
\widehat{Q}\right)  \right)  \left(  U_{\sigma\left(  1\right)  }%
,U_{\sigma\left(  2\right)  },...,U_{\sigma\left(  m\right)  }\right)
\right)  .$ Then, $P\left(  U_{1},U_{2},...,U_{m},V_{1},V_{2},...,V_{n}%
\right)  =\widehat{Q}\left(  Y_{1},Y_{2},...,Y_{n}\right)  $ yields $P\left(
U_{\sigma\left(  1\right)  },U_{\sigma\left(  2\right)  },...,U_{\sigma\left(
m\right)  },V_{1},V_{2},...,V_{n}\right)  =\widehat{Q}_{\sigma}\left(
Y_{1},Y_{2},...,Y_{n}\right)  $. But $P\left(  U_{1},U_{2},...,U_{m}%
,V_{1},V_{2},...,V_{n}\right)  =P\left(  U_{\sigma\left(  1\right)
},U_{\sigma\left(  2\right)  },...,U_{\sigma\left(  m\right)  },V_{1}%
,V_{2},...,V_{n}\right)  $ (since $P$ is UV-symmetric). Thus, $\widehat
{Q}\left(  Y_{1},Y_{2},...,Y_{n}\right)  =\widehat{Q}_{\sigma}\left(
Y_{1},Y_{2},...,Y_{n}\right)  $ for every permutation $\sigma$ of $\left\{
1,2,...,m\right\}  $. This yields $\widehat{Q}=\widehat{Q}_{\sigma}$ for every
permutation $\sigma$ of $\left\{  1,2,...,m\right\}  $ (since $Y_{1},$
$Y_{2},$ $...,$ $Y_{n}$ are algebraically independent over $K\left[
U_{1},U_{2},...,U_{m}\right]  $, as we can see by applying Theorem 4.1 to
$K\left[  U_{1},U_{2},...,U_{m}\right]  $ instead of $K$ and to $V_{1},$
$V_{2},$ $...,$ $V_{n}$ instead of $U_{1},$ $U_{2},$ $...,$ $U_{m}$). This
means that the coefficient of this polynomial $\widehat{Q}$ before $\beta
_{1}^{\lambda_{1}}\beta_{2}^{\lambda_{2}}...\beta_{n}^{\lambda_{n}}$ is a
symmetric polynomial in the variables $U_{1},$ $U_{2},$ $...,$ $U_{m}$ for
every $n$-tuple $\left(  \lambda_{1},\lambda_{2},...,\lambda_{n}\right)
\in\mathbb{N}^{n}$.}. Hence, by Theorem 4.1 \textbf{(a)}, there exists a
polynomial $R_{\left(  \lambda_{1},\lambda_{2},...,\lambda_{n}\right)  }%
\in\underbrace{K\left[  \alpha_{1},\alpha_{2},...,\alpha_{m}\right]
}_{\text{polynomial ring}}$ such that the coefficient of the polynomial
$\widehat{Q}$ before $\beta_{1}^{\lambda_{1}}\beta_{2}^{\lambda_{2}}%
...\beta_{n}^{\lambda_{n}}$ is $R_{\left(  \lambda_{1},\lambda_{2}%
,...,\lambda_{n}\right)  }\left(  X_{1},X_{2},...,X_{m}\right)  $. Thus,%
\begin{align*}
&  P\left(  U_{1},U_{2},...,U_{m},V_{1},V_{2},...,V_{n}\right)  =\widehat
{Q}\left(  Y_{1},Y_{2},...,Y_{n}\right) \\
&  =\sum_{\left(  \lambda_{1},\lambda_{2},...,\lambda_{n}\right)
\in\mathbb{N}^{n}}\left(  \text{the coefficient of the polynomial }\widehat
{Q}\text{ before }\beta_{1}^{\lambda_{1}}\beta_{2}^{\lambda_{2}}...\beta
_{n}^{\lambda_{n}}\right)  Y_{1}^{\lambda_{1}}Y_{2}^{\lambda_{2}}%
...Y_{n}^{\lambda_{n}}\\
&  =\sum_{\left(  \lambda_{1},\lambda_{2},...,\lambda_{n}\right)
\in\mathbb{N}^{n}}R_{\left(  \lambda_{1},\lambda_{2},...,\lambda_{n}\right)
}\left(  X_{1},X_{2},...,X_{m}\right)  Y_{1}^{\lambda_{1}}Y_{2}^{\lambda_{2}%
}...Y_{n}^{\lambda_{n}}.
\end{align*}
Thus, the polynomial $Q\in K\left[  \alpha_{1},\alpha_{2},...,\alpha_{m}%
,\beta_{1},\beta_{2},...,\beta_{n}\right]  $ defined by%
\[
Q=\sum_{\left(  \lambda_{1},\lambda_{2},...,\lambda_{n}\right)  \in
\mathbb{N}^{n}}R_{\left(  \lambda_{1},\lambda_{2},...,\lambda_{n}\right)
}\left(  \alpha_{1},\alpha_{2},...,\alpha_{m}\right)  \beta_{1}^{\lambda_{1}%
}\beta_{2}^{\lambda_{2}}...\beta_{n}^{\lambda_{n}}%
\]
satisfies $P\left(  U_{1},U_{2},...,U_{m},V_{1},V_{2},...,V_{n}\right)
=Q\left(  X_{1},X_{2},...,X_{m},Y_{1},Y_{2},...,Y_{m}\right)  $. It only
remains to prove that this is the only such polynomial. This amounts to
showing that $X_{1},$ $X_{2},$ $...,$ $X_{m},$ $Y_{1},$ $Y_{2},$ $...,$
$Y_{n}$ are algebraically independent over $K$. But this is clear since
$X_{1},$ $X_{2},$ $...,$ $X_{m}$ are algebraically independent over $K$ (by
Theorem 4.1) and the variables $Y_{1},$ $Y_{2},$ $...,$ $Y_{n}$ are
algebraically independent over $K\left[  X_{1},X_{2},...,X_{m}\right]  $ (by
Theorem 4.1, applied to $K\left[  X_{1},X_{2},...,X_{m}\right]  $ instead of
$K$ and to $V_{1},$ $V_{2},$ $...,$ $V_{n}$ instead of $U_{1},$ $U_{2},$
$...,$ $U_{m}$). [How does this implication work? Just write it down and see.]

For part \textbf{(b)}, we argue the same way as in \textbf{(a)}, but applying
Theorem 4.1 \textbf{(b)} along with Theorem 4.1 \textbf{(a)}.

Note that in the following, we are going to use Theorems 4.1 and 4.2 for
$K=\mathbb{Z}$ only.

Theorem 4.2 allows us to make the following definition:

\begin{quote}
\textbf{Definition.} Let $k\in\mathbb{N}$. Our goal now is to define a
polynomial $P_{k}\in\mathbb{Z}\left[  \alpha_{1},\alpha_{2},...,\alpha
_{k},\beta_{1},\beta_{2},...,\beta_{k}\right]  $ such that%
\begin{equation}
\sum_{\substack{S\subseteq\left\{  1,2,...,m\right\}  \times\left\{
1,2,...,n\right\}  ;\\\left\vert S\right\vert =k}}\prod_{\left(  i,j\right)
\in S}U_{i}V_{j}=P_{k}\left(  X_{1},X_{2},...,X_{k},Y_{1},Y_{2},...,Y_{k}%
\right)  \label{Pk1}%
\end{equation}
in the polynomial ring $\mathbb{Z}\left[  U_{1},U_{2},...,U_{m},V_{1}%
,V_{2},...,V_{n}\right]  $ for every $n\in\mathbb{N}$ and $m\in\mathbb{N}$,
where $X_{i}=\sum\limits_{\substack{S\subseteq\left\{  1,2,...,m\right\}
;\\\left\vert S\right\vert =i}}\prod\limits_{k\in S}U_{k}$ is the $i$-th
elementary symmetric polynomial in the variables $U_{1},$ $U_{2},$ $...,$
$U_{m}$ for every $i\in\mathbb{N}$, and $Y_{j}=\sum
\limits_{\substack{S\subseteq\left\{  1,2,...,n\right\}  ;\\\left\vert
S\right\vert =j}}\prod\limits_{k\in S}V_{k}$ is the $j$-th elementary
symmetric polynomial in the variables $V_{1},$ $V_{2},$ $...,$ $V_{n}$ for
every $j\in\mathbb{N}$.

In order to do this, we first fix some $n\in\mathbb{N}$ and $m\in\mathbb{N}$.
The polynomial%
\[
\sum_{\substack{S\subseteq\left\{  1,2,...,m\right\}  \times\left\{
1,2,...,n\right\}  ;\\\left\vert S\right\vert =k}}\prod_{\left(  i,j\right)
\in S}U_{i}V_{j}\in\mathbb{Z}\left[  U_{1},U_{2},...,U_{m},V_{1}%
,V_{2},...,V_{n}\right]
\]
is UV-symmetric. Thus, Theorem 4.2 \textbf{(a)} yields that there exists one
and only one polynomial $Q\in\mathbb{Z}\left[  \alpha_{1},\alpha
_{2},...,\alpha_{m},\beta_{1},\beta_{2},...,\beta_{n}\right]  $ such that%
\[
\sum_{\substack{S\subseteq\left\{  1,2,...,m\right\}  \times\left\{
1,2,...,n\right\}  ;\\\left\vert S\right\vert =k}}\prod_{\left(  i,j\right)
\in S}U_{i}V_{j}=Q\left(  X_{1},X_{2},...,X_{m},Y_{1},Y_{2},...,Y_{m}\right)
\]
in $\mathbb{Z}\left[  \alpha_{1},\alpha_{2},...,\alpha_{m},\beta_{1},\beta
_{2},...,\beta_{n}\right]  $. Since the polynomial $\sum
\limits_{\substack{S\subseteq\left\{  1,2,...,m\right\}  \times\left\{
1,2,...,n\right\}  ;\\\left\vert S\right\vert =k}}\prod_{\left(  i,j\right)
\in S}U_{i}V_{j}$ has total degree $\leq k$ in the variables $U_{1},$ $U_{2},$
$...,$ $U_{m}$ and of total degree $\leq k$ in the variables $V_{1},$ $V_{2},$
$...,$ $V_{n}$, Theorem 4.2 \textbf{(b)} yields that%
\[
\sum_{\substack{S\subseteq\left\{  1,2,...,m\right\}  \times\left\{
1,2,...,n\right\}  ;\\\left\vert S\right\vert =k}}\prod_{\left(  i,j\right)
\in S}U_{i}V_{j}=Q_{k,k}\left(  X_{1},X_{2},...,X_{k},Y_{1},Y_{2}%
,...,Y_{k}\right)  ,
\]
where $Q_{k,k}$ is the image of the polynomial $Q$ under the canonical
homomorphism $\mathbb{Z}\left[  \alpha_{1},\alpha_{2},...,\alpha_{m},\beta
_{1},\beta_{2},...,\beta_{n}\right]  \rightarrow\mathbb{Z}\left[  \alpha
_{1},\alpha_{2},...,\alpha_{k},\beta_{1},\beta_{2},...,\beta_{k}\right]  $.
However, this polynomial $Q_{k,k}$ is not independent of $n$ and $m$ yet (as
the polynomial $P_{k}$ that we intend to construct should be), so we call it
$Q_{k,k,\left[  n,m\right]  }$ rather than just $Q_{k,k}$.

Now we forget that we fixed $n\in\mathbb{N}$ and $m\in\mathbb{N}$. We have
learnt that%
\[
\sum_{\substack{S\subseteq\left\{  1,2,...,m\right\}  \times\left\{
1,2,...,n\right\}  ;\\\left\vert S\right\vert =k}}\prod_{\left(  i,j\right)
\in S}U_{i}V_{j}=Q_{k,k,\left[  n,m\right]  }\left(  X_{1},X_{2}%
,...,X_{k},Y_{1},Y_{2},...,Y_{k}\right)
\]
in the polynomial ring $\mathbb{Z}\left[  U_{1},U_{2},...,U_{m},V_{1}%
,V_{2},...,V_{n}\right]  $ for every $n\in\mathbb{N}$ and $m\in\mathbb{N}$.
Now, define a polynomial $P_{k}\in\mathbb{Z}\left[  \alpha_{1},\alpha
_{2},...,\alpha_{k},\beta_{1},\beta_{2},...,\beta_{k}\right]  $ by
$P_{k}=Q_{k,k,\left[  k,k\right]  }.$

\textbf{Theorem 4.3.} \textbf{(a)} The polynomial $P_{k}$ just defined
satisfies the equation (\ref{Pk1}) in the polynomial ring $\mathbb{Z}\left[
U_{1},U_{2},...,U_{m},V_{1},V_{2},...,V_{n}\right]  $ for every $n\in
\mathbb{N}$ and $m\in\mathbb{N}$. (Hence, the goal mentioned above in the
definition is actually achieved.)

\textbf{(b)} For every $n\in\mathbb{N}$ and $m\in\mathbb{N}$, we have%
\begin{equation}
\prod_{\left(  i,j\right)  \in\left\{  1,2,...,m\right\}  \times\left\{
1,2,...,n\right\}  }\left(  1+U_{i}V_{j}T\right)  =\sum_{k\in\mathbb{N}}%
P_{k}\left(  X_{1},X_{2},...,X_{k},Y_{1},Y_{2},...,Y_{k}\right)  T^{k}
\label{Pk2}%
\end{equation}
in the ring $\left(  \mathbb{Z}\left[  U_{1},U_{2},...,U_{m},V_{1}%
,V_{2},...,V_{n}\right]  \right)  \left[  \left[  T\right]  \right]  $. (Note
that the right hand side of this equation is a power series with coefficient
$1$ before $T^{0}$, since $P_{0}=1$.)
\end{quote}

\textit{Proof of Theorem 4.3.} \textbf{(a)} \textit{1st Step:} Fix
$n\in\mathbb{N}$ and $m\in\mathbb{N}$ such that $n\geq k$ and $m\geq k$. Then,
we claim that $Q_{k,k,\left[  n,m\right]  }=P_{k}$.

\textit{Proof.} The definition of $Q_{k,k,\left[  n,m\right]  }$ yields
\[
\sum_{\substack{S\subseteq\left\{  1,2,...,m\right\}  \times\left\{
1,2,...,n\right\}  ;\\\left\vert S\right\vert =k}}\prod_{\left(  i,j\right)
\in S}U_{i}V_{j}=Q_{k,k,\left[  n,m\right]  }\left(  X_{1},X_{2}%
,...,X_{k},Y_{1},Y_{2},...,Y_{k}\right)
\]
in the polynomial ring $\mathbb{Z}\left[  U_{1},U_{2},...,U_{m},V_{1}%
,V_{2},...,V_{n}\right]  $. Applying the canonical ring epimorphism
$\mathbb{Z}\left[  U_{1},U_{2},...,U_{m},V_{1},V_{2},...,V_{n}\right]
\rightarrow\mathbb{Z}\left[  U_{1},U_{2},...,U_{k},V_{1},V_{2},...,V_{k}%
\right]  $ (which maps every $U_{i}$ to $\left\{
\begin{array}
[c]{c}%
U_{i},\text{ if }i\leq k;\\
0,\text{ if }i>k
\end{array}
\right.  $ and every $V_{j}$ to $\left\{
\begin{array}
[c]{c}%
V_{j},\text{ if }j\leq k;\\
0,\text{ if }j>k
\end{array}
\right.  $) to this equation (and noticing that this epimorphism maps every
$X_{i}$ with $i\geq1$ to the corresponding $X_{i}$ of the image ring and every
$Y_{j}$ with $j\geq1$ to the corresponding $Y_{j}$ of the image ring!), we
obtain%
\[
\sum_{\substack{S\subseteq\left\{  1,2,...,k\right\}  \times\left\{
1,2,...,k\right\}  ;\\\left\vert S\right\vert =k}}\prod_{\left(  i,j\right)
\in S}U_{i}V_{j}=Q_{k,k,\left[  n,m\right]  }\left(  X_{1},X_{2}%
,...,X_{k},Y_{1},Y_{2},...,Y_{k}\right)
\]
in the polynomial ring $\mathbb{Z}\left[  U_{1},U_{2},...,U_{k},V_{1}%
,V_{2},...,V_{k}\right]  $. On the other hand, the definition of
$Q_{k,k,\left[  k,k\right]  }$ yields%
\[
\sum_{\substack{S\subseteq\left\{  1,2,...,k\right\}  \times\left\{
1,2,...,k\right\}  ;\\\left\vert S\right\vert =k}}\prod_{\left(  i,j\right)
\in S}U_{i}V_{j}=Q_{k,k,\left[  k,k\right]  }\left(  X_{1},X_{2}%
,...,X_{k},Y_{1},Y_{2},...,Y_{k}\right)
\]
in the same ring. These two equations yield%
\[
Q_{k,k,\left[  n,m\right]  }\left(  X_{1},X_{2},...,X_{k},Y_{1},Y_{2}%
,...,Y_{k}\right)  =Q_{k,k,\left[  k,k\right]  }\left(  X_{1},X_{2}%
,...,X_{k},Y_{1},Y_{2},...,Y_{k}\right)  .
\]
Since the elements $X_{1},$ $X_{2},$ $...,$ $X_{k},$ $Y_{1},$ $Y_{2},$ $...,$
$Y_{k}$ of $\mathbb{Z}\left[  U_{1},U_{2},...,U_{k},V_{1},V_{2},...,V_{k}%
\right]  $ are algebraically independent (by Theorem 4.2 \textbf{(a)}), this
yields $Q_{k,k,\left[  n,m\right]  }=Q_{k,k,\left[  k,k\right]  }.$ In other
words, $Q_{k,k,\left[  n,m\right]  }=P_{k},$ and the 1st Step is proven.

\textit{2nd Step:} For every $n\in\mathbb{N}$ and $m\in\mathbb{N}$, the
equation (\ref{Pk1}) is satisfied in the polynomial ring $\mathbb{Z}\left[
U_{1},U_{2},...,U_{m},V_{1},V_{2},...,V_{n}\right]  $.

\textit{Proof.} Let $n^{\prime}\in\mathbb{N}$ be such that $n^{\prime}\geq n$
and $n^{\prime}\geq k$. Let $m^{\prime}\in\mathbb{N}$ be such that $m^{\prime
}\geq m$ and $m^{\prime}\geq k$. Then, the 1st Step (applied to $n^{\prime}$
and $m^{\prime}$ instead of $n$ and $m$) yields that $Q_{k,k,\left[
n^{\prime},m^{\prime}\right]  }=P_{k}.$

The definition of $Q_{k,k,\left[  n^{\prime},m^{\prime}\right]  }$ yields
\[
\sum_{\substack{S\subseteq\left\{  1,2,...,m^{\prime}\right\}  \times\left\{
1,2,...,n^{\prime}\right\}  ;\\\left\vert S\right\vert =k}}\prod_{\left(
i,j\right)  \in S}U_{i}V_{j}=Q_{k,k,\left[  n^{\prime},m^{\prime}\right]
}\left(  X_{1},X_{2},...,X_{k},Y_{1},Y_{2},...,Y_{k}\right)
\]
in the polynomial ring $\mathbb{Z}\left[  U_{1},U_{2},...,U_{m^{\prime}}%
,V_{1},V_{2},...,V_{n^{\prime}}\right]  $. Applying the canonical ring
epimorphism $\mathbb{Z}\left[  U_{1},U_{2},...,U_{m^{\prime}},V_{1}%
,V_{2},...,V_{n^{\prime}}\right]  \rightarrow\mathbb{Z}\left[  U_{1}%
,U_{2},...,U_{m},V_{1},V_{2},...,V_{n}\right]  $ (which maps every $U_{i}$ to
$\left\{
\begin{array}
[c]{c}%
U_{i},\text{ if }i\leq m;\\
0,\text{ if }i>m
\end{array}
\right.  $ and every $V_{j}$ to $\left\{
\begin{array}
[c]{c}%
V_{j},\text{ if }j\leq n;\\
0,\text{ if }j>n
\end{array}
\right.  $) to this equation (and noticing that this epimorphism maps every
$X_{i}$ with $i\geq1$ to the corresponding $X_{i}$ of the image ring and every
$Y_{j}$ with $j\geq1$ to the corresponding $Y_{j}$ of the image ring!), we
obtain%
\[
\sum_{\substack{S\subseteq\left\{  1,2,...,m\right\}  \times\left\{
1,2,...,n\right\}  ;\\\left\vert S\right\vert =k}}\prod_{\left(  i,j\right)
\in S}U_{i}V_{j}=\underbrace{Q_{k,k,\left[  n^{\prime},m^{\prime}\right]  }%
}_{=P_{k}}\left(  X_{1},X_{2},...,X_{k},Y_{1},Y_{2},...,Y_{k}\right)
\]
in the polynomial ring $\mathbb{Z}\left[  U_{1},U_{2},...,U_{m},V_{1}%
,V_{2},...,V_{n}\right]  $. Hence, the equation (\ref{Pk1}) is satisfied in
the polynomial ring $\mathbb{Z}\left[  U_{1},U_{2},...,U_{m},V_{1}%
,V_{2},...,V_{n}\right]  .$ This completes the 2nd Step and proves Theorem 4.3
\textbf{(a)}.

\textbf{(b)} Immediately follows from part \textbf{(a)}.

Just as our above definition of the polynomials $P_{k}$ and Theorem 4.3 based
upon Theorem 4.2, we can make another definition basing upon Theorem 4.1:

\begin{quote}
\textbf{Definition.} For every set $H$ and every $j\in\mathbb{N}$, let us
denote by $\mathcal{P}_{j}\left(  H\right)  $ the set of all $j$-element
subsets of $H.$ (This is also often denoted as $\dbinom{H}{j}$.)

Let $j\in\mathbb{N}$. Let $k\in\mathbb{N}$. Our goal now is to define a
polynomial $P_{k,j}\in\mathbb{Z}\left[  \alpha_{1},\alpha_{2},...,\alpha
_{kj}\right]  $ such that%
\begin{equation}
\sum_{\substack{S\subseteq\mathcal{P}_{j}\left(  \left\{  1,2,...,m\right\}
\right)  ;\\\left\vert S\right\vert =k}}\prod_{I\in S}\prod_{i\in I}%
U_{i}=P_{k,j}\left(  X_{1},X_{2},...,X_{kj}\right)  \label{Pkj1}%
\end{equation}
in the polynomial ring $\mathbb{Z}\left[  U_{1},U_{2},...,U_{m}\right]  $ for
every $m\in\mathbb{N}$, where $X_{i}=\sum\limits_{\substack{S\subseteq\left\{
1,2,...,m\right\}  ;\\\left\vert S\right\vert =i}}\prod\limits_{k\in S}U_{k}$
is the $i$-th elementary symmetric polynomial in the variables $U_{1},$
$U_{2},$ $...,$ $U_{m}$ for every $i\in\mathbb{N}$.

In order to do this, we first fix some $m\in\mathbb{N}$. The polynomial%
\[
\sum_{\substack{S\subseteq\mathcal{P}_{j}\left(  \left\{  1,2,...,m\right\}
\right)  ;\\\left\vert S\right\vert =k}}\prod_{I\in S}\prod_{i\in I}U_{i}%
\in\mathbb{Z}\left[  U_{1},U_{2},...,U_{m}\right]
\]
is symmetric. Thus, Theorem 4.1 \textbf{(a)} yields that there exists one and
only one polynomial $Q\in\mathbb{Z}\left[  \alpha_{1},\alpha_{2}%
,...,\alpha_{m}\right]  $ such that%
\[
\sum_{\substack{S\subseteq\mathcal{P}_{j}\left(  \left\{  1,2,...,m\right\}
\right)  ;\\\left\vert S\right\vert =k}}\prod_{I\in S}\prod_{i\in I}%
U_{i}=Q\left(  X_{1},X_{2},...,X_{m}\right)  .
\]
Since the polynomial $\sum\limits_{\substack{S\subseteq\mathcal{P}_{j}\left(
\left\{  1,2,...,m\right\}  \right)  ;\\\left\vert S\right\vert =k}%
}\prod\limits_{I\in S}\prod\limits_{i\in I}U_{i}$ has total degree $\leq kj$
in the variables $U_{1},$ $U_{2},$ $...,$ $U_{m}$, Theorem 4.1 \textbf{(b)}
yields that%
\[
\sum_{\substack{S\subseteq\mathcal{P}_{j}\left(  \left\{  1,2,...,m\right\}
\right)  ;\\\left\vert S\right\vert =k}}\prod_{I\in S}\prod_{i\in I}%
U_{i}=Q_{k,j}\left(  X_{1},X_{2},...,X_{kj}\right)  ,
\]
where $Q_{k,j}$ is the image of the polynomial $Q$ under the canonical
homomorphism $\mathbb{Z}\left[  \alpha_{1},\alpha_{2},...,\alpha_{m}\right]
\rightarrow\mathbb{Z}\left[  \alpha_{1},\alpha_{2},...,\alpha_{kj}\right]  $.
However, this polynomial $Q_{k,j}$ is not independent of $m$ yet (as the
polynomial $P_{k,j}$ that we intend to construct should be), so we call it
$Q_{k,j,\left[  m\right]  }$ rather than just $Q_{k,j}$.

Now we forget that we fixed $m\in\mathbb{N}$. We have learnt that%
\[
\sum_{\substack{S\subseteq\mathcal{P}_{j}\left(  \left\{  1,2,...,m\right\}
\right)  ;\\\left\vert S\right\vert =k}}\prod_{I\in S}\prod_{i\in I}%
U_{i}=Q_{k,j,\left[  m\right]  }\left(  X_{1},X_{2},...,X_{kj}\right)  ,
\]
in the polynomial ring $\mathbb{Z}\left[  U_{1},U_{2},...,U_{m}\right]  $ for
every $m\in\mathbb{N}$. Now, define a polynomial $P_{k,j}\in\mathbb{Z}\left[
\alpha_{1},\alpha_{2},...,\alpha_{kj}\right]  $ by $P_{k,j}=Q_{k,j,\left[
kj\right]  }.$

\textbf{Theorem 4.4.} \textbf{(a)} The polynomial $P_{k,j}$ just defined
satisfies the equation (\ref{Pkj1}) in the polynomial ring $\mathbb{Z}\left[
U_{1},U_{2},...,U_{m}\right]  $ for every $m\in\mathbb{N}$. (Hence, the goal
mentioned above in the definition is actually achieved.)

\textbf{(b)} For every $m\in\mathbb{N}$ and $j\in\mathbb{N}$, we have%
\begin{equation}
\prod_{I\in\mathcal{P}_{j}\left(  \left\{  1,2,...,m\right\}  \right)
}\left(  1+\prod_{i\in I}U_{i}\cdot T\right)  =\sum_{k\in\mathbb{N}}%
P_{k,j}\left(  X_{1},X_{2},...,X_{kj}\right)  T^{k} \label{Pkj2}%
\end{equation}
in the ring $\left(  \mathbb{Z}\left[  U_{1},U_{2},...,U_{m}\right]  \right)
\left[  \left[  T\right]  \right]  $. (Note that the right hand side of this
equation is a power series with coefficient $1$ before $T^{0}$, since
$P_{0,j}=1$.)
\end{quote}

\textit{Proof of Theorem 4.4.} \textbf{(a)} \textit{1st Step:} Fix
$m\in\mathbb{N}$ such that $m\geq kj$. Then, we claim that $Q_{k,j,\left[
m\right]  }=P_{k,j}$.

\textit{Proof.} The definition of $Q_{k,j,\left[  m\right]  }$ yields
\[
\sum_{\substack{S\subseteq\mathcal{P}_{j}\left(  \left\{  1,2,...,m\right\}
\right)  ;\\\left\vert S\right\vert =k}}\prod_{I\in S}\prod_{i\in I}%
U_{i}=Q_{k,j,\left[  m\right]  }\left(  X_{1},X_{2},...,X_{kj}\right)
\]
in the polynomial ring $\mathbb{Z}\left[  U_{1},U_{2},...,U_{m}\right]  $.
Applying the canonical ring epimorphism $\mathbb{Z}\left[  U_{1}%
,U_{2},...,U_{m}\right]  \rightarrow\mathbb{Z}\left[  U_{1},U_{2}%
,...,U_{kj}\right]  $ (which maps every $U_{i}$ to $\left\{
\begin{array}
[c]{c}%
U_{i},\text{ if }i\leq kj;\\
0,\text{ if }i>kj
\end{array}
\right.  $) to this equation (and noticing that this epimorphism maps every
$X_{i}$ with $i\geq1$ to the corresponding $X_{i}$ of the image ring!), we
obtain%
\[
\sum_{\substack{S\subseteq\mathcal{P}_{j}\left(  \left\{  1,2,...,kj\right\}
\right)  ;\\\left\vert S\right\vert =k}}\prod_{I\in S}\prod_{i\in I}%
U_{i}=Q_{k,j,\left[  m\right]  }\left(  X_{1},X_{2},...,X_{kj}\right)
\]
in the polynomial ring $\mathbb{Z}\left[  U_{1},U_{2},...,U_{kj}\right]  $. On
the other hand, the definition of $Q_{k,j,\left[  kj\right]  }$ yields%
\[
\sum_{\substack{S\subseteq\mathcal{P}_{j}\left(  \left\{  1,2,...,kj\right\}
\right)  ;\\\left\vert S\right\vert =k}}\prod_{I\in S}\prod_{i\in I}%
U_{i}=Q_{k,j,\left[  kj\right]  }\left(  X_{1},X_{2},...,X_{kj}\right)
\]
in the same ring. These two equations yield%
\[
Q_{k,j,\left[  m\right]  }\left(  X_{1},X_{2},...,X_{kj}\right)
=Q_{k,j,\left[  kj\right]  }\left(  X_{1},X_{2},...,X_{kj}\right)  .
\]
Since the elements $X_{1},$ $X_{2},$ $...,$ $X_{kj}$ of $\mathbb{Z}\left[
U_{1},U_{2},...,U_{kj}\right]  $ are algebraically independent (by Theorem 4.1
\textbf{(a)}), this yields $Q_{k,j,\left[  m\right]  }=Q_{k,j,\left[
kj\right]  }.$ In other words, $Q_{k,j,\left[  m\right]  }=P_{k,j},$ and the
1st Step is proven.

\textit{2nd Step:} For every $m\in\mathbb{N}$, the equation (\ref{Pkj1}) is
satisfied in the polynomial ring $\mathbb{Z}\left[  U_{1},U_{2},...,U_{m}%
\right]  $.

\textit{Proof.} Let $m^{\prime}\in\mathbb{N}$ be such that $m^{\prime}\geq m$
and $m^{\prime}\geq kj$. Then, the 1st Step (applied to $m^{\prime}$ instead
of $m$) yields that $Q_{k,j,\left[  m^{\prime}\right]  }=P_{k,j}.$

The definition of $Q_{k,j,\left[  m^{\prime}\right]  }$ yields
\[
\sum_{\substack{S\subseteq\mathcal{P}_{j}\left(  \left\{  1,2,...,m^{\prime
}\right\}  \right)  ;\\\left\vert S\right\vert =k}}\prod_{I\in S}\prod_{i\in
I}U_{i}=Q_{k,j,\left[  m^{\prime}\right]  }\left(  X_{1},X_{2},...,X_{kj}%
\right)
\]
in the polynomial ring $\mathbb{Z}\left[  U_{1},U_{2},...,U_{m^{\prime}%
}\right]  $. Applying the canonical ring epimorphism $\mathbb{Z}\left[
U_{1},U_{2},...,U_{m^{\prime}}\right]  \rightarrow\mathbb{Z}\left[
U_{1},U_{2},...,U_{m}\right]  $ (which maps every $U_{i}$ to $\left\{
\begin{array}
[c]{c}%
U_{i},\text{ if }i\leq m;\\
0,\text{ if }i>m
\end{array}
\right.  $) to this equation (and noticing that this epimorphism maps every
$X_{i}$ with $i\geq1$ to the corresponding $X_{i}$ of the image ring!), we
obtain%
\[
\sum_{\substack{S\subseteq\mathcal{P}_{j}\left(  \left\{  1,2,...,m\right\}
\right)  ;\\\left\vert S\right\vert =k}}\prod_{I\in S}\prod_{i\in I}%
U_{i}=\underbrace{Q_{k,j,\left[  m^{\prime}\right]  }}_{=P_{k,j}}\left(
X_{1},X_{2},...,X_{kj}\right)
\]
in the polynomial ring $\mathbb{Z}\left[  U_{1},U_{2},...,U_{m}\right]  .$
This means that the equation (\ref{Pkj1}) is satisfied in the polynomial ring
$\mathbb{Z}\left[  U_{1},U_{2},...,U_{m}\right]  .$ This completes the 2nd
Step and proves Theorem 4.4 \textbf{(a)}.

\textbf{(b)} Immediately follows from part \textbf{(a)}.

\begin{quotation}
\textit{Exercise 4.1. (Computing }$P_{k}$ \textit{and }$P_{k,j}$ \textit{as
coefficients of determinants.)} The definitions of the polynomials $P_{k}$ and
$P_{k,j}$ provide a possibility to recursively compute them for given values
of $P_{k}$ and $P_{k,j}$ (at least if one knows the constructive proof of
Theorem 4.1, which is fortunately the one given in most books). In this
exercise, we will show another way to compute explicite formulas for $P_{k}$
and $P_{k,j}$:

\textbf{(a)} Let $m\in\mathbb{N}$. In the polynomial ring $\mathbb{Z}\left[
U_{1},U_{2},...,U_{m}\right]  $, let $X_{i}=\sum\limits_{\substack{S\subseteq
\left\{  1,2,...,m\right\}  ;\\\left\vert S\right\vert =i}}\prod\limits_{k\in
S}U_{k}$ be the $i$-th elementary symmetric polynomial in the variables
$U_{1},$ $U_{2},$ $...,$ $U_{m}$ for every $i\in\mathbb{N}$.

Define a matrix $F_{U}\in\left(  \mathbb{Z}\left[  X_{1},X_{2},...,X_{m}%
\right]  \right)  ^{m\times m}$ by%
\[
F_{U}=\left(
\begin{array}
[c]{cccccc}%
0 & 1 & 0 & 0 & \cdots & 0\\
0 & 0 & 1 & 0 & \cdots & 0\\
0 & 0 & 0 & 1 & \cdots & 0\\
\vdots & \vdots & \vdots & \vdots & \vdots & \vdots\\
0 & 0 & 0 & 0 & \cdots & 1\\
\left(  -1\right)  ^{m-1}X_{m} & \left(  -1\right)  ^{m-2}X_{m-1} & \left(
-1\right)  ^{m-3}X_{m-2} & \left(  -1\right)  ^{m-4}X_{m-3} & \cdots & \left(
-1\right)  ^{0}X_{1}%
\end{array}
\right)  .
\]
Prove that the polynomial $\det\left(  TU+I_{m}\right)  \in\left(
\mathbb{Z}\left[  X_{1},X_{2},...,X_{m}\right]  \right)  \left[  T\right]  $
equals $\prod\limits_{i=1}^{m}\left(  1+U_{i}T\right)  $.

\textbf{(b)} Let $m\in\mathbb{N}$ and $n\in\mathbb{N}$. In the polynomial ring
$\mathbb{Z}\left[  U_{1},U_{2},...,U_{m},V_{1},V_{2},...,V_{n}\right]  $, let
$X_{i}=\sum\limits_{\substack{S\subseteq\left\{  1,2,...,m\right\}
;\\\left\vert S\right\vert =i}}\prod\limits_{k\in S}U_{k}$ be the $i$-th
elementary symmetric polynomial in the variables $U_{1},$ $U_{2},$ $...,$
$U_{m}$ for every $i\in\mathbb{N}$, and $Y_{j}=\sum
\limits_{\substack{S\subseteq\left\{  1,2,...,n\right\}  ;\\\left\vert
S\right\vert =j}}\prod\limits_{k\in S}V_{k}$ be the $j$-th elementary
symmetric polynomial in the variables $V_{1},$ $V_{2},$ $...,$ $V_{n}$ for
every $j\in\mathbb{N}$.

Similarly to the matrix $F_{U}$ defined in part \textbf{(a)}, we can define a
matrix $F_{V}\in\left(  \mathbb{Z}\left[  Y_{1},Y_{2},...,Y_{n}\right]
\right)  ^{n\times n}$ by%
\[
F_{V}=\left(
\begin{array}
[c]{cccccc}%
0 & 1 & 0 & 0 & \cdots & 0\\
0 & 0 & 1 & 0 & \cdots & 0\\
0 & 0 & 0 & 1 & \cdots & 0\\
\vdots & \vdots & \vdots & \vdots & \vdots & \vdots\\
0 & 0 & 0 & 0 & \cdots & 1\\
\left(  -1\right)  ^{n-1}Y_{n} & \left(  -1\right)  ^{n-2}Y_{n-1} & \left(
-1\right)  ^{n-3}Y_{n-2} & \left(  -1\right)  ^{n-4}Y_{n-3} & \cdots & \left(
-1\right)  ^{0}Y_{1}%
\end{array}
\right)  .
\]
Prove that the polynomial $\det\left(  -T\left(  U\otimes V\right)
+I_{mn}\right)  \in\left(  \mathbb{Z}\left[  X_{1},X_{2},...,X_{m},Y_{1}%
,Y_{2},...,Y_{n}\right]  \right)  \left[  T\right]  $ equals $\prod
\limits_{\left(  i,j\right)  \in\left\{  1,2,...,m\right\}  \times\left\{
1,2,...,n\right\}  }\left(  1+U_{i}V_{j}T\right)  .$ Conclude that the
coefficient of this polynomial before $T^{k}$ equals to $Q_{k,k,\left[
n,m\right]  }\left(  X_{1},X_{2},...,X_{k},Y_{1},Y_{2},...,Y_{k}\right)  $
defined in the definition of $P_{k}$. How to compute $P_{k}$ now? (Don't
forget to choose $n$ and $m$ such that $n\geq k$ and $m\geq k$.)

\textbf{(c)} Let $m\in\mathbb{N}$ and $j\in\mathbb{N}$. In part \textbf{(a)},
prove that the polynomial $\det\left(  \left(  -1\right)  ^{j}T\left(
\wedge^{j}U\right)  +I_{m}\right)  \in\left(  \mathbb{Z}\left[  X_{1}%
,X_{2},...,X_{m}\right]  \right)  \left[  T\right]  $ equals $\prod
\limits_{I\in\mathcal{P}_{j}\left(  \left\{  1,2,...,m\right\}  \right)
}\left(  1+\prod\limits_{i\in I}U_{i}\cdot T\right)  $. Conclude that the
coefficient of this polynomial before $T^{k}$ equals to $Q_{k,j,\left[
m\right]  }\left(  X_{1},X_{2},...,X_{kj}\right)  $ defined in the definition
of $P_{k,j}$. How to compute $P_{k,j}$ now? (Don't forget to choose $m$ such
that $m\geq k$.)
\end{quotation}

\begin{center}
\fbox{\textbf{5. A }$\lambda$\textbf{-ring structure on }$\Lambda\left(
K\right)  =1+K\left[  \left[  T\right]  \right]  ^{+}$}
\end{center}

Now we are going to introduce a $\lambda$-ring structure on a particular set
defined for any given ring $K.$

\begin{quote}
\textbf{Definition.} Let $K$ be a ring. Let $K\left[  \left[  T\right]
\right]  ^{+}$ denote the subset%
\begin{align*}
TK\left[  \left[  T\right]  \right]   &  =\left\{  \sum_{i\in\mathbb{N}}%
a_{i}T^{i}\in K\left[  \left[  T\right]  \right]  \ \mid\ a_{i}\in K\text{ for
all }i,\text{ and }a_{0}=0\right\} \\
&  =\left\{  p\in K\left[  \left[  T\right]  \right]  \ \mid\ p\text{ is a
power series with constant term }0\right\}
\end{align*}
We are going to define a ring structure on the set
\begin{align*}
1+K\left[  \left[  T\right]  \right]  ^{+}  &  =\left\{  1+u\mid u\in K\left[
\left[  T\right]  \right]  ^{+}\right\} \\
&  =\left\{  p\in K\left[  \left[  T\right]  \right]  \ \mid\ p\text{ is a
power series with constant term }1\right\}  .
\end{align*}
First, we define an Abelian group structure on this set:

Define an addition $\widehat{+}$ on the set $1+K\left[  \left[  T\right]
\right]  ^{+}$ by $u\widehat{+}v=uv$ for every $u\in1+K\left[  \left[
T\right]  \right]  ^{+}$ and $v\in1+K\left[  \left[  T\right]  \right]  ^{+}$.
In other words, addition on $1+K\left[  \left[  T\right]  \right]  ^{+}$ is
defined as multiplication of power series. The zero of $1+K\left[  \left[
T\right]  \right]  ^{+}$ will be $1$. The subtraction $\widehat{-}$ on the set
$1+K\left[  \left[  T\right]  \right]  ^{+}$ is given by $u\widehat{-}%
v=\dfrac{u}{v}$ for every $u\in1+K\left[  \left[  T\right]  \right]  ^{+}$ and
$v\in1+K\left[  \left[  T\right]  \right]  ^{+}$ (since every $v\in1+K\left[
\left[  T\right]  \right]  ^{+}$ is an invertible power series).

Then, clearly, $\left(  1+K\left[  \left[  T\right]  \right]  ^{+},\widehat
{+}\right)  $ is an Abelian group with zero $1$.

Now, define a multiplication $\widehat{\cdot}$ on the set $1+K\left[  \left[
T\right]  \right]  ^{+}$ by%
\[
\left(  \sum_{i\in\mathbb{N}}a_{i}T^{i}\right)  \widehat{\cdot}\left(
\sum_{i\in\mathbb{N}}b_{i}T^{i}\right)  =\sum_{k\in\mathbb{N}}P_{k}\left(
a_{1},a_{2},...,a_{k},b_{1},b_{2},...,b_{k}\right)  T^{k}%
\]
(here, the $\sum\limits_{k\in\mathbb{N}}$ sign means addition in $K\left[
\left[  T\right]  \right]  $, not in $1+K\left[  \left[  T\right]  \right]
^{+}$).

The multiplicative unity of the ring $1+K\left[  \left[  T\right]  \right]
^{+}$ will be $1+T$.

Also, for every $j\in\mathbb{N}$, define a mapping $\widehat{\lambda}%
^{j}:1+K\left[  \left[  T\right]  \right]  ^{+}\rightarrow1+K\left[  \left[
T\right]  \right]  ^{+}$ by%
\[
\widehat{\lambda}^{j}\left(  \sum_{i\in\mathbb{N}}a_{i}T^{i}\right)
=\sum_{k\in\mathbb{N}}P_{k,j}\left(  a_{1},a_{2},...,a_{kj}\right)  T^{k}.
\]



\end{quote}

Note that we have denoted the newly-defined addition, subtraction and
multiplication on the set $1+K\left[  \left[  T\right]  \right]  ^{+}$ by
$\widehat{+},$ $\widehat{-}$ and $\widehat{\cdot}$ in order to distinguish
them from the addition $+,$ subtraction $-$ and multiplication $\cdot$
inherited from $K\left[  \left[  T\right]  \right]  $. We will later continue
in this spirit (for instance, we will denote a finite sum with respect to the
addition $\widehat{+}$ by the sign $\widehat{\sum}$, while a finite sum with
respect to the addition $+$ will be written using the normal $\sum$
sign).\footnote{In [2], Knutson writes $"+"$, $"-"$ and $"\cdot"$ instead of
$\widehat{+},$ $\widehat{-}$ and $\widehat{\cdot}$ for the newly-defined
operations. In [1], Fulton and Lang simply write $+,$ $-$ and $\cdot$ for
$\widehat{+},$ $\widehat{-}$ and $\widehat{\cdot}$, approving the danger of
confusion with the "old" operations $+,$ $-$ and $\cdot$ inherited from
$K\left[  \left[  T\right]  \right]  $.}

\begin{quote}
\textbf{Theorem 5.1.} \textbf{(a)} The multiplication $\widehat{\cdot}$ just
defined makes $\left(  1+K\left[  \left[  T\right]  \right]  ^{+},\widehat
{+},\widehat{\cdot}\right)  $ a ring with multiplicative unity $1+T$. We will
call this ring $\Lambda\left(  K\right)  $.

\textbf{(b)} The above defined maps $\widehat{\lambda}^{j}$ make $\left(
\Lambda\left(  K\right)  ,\left(  \widehat{\lambda}^{i}\right)  _{i\in
\mathbb{N}}\right)  $ a $\lambda$-ring.
\end{quote}

Before we prove this Theorem 5.1, we try to motivate the above definition of
$\Lambda\left(  K\right)  $:

The set $1+K\left[  T\right]  ^{+}$, defined by%
\begin{align*}
K\left[  T\right]  ^{+}  &  =TK\left[  T\right]  =\left\{  \sum_{i\in
\mathbb{N}}a_{i}T^{i}\in K\left[  T\right]  \ \mid\ a_{i}\in K\text{ for all
}i,\text{ and }a_{0}=0\right\} \\
&  =\left\{  p\in K\left[  T\right]  \ \mid\ p\text{ is a polynomial with
constant term }0\right\}  ,
\end{align*}
is a subset of $1+K\left[  \left[  T\right]  \right]  ^{+}.$ The elements of
$1+K\left[  T\right]  ^{+}$ are polynomials. We have:

\begin{quote}
\textbf{Theorem 5.2.} Let $K$ be a ring. For every element $p\in1+K\left[
T\right]  ^{+}$, there exists an integer $n$ (the degree of the polynomial
$p$), a finite-free extension ring $K_{p}$ of the ring $K$ and $n$ elements
$p_{1},$ $p_{2},$ $...,$ $p_{n}$ of this extension ring $K_{p}$ such that
$p=\prod\limits_{i=1}^{n}\left(  1+p_{i}T\right)  $ in $K_{p}\left[  T\right]
$.
\end{quote}

\textit{Proof of Theorem 5.2.} Let $p=\sum\limits_{i=0}^{n}a_{i}T^{i}$, where
$n=\deg p$. Define a new polynomial $\widetilde{p}=\sum\limits_{i=0}%
^{n}a_{n-i}T^{i}\in K\left[  T\right]  $. Then, the polynomial $\widetilde{p}$
is monic (since $p\in1+K\left[  T\right]  ^{+}$), and the equation
$p=\prod\limits_{i=1}^{n}\left(  1+p_{i}T\right)  $ becomes equivalent to
$\widetilde{p}=\prod\limits_{i=1}^{n}\left(  p_{i}+T\right)  $, so that
Theorem 5.2 simply claims that for every monic polynomial over a ring, we can
find a finite-free extension of the ring over which the polynomial splits into
linear polynomials. But this is an easy fact (proven in Exercise 5.1). Thus,
Theorem 5.2 is proven.

Let us introduce some notation again:

\begin{quote}
\textbf{Definition.} For every set $H$, let $\mathcal{P}_{\operatorname*{fin}%
}^{\ast}\left(  H\right)  $ denote the set of all finite multisets which
consist of elements of $H$. Also, we recall that we denote the multiset formed
by the elements $u_{1},$ $u_{2},$ $...,$ $u_{n}$ (with multiplicity) by
$\left[  u_{1},u_{2},...,u_{n}\right]  $.

For our ring $K$, let $\operatorname*{Exten}K$ be the set of all finite-free
extension rings of $K$. (Again, this is not a set. Again, we don't care.
Basically it is enough to consider all finite-free extension rings of the form
$K\left[  X_{1},X_{2},...,X_{n}\right]  \diagup I$ with $I$ being an ideal of
$K\left[  X_{1},X_{2},...,X_{n}\right]  $, and \textit{these} extension rings
do form a set.)

Let $K^{\operatorname*{int}}$ be the subset%
\[
\left\{  \left(  \widetilde{K},\left[  u_{1},u_{2},...,u_{n}\right]  \right)
\in\bigcup\limits_{K^{\prime}\subseteq\operatorname*{Exten}K}^{\cdot
}\mathcal{P}_{\operatorname*{fin}}^{\ast}\left(  K^{\prime}\right)
\ \ \mid\ \ \prod\limits_{i=1}^{n}\left(  1+u_{i}T\right)  \in K\left[
T\right]  \right\}
\]
of $\bigcup\limits_{K^{\prime}\subseteq\operatorname*{Exten}K}^{\cdot
}\mathcal{P}_{\operatorname*{fin}}^{\ast}\left(  K^{\prime}\right)  $ (where
$\bigcup\limits_{K^{\prime}\subseteq\operatorname*{Exten}K}^{\cdot}%
\mathcal{P}_{\operatorname*{fin}}^{\ast}\left(  K^{\prime}\right)  $ denotes
the disjoint union of the sets $\mathcal{P}_{\operatorname*{fin}}^{\ast
}\left(  K^{\prime}\right)  $ over all $K^{\prime}\subseteq
\operatorname*{Exten}K$, defined by $\bigcup\limits_{K^{\prime}\subseteq
\operatorname*{Exten}K}^{\cdot}\mathcal{P}_{\operatorname*{fin}}^{\ast}\left(
K^{\prime}\right)  =\bigcup\limits_{K^{\prime}\subseteq\operatorname*{Exten}%
K}\left\{  K^{\prime}\right\}  \times\mathcal{P}_{\operatorname*{fin}}^{\ast
}\left(  K^{\prime}\right)  $). We can then define a map%
\[
\Pi:K^{\operatorname*{int}}\rightarrow1+K\left[  T\right]  ^{+}%
\]
through%
\[
\Pi\left(  \widetilde{K},\left[  u_{1},u_{2},...,u_{n}\right]  \right)
=\prod\limits_{i=1}^{n}\left(  1+u_{i}T\right)  \in1+K\left[  T\right]
^{+}\ \ \ \ \ \ \ \ \ \ \text{for every }\left(  \widetilde{K},\left[
u_{1},u_{2},...,u_{n}\right]  \right)  \in K^{\operatorname*{int}}.
\]


We also define a map%
\[
r:1+K\left[  T\right]  ^{+}\rightarrow K^{\operatorname*{int}}%
\]
(the $r$ stands for "roots" here) through
\[
r\left(  p\right)  =\left(  K_{p},\left[  p_{1},p_{2},...,p_{n}\right]
\right)  \ \ \ \ \ \ \ \ \ \ \text{for every }p\in1+K\left[  T\right]  ^{+},
\]
where $K_{p}$ and $\left[  p_{1},p_{2},...,p_{n}\right]  $ are defined as in
Theorem 5.2.

Clearly, $\Pi\circ r=\operatorname*{id}$, so that every polynomial
$p\in1+K\left[  T\right]  ^{+}$ can be written as $p=\Pi\left(  \widetilde
{K},\left[  u_{1},u_{2},...,u_{n}\right]  \right)  $ for some $\left(
\widetilde{K},\left[  u_{1},u_{2},...,u_{n}\right]  \right)  \in
K^{\operatorname*{int}}$.
\end{quote}

This way, we have a correspondence between elements of $1+K\left[  T\right]
^{+}$ and multisets of elements of an extension ring of $K$. This
correspondence is neither injective nor surjective, but it reminds us of the
correspondence between polynomials over a field and their roots over
extensions of that field (and the proof of Theorem 5.2 explains why), and it
will help us to understand $\widehat{+},$ $\widehat{\cdot}$ and $\widehat
{\lambda}^{j}$ better.

In fact, we have the following fact:

\begin{quote}
\textbf{Theorem 5.3.} Let $K$ be a ring.

Let $u\in1+K\left[  T\right]  ^{+}$ and $v\in1+K\left[  T\right]  ^{+}$.
Assume that $u=\Pi\left(  \widetilde{K}_{u},\left[  u_{1},u_{2},...,u_{m}%
\right]  \right)  $ for some $\left(  \widetilde{K}_{u},\left[  u_{1}%
,u_{2},...,u_{m}\right]  \right)  \in K^{\operatorname*{int}}$ and that
$v=\Pi\left(  \widetilde{K}_{v},\left[  v_{1},v_{2},...,v_{n}\right]  \right)
$ for some $\left(  \widetilde{K}_{v},\left[  v_{1},v_{2},...,v_{n}\right]
\right)  \in K^{\operatorname*{int}}$. This, in particular, implies that
$\widetilde{K}_{u}$ and $\widetilde{K}_{v}$ are finite-free extension rings of
$K$. Let $\widetilde{K}_{u,v}$ be a finite-free extension ring of $K$ which
contains both $\widetilde{K}_{u}$ and $\widetilde{K}_{v}$ as subrings.

\textbf{(a)} Such a ring $\widetilde{K}_{u,v}$ always exists. For instance,
$\widetilde{K}_{u}\otimes\widetilde{K}_{v}$ is a finite-free extension ring of
$K$, and we can canonically identify $\widetilde{K}_{u}$ with the subring
$\widetilde{K}_{u}\otimes1$ of $\widetilde{K}_{u}\otimes\widetilde{K}_{v}$,
and $\widetilde{K}_{v}$ with the subring $1\otimes\widetilde{K}_{v}$ of
$\widetilde{K}_{u}\otimes\widetilde{K}_{v}$; hence, we can set $\widetilde
{K}_{u,v}=\widetilde{K}_{u}\otimes\widetilde{K}_{v}$.

\textbf{(b)} We have $u\widehat{+}v=\Pi\left(  \widetilde{K}_{u,v},\left[
u_{1},u_{2},...,u_{m},v_{1},v_{2},...,v_{n}\right]  \right)  $.

\textbf{(c)} Also, $u\widehat{\cdot}v=\Pi\left(  \widetilde{K}_{u,v},\left[
u_{i}v_{j}\mid\left(  i,j\right)  \in\left\{  1,2,...,m\right\}
\times\left\{  1,2,...,n\right\}  \right]  \right)  $.

\textbf{(d)} Let $j\in\mathbb{N}$. Then, $\widehat{\lambda}^{j}\left(
u\right)  =\Pi\left(  \widetilde{K}_{u},\left[  \prod\limits_{i\in I}%
u_{i}\ \mid\ I\in\mathcal{P}_{j}\left(  \left\{  1,2,...,m\right\}  \right)
\right]  \right)  $.
\end{quote}

\textit{Proof of Theorem 5.3.} The assumption that $u=\Pi\left(  \widetilde
{K}_{u},\left[  u_{1},u_{2},...,u_{m}\right]  \right)  $ is just a different
way to say that $u=\prod\limits_{i=1}^{m}\left(  1+u_{i}T\right)  $.
Similarly, $v=\prod\limits_{j=1}^{n}\left(  1+v_{j}T\right)  $. Let
$u=\sum\limits_{i\in\mathbb{N}}a_{i}T^{i}$ and $v=\sum\limits_{i\in\mathbb{N}%
}b_{i}T^{i}$.

\textbf{(a)} Obvious.

\textbf{(b)} The equations $u=\prod\limits_{i=1}^{m}\left(  1+u_{i}T\right)  $
and $v=\prod\limits_{j=1}^{n}\left(  1+v_{j}T\right)  $ yield $u\widehat
{+}v=uv=\prod\limits_{i=1}^{m}\left(  1+u_{i}T\right)  \prod\limits_{j=1}%
^{n}\left(  1+v_{j}T\right)  $, what rewrites as $u\widehat{+}v=\Pi\left(
\widetilde{K}_{u,v},\left[  u_{1},u_{2},...,u_{m},v_{1},v_{2},...,v_{n}%
\right]  \right)  .$

\textbf{(c)} Consider the ring $\mathbb{Z}\left[  U_{1},U_{2},...,U_{m}%
,V_{1},V_{2},...,V_{n}\right]  $ (the polynomial ring in $m+n$ indeterminates
$U_{1},$ $U_{2},$ $...,$ $U_{m},$ $V_{1},$ $V_{2},$ $...,$ $V_{n}$ over the
ring $\mathbb{Z}$). For every $i\in\mathbb{N}$, let $X_{i}=\sum
\limits_{\substack{S\subseteq\left\{  1,2,...,m\right\}  ;\\\left\vert
S\right\vert =i}}\prod\limits_{k\in S}U_{k}$ be the $i$-th elementary
symmetric polynomial in the variables $U_{1},$ $U_{2},$ $...,$ $U_{m}$. For
every $j\in\mathbb{N}$, let $Y_{j}=\sum\limits_{\substack{S\subseteq\left\{
1,2,...,n\right\}  ;\\\left\vert S\right\vert =j}}\prod\limits_{k\in S}V_{k}$
be the $j$-th elementary symmetric polynomial in the variables $V_{1},$
$V_{2},$ $...,$ $V_{n}$.

There exists a ring homomorphism%
\[
\mathbb{Z}\left[  U_{1},U_{2},...,U_{m},V_{1},V_{2},...,V_{n}\right]
\rightarrow\widetilde{K}_{u,v}%
\]
which maps $U_{i}$ to $u_{i}$ for every $i$ and $V_{j}$ to $v_{j}$ for every
$j$. This homomorphism maps $X_{i}$ to $a_{i}$ for every $i\in\mathbb{N}$
(because $a_{i}$ is the $i$-th elementary symmetric polynomial applied to
$u_{1},$ $u_{2},$ $...,$ $u_{m}$, since $\sum\limits_{i\in\mathbb{N}}%
a_{i}T^{i}=u=\prod\limits_{i=1}^{m}\left(  1+u_{i}T\right)  $) and $Y_{j}$ to
$b_{j}$ for every $j\in\mathbb{N}$ (for a similar reason). Hence, applying
this homomorphism to (\ref{Pk2}), we obtain%
\[
\prod_{\left(  i,j\right)  \in\left\{  1,2,...,m\right\}  \times\left\{
1,2,...,n\right\}  }\left(  1+u_{i}v_{j}T\right)  =\sum_{k\in\mathbb{N}}%
P_{k}\left(  a_{1},a_{2},...,a_{k},b_{1},b_{2},...,b_{k}\right)  T^{k}.
\]
But%
\[
\sum_{k\in\mathbb{N}}P_{k}\left(  a_{1},a_{2},...,a_{k},b_{1},b_{2}%
,...,b_{k}\right)  T^{k}=\left(  \sum_{i\in\mathbb{N}}a_{i}T^{i}\right)
\widehat{\cdot}\left(  \sum_{i\in\mathbb{N}}b_{i}T^{i}\right)  =u\widehat
{\cdot}v,
\]
so this becomes%
\[
\prod_{\left(  i,j\right)  \in\left\{  1,2,...,m\right\}  \times\left\{
1,2,...,n\right\}  }\left(  1+u_{i}v_{j}T\right)  =u\widehat{\cdot}v,
\]
and thus%
\[
u\widehat{\cdot}v=\prod_{\left(  i,j\right)  \in\left\{  1,2,...,m\right\}
\times\left\{  1,2,...,n\right\}  }\left(  1+u_{i}v_{j}T\right)  =\Pi\left(
\widetilde{K}_{u,v},\left[  u_{i}v_{j}\mid\left(  i,j\right)  \in\left\{
1,2,...,m\right\}  \times\left\{  1,2,...,n\right\}  \right]  \right)  ,
\]
proving Theorem 5.3 \textbf{(c)}.

\textbf{(d)} Consider the polynomial ring $\mathbb{Z}\left[  U_{1}%
,U_{2},...,U_{m}\right]  $. For every $i\in\mathbb{N}$, let $X_{i}%
=\sum\limits_{\substack{S\subseteq\left\{  1,2,...,m\right\}  ;\\\left\vert
S\right\vert =i}}\prod\limits_{k\in S}U_{k}$ be the $i$-th elementary
symmetric polynomial in the variables $U_{1},$ $U_{2},$ $...,$ $U_{m}$.

There exists a ring homomorphism $\mathbb{Z}\left[  U_{1},U_{2},...,U_{m}%
\right]  \rightarrow\widetilde{K}_{u}$ which maps $U_{i}$ to $u_{i}$ for every
$i$. This homomorphism maps $X_{i}$ to $a_{i}$ for every $i\in\mathbb{N}$
(because $a_{i}$ is the $i$-th elementary symmetric polynomial applied to
$u_{1},$ $u_{2},$ $...,$ $u_{m}$, since $\sum\limits_{i\in\mathbb{N}}%
a_{i}T^{i}=u=\prod\limits_{i=1}^{m}\left(  1+u_{i}T\right)  $). Hence,
applying this homomorphism to (\ref{Pkj2}), we obtain%
\[
\prod_{I\in\mathcal{P}_{j}\left(  \left\{  1,2,...,m\right\}  \right)
}\left(  1+\prod_{i\in I}u_{i}\cdot T\right)  =\sum_{k\in\mathbb{N}}%
P_{k,j}\left(  a_{1},a_{2},...,a_{kj}\right)  T^{k}.
\]
But%
\[
\sum_{k\in\mathbb{N}}P_{k,j}\left(  a_{1},a_{2},...,a_{kj}\right)
T^{k}=\widehat{\lambda}^{j}\left(  \sum_{i\in\mathbb{N}}a_{i}T^{i}\right)
=\widehat{\lambda}^{j}\left(  u\right)  ,
\]
so this becomes%
\[
\prod_{I\in\mathcal{P}_{j}\left(  \left\{  1,2,...,m\right\}  \right)
}\left(  1+\prod_{i\in I}u_{i}\cdot T\right)  =\widehat{\lambda}^{j}\left(
u\right)  ,
\]
and thus%
\[
\widehat{\lambda}^{j}\left(  u\right)  =\prod_{I\in\mathcal{P}_{j}\left(
\left\{  1,2,...,m\right\}  \right)  }\left(  1+\prod_{i\in I}u_{i}\cdot
T\right)  =\Pi\left(  \widetilde{K}_{u},\left[  \prod\limits_{i\in I}%
u_{i}\ \mid\ I\in\mathcal{P}_{j}\left(  \left\{  1,2,...,m\right\}  \right)
\right]  \right)  ,
\]
proving Theorem 5.3 \textbf{(d)}.

\begin{quote}
\textbf{Corollary 5.4.} Let $K$ be a ring. Let $\widetilde{K}$ be a
finite-free extension ring of $K$. Let $I$ be some finite set, and let $T_{i}$
be a finite set for every $i\in I$. Let $u_{i,j}$ be an element of
$\widetilde{K}$ for every $i\in I$ and every $j\in T_{i}$. We will write
$\left[  u_{i,j}\mid i\in I\text{ and }j\in T_{i}\right]  $ for the multiset
formed by all these $u_{i,j}$ (where each element occurs as often as it occurs
among these $u_{i,j}$).

\textbf{(a)} Then,%
\[
\widehat{\sum_{i\in I}}\Pi\left(  \widetilde{K},\left[  u_{i,j}\mid j\in
T_{i}\right]  \right)  =\Pi\left(  \widetilde{K},\left[  u_{i,j}\mid i\in
I\text{ and }j\in T_{i}\right]  \right)  .
\]
Here, the sign $\widehat{\sum\limits_{i\in I}}$ means a finite sum based on
the addition $\widehat{+}$ of the ring $\Lambda\left(  K\right)  $ (for
instance, $\widehat{\sum\limits_{i\in\left\{  1,2,3\right\}  }}a_{i}$ means
$a_{1}\widehat{+}a_{2}\widehat{+}a_{3}$ and not $a_{1}+a_{2}+a_{3}$).

\textbf{(b)} Also,%
\[
\widehat{\prod_{i\in I}}\Pi\left(  \widetilde{K},\left[  u_{i,j}\mid j\in
T_{i}\right]  \right)  =\Pi\left(  \widetilde{K},\left[  \prod_{i\in
I}u_{i,j_{i}}\mid\left(  j_{i}\right)  _{i\in I}\in\prod_{i\in I}T_{i}\right]
\right)  .
\]
Here, the sign $\widehat{\prod\limits_{i\in I}}$ means a finite product based
on the multiplication $\widehat{\cdot}$ of the ring $\Lambda\left(  K\right)
$ (for instance, $\widehat{\prod\limits_{i\in\left\{  1,2,3\right\}  }}a_{i}$
means $a_{1}\widehat{\cdot}a_{2}\widehat{\cdot}a_{3}$ and not $a_{1}\cdot
a_{2}\cdot a_{3}$).
\end{quote}

\textit{Proof of Corollary 5.4.} Part \textbf{(a)} follows by induction from
Theorem 5.3 \textbf{(b)}, and part \textbf{(b)} follows by induction from
Theorem 5.3 \textbf{(c)}.

We are approaching the proof of Theorem 5.1. The idea of the proof is: We have
to show some identities for elements of $1+K\left[  \left[  T\right]  \right]
^{+}$ (such as associativity of multiplication). Computing with elements of
$1+K\left[  \left[  T\right]  \right]  ^{+}$ is very difficult, but computing
with elements of $1+K\left[  T\right]  ^{+}$ is rather easy thanks to Theorem
5.3. Hence, we are going to reduce Theorem 5.1 to the case when our elements
are in $1+K\left[  T\right]  ^{+}$. The reader is encouraged to try doing this
on his own. In practice, it is a matter of noticing that only the first so and
so many coefficients of the power series $u$ and $v$ matter when computing the
$k$-th element of $u\widehat{\cdot}v$ (for instance), and thus we can truncate
the power series at these coefficients, thus turning it into a polynomial. The
abstract algebraical way to formulate this argument is by introducing the
so-called $\left(  T\right)  $\textit{-topology} (also called the $\left(
T\right)  $\textit{-adic topology}) on $K\left[  \left[  T\right]  \right]  $:

\begin{quote}
\textbf{Definition.} Let $K$ be a ring. As a $K$-module, $K\left[  \left[
T\right]  \right]  =\bigoplus\limits_{k\in\mathbb{N}}KT^{k}$. Now, we define
the so-called $\left(  T\right)  $\textit{-topology} on the ring $K\left[
\left[  T\right]  \right]  $ as the topology generated by%
\[
\left\{  0,\varnothing\right\}  \cup\left\{  u+\sum_{k\geq N}KT^{k}%
\ \mid\ u\in K\left[  \left[  T\right]  \right]  \text{ and }N\in
\mathbb{N}\right\}  .
\]
In other words, the open sets of this topology should be $0$, $\varnothing$
and all translates of the submodules $\sum\limits_{k\geq N}KT^{k}$ for
$N\in\mathbb{N}$, as well as the unions of these sets. (Note that the sum
$\sum\limits_{k\geq N}KT^{k}$ is actually a direct sum. Also note that every
translate of the submodule $\sum\limits_{k\geq N}KT^{k}$ for $N\in\mathbb{N}$
actually has the form $p+\sum\limits_{k\geq N}KT^{k}$ for a polynomial $p\in
K\left[  T\right]  $ of degree $<N$, and this polynomial is uniquely determined.)
\end{quote}

Now, an easy fact:

\begin{quote}
\textbf{Theorem 5.5.} Let $K$ be a ring. The $\left(  T\right)  $-topology on
the ring $K\left[  \left[  T\right]  \right]  $ restricts to a topology on its
subset $1+K\left[  \left[  T\right]  \right]  ^{+}$; we call this topology the
$\left(  T\right)  $\textit{-topology} again. Whenever we say "open",
"continuous", "dense", etc., we are referring to this topology.

\textbf{(a)} The subset $1+K\left[  T\right]  ^{+}$ is dense in $1+K\left[
\left[  T\right]  \right]  ^{+}$.

\textbf{(b)} Let $f:1+K\left[  \left[  T\right]  \right]  ^{+}\rightarrow
1+K\left[  \left[  T\right]  \right]  ^{+}$ be a map such that for every
$n\in\mathbb{N}$ there exists some $N\in\mathbb{N}$ such that the first $n$
coefficients of the image of a formal power series under $f$ depend only on
the first $N$ coefficients of the series itself (and not on the remaining
ones). Then, $f$ is continuous.

\textbf{(c)} Let $g:\left(  1+K\left[  \left[  T\right]  \right]  ^{+}\right)
\times\left(  1+K\left[  \left[  T\right]  \right]  ^{+}\right)
\rightarrow1+K\left[  \left[  T\right]  \right]  ^{+}$ be a map such that for
every $n\in\mathbb{N}$ there exists some $N\in\mathbb{N}$ such that the first
$n$ coefficients of the image of a pair of formal power series under $f$
depend only on the first $N$ coefficients of the two series itself (and not on
the remaining ones). Then, $g$ is continuous.

\textbf{(d)} The map%
\begin{align*}
\left(  1+K\left[  \left[  T\right]  \right]  ^{+}\right)  \times\left(
1+K\left[  \left[  T\right]  \right]  ^{+}\right)   &  \rightarrow1+K\left[
\left[  T\right]  \right]  ^{+},\\
\left(  u,v\right)   &  \mapsto u\widehat{+}v,
\end{align*}
the map%
\begin{align*}
\left(  1+K\left[  \left[  T\right]  \right]  ^{+}\right)  \times\left(
1+K\left[  \left[  T\right]  \right]  ^{+}\right)   &  \rightarrow1+K\left[
\left[  T\right]  \right]  ^{+},\\
\left(  u,v\right)   &  \mapsto u\widehat{-}v,
\end{align*}
the map%
\begin{align*}
\left(  1+K\left[  \left[  T\right]  \right]  ^{+}\right)  \times\left(
1+K\left[  \left[  T\right]  \right]  ^{+}\right)   &  \rightarrow1+K\left[
\left[  T\right]  \right]  ^{+},\\
\left(  u,v\right)   &  \mapsto u\widehat{\cdot}v,
\end{align*}
and the map $\widehat{\lambda}^{j}:1+K\left[  \left[  T\right]  \right]
^{+}\rightarrow1+K\left[  \left[  T\right]  \right]  ^{+}$ for every
$j\in\mathbb{N}$ are continuous.
\end{quote}

Note that Theorem 5.5 \textbf{(d)} yields that any finite compositions of the
maps $\widehat{+},$ $\widehat{-},$ $\widehat{\cdot}$ and $\widehat{\lambda
}^{j}$ are continuous (since finite compositions of continuous functions are
continuous). In particular, any polynomial with integral coefficients acts on
$1+K\left[  \left[  T\right]  \right]  ^{+}$ as a continuous map.

\textit{Proof of Theorem 5.5.} \textbf{(a)} is done in any commutative algebra
book such as [3], Chapter 10.

\textbf{(b)} and \textbf{(c)} are basic exercises in topology.

\textbf{(d)} follows from \textbf{(b)} and \textbf{(c)} together with the
definitions of $\widehat{+},$ $\widehat{\cdot}$ and $\widehat{\lambda}^{j}$.

Now it comes:

\textit{Proof of Theorem 5.1.} \textbf{(a)} We have to prove the ring axioms
for $\left(  1+K\left[  \left[  T\right]  \right]  ^{+},\widehat{+}%
,\widehat{\cdot}\right)  $ (including the unity axiom for $1+T$). There are
several axioms to be checked, but the idea is always the same, so we will only
check the associativity of $\widehat{\cdot}$ and leave the rest to the reader.

In order to prove that the operation $\widehat{\cdot}$ is associative, we must
show that $u\widehat{\cdot}\left(  v\widehat{\cdot}w\right)  =\left(
u\widehat{\cdot}v\right)  \widehat{\cdot}w$ for all $u,v,w\in1+K\left[
\left[  T\right]  \right]  ^{+}$. Since the operation $\widehat{\cdot}$ is
continuous (by Theorem 5.5 \textbf{(d)}), and $1+K\left[  T\right]  ^{+}$ is a
dense subset of $1+K\left[  \left[  T\right]  \right]  ^{+}$ (by Theorem 5.5
\textbf{(a)}), this needs only to be shown for all $u,v,w\in1+K\left[
T\right]  ^{+}$. So let us assume that $u,v,w\in1+K\left[  T\right]  ^{+}$.
Then, there exist

\begin{itemize}
\item some $\left(  \widetilde{K}_{u},\left[  u_{1},u_{2},...,u_{m}\right]
\right)  \in K^{\operatorname*{int}}$ such that $u=\Pi\left(  \widetilde
{K}_{u},\left[  u_{1},u_{2},...,u_{m}\right]  \right)  ,$

\item some $\left(  \widetilde{K}_{v},\left[  v_{1},v_{2},...,v_{n}\right]
\right)  \in K^{\operatorname*{int}}$ such that $v=\Pi\left(  \widetilde
{K}_{v},\left[  v_{1},v_{2},...,v_{n}\right]  \right)  ,$

\item some $\left(  \widetilde{K}_{w},\left[  w_{1},w_{2},...,w_{\ell}\right]
\right)  \in K^{\operatorname*{int}}$ such that $w=\Pi\left(  \widetilde
{K}_{w},\left[  w_{1},w_{2},...,w_{\ell}\right]  \right)  $.
\end{itemize}

Let $L=\widetilde{K}_{u}\otimes\widetilde{K}_{v}\otimes\widetilde{K}_{w}$. We
identify the rings $\widetilde{K}_{u},$ $\widetilde{K}_{v},$ $\widetilde
{K}_{w}$ with the subrings $\widetilde{K}_{u}\otimes1\otimes1,$ $1\otimes
\widetilde{K}_{v}\otimes1,$ $1\otimes1\otimes\widetilde{K}_{w}$ of the ring
$L=\widetilde{K}_{u}\otimes\widetilde{K}_{v}\otimes\widetilde{K}_{w},$
respectively. This way, $L$ becomes an extension ring of $K$ which contains
all three rings $\widetilde{K}_{u},$ $\widetilde{K}_{v},$ $\widetilde{K}_{w}.$
Now, Theorem 5.3 \textbf{(c)} yields%
\begin{align*}
u\widehat{\cdot}v  &  =\Pi\left(  L,\left[  u_{i}v_{j}\mid\left(  i,j\right)
\in\left\{  1,2,...,m\right\}  \times\left\{  1,2,...,n\right\}  \right]
\right)  ;\\
\left(  u\widehat{\cdot}v\right)  \widehat{\cdot}w  &  =\Pi\left(  L,\left[
\left(  u_{i}v_{j}\right)  w_{k}\mid\left(  \left(  i,j\right)  ,k\right)
\in\left(  \left\{  1,2,...,m\right\}  \times\left\{  1,2,...,n\right\}
\right)  \times\left\{  1,2,...,\ell\right\}  \right]  \right)  ,
\end{align*}
as well as%
\begin{align*}
v\widehat{\cdot}w  &  =\Pi\left(  L,\left[  v_{j}w_{k}\mid\left(  j,k\right)
\in\left\{  1,2,...,n\right\}  \times\left\{  1,2,...,\ell\right\}  \right]
\right)  ;\\
u\widehat{\cdot}\left(  v\widehat{\cdot}w\right)   &  =\Pi\left(  L,\left[
u_{i}\left(  v_{j}w_{k}\right)  \mid\left(  i,\left(  j,k\right)  \right)
\in\left\{  1,2,...,m\right\}  \times\left(  \left\{  1,2,...,n\right\}
\times\left\{  1,2,...,\ell\right\}  \right)  \right]  \right)  .
\end{align*}
Now, $\left(  u\widehat{\cdot}v\right)  \widehat{\cdot}w=u\widehat{\cdot
}\left(  v\widehat{\cdot}w\right)  $ follows from $\left(  u_{i}v_{j}\right)
w_{k}=u_{i}\left(  v_{j}w_{k}\right)  $ and
\[
\left(  \left\{  1,2,...,m\right\}  \times\left\{  1,2,...,n\right\}  \right)
\times\left\{  1,2,...,\ell\right\}  \cong\left\{  1,2,...,m\right\}
\times\left(  \left\{  1,2,...,n\right\}  \times\left\{  1,2,...,\ell\right\}
\right)
\]
(which is a canonical isomorphism, mapping every $\left(  \left(  i,j\right)
,k\right)  $ to $\left(  i,\left(  j,k\right)  \right)  $). This proves the
associativity of $u\widehat{\cdot}\left(  v\widehat{\cdot}w\right)  =\left(
u\widehat{\cdot}v\right)  \widehat{\cdot}w$.

\textbf{(b)} It is easy to see that $\widehat{\lambda}^{0}\left(  x\right)
=1$ and $\widehat{\lambda}^{1}\left(  x\right)  =x$ for every $x\in
\Lambda\left(  K\right)  $. It now remains to prove that%
\begin{equation}
\widehat{\lambda}^{j}\left(  u+v\right)  =\widehat{\sum_{i=0}^{j}}%
\widehat{\lambda}^{i}\left(  u\right)  \widehat{\cdot}\widehat{\lambda}%
^{j-i}\left(  v\right)  \label{SpezLemma1}%
\end{equation}
for every $j\in\mathbb{N},$ $u\in\Lambda\left(  K\right)  $ and $v\in
\Lambda\left(  K\right)  .$ Here, the sign $\widehat{\sum\limits_{i=0}^{j}}$
means that the summation is based on the addition $\widehat{+}$ of the ring
$\Lambda\left(  K\right)  $.

Let us fix some $j\in\mathbb{N}$. Since the addition $\widehat{+},$ the
multiplication $\widehat{\cdot}$ and the map $\widehat{\lambda}^{i}$ for every
$i\in\mathbb{N}$ are continuous (by Theorem 5.5 \textbf{(d)}), and $1+K\left[
T\right]  ^{+}$ is a dense subset of $1+K\left[  \left[  T\right]  \right]
^{+}$ (by Theorem 5.5 \textbf{(a)}), we only need to check (\ref{SpezLemma1})
for all $u,v\in1+K\left[  T\right]  ^{+}$. So let us assume that
$u,v\in1+K\left[  T\right]  ^{+}$. Then, there exist some $\left(
\widetilde{K}_{u},\left[  u_{1},u_{2},...,u_{m}\right]  \right)  \in
K^{\operatorname*{int}}$ such that $u=\Pi\left(  \widetilde{K}_{u},\left[
u_{1},u_{2},...,u_{m}\right]  \right)  ,$ and some $\left(  \widetilde{K}%
_{v},\left[  v_{1},v_{2},...,v_{n}\right]  \right)  \in K^{\operatorname*{int}%
}$ such that $v=\Pi\left(  \widetilde{K}_{v},\left[  v_{1},v_{2}%
,...,v_{n}\right]  \right)  $. Let $\widetilde{K}_{u,v}$ be a finite-free
extension ring of $K$ which contains both $\widetilde{K}_{u}$ and
$\widetilde{K}_{v}$ as subrings. (Such an extension ring exists, as was proven
in Theorem 5.3 \textbf{(a)}.) Theorem 5.3 \textbf{(b)} yields%
\[
u\widehat{+}v=\Pi\left(  \widetilde{K}_{u,v},\left[  u_{1},u_{2}%
,...,u_{m},v_{1},v_{2},...,v_{n}\right]  \right)  .
\]
In other words, if we define $m+n$ elements $w_{1},$ $w_{2},$ $...,$ $w_{m+n}$
by $w_{i}=\left\{
\begin{array}
[c]{c}%
u_{i},\text{ if }i\leq m;\\
v_{i-m},\text{ if }i>m
\end{array}
\right.  $ for every $i\in\left\{  1,2,...,m+n\right\}  $, then%
\[
u\widehat{+}v=\Pi\left(  \widetilde{K}_{u,v},\left[  w_{1},w_{2}%
,...,w_{m+n}\right]  \right)  .
\]
Consequently, Theorem 5.3 \textbf{(d)} yields%
\begin{align*}
&  \widehat{\lambda}^{j}\left(  u\widehat{+}v\right)  =\Pi\left(
\widetilde{K}_{u,v},\left[  \prod\limits_{i\in I}w_{i}\ \mid\ I\in
\mathcal{P}_{j}\left(  \left\{  1,2,...,m+n\right\}  \right)  \right]  \right)
\\
&  =\Pi\left(  \widetilde{K}_{u,v},\left[  \prod\limits_{\alpha\in J}%
w_{\alpha}\prod\limits_{\beta\in K^{\prime}}w_{\beta}\ \mid\ i\in\left\{
0,1,...,j\right\}  \text{, }J\in\mathcal{P}_{i}\left(  \left\{
1,2,...,m\right\}  \right)  \text{, }K^{\prime}\in\mathcal{P}_{j-i}\left(
\left\{  m+1,m+2,...,m+n\right\}  \right)  \right]  \right) \\
&  \ \ \ \ \ \ \ \ \ \ \left(
\begin{array}
[c]{c}%
\text{here, we have split the set }I\in\mathcal{P}_{j}\left(  \left\{
1,2,...,m+n\right\}  \right)  \text{ into }J=I\cap\left\{  1,2,...,m\right\}
\text{ and}\\
K^{\prime}=I\cap\left\{  m+1,m+2,...,m+n\right\}
\end{array}
\right) \\
&  =\Pi\left(  \widetilde{K}_{u,v},\left[  \prod\limits_{\alpha\in J}%
w_{\alpha}\prod\limits_{\beta\in K}w_{m+\beta}\ \mid\ i\in\left\{
0,1,...,j\right\}  \text{, }J\in\mathcal{P}_{i}\left(  \left\{
1,2,...,m\right\}  \right)  \text{, }K\in\mathcal{P}_{j-i}\left(  \left\{
1,2,...,n\right\}  \right)  \right]  \right) \\
&  \ \ \ \ \ \ \ \ \ \ \left(  \text{here, we have substituted }K=\left\{
u-m\mid u\in K^{\prime}\right\}  \text{ for }K^{\prime}\right) \\
&  =\Pi\left(  \widetilde{K}_{u,v},\left[  \prod\limits_{\alpha\in J}%
u_{\alpha}\prod\limits_{\beta\in K}v_{\beta}\ \mid\ i\in\left\{
0,1,...,j\right\}  \text{, }J\in\mathcal{P}_{i}\left(  \left\{
1,2,...,m\right\}  \right)  \text{, }K\in\mathcal{P}_{j-i}\left(  \left\{
1,2,...,n\right\}  \right)  \right]  \right) \\
&  \ \ \ \ \ \ \ \ \ \ \left(  \text{since }w_{i}=\left\{
\begin{array}
[c]{c}%
u_{i},\text{ if }i\leq m;\\
v_{i-m},\text{ if }i>m
\end{array}
\right.  \right) \\
&  =\widehat{\sum_{i=0}^{j}}\Pi\left(  \widetilde{K}_{u,v},\left[
\prod\limits_{\alpha\in J}u_{\alpha}\prod\limits_{\beta\in K}v_{\beta}%
\ \mid\ J\in\mathcal{P}_{i}\left(  \left\{  1,2,...,m\right\}  \right)
\text{, }K\in\mathcal{P}_{j-i}\left(  \left\{  1,2,...,n\right\}  \right)
\right]  \right) \\
&  \ \ \ \ \ \ \ \ \ \ \left(  \text{by Corollary 5.4 \textbf{(a)}}\right) \\
&  =\widehat{\sum_{i=0}^{j}}\underbrace{\Pi\left(  \widetilde{K}_{u},\left[
\prod\limits_{\alpha\in J}u_{\alpha}\ \mid\ J\in\mathcal{P}_{i}\left(
\left\{  1,2,...,m\right\}  \right)  \right]  \right)  }_{=\widehat{\lambda
}^{i}\left(  u\right)  \text{ by Theorem 5.3 \textbf{(d)}}}\widehat{\cdot
}\underbrace{\Pi\left(  \widetilde{K}_{v},\left[  \prod\limits_{\beta\in
K}v_{\beta}\ \mid\ \text{ }K\in\mathcal{P}_{j-i}\left(  \left\{
1,2,...,n\right\}  \right)  \right]  \right)  }_{=\widehat{\lambda}%
^{j-i}\left(  v\right)  \text{ by Theorem 5.3 \textbf{(d)}}}\\
&  \ \ \ \ \ \ \ \ \ \ \left(  \text{by Theorem 5.3 \textbf{(c)}}\right) \\
&  =\widehat{\sum_{i=0}^{j}}\widehat{\lambda}^{i}\left(  u\right)
\widehat{\cdot}\widehat{\lambda}^{j-i}\left(  v\right)  ,
\end{align*}
proving (\ref{SpezLemma1}). Theorem 5.1 \textbf{(b)} is proven.

\begin{quote}
\textbf{Theorem 5.6.} Let $\left(  K,\left(  \lambda^{i}\right)
_{i\in\mathbb{N}}\right)  $ be a $\lambda$-ring. Consider the map $\lambda
_{T}:K\rightarrow\Lambda\left(  K\right)  $ defined by
\[
\lambda_{T}\left(  x\right)  =\sum\limits_{i\in\mathbb{N}}\lambda^{i}\left(
x\right)  T^{i}\ \ \ \ \ \ \ \ \ \ \text{for every }x\in K.
\]
Then, $\lambda_{T}$ is an additive group homomorphism (where the additive
group structure on $\Lambda\left(  K\right)  $ is given by $\widehat{+}$).
\end{quote}

\textit{Proof of Theorem 5.6.} The map $\lambda_{T}$ is well-defined (i. e.
every $x\in K$ satisfies $\sum\limits_{i\in\mathbb{N}}\lambda^{i}\left(
x\right)  T^{i}\in\Lambda\left(  K\right)  $) because $\lambda^{0}\left(
x\right)  =1$ for every $x\in K$. The assertion that $\lambda_{T}$ is an
additive group homomorphism follows from Theorem 2.1. Theorem 5.6 is thus proven.

The following properties of the map $\operatorname*{ev}$ defined in Section 3
will turn out useful to us later:

\begin{quote}
\textbf{Theorem 5.7.} Let $K$ be a ring.

\textbf{(a)} For every $\mu\in K$, the map $\operatorname*{ev}_{\mu
T}:K\left[  \left[  T\right]  \right]  \rightarrow K\left[  \left[  T\right]
\right]  $ is continuous (with respect to the $\left(  T\right)  $-topology).

\textbf{(b)} Let $u\in1+K\left[  T\right]  ^{+}$. Assume that $u=\Pi\left(
\widetilde{K}_{u},\left[  u_{1},u_{2},...,u_{m}\right]  \right)  $ for some
$\left(  \widetilde{K}_{u},\left[  u_{1},u_{2},...,u_{m}\right]  \right)  \in
K^{\operatorname*{int}}$. Let $\mu\in K$. Then, $\operatorname*{ev}_{\mu
T}\left(  u\right)  =\Pi\left(  \widetilde{K}_{u},\left[  \mu u_{1},\mu
u_{2},...,\mu u_{m}\right]  \right)  =\Pi\left(  \widetilde{K}_{u},\left[  \mu
u_{i}\mid i\in\left\{  1,2,...,m\right\}  \right]  \right)  $.

\textbf{(c)} Let $u\in\Lambda\left(  K\right)  $ and $v\in\Lambda\left(
K\right)  $. Let $\mu\in K$. Then, $\operatorname*{ev}_{\mu T}\left(
u\right)  \widehat{+}\operatorname*{ev}_{\mu T}\left(  v\right)
=\operatorname*{ev}_{\mu T}\left(  u\widehat{+}v\right)  $.

\textbf{(d)} Let $u\in\Lambda\left(  K\right)  $ and $v\in\Lambda\left(
K\right)  $. Let $\mu\in K$ and $\nu\in K$. Then, $\operatorname*{ev}_{\mu
T}\left(  u\right)  \widehat{\cdot}\operatorname*{ev}_{\nu T}\left(  v\right)
=\operatorname*{ev}_{\mu\nu T}\left(  u\widehat{\cdot}v\right)  $.

\textbf{(e)} Let $u\in\Lambda\left(  K\right)  $. Let $\mu\in K$. Let
$k\in\mathbb{N}$. Then, $\widehat{\lambda}^{k}\left(  \operatorname*{ev}_{\mu
T}\left(  u\right)  \right)  =\operatorname*{ev}_{\mu^{k}T}\left(
\widehat{\lambda}^{k}\left(  u\right)  \right)  $.
\end{quote}

\textit{Proof of Theorem 5.7.} \textbf{(a)} Obvious from Theorem 5.5
\textbf{(b)}.

\textbf{(b)} By assumption, $u=\Pi\left(  \widetilde{K},\left[  u_{1}%
,u_{2},...,u_{n}\right]  \right)  =\prod\limits_{i=1}^{n}\left(
1+u_{i}T\right)  ,$ so that $\operatorname*{ev}_{\mu T}\left(  u\right)
=\prod\limits_{i=1}^{n}\left(  1+\mu u_{i}T\right)  =\Pi\left(  \widetilde
{K}_{u},\left[  \mu u_{1},\mu u_{2},...,\mu u_{m}\right]  \right)  $, and
Theorem 5.7 \textbf{(b)} is proven.

\textbf{(d)} Since the operation $\widehat{\cdot}$ and the map
$\operatorname*{ev}_{mT}$ are continuous (by Theorem 5.5 \textbf{(d) }and
Theorem 5.7 \textbf{(a)}), and $1+K\left[  T\right]  ^{+}$ is a dense subset
of $1+K\left[  \left[  T\right]  \right]  ^{+}$ (by Theorem 5.5 \textbf{(a)}),
this needs only to be shown for all $u,v\in1+K\left[  T\right]  ^{+}$. So let
us assume that $u,v\in1+K\left[  T\right]  ^{+}$. Then, there exist some
$\left(  \widetilde{K}_{u},\left[  u_{1},u_{2},...,u_{m}\right]  \right)  \in
K^{\operatorname*{int}}$ such that $u=\Pi\left(  \widetilde{K}_{u},\left[
u_{1},u_{2},...,u_{m}\right]  \right)  ,$ and some $\left(  \widetilde{K}%
_{v},\left[  v_{1},v_{2},...,v_{n}\right]  \right)  \in K^{\operatorname*{int}%
}$ such that $v=\Pi\left(  \widetilde{K}_{v},\left[  v_{1},v_{2}%
,...,v_{n}\right]  \right)  $. Theorem 5.3 \textbf{(a)} says that there exists
an extension ring $\widetilde{K}_{u,v}$ containing both $\widetilde{K}_{u}$
and $\widetilde{K}_{v}$ as subrings. Now, Theorem 5.3 \textbf{(c)} yields
$u\widehat{\cdot}v=\Pi\left(  \widetilde{K}_{u,v},\left[  u_{i}v_{j}%
\mid\left(  i,j\right)  \in\left\{  1,2,...,m\right\}  \times\left\{
1,2,...,n\right\}  \right]  \right)  $, so that we conclude%
\begin{align*}
\operatorname*{ev}\nolimits_{\mu\nu T}\left(  u\widehat{\cdot}v\right)   &
=\Pi\left(  \widetilde{K}_{u,v},\left[  \mu\nu u_{i}v_{j}\mid\left(
i,j\right)  \in\left\{  1,2,...,m\right\}  \times\left\{  1,2,...,n\right\}
\right]  \right) \\
&  =\Pi\left(  \widetilde{K}_{u,v},\left[  \mu u_{i}\cdot\nu v_{j}\mid\left(
i,j\right)  \in\left\{  1,2,...,m\right\}  \times\left\{  1,2,...,n\right\}
\right]  \right)
\end{align*}
using Theorem 5.7 \textbf{(b)}.

On the other hand, Theorem 5.7 \textbf{(b)} yields $\operatorname*{ev}_{\mu
T}\left(  u\right)  =\Pi\left(  \widetilde{K}_{u},\left[  \mu u_{1},\mu
u_{2},...,\mu u_{m}\right]  \right)  $ and (similarly) $\operatorname*{ev}%
_{\nu T}\left(  v\right)  =\Pi\left(  \widetilde{K}_{v},\left[  \nu v_{1},\nu
v_{2},...,\nu v_{m}\right]  \right)  $. Thus, Theorem 5.3 \textbf{(c)} yields%
\[
\operatorname*{ev}\nolimits_{\mu T}\left(  u\right)  \widehat{\cdot
}\operatorname*{ev}\nolimits_{\nu T}\left(  v\right)  =\Pi\left(
\widetilde{K}_{u,v},\left[  \mu u_{i}\cdot\nu v_{j}\mid\left(  i,j\right)
\in\left\{  1,2,...,m\right\}  \times\left\{  1,2,...,n\right\}  \right]
\right)  .
\]
Hence, $\operatorname*{ev}_{\mu T}\left(  u\right)  \widehat{\cdot
}\operatorname*{ev}_{\nu T}\left(  v\right)  =\operatorname*{ev}_{\mu\nu
T}\left(  u\widehat{\cdot}v\right)  $. This proves Theorem 5.7 \textbf{(d)}.

\textbf{(c)} Similar to \textbf{(d)}.

\textbf{(e)} Since the maps $\widehat{\lambda}^{k}$ and $\operatorname*{ev}%
_{mT}$ are continuous (by Theorem 5.5 \textbf{(d) }and Theorem 5.7
\textbf{(a)}), and $1+K\left[  T\right]  ^{+}$ is a dense subset of
$1+K\left[  \left[  T\right]  \right]  ^{+}$ (by Theorem 5.5 \textbf{(a)}),
this needs only to be shown for all $u\in1+K\left[  T\right]  ^{+}$. So, from
now on we assume that $u\in1+K\left[  T\right]  ^{+}$. Then, there exists some
$\left(  \widetilde{K}_{u},\left[  u_{1},u_{2},...,u_{m}\right]  \right)  \in
K^{\operatorname*{int}}$ such that $u=\Pi\left(  \widetilde{K}_{u},\left[
u_{1},u_{2},...,u_{m}\right]  \right)  $. Theorem 5.3 \textbf{(d)} then yields%
\[
\widehat{\lambda}^{k}\left(  u\right)  =\Pi\left(  \widetilde{K}_{u},\left[
\prod\limits_{i\in I}u_{i}\ \mid\ I\in\mathcal{P}_{k}\left(  \left\{
1,2,...,m\right\}  \right)  \right]  \right)  =\prod\limits_{I\in
\mathcal{P}_{k}\left(  \left\{  1,2,...,m\right\}  \right)  }\left(
1+\prod\limits_{i\in I}u_{i}\cdot T\right)  ,
\]
so that%
\begin{align*}
\operatorname*{ev}\nolimits_{\mu^{k}T}\left(  \widehat{\lambda}^{k}\left(
u\right)  \right)   &  =\prod\limits_{I\in\mathcal{P}_{k}\left(  \left\{
1,2,...,m\right\}  \right)  }\left(  1+\prod\limits_{i\in I}u_{i}\cdot\mu
^{k}T\right)  =\prod\limits_{I\in\mathcal{P}_{k}\left(  \left\{
1,2,...,m\right\}  \right)  }\left(  1+\prod\limits_{i\in I}\left(  \mu
u_{i}\right)  \cdot T\right) \\
&  =\Pi\left(  \widetilde{K}_{u},\left[  \prod\limits_{i\in I}\left(  \mu
u_{i}\right)  \ \mid\ I\in\mathcal{P}_{k}\left(  \left\{  1,2,...,m\right\}
\right)  \right]  \right)  .
\end{align*}
On the other hand, Theorem 5.7 \textbf{(b)} yields $\operatorname*{ev}_{\mu
T}\left(  u\right)  =\Pi\left(  \widetilde{K}_{u},\left[  \mu u_{1},\mu
u_{2},...,\mu u_{m}\right]  \right)  $ and thus, by Theorem 5.3 \textbf{(d)}
again,%
\[
\widehat{\lambda}^{k}\left(  \operatorname*{ev}\nolimits_{\mu T}\left(
u\right)  \right)  =\Pi\left(  \widetilde{K}_{u},\left[  \prod\limits_{i\in
I}\left(  \mu u_{i}\right)  \ \mid\ I\in\mathcal{P}_{k}\left(  \left\{
1,2,...,m\right\}  \right)  \right]  \right)  ,
\]
so that we conclude $\widehat{\lambda}^{k}\left(  \operatorname*{ev}_{\mu
T}\left(  u\right)  \right)  =\operatorname*{ev}_{\mu^{k}T}\left(
\widehat{\lambda}^{k}\left(  u\right)  \right)  $, and Theorem 5.7
\textbf{(e)} is proven.

Finally, a small definition that turns $\Lambda$ into a functor:

\begin{quote}
\textbf{Definition.} Every homomorphism $\varphi:K\rightarrow L$ of rings
canonically induces a $\lambda$-ring homomorphism $\Lambda\left(  K\right)
\rightarrow\Lambda\left(  L\right)  $ (which sends every $\sum\limits_{i\in
\mathbb{N}}a_{i}T^{i}\in\Lambda\left(  K\right)  $ to $\sum\limits_{i\in
\mathbb{N}}\varphi\left(  a_{i}\right)  T^{i}\in\Lambda\left(  L\right)  $).
This homomorphism $\Lambda\left(  K\right)  \rightarrow\Lambda\left(
L\right)  $ will be denoted by $\Lambda\left(  \varphi\right)  $.
\end{quote}

It is easy to see that this $\Lambda\left(  \varphi\right)  $ indeed is a
$\lambda$-ring homomorphism.\footnote{Basically, this is because the $P_{k}$
and $P_{k,j}$ are polynomials, and polynomials commute with ring
homomorphisms.} Besides, it has some obvious properties: If $\varphi$ is
surjective, then so is $\Lambda\left(  \varphi\right)  $. If $\varphi$ is
injective, then $\Lambda\left(  \varphi\right)  $ is injective as well; thus,
if $L$ is an extension ring of $K$, then $\Lambda\left(  L\right)  $ can be
canonically considered an extension ring of $\Lambda\left(  K\right)  $.

\begin{quotation}
\textit{Exercise 5.1.} Let $K$ be a ring. For every monic polynomial $P\in
K\left[  T\right]  $, there exists a finite-free extension ring $K_{p}$ of the
ring $K$ and $n$ elements $p_{1},$ $p_{2},$ $...,$ $p_{n}$ of this extension
ring $K_{p}$ such that $p=\prod\limits_{i=1}^{n}\left(  T-p_{i}\right)  $ in
$K_{p}\left[  T\right]  $, where $n=\deg P$.

\textit{Exercise 5.2.} Let $K$ be a ring, and $L$ an extension ring of $K$.
For some $n\in\mathbb{N}$, an element $u$ of $L$ is said to be $n$%
\textit{-integral} over $K$ if there exists a monic polynomial $P\in K\left[
X\right]  $ such that $\deg P=n$ and $P\left(  u\right)  =0$.

Let $n\in\mathbb{N}$ and $m\in\mathbb{N}$. Let $\alpha$ and $\beta$ be two
elements of $L$ such that $\alpha$ is $n$-integral over $K$ and $\beta$ is
$m$-integral over $K$. Prove that $\alpha\beta$ is $nm$-integral over $K$.

[This is a known fact, but it turns out to also be a simple corollary of our
construction of the polynomials $P_{k}$ further above.]\ \ \ \ 
\end{quotation}

\begin{center}
\fbox{\textbf{6. Special }$\lambda$\textbf{-rings}}
\end{center}

Now we will define a particular subclass of $\lambda$-rings that we will be
interested in from now on:

\begin{quote}
\textbf{Definition.} \textbf{1)} Let $\left(  K,\left(  \lambda^{i}\right)
_{i\in\mathbb{N}}\right)  $ be a $\lambda$-ring. The map $\lambda_{T}$ defined
in Theorem 5.6 is an additive group homomorphism (by Theorem 5.6). We call
$\left(  K,\left(  \lambda^{i}\right)  _{i\in\mathbb{N}}\right)  $ a
\textit{special }$\lambda$\textit{-ring} if this map $\lambda_{T}:\left(
K,\left(  \lambda^{i}\right)  _{i\in\mathbb{N}}\right)  \rightarrow\left(
\Lambda\left(  K\right)  ,\left(  \widehat{\lambda}^{i}\right)  _{i\in
\mathbb{N}}\right)  $ is a $\lambda$-ring homomorphism.

\textbf{2)} Let $\left(  K,\left(  \lambda^{i}\right)  _{i\in\mathbb{N}%
}\right)  $ be a $\lambda$-ring. Let $L$ be a sub-$\lambda$-ring of $K$. If
$\left(  L,\left(  \lambda^{i}\mid_{L}\right)  _{i\in\mathbb{N}}\right)  $ is
a special $\lambda$-ring, then we call $L$ a \textit{special sub-}$\lambda
$\textit{-ring} of $K$.
\end{quote}

A different, more down-to-earth characterization of special $\lambda$-rings:

\begin{quote}
\textbf{Theorem 6.1.} Let $\left(  K,\left(  \lambda^{i}\right)
_{i\in\mathbb{N}}\right)  $ be a $\lambda$-ring.

Then, $\left(  K,\left(  \lambda^{i}\right)  _{i\in\mathbb{N}}\right)  $ is a
special $\lambda$-ring if and only if%
\begin{align}
&  \lambda^{k}\left(  xy\right)  =P_{k}\left(  \lambda^{1}\left(  x\right)
,\lambda^{2}\left(  x\right)  ,...,\lambda^{k}\left(  x\right)  ,\lambda
^{1}\left(  y\right)  ,\lambda^{2}\left(  y\right)  ,...,\lambda^{k}\left(
y\right)  \right) \nonumber\\
&  \ \ \ \ \ \ \ \ \ \ \text{for every }k\in\mathbb{N}\text{, }x\in K\text{
and }y\in K \label{Lkxy}%
\end{align}
and%
\begin{align}
&  \lambda^{k}\left(  \lambda^{j}\left(  x\right)  \right)  =P_{k,j}\left(
\lambda^{1}\left(  x\right)  ,\lambda^{2}\left(  x\right)  ,...,\lambda
^{kj}\left(  x\right)  \right) \nonumber\\
&  \ \ \ \ \ \ \ \ \ \ \text{for every }k\in\mathbb{N}\text{, }j\in
\mathbb{N}\text{ and }x\in K. \label{LkLjx}%
\end{align}



\end{quote}

\textit{Proof of Theorem 6.1.} According to the preceeding definition,
$\left(  K,\left(  \lambda^{i}\right)  _{i\in\mathbb{N}}\right)  $ is a
special $\lambda$-ring if and only if the map $\lambda_{T}$ is a $\lambda
$-ring homomorphism. This map is always an additive group homomorphism (by
Theorem 5.6); hence, it is a $\lambda$-ring homomorphism if and only if it
satisfies the three conditions%
\begin{align*}
\lambda_{T}\left(  xy\right)   &  =\lambda_{T}\left(  x\right)  \widehat
{\cdot}\lambda_{T}\left(  y\right)  \ \ \ \ \ \ \ \ \ \ \text{for every }x\in
K\text{ and }y\in K,\\
\lambda_{T}\left(  1\right)   &  =1+T,\ \ \ \ \ \ \ \ \ \ \text{and}\\
\lambda_{T}\left(  \lambda^{j}\left(  x\right)  \right)   &  =\widehat
{\lambda}^{j}\left(  \lambda_{T}\left(  x\right)  \right)
\ \ \ \ \ \ \ \ \ \ \text{for every }j\in\mathbb{N}\text{ and }x\in K
\end{align*}
(note that $1+T$ is the multiplicative unity of $\Lambda\left(  K\right)  $).
The second of these three conditions actually follows from the third one
(since $\lambda_{T}\left(  \lambda^{j}\left(  x\right)  \right)
=\widehat{\lambda}^{j}\left(  \lambda_{T}\left(  x\right)  \right)  $, applied
to $j=0$, yields $\lambda_{T}\left(  1\right)  =1+T$), so we see that the map
$\lambda_{T}$ is a $\lambda$-ring homomorphism if and only if it satisfies the
two conditions%
\begin{align*}
\lambda_{T}\left(  xy\right)   &  =\lambda_{T}\left(  x\right)  \widehat
{\cdot}\lambda_{T}\left(  y\right)  \ \ \ \ \ \ \ \ \ \ \text{for every }x\in
K\text{ and }y\in K,\ \ \ \ \ \ \ \ \ \ \text{and}\\
\lambda_{T}\left(  \lambda^{j}\left(  x\right)  \right)   &  =\widehat
{\lambda}^{j}\left(  \lambda_{T}\left(  x\right)  \right)
\ \ \ \ \ \ \ \ \ \ \text{for every }j\in\mathbb{N}\text{ and }x\in K.
\end{align*}


But these two conditions are equivalent to (\ref{Lkxy}) and (\ref{LkLjx}),
respectively (because of the definitions of $\widehat{\cdot}$ and
$\widehat{\lambda}^{j}$ and because two formal power series are equal if and
only if their respective coefficients are equal). This proves Theorem 6.1.

\begin{quote}
\textbf{Theorem 6.2 (Grothendieck).} Let $K$ be a ring. Then, $\left(
\Lambda\left(  K\right)  ,\left(  \widehat{\lambda}^{i}\right)  _{i\in
\mathbb{N}}\right)  $ is a special $\lambda$-ring.
\end{quote}

\textit{Proof of Theorem 6.2.} According to Theorem 6.1, we only have to prove
that%
\begin{align}
&  \widehat{\lambda}^{k}\left(  u\widehat{\cdot}v\right)  =\widehat{P_{k}%
}\left(  \widehat{\lambda}^{1}\left(  u\right)  ,\widehat{\lambda}^{2}\left(
u\right)  ,...,\widehat{\lambda}^{k}\left(  u\right)  ,\widehat{\lambda}%
^{1}\left(  v\right)  ,\widehat{\lambda}^{2}\left(  v\right)  ,...,\widehat
{\lambda}^{k}\left(  v\right)  \right) \nonumber\\
&  \ \ \ \ \ \ \ \ \ \ \text{for every }k\in\mathbb{N}\text{, }u\in
\Lambda\left(  K\right)  \text{ and }v\in\Lambda\left(  K\right)  ,
\label{6.2.P.Lkxy}%
\end{align}
and%
\begin{align}
&  \widehat{\lambda}^{k}\left(  \widehat{\lambda}^{j}\left(  u\right)
\right)  =\widehat{P_{k,j}}\left(  \widehat{\lambda}^{1}\left(  u\right)
,\widehat{\lambda}^{2}\left(  u\right)  ,...,\widehat{\lambda}^{kj}\left(
u\right)  \right) \nonumber\\
&  \ \ \ \ \ \ \ \ \ \ \text{for every }k\in\mathbb{N}\text{, }j\in
\mathbb{N}\text{ and }u\in\Lambda\left(  K\right)  . \label{6.2.P.LkLjx}%
\end{align}
Here, we are using the following \textit{notation:} If $S\in\mathbb{Z}\left[
\alpha_{1},\alpha_{2},...,\alpha_{kj}\right]  $ is a polynomial, then
$\widehat{S}\left(  \widehat{\lambda}^{1}\left(  u\right)  ,\widehat{\lambda
}^{2}\left(  u\right)  ,...,\widehat{\lambda}^{kj}\left(  u\right)  \right)  $
denotes the polynomial $S$ applied to $\widehat{\lambda}^{1}\left(  u\right)
,$ $\widehat{\lambda}^{2}\left(  u\right)  ,$ $...,$ $\widehat{\lambda}%
^{kj}\left(  u\right)  $ \textit{as elements of the ring }$\Lambda\left(
K\right)  $ (and not as elements of the ring $K\left[  \left[  T\right]
\right]  $). For instance, if $S=\alpha_{1}+\alpha_{2}+...+\alpha_{kj},$ then
$\widehat{S}\left(  \widehat{\lambda}^{1}\left(  u\right)  ,\widehat{\lambda
}^{2}\left(  u\right)  ,...,\widehat{\lambda}^{kj}\left(  u\right)  \right)  $
means $\widehat{\lambda}^{1}\left(  u\right)  \widehat{+}\widehat{\lambda}%
^{2}\left(  u\right)  \widehat{+}...\widehat{+}\widehat{\lambda}^{kj}\left(
u\right)  $ (and not $\widehat{\lambda}^{1}\left(  u\right)  +\widehat
{\lambda}^{2}\left(  u\right)  +...+\widehat{\lambda}^{kj}\left(  u\right)  $,
where $+$ denotes the addition in the ring $K\left[  \left[  T\right]
\right]  $). This explains how the right hand sides of the equations
(\ref{6.2.P.Lkxy}) and (\ref{6.2.P.LkLjx}) should be understood.

Let us prove (\ref{6.2.P.Lkxy}) here. Since the subset $1+K\left[  T\right]
^{+}$ is dense in $1+K\left[  \left[  T\right]  \right]  ^{+}=\Lambda\left(
K\right)  $ (by Theorem 5.5 \textbf{(a)}), and since $\widehat{\cdot}$ and
$\widehat{\lambda}^{i}$ are continuous (by Theorem 5.5 \textbf{(d)}), it will
be enough to verify (\ref{6.2.P.Lkxy}) for $u\in1+K\left[  T\right]  ^{+}$ and
$v\in1+K\left[  T\right]  ^{+}$. Then, there exist some $\left(  \widetilde
{K}_{u},\left[  u_{1},u_{2},...,u_{m}\right]  \right)  \in
K^{\operatorname*{int}}$ such that $u=\Pi\left(  \widetilde{K},\left[
u_{1},u_{2},...,u_{m}\right]  \right)  ,$ and some $\left(  \widetilde{K}%
_{v},\left[  v_{1},v_{2},...,v_{n}\right]  \right)  \in K^{\operatorname*{int}%
}$ such that $v=\Pi\left(  \widetilde{K},\left[  v_{1},v_{2},...,v_{n}\right]
\right)  .$ By Theorem 5.3 \textbf{(a)}, there exists a finite-free extension
ring $\widetilde{K}_{u,v}$ of $K$ which contains both $\widetilde{K}_{u}$ and
$\widetilde{K}_{v}$ as subrings. We replace $K$ by $\widetilde{K}_{u,v}$ now
(silently using the obvious fact that the injection $K\rightarrow\widetilde
{K}_{u,v}$ canonically yields an injection $\Lambda\left(  K\right)
\rightarrow\Lambda\left(  \widetilde{K}_{u,v}\right)  $). Hence, we can now
assume that $u_{1},$ $u_{2},$ $...,$ $u_{m},$ $v_{1},$ $v_{2},$ $...,$ $v_{n}$
all lie in $K$. Theorem 5.3 \textbf{(c)} yields $u\widehat{\cdot}v=\Pi\left(
\widetilde{K}_{u,v},\left[  u_{i}v_{j}\mid\left(  i,j\right)  \in\left\{
1,2,...,m\right\}  \times\left\{  1,2,...,n\right\}  \right]  \right)  $.

There exists a ring homomorphism%
\[
\mathbb{Z}\left[  U_{1},U_{2},...,U_{m},V_{1},V_{2},...,V_{n}\right]
\rightarrow\Lambda\left(  K\right)
\]
which maps $U_{i}$ to $1+u_{i}T$ for every $i$ and $V_{j}$ to $1+v_{j}T$ for
every $j$. This homomorphism maps $X_{i}=\sum\limits_{\substack{S\subseteq
\left\{  1,2,...,m\right\}  ;\\\left\vert S\right\vert =i}}\prod\limits_{k\in
S}U_{k}$ to%
\begin{align*}
\widehat{\sum\limits_{\substack{S\subseteq\left\{  1,2,...,m\right\}
;\\\left\vert S\right\vert =i}}}\widehat{\prod\limits_{k\in S}}\underbrace
{\left(  1+u_{k}T\right)  }_{=\Pi\left(  K,\left[  u_{k}\right]  \right)  }
&  =\widehat{\sum\limits_{\substack{S\subseteq\left\{  1,2,...,m\right\}
;\\\left\vert S\right\vert =i}}}\underbrace{\widehat{\prod\limits_{k\in S}}%
\Pi\left(  K,\left[  u_{k}\right]  \right)  }_{\substack{=\Pi\left(  K,\left[
\prod\limits_{k\in S}u_{k}\right]  \right)  \\\text{according to Corollary 5.4
\textbf{(b)}}}}=\widehat{\sum\limits_{\substack{S\subseteq\left\{
1,2,...,m\right\}  ;\\\left\vert S\right\vert =i}}}\Pi\left(  K,\left[
\prod\limits_{k\in S}u_{k}\right]  \right) \\
&  =\Pi\left(  K,\left[  \prod\limits_{k\in S}u_{k}\mid S\subseteq\left\{
1,2,...,m\right\}  ;\ \left\vert S\right\vert =i\right]  \right) \\
&  =\Pi\left(  K,\left[  \prod\limits_{k\in S}u_{k}\mid S\in\mathcal{P}%
_{i}\left(  \left\{  1,2,...,m\right\}  \right)  \right]  \right) \\
&  =\widehat{\lambda}^{i}\left(  u\right)  \ \ \ \ \ \ \ \ \ \ \left(
\text{after Theorem 5.3 \textbf{(d)}}\right)
\end{align*}
and $Y_{j}$ to $\widehat{\lambda}^{j}\left(  v\right)  $ for every
$j\in\mathbb{N}$ (according to a similar argument). Hence, applying this
homomorphism to (\ref{Pk1}), we obtain%
\[
\widehat{\sum_{\substack{S\subseteq\left\{  1,2,...,m\right\}  \times\left\{
1,2,...,n\right\}  ;\\\left\vert S\right\vert =k}}}\widehat{\prod_{\left(
i,j\right)  \in S}}\left(  1+u_{i}T\right)  \widehat{\cdot}\left(
1+v_{j}T\right)  =\widehat{P_{k}}\left(  \widehat{\lambda}^{1}\left(
u\right)  ,\widehat{\lambda}^{2}\left(  u\right)  ,...,\widehat{\lambda}%
^{k}\left(  u\right)  ,\widehat{\lambda}^{1}\left(  v\right)  ,\widehat
{\lambda}^{2}\left(  v\right)  ,...,\widehat{\lambda}^{k}\left(  v\right)
\right)  .
\]
But%
\begin{align*}
&  \widehat{\sum_{\substack{S\subseteq\left\{  1,2,...,m\right\}
\times\left\{  1,2,...,n\right\}  ;\\\left\vert S\right\vert =k}}}%
\widehat{\prod_{\left(  i,j\right)  \in S}}\underbrace{\left(  1+u_{i}%
T\right)  \widehat{\cdot}\left(  1+v_{j}T\right)  }_{\substack{=\Pi\left(
K,\left[  u_{i}\right]  \right)  \widehat{\cdot}\Pi\left(  K,\left[
v_{j}\right]  \right)  \\=\Pi\left(  K,\left[  u_{i}v_{j}\right]  \right)
\text{ after}\\\text{Theorem 5.3 \textbf{(c)}}}}=\widehat{\sum
_{\substack{S\subseteq\left\{  1,2,...,m\right\}  \times\left\{
1,2,...,n\right\}  ;\\\left\vert S\right\vert =k}}}\underbrace{\widehat
{\prod_{\left(  i,j\right)  \in S}}\Pi\left(  K,\left[  u_{i}v_{j}\right]
\right)  }_{\substack{=\Pi\left(  K,\left[  \prod\limits_{\left(  i,j\right)
\in S}u_{i}v_{j}\right]  \right)  \\\text{after Corollary 5.4 \textbf{(b)}}%
}}\\
&  =\widehat{\sum_{\substack{S\subseteq\left\{  1,2,...,m\right\}
\times\left\{  1,2,...,n\right\}  ;\\\left\vert S\right\vert =k}}}\Pi\left(
K,\left[  \prod\limits_{\left(  i,j\right)  \in S}u_{i}v_{j}\right]  \right)
\\
&  =\Pi\left(  K,\left[  \prod_{\left(  i,j\right)  \in S}u_{i}v_{j}\mid
S\subseteq\left\{  1,2,...,m\right\}  \times\left\{  1,2,...,n\right\}
;\ \left\vert S\right\vert =k\right]  \right)  \ \ \ \ \ \ \ \ \ \ \left(
\text{after Corollary 5.4 \textbf{(a)}}\right) \\
&  =\Pi\left(  K,\left[  \prod_{\left(  i,j\right)  \in S}u_{i}v_{j}\mid
S\in\mathcal{P}_{k}\left(  \left\{  1,2,...,m\right\}  \times\left\{
1,2,...,n\right\}  \right)  \right]  \right) \\
&  =\widehat{\lambda}^{k}\left(  u\widehat{\cdot}v\right)
\end{align*}
after Theorem 5.3 \textbf{(d)}, applied to $u\widehat{\cdot}v$ instead of $u$
(and keeping in mind that $u\widehat{\cdot}v=\Pi\left(  \widetilde{K}%
_{u,v},\left[  u_{i}v_{j}\mid\left(  i,j\right)  \in\left\{
1,2,...,m\right\}  \times\left\{  1,2,...,n\right\}  \right]  \right)  $).
Thus, (\ref{6.2.P.Lkxy}) is proven. Similarly we can show (\ref{6.2.P.LkLjx})
(the details are left to the reader). Theorem 6.2 is thus proven.

Theorem 6.1 gives us an alternative definition of special $\lambda$-rings via
the polynomials $P_{k}$ and $P_{k,j}$. Why, then, did we define the notion of
special $\lambda$-rings via the map $\lambda_{T}:\left(  K,\left(  \lambda
^{i}\right)  _{i\in\mathbb{N}}\right)  \rightarrow\left(  \Lambda\left(
K\right)  ,\left(  \widehat{\lambda}^{i}\right)  _{i\in\mathbb{N}}\right)  $
rather than using Theorem 6.1? The reason is that while Theorem 6.1 provides
an easy-to-formulate definition of special $\lambda$-rings, it is rather hard
to work with. In order to check that some given ring is a special $\lambda
$-ring using Theorem 6.1, we would have to prove the identities (\ref{Lkxy})
and (\ref{LkLjx}), which is a difficult task since the polynomials $P_{k}$ and
$P_{k,j}$ are very hard to compute explicitely. Using the definition that we
gave, we would have to check that $\lambda_{T}:\left(  K,\left(  \lambda
^{i}\right)  _{i\in\mathbb{N}}\right)  \rightarrow\left(  \Lambda\left(
K\right)  ,\left(  \widehat{\lambda}^{i}\right)  _{i\in\mathbb{N}}\right)  $
is a $\lambda$-ring homomorphism, what seems to be harder, but turns out not
that hard since Exercise 2.1 reduces this to checking some identity at
$\mathbb{Z}$-module generators of $K$.

\begin{quotation}
\textit{Exercise 6.1.} Let $K$ be a ring. Consider the localization $\left(
1+K\left[  T\right]  ^{+}\right)  ^{-1}K\left[  T\right]  $ of the polynomial
ring $K\left[  T\right]  $ at the multiplicatively closed subset $1+K\left[
T\right]  ^{+}$. \ \ \ \ \footnote{When $K$ is a field, this localization is
simply the (local) ring of the (so-called) rational functions in one variable
over $K$. ("Rational function" is an evil misnomer, in this case, but at least
people know what it means.)} This localization $\left(  1+K\left[  T\right]
^{+}\right)  ^{-1}K\left[  T\right]  $ can be considered a subring of
$K\left[  \left[  T\right]  \right]  $ (since $K\left[  T\right]  \subseteq
K\left[  \left[  T\right]  \right]  $, and every element of $1+K\left[
T\right]  ^{+}$ is invertible in $K\left[  \left[  T\right]  \right]  $).
Prove that the set $\left(  1+K\left[  T\right]  ^{+}\right)  ^{-1}K\left[
T\right]  \cap\Lambda\left(  K\right)  $ is a special sub-$\lambda$-ring of
$\Lambda\left(  K\right)  $.

\textit{Exercise 6.2.} Let $\left(  K,\left(  \lambda^{i}\right)
_{i\in\mathbb{N}}\right)  $ be a special $\lambda$-ring. Then, prove that none
of the elements $1,$ $2,$ $3,$ $...$ of the ring $K$ equals zero in $K$.

\textit{Exercise 6.3.} Consider the ring $\mathbb{Z}\left[  X\right]
\diagup\left(  X^{2},2X\right)  =\mathbb{Z}\left[  x\right]  $, where $x$
denotes the residue class of $X$ modulo the ideal $\left(  X^{2},2X\right)  $.

Define a map $\lambda_{T}:\mathbb{Z}\left[  x\right]  \rightarrow\left(
\mathbb{Z}\left[  x\right]  \right)  \left[  \left[  T\right]  \right]  $ by
$\lambda_{T}\left(  a+bx\right)  =\left(  1+T\right)  ^{a}\left(  1+xT\right)
^{b}$ for every $a\in\mathbb{Z}$ and $b\in\mathbb{Z}$.

Define a map $\lambda^{i}:\mathbb{Z}\left[  x\right]  \rightarrow
\mathbb{Z}\left[  x\right]  $ for every $i\in\mathbb{N}$ through the condition
$\lambda_{T}\left(  x\right)  =\sum\limits_{i\in\mathbb{N}}\lambda^{i}\left(
x\right)  T^{i}$ for every $x\in K$.

Prove that $\left(  \mathbb{Z}\left[  x\right]  ,\left(  \lambda^{i}\right)
_{i\in\mathbb{N}}\right)  $ is a special $\lambda$-ring. [This way, we see
that the additive group of a special $\lambda$-ring needs not be torsion-free.]

\textit{Exercise 6.4.} Let $\left(  K,\left(  \lambda^{i}\right)
_{i\in\mathbb{N}}\right)  $ be a $\lambda$-ring. Let $E$ be a generating set
of the $\mathbb{Z}$-module $K$.

Prove that the $\lambda$-ring $\left(  K,\left(  \lambda^{i}\right)
_{i\in\mathbb{N}}\right)  $ is special if and only if it satisfies%
\begin{align}
&  \lambda^{k}\left(  xy\right)  =P_{k}\left(  \lambda^{1}\left(  x\right)
,\lambda^{2}\left(  x\right)  ,...,\lambda^{k}\left(  x\right)  ,\lambda
^{1}\left(  y\right)  ,\lambda^{2}\left(  y\right)  ,...,\lambda^{k}\left(
y\right)  \right) \nonumber\\
&  \ \ \ \ \ \ \ \ \ \ \text{for every }k\in\mathbb{N}\text{, }x\in E\text{
and }y\in E \label{LkxyE}%
\end{align}
and%
\begin{align}
&  \lambda^{k}\left(  \lambda^{j}\left(  x\right)  \right)  =P_{k,j}\left(
\lambda^{1}\left(  x\right)  ,\lambda^{2}\left(  x\right)  ,...,\lambda
^{kj}\left(  x\right)  \right) \nonumber\\
&  \ \ \ \ \ \ \ \ \ \ \text{for every }k\in\mathbb{N}\text{, }j\in
\mathbb{N}\text{ and }x\in E. \label{LkLjxE}%
\end{align}


\textit{Exercise 6.5.} Let $K$ be a ring. Let $i\in\mathbb{N}$. Define a
mapping $\operatorname*{coeff}\nolimits_{i}:\Lambda\left(  K\right)
\rightarrow K$ by $\operatorname*{coeff}\nolimits_{i}\left(  \sum
\limits_{j\in\mathbb{N}}a_{j}T^{j}\right)  =a_{i}$ for every $\sum
\limits_{j\in\mathbb{N}}a_{j}T^{j}\in\Lambda\left(  K\right)  $ (with
$a_{j}\in K$ for every $j\in\mathbb{N}$). (In other words,
$\operatorname*{coeff}\nolimits_{i}$ is the mapping that takes a power series
and returns its coefficient before $T^{i}.$)

Prove that%
\[
\operatorname*{coeff}\nolimits_{i}\left(  u\right)  =\operatorname*{coeff}%
\nolimits_{1}\left(  \widehat{\lambda}^{i}\left(  u\right)  \right)
\ \ \ \ \ \ \ \ \ \ \text{for every }u\in\Lambda\left(  K\right)  .
\]


\textit{Exercise 6.6.} Let $\left(  K,\left(  \lambda^{i}\right)
_{i\in\mathbb{N}}\right)  $ be a special $\lambda$-ring, and $A$ be a ring.
Let $\varphi:K\rightarrow A$ be a ring homomorphism, and let
$\operatorname*{coeff}\nolimits_{1}^{A}:\Lambda\left(  A\right)  \rightarrow
A$ be the mapping defined by $\operatorname*{coeff}\nolimits_{1}^{A}\left(
\sum\limits_{j\in\mathbb{N}}a_{j}T^{j}\right)  =a_{1}$ for every
$\sum\limits_{j\in\mathbb{N}}a_{j}T^{j}\in\Lambda\left(  A\right)  $ (with
$a_{j}\in A$ for every $j\in\mathbb{N}$). (In other words,
$\operatorname*{coeff}\nolimits_{1}^{A}$ is the mapping that takes a power
series and returns its coefficient before $T^{1}.$)

As Theorem 5.1 \textbf{(b)} states, $\left(  \Lambda\left(  A\right)  ,\left(
\widehat{\lambda}_{A}^{i}\right)  _{i\in\mathbb{N}}\right)  $ is a $\lambda
$-ring, where the maps $\widehat{\lambda}_{A}^{i}:\Lambda\left(  A\right)
\rightarrow\Lambda\left(  A\right)  $ are defined in the same way as the maps
$\widehat{\lambda}^{i}:\Lambda\left(  K\right)  \rightarrow\Lambda\left(
K\right)  $ (which we have defined in Section 5) but for the ring $K$ instead
of $A$.

Prove that there exists one and only one $\lambda$-ring homomorphism
$\widetilde{\varphi}:K\rightarrow\Lambda\left(  A\right)  $ such that
$\operatorname*{coeff}\nolimits_{1}^{A}\circ\widetilde{\varphi}=\varphi.$
\end{quotation}

\begin{center}
\fbox{\textbf{7. Examples of special }$\lambda$\textbf{-rings}}
\end{center}

We have learned a lot of examples for $\lambda$-rings, but which of them are
special? Of course, the trivial ring $0$ with the trivial maps $\lambda
^{i}:0\rightarrow0$ is a special $\lambda$-ring. Also, we know a vast class of
special $\lambda$-rings from Theorem 6.2. Obviously, every sub-$\lambda$-ring
of a special $\lambda$-ring is special. On the other hand, the $\lambda$-ring
$\left(  K,\left(  \lambda^{i}\right)  _{i\in\mathbb{N}}\right)  $ defined in
Exercise 3.3 \textbf{(a)} is not special unless $p=1$. What happens to the
other examples from Section 3?

\begin{quote}
\textbf{Theorem 7.1.} The $\lambda$-ring $\left(  \mathbb{Z},\left(
\lambda^{i}\right)  _{i\in\mathbb{N}}\right)  $ defined in Theorem 3.1 is special.
\end{quote}

\textit{Proof of Theorem 7.1.} According to Theorem 6.1, we just have to
verify the identities (\ref{Lkxy}) and (\ref{LkLjx}) for $K=\mathbb{Z}$. In
other words, we have to prove that%
\begin{equation}
\dbinom{xy}{k}=P_{k}\left(  \dbinom{x}{1},\dbinom{x}{2},...,\dbinom{x}%
{k},\dbinom{y}{1},\dbinom{y}{2},...,\dbinom{y}{k}\right)  \label{7.1.Lkxy}%
\end{equation}
for every $k\in\mathbb{N}$, $x\in\mathbb{Z}$ and $y\in\mathbb{Z}$ and%
\begin{equation}
\dbinom{\dbinom{x}{j}}{k}=P_{k,j}\left(  \dbinom{x}{1},\dbinom{x}%
{2},...,\dbinom{x}{kj}\right)  \label{7.1.LkLjx}%
\end{equation}
for every $k\in\mathbb{N}$, $j\in\mathbb{N}$ and $x\in\mathbb{Z}.$

Let us prove (\ref{7.1.Lkxy}): Fix $k\in\mathbb{N}$. Then, (\ref{7.1.Lkxy}) is
a polynomial identity in $x$ and in $y.$ Hence, (for the same reason as in the
proof of Theorem 3.1) it is enough to prove (\ref{7.1.Lkxy}) for all natural
$x$ and $y$. In this case, let $m=x$ and $n=y$. There exists a ring
homomorphism $\mathbb{Z}\left[  U_{1},U_{2},...,U_{m},V_{1},V_{2}%
,...,V_{n}\right]  \rightarrow\mathbb{Z}$ mapping every $U_{i}$ to $1$ and
every $V_{j}$ to $1$. This homomorphism maps $X_{i}=\sum
\limits_{\substack{S\subseteq\left\{  1,2,...,m\right\}  ;\\\left\vert
S\right\vert =i}}\prod\limits_{k\in S}U_{k}$ to%
\[
\sum\limits_{\substack{S\subseteq\left\{  1,2,...,m\right\}  ;\\\left\vert
S\right\vert =i}}\prod\limits_{k\in S}1=\sum\limits_{\substack{S\subseteq
\left\{  1,2,...,m\right\}  ;\\\left\vert S\right\vert =i}}1=\sum
\limits_{S\in\mathcal{P}_{i}\left(  \left\{  1,2,...,m\right\}  \right)
}1=\left\vert \mathcal{P}_{i}\left(  \left\{  1,2,...,m\right\}  \right)
\right\vert =\dbinom{m}{i}%
\]
for every $i\in\mathbb{N},$ and maps $Y_{j}$ to $\dbinom{n}{j}$ for every
$j\in\mathbb{N}$. Thus, applying this homomorphism to the polynomial identity
(\ref{Pk1}), we obtain%

\[
\sum_{\substack{S\subseteq\left\{  1,2,...,m\right\}  \times\left\{
1,2,...,n\right\}  ;\\\left\vert S\right\vert =k}}\prod_{\left(  i,j\right)
\in S}1\cdot1=P_{k}\left(  \dbinom{m}{1},\dbinom{m}{2},...,\dbinom{m}%
{k},\dbinom{n}{1},\dbinom{n}{2},...,\dbinom{n}{k}\right)  .
\]
Since $m=x,$ $n=y$ and
\begin{align*}
\sum_{\substack{S\subseteq\left\{  1,2,...,m\right\}  \times\left\{
1,2,...,n\right\}  ;\\\left\vert S\right\vert =k}}\underbrace{\prod_{\left(
i,j\right)  \in S}1\cdot1}_{=1}  &  =\sum_{\substack{S\subseteq\left\{
1,2,...,m\right\}  \times\left\{  1,2,...,n\right\}  ;\\\left\vert
S\right\vert =k}}1=\sum_{S\in\mathcal{P}_{k}\left(  \left\{
1,2,...,m\right\}  \times\left\{  1,2,...,n\right\}  \right)  }\\
&  =\left\vert \mathcal{P}_{k}\left(  \left\{  1,2,...,m\right\}
\times\left\{  1,2,...,n\right\}  \right)  \right\vert =\dbinom{mn}{k}%
=\dbinom{xy}{k},
\end{align*}
this equality transforms into (\ref{7.1.Lkxy}). Hence, (\ref{7.1.Lkxy}) is
proven (since, as we said, once it is proven for natural $x$ and $y$, it
follows for all integers $x$ and $y$). Just as we have derived (\ref{7.1.Lkxy}%
) from (\ref{Pk1}), we can derive (\ref{LkLjx}) from (\ref{Pkj1}), and Theorem
7.1 is proven.

This generalizes:

\begin{quote}
\textbf{Theorem 7.2.} Let $K$ be a binomial ring. The $\lambda$-ring $\left(
K,\left(  \lambda^{i}\right)  _{i\in\mathbb{N}}\right)  $ defined in Theorem
3.2 is special.
\end{quote}

\textit{Proof of Theorem 7.2.} This follows from our proof of Theorem 7.1 in
the same way as Theorem 3.2 followed from our proof of Theorem 3.1. To be more
precise: According to Theorem 6.1, the $\lambda$-ring $\left(  K,\left(
\lambda^{i}\right)  _{i\in\mathbb{N}}\right)  $ is special if it satisfies the
identities (\ref{Lkxy}) and (\ref{LkLjx}). This means (\ref{7.1.Lkxy}) for
every $k\in\mathbb{N}$, $x\in K$ and $y\in K$ and (\ref{7.1.LkLjx}) for every
$k\in\mathbb{N}$, $j\in\mathbb{N}$ and $x\in K$. In the proof of Theorem 7.1,
we have proven these identities for $x\in\mathbb{Z}$ and $y\in\mathbb{Z}$; but
being polynomial identities (for fixed $k$), they become identities for $x\in
K$ and $y\in K$, and Theorem 7.2 is proven.

Theorem 3.3 has a special version as well:

\begin{quote}
\textbf{Theorem 7.3.} Let $\left(  K,\left(  \lambda^{i}\right)
_{i\in\mathbb{N}}\right)  $ be a special $\lambda$-ring. Then, the $\lambda
$-ring $\left(  K\left[  S\right]  ,\left(  \overline{\lambda}^{i}\right)
_{i\in\mathbb{N}}\right)  $ defined in Theorem 3.3 is special.
\end{quote}

\textit{Proof of Theorem 7.3.} As in the proof of Theorem 3.3, we can define a
map $\overline{\lambda}_{T}:K\left[  S\right]  \rightarrow\left(  K\left[
S\right]  \right)  \left[  \left[  T\right]  \right]  $ by $\overline{\lambda
}_{T}\left(  u\right)  =\sum\limits_{i\in\mathbb{N}}\overline{\lambda}%
^{i}\left(  u\right)  T^{j}$ for every $u\in K\left[  S\right]  $. Noting that
$\overline{\lambda}_{T}\left(  u\right)  \in\Lambda\left(  K\left[  S\right]
\right)  $ for every $u\in K\left[  S\right]  $ (since $\left(  K,\left(
\lambda^{i}\right)  _{i\in\mathbb{N}}\right)  $ is a $\lambda$-ring), we see
that we can actually consider $\overline{\lambda}_{T}$ as a map $K\left[
S\right]  \rightarrow\Lambda\left(  K\left[  S\right]  \right)  $.

Theorem 5.6 yields that the map $\overline{\lambda}_{T}$ is an additive group
homomorphism. In order to show that the $\lambda$-ring $\left(  K\left[
S\right]  ,\left(  \overline{\lambda}^{i}\right)  _{i\in\mathbb{N}}\right)  $
is special, we must prove that this map $\overline{\lambda}_{T}$ is a
$\lambda$-ring homomorphism.

Let $E=\left\{  aS^{\alpha}\mid a\in K,\ \alpha\in\mathbb{N}\right\}  $.
Obviously, $E$ is a generating set of the $\mathbb{Z}$-module $K\left[
S\right]  $. Notice that $\overline{\lambda}_{T}\left(  aS^{\alpha}\right)
=\lambda_{S^{\alpha}T}\left(  a\right)  $ for every $a\in K$ and $\alpha
\in\mathbb{N}$ (as shown in the proof of Theorem 3.3 \textbf{(b)}).

For every $a\in K,$ $\alpha\in\mathbb{N}$, $b\in K$ and $\beta\in\mathbb{N}$,
we have%
\begin{align*}
&  \overline{\lambda}_{T}\left(  aS^{\alpha}\right)  \widehat{\cdot}%
\overline{\lambda}_{T}\left(  bS^{\beta}\right)  =\lambda_{S^{\alpha}T}\left(
a\right)  \widehat{\cdot}\lambda_{S^{\beta}T}\left(  b\right)
=\operatorname*{ev}\nolimits_{S^{\alpha}T}\left(  \lambda_{T}\left(  a\right)
\right)  \widehat{\cdot}\operatorname*{ev}\nolimits_{S^{\beta}T}\left(
\lambda_{T}\left(  b\right)  \right)  =\operatorname*{ev}\nolimits_{S^{\alpha
}S^{\beta}T}\left(  \lambda_{T}\left(  a\right)  \widehat{\cdot}\lambda
_{T}\left(  b\right)  \right) \\
&  \ \ \ \ \ \ \ \ \ \ \left(  \text{by Theorem 5.7 \textbf{(d)}, applied to
}K\left[  S\right]  ,\text{ }\lambda_{T}\left(  a\right)  ,\text{ }\lambda
_{T}\left(  b\right)  ,\text{ }S^{\alpha}\text{ and }S^{\beta}\text{ instead
of }K,\text{ }u,\text{ }v,\text{ }\mu\text{ and }\nu\right) \\
&  =\operatorname*{ev}\nolimits_{S^{\alpha}S^{\beta}T}\left(  \lambda
_{T}\left(  ab\right)  \right)  \ \ \ \ \ \ \ \ \ \ \left(  \text{since
}\lambda_{T}\text{ is a ring homomorphism, as }\left(  K,\left(  \lambda
^{i}\right)  _{i\in\mathbb{N}}\right)  \text{ is a special }\lambda
\text{-ring}\right) \\
&  =\lambda_{S^{\alpha}S^{\beta}T}\left(  ab\right)  =\lambda_{S^{\alpha
+\beta}T}\left(  ab\right)
\end{align*}
and $\overline{\lambda}_{T}\left(  aS^{\alpha}\cdot bS^{\beta}\right)
=\overline{\lambda}_{T}\left(  ab\cdot S^{\alpha+\beta}\right)  =\lambda
_{S^{\alpha+\beta}T}\left(  ab\right)  ,$ so that $\overline{\lambda}%
_{T}\left(  aS^{\alpha}\right)  \widehat{\cdot}\overline{\lambda}_{T}\left(
bS^{\beta}\right)  =\overline{\lambda}_{T}\left(  aS^{\alpha}\cdot bS^{\beta
}\right)  $. In other words, $\overline{\lambda}_{T}\left(  e\right)
\widehat{\cdot}\overline{\lambda}_{T}\left(  f\right)  =\overline{\lambda}%
_{T}\left(  ef\right)  $ for any two elements $e$ and $f$ of $E$. Since $E$ is
a generating set of the $\mathbb{Z}$-module $K\left[  S\right]  $, and since
$\overline{\lambda}_{T}$ is already knows to be an additive group
homomorphism, it thus follows that $\overline{\lambda}_{T}\left(  x\right)
\widehat{\cdot}\overline{\lambda}_{T}\left(  y\right)  =\overline{\lambda}%
_{T}\left(  xy\right)  $ for any two elements $x$ and $y$ of $K\left[
S\right]  $. Since $\overline{\lambda}_{T}$ also maps the multiplicative unity
$1$ of $K\left[  S\right]  $ to the multiplicative unity $1+T$ of
$\Lambda\left(  K\left[  S\right]  \right)  $ (this follows from
$\overline{\lambda}_{T}\left(  aS^{\alpha}\right)  =\lambda_{S^{\alpha}%
T}\left(  a\right)  $, applied to $a=1$ and $\alpha=0$), it thus follows that
$\overline{\lambda}_{T}:K\left[  S\right]  \rightarrow\Lambda\left(  K\left[
S\right]  \right)  $ is a ring homomorphism.

Now, for every $i\in\mathbb{N}$, let us define a map $\widehat{\overline
{\lambda}}^{i}:\Lambda\left(  K\left[  S\right]  \right)  \rightarrow
\Lambda\left(  K\left[  S\right]  \right)  $ in the same way as the map
$\widehat{\lambda}^{i}:\Lambda\left(  K\right)  \rightarrow\Lambda\left(
K\right)  $ was defined in Section 5 (but with $K$ replaced by $\Lambda\left(
K\right)  $). Then, the diagram%
\begin{equation}
\xymatrixcolsep{4pc}\xymatrix{ \Lambda\left(K\right) \ar@{^{(}->}[d] \ar[r]^{\widehat{\lambda}^i} & \Lambda\left(K\right) \ar@{^{(}->}[d] \\ \Lambda\left(K\left[S\right]\right) \ar[r]_{\widehat{\overline{\lambda}}^i} & \Lambda\left(K\left[S\right]\right) }
\label{7.3.commdiag}%
\end{equation}
(where the vertical arrows are induced by the canonical inclusion
$K\rightarrow K\left[  S\right]  $) is commutative (since the maps
$\widehat{\lambda}^{i}:\Lambda\left(  K\right)  \rightarrow\Lambda\left(
K\right)  $ and $\widehat{\overline{\lambda}}^{i}:\Lambda\left(  K\left[
S\right]  \right)  \rightarrow\Lambda\left(  K\left[  S\right]  \right)  $
were defined in the same natural way).

For every $a\in K$ and $\alpha\in\mathbb{N}$, we have%
\begin{align*}
&  \left(  \widehat{\overline{\lambda}}^{i}\circ\overline{\lambda}_{T}\right)
\left(  aS^{\alpha}\right)  =\widehat{\overline{\lambda}}^{i}\left(
\overline{\lambda}_{T}\left(  aS^{\alpha}\right)  \right)  =\widehat
{\overline{\lambda}}^{i}\left(  \operatorname*{ev}\nolimits_{S^{\alpha}%
T}\left(  \lambda_{T}\left(  a\right)  \right)  \right)
\ \ \ \ \ \ \ \ \ \ \left(  \text{since }\overline{\lambda}_{T}\left(
aS^{\alpha}\right)  =\lambda_{S^{\alpha}T}\left(  a\right)
=\operatorname*{ev}\nolimits_{S^{\alpha}T}\left(  \lambda_{T}\left(  a\right)
\right)  \right) \\
&  =\operatorname*{ev}\nolimits_{\left(  S^{\alpha}\right)  ^{i}T}\left(
\widehat{\overline{\lambda}}^{i}\left(  \lambda_{T}\left(  a\right)  \right)
\right) \\
&  \ \ \ \ \ \ \ \ \ \ \left(  \text{by Theorem 5.7 \textbf{(e)}, applied to
}K\left[  S\right]  ,\text{ }\lambda_{T}\left(  a\right)  ,\text{ }S^{\alpha
}\text{ and }i\text{ instead of }K,\text{ }u,\text{ }\mu\text{ and }k\right)
\\
&  =\operatorname*{ev}\nolimits_{\left(  S^{\alpha}\right)  ^{i}T}\left(
\widehat{\lambda}^{i}\left(  \lambda_{T}\left(  a\right)  \right)  \right)
\ \ \ \ \ \ \ \ \ \ \left(  \text{due to the commutative diagramm
(\ref{7.3.commdiag})}\right) \\
&  =\operatorname*{ev}\nolimits_{\left(  S^{\alpha}\right)  ^{i}T}\left(
\lambda_{T}\left(  \lambda^{i}\left(  a\right)  \right)  \right)
\ \ \ \ \ \ \ \ \ \ \left(
\begin{array}
[c]{c}%
\text{since }\widehat{\lambda}^{i}\circ\lambda_{T}=\lambda_{T}\circ\lambda
^{i}\text{, because }\lambda_{T}\text{ is a }\lambda\text{-ring homomorphism,}%
\\
\text{since }\left(  K,\left(  \lambda^{i}\right)  _{i\in\mathbb{N}}\right)
\text{ is a special }\lambda\text{-ring}%
\end{array}
\right) \\
&  =\lambda_{\left(  S^{\alpha}\right)  ^{i}T}\left(  \lambda^{i}\left(
a\right)  \right)  =\lambda_{S^{\alpha i}T}\left(  \lambda^{i}\left(
a\right)  \right)  =\overline{\lambda}_{T}\left(  \underbrace{\lambda
^{i}\left(  a\right)  S^{\alpha i}}_{\substack{=\overline{\lambda}^{i}\left(
aS^{\alpha}\right)  \text{ by}\\\text{Theorem 3.3 \textbf{(b)}}}}\right)
=\overline{\lambda}_{T}\left(  \overline{\lambda}^{i}\left(  aS^{\alpha
}\right)  \right)  =\left(  \overline{\lambda}_{T}\circ\overline{\lambda}%
^{i}\right)  \left(  aS^{\alpha}\right)  .
\end{align*}
In other words, every $e\in E$ satisfies $\left(  \widehat{\overline{\lambda}%
}^{i}\circ\overline{\lambda}_{T}\right)  \left(  e\right)  =\left(
\overline{\lambda}_{T}\circ\overline{\lambda}^{i}\right)  \left(  e\right)  $.

Altogether, we now know that $\overline{\lambda}_{T}:K\left[  S\right]
\rightarrow\Lambda\left(  K\left[  S\right]  \right)  $ is a ring
homomorphism, that $E$ is a generating set of the $\mathbb{Z}$-module
$K\left[  S\right]  $, and that every $e\in E$ satisfies $\left(
\widehat{\overline{\lambda}}^{i}\circ\overline{\lambda}_{T}\right)  \left(
e\right)  =\left(  \overline{\lambda}_{T}\circ\overline{\lambda}^{i}\right)
\left(  e\right)  $. Thus, by Exercise 2.1 \textbf{(b)}, it follows that
$\overline{\lambda}_{T}$ is a $\lambda$-ring homomorphism. This proves Theorem 7.3.

\begin{quotation}
\textit{Exercise 7.1.} Let $M$ be a commutative monoid with neutral element.
Prove that the $\lambda$-ring $\left(  \mathbb{Z}\left[  M\right]  ,\left(
\lambda^{i}\right)  _{i\in\mathbb{N}}\right)  $ defined in Exercise 3.4 is special.

\textit{Exercise 7.2.} Let $M$ be a commutative monoid with neutral element.
Let $\left(  K,\left(  \lambda^{i}\right)  _{i\in\mathbb{N}}\right)  $ be a
special $\lambda$-ring. Prove that the $\lambda$-ring $\left(  K\left[
M\right]  ,\left(  \overline{\lambda}^{i}\right)  _{i\in\mathbb{N}}\right)  $
defined in Exercise 3.5 \textbf{(a)} is special.
\end{quotation}

\begin{center}
\fbox{\textbf{8. The }$\lambda$\textbf{-verification principle}}
\end{center}

In Section 5, we have constructed a family of $\lambda$-rings $\Lambda\left(
K\right)  $ which are halfways nice to work with: If you want to prove an
identity involving the ring structure of $\Lambda\left(  K\right)  $ (the
addition $\widehat{+}$, the corresponding subtraction $\widehat{-}$, the zero
$1$, the multiplication $\widehat{\cdot}$, and the multiplicative unity $1+T$)
and the mappings $\widehat{\lambda}^{i}$, then it is enough to verify it for
elements of $1+K\left[  T\right]  ^{+}$ only (by continuity, according to
Theorem 5.5); and this is usually much easier since we know what $\widehat{+}%
$, $\widehat{\cdot}$ and $\widehat{\lambda}^{i}$ mean for elements of
$1+K\left[  T\right]  ^{+}$ (this is what Theorem 5.3 is for).

As a consequence of this, it is no wonder that often an identity is more
easily proven in $\Lambda\left(  K\right)  $ than in arbitrary $\lambda
$-rings. However, it turns out that if an identity can be proven in
$\Lambda\left(  K\right)  $, then it automatically holds for arbitrary
$\lambda$-rings! This is one of the so-called $\lambda$\textit{-verification
principles}\footnote{We are following [2], pp. 25-27 here, though our Theorem
8.1 is not exactly what [2] calls "verification principle".}. Before we
formulate this principle, let us first formally define what kind of identities
it will hold for:

\begin{quote}
\textbf{Definition.} Let $\operatorname*{Rng}^{\operatorname*{S}\Lambda}$
denote the so-called \textit{category of special }$\lambda$\textit{-rings},
which is defined as the category whose objects are the special $\lambda$-rings
and whose morphisms are $\lambda$-ring homomorphisms between its objects.

Let $\operatorname*{USet}:\operatorname*{Rng}^{\operatorname*{S}\Lambda
}\rightarrow\operatorname*{Set}$ be the functor which maps every special
$\lambda$-ring to its underlying set. Let $n\in\mathbb{N}$. An $n$%
-\textit{operation of special }$\lambda$\textit{-rings} will mean a natural
transformation from the functor $\operatorname*{USet}^{n}$ to
$\operatorname*{USet}$ (where the functor $\operatorname*{USet}^{n}$ is
defined as the functor from $\operatorname*{Rng}^{\operatorname*{S}\Lambda}$
to $\operatorname*{Set}^{n}$ which maps every special $\lambda$-ring $\left(
K,\left(  \lambda^{i}\right)  _{i\in\mathbb{N}}\right)  $ to the family
$\left(  \underbrace{K,K,...,K}_{n\text{ times}}\right)  $ of sets).

In other words, an $n$-operation $m$ of special $\lambda$-rings is a family of
mappings\footnote{Here, "mapping" actually means "mapping" and not "group
homomorphism" of "ring homomorphism".} $m_{\left(  K,\left(  \lambda
^{i}\right)  _{i\in\mathbb{N}}\right)  }:K^{n}\rightarrow K$ for every special
$\lambda$-ring $\left(  K,\left(  \lambda^{i}\right)  _{i\in\mathbb{N}%
}\right)  $ such that the diagram%
\begin{equation}
\xymatrixcolsep{5pc}\xymatrix{ K^n \ar[r]^{f^{\times n}} \ar[d]_{m_{\left(K,\left(\lambda^i\right)_{i\in\mathbb{N}}\right)}} & L^n \ar[d]^{m_{\left(L,\left(\mu^i\right)_{i\in\mathbb{N}}\right)}} \\ K \ar[r]_f & L }
\label{I-oper}%
\end{equation}
commutes for any two special $\lambda$-rings $\left(  K,\left(  \lambda
^{i}\right)  _{i\in\mathbb{N}}\right)  $ and $\left(  L,\left(  \mu
^{i}\right)  _{i\in\mathbb{N}}\right)  $ and any $\lambda$-ring homomorphism
$f:\left(  K,\left(  \lambda^{i}\right)  _{i\in\mathbb{N}}\right)
\rightarrow\left(  L,\left(  \mu^{i}\right)  _{i\in\mathbb{N}}\right)  $.
Here, $f^{\times n}$ means the map from $K^{n}$ to $L^{n}$ which equals $f$ on
each coordinate.
\end{quote}

In practice, what are $n$-operations of special $\lambda$-rings? The answer
is: Pretty much every map $K^{n}\rightarrow K$ which is defined for every
special $\lambda$-ring $\left(  K,\left(  \lambda^{i}\right)  _{i\in
\mathbb{N}}\right)  $ just using addition, subtraction, multiplication, $0$
and $1$ and the maps $\lambda^{i}$ is an $n$-operation. In particular, every
polynomial map (where the polynomial has integer coefficients) is an
$n$-operation, and so are the maps $\lambda^{i}:K\rightarrow K$. To give a
different example, the family of maps $m_{\left(  K,\left(  \lambda
^{i}\right)  _{i\in\mathbb{N}}\right)  }:K^{3}\rightarrow K$ for every special
$\lambda$-ring $\left(  K,\left(  \lambda^{i}\right)  _{i\in\mathbb{N}%
}\right)  $ defined by%
\[
m_{\left(  K,\left(  \lambda^{i}\right)  _{i\in\mathbb{N}}\right)  }\left(
a_{1},a_{2},a_{3}\right)  =\lambda^{5}\left(  \lambda^{2}\left(  a_{1}\right)
-\lambda^{4}\left(  a_{2}\right)  \cdot a_{3}\right)
\]
is a $3$-operation of special $\lambda$-rings.

Now, here is the theorem we came for:

\begin{quote}
\textbf{Theorem 8.1 (}$\lambda$\textbf{-verification principle).} Let $\left(
K,\left(  \lambda^{i}\right)  _{i\in\mathbb{N}}\right)  $ be a special
$\lambda$-ring. Let $n\in\mathbb{N}$. Let $m$ and $m^{\prime}$ be two
$n$-operations of special $\lambda$-rings.

Assume that $m_{\left(  \Lambda\left(  K\right)  ,\left(  \widehat{\lambda
}^{i}\right)  _{i\in\mathbb{N}}\right)  }=m_{\left(  \Lambda\left(  K\right)
,\left(  \widehat{\lambda}^{i}\right)  _{i\in\mathbb{N}}\right)  }^{\prime}$.
Then, $m_{\left(  K,\left(  \lambda^{i}\right)  _{i\in\mathbb{N}}\right)
}=m_{\left(  K,\left(  \lambda^{i}\right)  _{i\in\mathbb{N}}\right)  }%
^{\prime}$.
\end{quote}

The proof of this result turns out to be surprisingly simple. First a trivial lemma:

\begin{quote}
\textbf{Theorem 8.2.} Let $\left(  K,\left(  \lambda^{i}\right)
_{i\in\mathbb{N}}\right)  $ be a $\lambda$-ring. Define a mapping
$\operatorname*{coeff}\nolimits_{1}:\Lambda\left(  K\right)  \rightarrow K$ by
$\operatorname*{coeff}\nolimits_{1}\left(  \sum\limits_{j\in\mathbb{N}}%
a_{j}T^{j}\right)  =a_{1}$ for every $\sum\limits_{j\in\mathbb{N}}a_{j}%
T^{j}\in\Lambda\left(  K\right)  $ (with $a_{j}\in K$ for every $j\in
\mathbb{N}$). (In other words, $\operatorname*{coeff}\nolimits_{1}$ is the
mapping that takes a power series and returns its coefficient before $T^{1}.$)

Then, $\operatorname*{coeff}\nolimits_{1}\circ\lambda_{T}=\operatorname*{id}%
_{K}$.
\end{quote}

\textit{Proof of Theorem 8.2.} This is clear, since $\left(
\operatorname*{coeff}\nolimits_{1}\circ\lambda_{T}\right)  \left(  x\right)
=\operatorname*{coeff}\nolimits_{1}\left(  \lambda_{T}\left(  x\right)
\right)  =\operatorname*{coeff}\nolimits_{1}\left(  \sum\limits_{i\in
\mathbb{N}}\lambda^{i}\left(  x\right)  \right)  =\lambda^{1}\left(  x\right)
=x$ for every $x\in K$. Theorem 8.2 is now proven.

\textit{Proof of Theorem 8.1.} Since $\left(  K,\left(  \lambda^{i}\right)
_{i\in\mathbb{N}}\right)  $ is a special $\lambda$-ring, the map $\lambda
_{T}:K\rightarrow\Lambda\left(  K\right)  $ is a $\lambda$-ring homomorphism.
According to (\ref{I-oper}), we thus have the two commutative diagrams%
\[
\xymatrixrowsep{5pc}\xymatrixcolsep{4pc}\xymatrix{
K^n \ar[r]^{\left(\lambda_T\right)^{\times n}} \ar[d]_{m_{\left(K,\left(\lambda^i\right)_{i\in\mathbb{N}}\right)}} & \left(\Lambda\left(K\right)\right)^n \ar[d]^{m_{\left(\Lambda\left(K\right),\left(\widehat{\lambda}^i\right)_{i\in\mathbb{N}}\right)}} \\
K \ar[r]_{\lambda_T} & \Lambda\left(K\right)
}\ \ \ \ \ \ \ \ \ \ \text{and}%
\ \ \ \ \ \ \ \ \ \ \xymatrixrowsep{5pc}\xymatrixcolsep{4pc}\xymatrix{
K^n \ar[r]^{\left(\lambda_T\right)^{\times n}} \ar[d]_{m^{\prime}_{\left(K,\left(\lambda^i\right)_{i\in\mathbb{N}}\right)}} & \left(\Lambda\left(K\right)\right)^n \ar[d]^{m^{\prime}_{\left(\Lambda\left(K\right),\left(\widehat{\lambda}^i\right)_{i\in\mathbb{N}}\right)}} \\
K \ar[r]_{\lambda_T} & \Lambda\left(K\right)
}.
\]
Hence, $m_{\left(  \Lambda\left(  K\right)  ,\left(  \widehat{\lambda}%
^{i}\right)  _{i\in\mathbb{N}}\right)  }=m_{\left(  \Lambda\left(  K\right)
,\left(  \widehat{\lambda}^{i}\right)  _{i\in\mathbb{N}}\right)  }^{\prime}$
yields%
\[
\lambda_{T}\circ m_{\left(  K,\left(  \lambda^{i}\right)  _{i\in\mathbb{N}%
}\right)  }=m_{\left(  \Lambda\left(  K\right)  ,\left(  \widehat{\lambda}%
^{i}\right)  _{i\in\mathbb{N}}\right)  }\circ\left(  \lambda_{T}\right)
^{\times n}=m_{\left(  \Lambda\left(  K\right)  ,\left(  \widehat{\lambda}%
^{i}\right)  _{i\in\mathbb{N}}\right)  }^{\prime}\circ\left(  \lambda
_{T}\right)  ^{\times n}=\lambda_{T}\circ m_{\left(  K,\left(  \lambda
^{i}\right)  _{i\in\mathbb{N}}\right)  }^{\prime}.
\]
Hence, $m_{\left(  K,\left(  \lambda^{i}\right)  _{i\in\mathbb{N}}\right)
}=m_{\left(  K,\left(  \lambda^{i}\right)  _{i\in\mathbb{N}}\right)  }%
^{\prime}$ because $\lambda_{T}$ is injective (due to Theorem 8.2). Theorem
8.1 is thus proven!

Before we move on to concrete properties of special $\lambda$-rings, let us
merge Theorems 8.1 and 5.5 into one simple principle for proving facts about
$\lambda$-rings -- our Theorem 8.4 below. Before we formulate it, let us
define the notion of $1$\textit{-dimensional} elements of a $\lambda$-ring.

\begin{quote}
\textbf{Definition.} Let $\left(  K,\left(  \lambda^{i}\right)  _{i\in
\mathbb{N}}\right)  $ be a $\lambda$-ring, and let $x\in K$ be an element of
$K$. Then, $x$ is said to be $1$\textit{-dimensional} if and only if
$\lambda^{i}\left(  x\right)  =0$ for every integer $i>1$.

\textbf{Theorem 8.3.}

\textbf{(a)} Let $\left(  K,\left(  \lambda^{i}\right)  _{i\in\mathbb{N}%
}\right)  $ be a $\lambda$-ring. Let $x\in K$ be an element of $K$. The
element $x$ is $1$-dimensional if and only if $\lambda_{T}\left(  x\right)
=1+xT$ (where $\lambda_{T}:K\rightarrow K\left[  \left[  T\right]  \right]  $
is the map defined in Theorem 2.1).

\textbf{(b)} Let $\left(  K,\left(  \lambda^{i}\right)  _{i\in\mathbb{N}%
}\right)  $ be a special $\lambda$-ring. Let $x$ and $y$ be two $1$%
-dimensional elements of $K$. Then, $xy$ is $1$-dimensional as well.

\textbf{(c)} Let $K$ be a ring. Let $e\in K$. Then, the element $1+eT$ of the
$\lambda$-ring $\Lambda\left(  K\right)  $ is $1$-dimensional.
\end{quote}

\textit{Proof of Theorem 8.3.} \textbf{(a)} In fact,%
\[
\lambda_{T}\left(  x\right)  =\sum\limits_{i\in\mathbb{N}}\lambda^{i}\left(
x\right)  T^{i}=\underbrace{\lambda^{0}\left(  x\right)  }_{=1}+\underbrace
{\lambda^{1}\left(  x\right)  }_{=x}T+\sum
\limits_{\substack{i>1\\\text{integer}}}\lambda^{i}\left(  x\right)
T^{i}=1+xT+\sum\limits_{\substack{i>1\\\text{integer}}}\lambda^{i}\left(
x\right)  T^{i}.
\]
Hence, $\lambda_{T}\left(  x\right)  =1+xT$ if and only if $\lambda^{i}\left(
x\right)  =0$ for every integer $i>1$ (which means that $x$ is $1$%
-dimensional). Theorem 8.3 \textbf{(a)} is thus proven.

\textbf{(b)} Since the $\lambda$-ring $\left(  K,\left(  \lambda^{i}\right)
_{i\in\mathbb{N}}\right)  $ is special, the map $\lambda_{T}$, seen as a map
from $K$ to $\Lambda\left(  K\right)  $, is a ring homomorphism, so that
$\lambda_{T}\left(  xy\right)  =\lambda_{T}\left(  x\right)  \widehat{\cdot
}\lambda_{T}\left(  y\right)  $. But Theorem 8.3 \textbf{(a)} yields
$\lambda_{T}\left(  x\right)  =1+xT=\Pi\left(  K,\left[  x\right]  \right)  $.
Similarly, $\lambda_{T}\left(  y\right)  =\Pi\left(  K,\left[  y\right]
\right)  $. Thus,%
\begin{align*}
\lambda_{T}\left(  xy\right)   &  =\lambda_{T}\left(  x\right)  \widehat
{\cdot}\lambda_{T}\left(  y\right)  =\Pi\left(  K,\left[  x\right]  \right)
\widehat{\cdot}\Pi\left(  K,\left[  y\right]  \right)  =\Pi\left(  K,\left[
xy\right]  \right)  \ \ \ \ \ \ \ \ \ \ \left(  \text{after Theorem 5.3
\textbf{(c)}}\right) \\
&  =1+xyT,
\end{align*}
and Theorem 8.3 \textbf{(b)} is proven.

\textbf{(c)} For every integer $i>1$, the element%
\begin{align*}
\widehat{\lambda}^{i}\left(  1+eT\right)   &  =\Pi\left(  K^{\prime
},\underbrace{\left[  \prod_{i\in I}e\mid I\in\mathcal{P}_{i}\left(  \left\{
1\right\}  \right)  \right]  }_{\substack{\text{empty multiset,}\\\text{since
}i>1\text{ yields }\mathcal{P}_{i}\left(  \left\{  1\right\}  \right)
=\varnothing}}\right)  \ \ \ \ \ \ \ \ \ \ \left(  \text{by Theorem 5.3
\textbf{(d)}, since }1+eT=\Pi\left(  K^{\prime},\left[  e\right]  \right)
\right) \\
&  =\Pi\left(  K^{\prime},\text{ empty multiset}\right)  =1
\end{align*}
is the zero of $\Lambda\left(  K\right)  $. Thus, $1+eT$ is $1$-dimensional.
Theorem 8.3 \textbf{(c)} is proven.

Now, we can formulate the desired result:

\begin{quote}
\textbf{Theorem 8.4 (continuous splitting }$\lambda$\textbf{-verification
principle).} Let $n\in\mathbb{N}$. Let $m$ and $m^{\prime}$ be two
$n$-operations of special $\lambda$-rings.

Assume that the following two assumptions hold:

\textit{Continuity assumption:} The maps $m_{\left(  \Lambda\left(  K\right)
,\left(  \widehat{\lambda}^{i}\right)  _{i\in\mathbb{N}}\right)  }:\left(
\Lambda\left(  K\right)  \right)  ^{n}\rightarrow\Lambda\left(  K\right)  $
and $m_{\left(  \Lambda\left(  K\right)  ,\left(  \widehat{\lambda}%
^{i}\right)  _{i\in\mathbb{N}}\right)  }^{\prime}:\left(  \Lambda\left(
K\right)  \right)  ^{n}\rightarrow\Lambda\left(  K\right)  $ are continuous
with respect to the $\left(  T\right)  $-adic topology for every special
$\lambda$-ring $\left(  K,\left(  \lambda^{i}\right)  _{i\in\mathbb{N}%
}\right)  $.

\textit{Split equality assumption:} For every special $\lambda$-ring $\left(
K,\left(  \lambda^{i}\right)  _{i\in\mathbb{N}}\right)  $ and every $\left(
u_{1},u_{2},...,u_{n}\right)  \in K^{n}$ such that $u_{i}$ is the sum of
finitely many $1$-dimensional elements of $K$ for every $i\in\left\{
1,2,...,n\right\}  $, we have $m_{\left(  K,\left(  \lambda^{i}\right)
_{i\in\mathbb{N}}\right)  }\left(  u_{1},u_{2},...,u_{n}\right)  =m_{\left(
K,\left(  \lambda^{i}\right)  _{i\in\mathbb{N}}\right)  }^{\prime}\left(
u_{1},u_{2},...,u_{n}\right)  $.

Then, $m=m^{\prime}$.
\end{quote}

\textit{Proof of Theorem 8.4.} We have to prove that $m=m^{\prime}$. In other
words, we must show that $m_{\left(  K,\left(  \lambda^{i}\right)
_{i\in\mathbb{N}}\right)  }=m_{\left(  K,\left(  \lambda^{i}\right)
_{i\in\mathbb{N}}\right)  }^{\prime}$ for every special $\lambda$-ring
$\left(  K,\left(  \lambda^{i}\right)  _{i\in\mathbb{N}}\right)  $. According
to Theorem 8.1, this will immediately follow once we have shown that
$m_{\left(  \Lambda\left(  K\right)  ,\left(  \widehat{\lambda}^{i}\right)
_{i\in\mathbb{N}}\right)  }=m_{\left(  \Lambda\left(  K\right)  ,\left(
\widehat{\lambda}^{i}\right)  _{i\in\mathbb{N}}\right)  }^{\prime}$ for every
special $\lambda$-ring $\left(  K,\left(  \lambda^{i}\right)  _{i\in
\mathbb{N}}\right)  $. So it remains to prove this.

Consider a special $\lambda$-ring $\left(  K,\left(  \lambda^{i}\right)
_{i\in\mathbb{N}}\right)  $. We must prove that $m_{\left(  \Lambda\left(
K\right)  ,\left(  \widehat{\lambda}^{i}\right)  _{i\in\mathbb{N}}\right)
}=m_{\left(  \Lambda\left(  K\right)  ,\left(  \widehat{\lambda}^{i}\right)
_{i\in\mathbb{N}}\right)  }^{\prime}$.

Consider the $\left(  T\right)  $-adic topology on $\Lambda\left(  K\right)
$. The maps $m_{\left(  \Lambda\left(  K\right)  ,\left(  \widehat{\lambda
}^{i}\right)  _{i\in\mathbb{N}}\right)  }$ and $m_{\left(  \Lambda\left(
K\right)  ,\left(  \widehat{\lambda}^{i}\right)  _{i\in\mathbb{N}}\right)
}^{\prime}$ are continuous, while the subset $1+K\left[  T\right]  ^{+}$ of
$1+K\left[  \left[  T\right]  \right]  ^{+}=\Lambda\left(  K\right)  $ is
dense (by Theorem 5.5 \textbf{(a)}). Hence, in order to prove that $m_{\left(
\Lambda\left(  K\right)  ,\left(  \widehat{\lambda}^{i}\right)  _{i\in
\mathbb{N}}\right)  }=m_{\left(  \Lambda\left(  K\right)  ,\left(
\widehat{\lambda}^{i}\right)  _{i\in\mathbb{N}}\right)  }^{\prime}$, it will
be enough to show that
\begin{equation}
m_{\left(  \Lambda\left(  K\right)  ,\left(  \widehat{\lambda}^{i}\right)
_{i\in\mathbb{N}}\right)  }\left(  u_{1},u_{2},...,u_{n}\right)  =m_{\left(
\Lambda\left(  K\right)  ,\left(  \widehat{\lambda}^{i}\right)  _{i\in
\mathbb{N}}\right)  }^{\prime}\left(  u_{1},u_{2},...,u_{n}\right)
\label{8.4.goal}%
\end{equation}
for every $\left(  u_{1},u_{2},...,u_{n}\right)  \in\left(  1+K\left[
T\right]  ^{+}\right)  ^{n}.$

Fix some $\left(  u_{1},u_{2},...,u_{n}\right)  \in\left(  1+K\left[
T\right]  ^{+}\right)  ^{n}$. For every $i\in\left\{  1,2,...,n\right\}  $,
there exists some $\left(  \widetilde{K}_{u_{i}},\left[  \left(  u_{i}\right)
_{1},\left(  u_{i}\right)  _{2},...,\left(  u_{i}\right)  _{n_{i}}\right]
\right)  \in K^{\operatorname*{int}}$ such that $u_{i}=\Pi\left(
\widetilde{K}_{u_{i}},\left[  \left(  u_{i}\right)  _{1},\left(  u_{i}\right)
_{2},...,\left(  u_{i}\right)  _{n_{i}}\right]  \right)  $. According to
Theorem 5.3 \textbf{(a)} (applied several times), there exists a finite-free
extension ring $K^{\prime}$ of $K$ which contains the $\widetilde{K}_{u_{i}}$
for all $i\in\left\{  1,2,...,n\right\}  $ as subrings. Hence, $u_{i}%
=\Pi\left(  K^{\prime},\left[  \left(  u_{i}\right)  _{1},\left(
u_{i}\right)  _{2},...,\left(  u_{i}\right)  _{n_{i}}\right]  \right)  $ for
every $i\in\left\{  1,2,...,n\right\}  $.

In the ring $\Lambda\left(  K^{\prime}\right)  $ (which is an extension ring
of $\Lambda\left(  K\right)  $, since $K\subseteq K^{\prime}$ and since
$\Lambda$ is a functor), this yields%
\begin{align}
u_{i}  &  =\Pi\left(  K^{\prime},\left[  \left(  u_{i}\right)  _{1},\left(
u_{i}\right)  _{2},...,\left(  u_{i}\right)  _{n_{i}}\right]  \right)
=\prod_{j=1}^{n_{i}}\left(  1+\left(  u_{i}\right)  _{j}T\right)
=\widehat{\sum_{j=1}^{n_{i}}}\left(  1+\left(  u_{i}\right)  _{j}T\right)
\nonumber\\
&  \ \ \ \ \ \ \ \ \ \ \left(  \text{since addition in }\Lambda\left(
K^{\prime}\right)  \text{ is multiplication in }K^{\prime}\left[  \left[
T\right]  \right]  \right)  . \label{8.4.hilf}%
\end{align}
On the other hand, for every $j\in\left\{  1,2,...,n_{i}\right\}  $, the
element $1+\left(  u_{i}\right)  _{j}T$ of $\Lambda\left(  K^{\prime}\right)
$ is $1$-dimensional (by Theorem 8.3 \textbf{(c)}, applied to $e=\left(
u_{i}\right)  _{j}$). Thus, (\ref{8.4.hilf}) shows that $u_{i}$ is a sum of
$1$-dimensional elements of $\Lambda\left(  K^{\prime}\right)  $ for every
$i\in\left\{  1,2,...,n\right\}  $. Hence, applying the split equality
assumption to the special $\lambda$-ring $\left(  \Lambda\left(  K^{\prime
}\right)  ,\left(  \widehat{\lambda}^{i}\right)  _{i\in\mathbb{N}}\right)  $
instead of $\left(  K,\left(  \lambda^{i}\right)  _{i\in\mathbb{N}}\right)  $,
we see that%
\[
m_{\left(  \Lambda\left(  K\right)  ,\left(  \widehat{\lambda}^{i}\right)
_{i\in\mathbb{N}}\right)  }\left(  u_{1},u_{2},...,u_{n}\right)  =m_{\left(
\Lambda\left(  K\right)  ,\left(  \widehat{\lambda}^{i}\right)  _{i\in
\mathbb{N}}\right)  }^{\prime}\left(  u_{1},u_{2},...,u_{n}\right)  .
\]
This is an equality in the ring $\Lambda\left(  K^{\prime}\right)  $, but
since $\Lambda\left(  K\right)  $ can be canonically seen as a subring of
$\Lambda\left(  K^{\prime}\right)  $ (because $K$ is a subring of $K^{\prime}%
$), this yields (\ref{8.4.goal}). This proves Theorem 8.4.

Roughly speaking, Theorem 8.4 says that whether some identity holds on every
special $\lambda$-rings or not can be checked just by looking at the sums of
$1$-dimensional elements. This is why it is worthwhile to study such sums. Let
us record a property of these:

\begin{quote}
\textbf{Theorem 8.5.} Let $\left(  K,\left(  \lambda^{i}\right)
_{i\in\mathbb{N}}\right)  $ be a $\lambda$-ring. Let $u_{1},$ $u_{2},$ $...,$
$u_{m}$ be $1$-dimensional elements of $K$. Let $i\in\mathbb{N}$. Then,%
\[
\lambda^{i}\left(  u_{1}+u_{2}+...+u_{m}\right)  =\sum_{\substack{S\subseteq
\left\{  1,2,...,m\right\}  ;\\\left\vert S\right\vert =i}}\prod_{k\in S}%
u_{k}.
\]



\end{quote}

\textit{Proof of Theorem 8.5.} We have%
\begin{align*}
&  \sum_{i\in\mathbb{N}}\lambda^{i}\left(  u_{1}+u_{2}+...+u_{m}\right)
T^{i}=\lambda_{T}\left(  u_{1}+u_{2}+...+u_{m}\right) \\
&  =\prod_{j=1}^{m}\lambda_{T}\left(  u_{j}\right)
\ \ \ \ \ \ \ \ \ \ \left(  \text{by Theorem 2.1 \textbf{(a)}, applied several
times}\right) \\
&  =\prod_{j=1}^{m}\left(  1+u_{j}T\right)  \ \ \ \ \ \ \ \ \ \ \left(
\text{since the element }u_{j}\text{ is }1\text{-dimensional and thus
satisfies }\lambda_{T}\left(  u_{j}\right)  =1+u_{j}T\right) \\
&  =\sum_{i\in\mathbb{N}}\sum\limits_{\substack{S\subseteq\left\{
1,2,...,m\right\}  ;\\\left\vert S\right\vert =i}}\prod\limits_{k\in S}%
u_{k}\cdot T^{i}.
\end{align*}
Comparing coefficients yields the assertion of Theorem 8.5.

\begin{quotation}
\textit{Exercise 8.1.} Let $\left(  K,\left(  \lambda^{i}\right)
_{i\in\mathbb{N}}\right)  $ be a special $\lambda$-ring. Define the mapping
$\operatorname*{coeff}\nolimits_{1}:\Lambda\left(  K\right)  \rightarrow K$ as
in Theorem 8.2. Then, show that $\operatorname*{coeff}\nolimits_{1}%
:\Lambda\left(  K\right)  \rightarrow K$ is a ring homomorphism\footnote{but,
generally, not a $\lambda$-ring homomorphism!}.

\textit{Exercise 8.2.} Let $\left(  K,\left(  \lambda^{i}\right)
_{i\in\mathbb{N}}\right)  $ be a special $\lambda$-ring. If $x\in K$ is an
invertible $1$-dimensional element of $K$, then prove that $x^{-1}$ is
$1$-dimensional as well.

\textit{Exercise 8.3.} Let $\left(  K,\left(  \lambda^{i}\right)
_{i\in\mathbb{N}}\right)  $ be a $\lambda$-ring. Let $E$ be a generating set
of the $\mathbb{Z}$-module $K$ such that every element $e\in E$ is $1$-dimensional.

Prove that the $\lambda$-ring $\left(  K,\left(  \lambda^{i}\right)
_{i\in\mathbb{N}}\right)  $ is special.
\end{quotation}

\begin{center}
\fbox{\textbf{9. Adams operations}}
\end{center}

We are now ready to define Adams operations of special $\lambda$-rings. There
are two different ways to do this; we will take one of these as the definition
and the other one as a theorem.

Remember how we defined the "universal" symmetric polynomials $P_{k}$ and
$P_{k,j}$ in Section 4? Prepare for some more:

\begin{quote}
\textbf{Definition.} Let $j\in\mathbb{N}\setminus\left\{  0\right\}  $. Our
goal is to define a polynomial $N_{j}\in\mathbb{Z}\left[  \alpha_{1}%
,\alpha_{2},...,\alpha_{j}\right]  $ such that%
\begin{equation}
\sum_{i=1}^{m}U_{i}^{j}=N_{j}\left(  X_{1},X_{2},...,X_{j}\right)  \label{Nj1}%
\end{equation}
in the polynomial ring $\mathbb{Z}\left[  U_{1},U_{2},...,U_{m}\right]  $ for
every $m\in\mathbb{N}$, where $X_{i}=\sum\limits_{\substack{S\subseteq\left\{
1,2,...,m\right\}  ;\\\left\vert S\right\vert =i}}\prod\limits_{k\in S}U_{k}$
is the $i$-th elementary symmetric polynomial in the variables $U_{1},$
$U_{2},$ $...,$ $U_{m}$ for every $i\in\mathbb{N}$.

In order to do this, we first fix some $m\in\mathbb{N}$. The polynomial
$\sum\limits_{i=1}^{m}U_{i}^{j}\in\mathbb{Z}\left[  U_{1},U_{2},...,U_{m}%
\right]  $ is symmetric. Thus, Theorem 4.1 \textbf{(a)} yields that there
exists one and only one polynomial $Q\in\mathbb{Z}\left[  \alpha_{1}%
,\alpha_{2},...,\alpha_{m}\right]  $ such that $\sum\limits_{i=1}^{m}U_{i}%
^{j}=Q\left(  X_{1},X_{2},...,X_{m}\right)  .$Since the polynomial
$\sum\limits_{i=1}^{m}U_{i}^{j}$ has total degree $\leq j$ in the variables
$U_{1},$ $U_{2},$ $...,$ $U_{m}$, Theorem 4.1 \textbf{(b)} yields that%
\[
\sum_{i=1}^{m}U_{i}^{j}=Q_{j}\left(  X_{1},X_{2},...,X_{j}\right)  ,
\]
where $Q_{j}$ is the image of the polynomial $Q$ under the canonical
homomorphism $\mathbb{Z}\left[  \alpha_{1},\alpha_{2},...,\alpha_{m}\right]
\rightarrow\mathbb{Z}\left[  \alpha_{1},\alpha_{2},...,\alpha_{j}\right]  $.
However, this polynomial $Q_{j}$ is not independent of $m$ yet (as the
polynomial $N_{j}$ that we intend to construct should be), so we call it
$Q_{j,\left[  m\right]  }$ rather than just $Q_{j}$.

Now we forget that we fixed $m\in\mathbb{N}$. We have learnt that%
\[
\sum_{i=1}^{m}U_{i}^{j}=Q_{j,\left[  m\right]  }\left(  X_{1},X_{2}%
,...,X_{j}\right)  ,
\]
in the polynomial ring $\mathbb{Z}\left[  U_{1},U_{2},...,U_{m}\right]  $ for
every $m\in\mathbb{N}$. Now, define a polynomial $N_{j}\in\mathbb{Z}\left[
\alpha_{1},\alpha_{2},...,\alpha_{kj}\right]  $ by $N_{j}=Q_{j,\left[
j\right]  }.$

This polynomial $N_{j}$ is called the $j$\textit{-th Hirzebruch-Newton
polynomial}.\footnote{The "Newton" in the name of this polynomial $N_{j}$
probably refers to the fact that the explicit form of $N_{j}$ can be easily
computed (recursively) from the so-called Newton identities (which relate the
power sums and the elementary symmetric polynomials).}

\textbf{Theorem 9.1.} \textbf{(a)} The polynomial $N_{j}$ just defined
satisfies the equation (\ref{Nj1}) in the polynomial ring $\mathbb{Z}\left[
U_{1},U_{2},...,U_{m}\right]  $ for every $m\in\mathbb{N}$. (Hence, the goal
mentioned above in the definition is actually achieved.)

\textbf{(b)} For every $m\in\mathbb{N}$, we have%
\begin{equation}
T\sum_{i=1}^{m}\dfrac{U_{i}}{1-U_{i}T}=\sum_{j\in\mathbb{N}\setminus\left\{
0\right\}  }N_{j}\left(  X_{1},X_{2},...,X_{j}\right)  T^{j} \label{Nj2}%
\end{equation}
in the ring $\left(  \mathbb{Z}\left[  U_{1},U_{2},...,U_{m}\right]  \right)
\left[  \left[  T\right]  \right]  $.
\end{quote}

\textit{Proof of Theorem 9.1.} \textbf{(a)} This proof is going to be very
similar to that of Theorem 4.4 \textbf{(a)}.

\textit{1st Step:} Fix $m\in\mathbb{N}$ such that $m\geq j$. Then, we claim
that $Q_{j,\left[  m\right]  }=N_{j}$.

\textit{Proof.} By the definition of $Q_{j,\left[  m\right]  }$, we have%
\[
\sum_{i=1}^{m}U_{i}^{j}=Q_{j,\left[  m\right]  }\left(  X_{1},X_{2}%
,...,X_{j}\right)
\]
in the polynomial ring $\mathbb{Z}\left[  U_{1},U_{2},...,U_{m}\right]  $.
Applying the canonical ring epimorphism $\mathbb{Z}\left[  U_{1}%
,U_{2},...,U_{m}\right]  \rightarrow\mathbb{Z}\left[  U_{1},U_{2}%
,...,U_{j}\right]  $ (which maps every $U_{i}$ to $\left\{
\begin{array}
[c]{c}%
U_{i},\text{ if }i\leq j;\\
0,\text{ if }i>j
\end{array}
\right.  $) to this equation (and noticing that this epimorphism maps every
$X_{i}$ with $i\geq1$ to the corresponding $X_{i}$ of the image ring), we
obtain%
\[
\sum_{i=1}^{j}U_{i}^{j}=Q_{j,\left[  m\right]  }\left(  X_{1},X_{2}%
,...,X_{j}\right)
\]
in the polynomial ring $\mathbb{Z}\left[  U_{1},U_{2},...,U_{j}\right]  $. On
the other hand, the definition of $Q_{j,\left[  j\right]  }$ yields%
\[
\sum_{i=1}^{j}U_{i}^{j}=Q_{j,\left[  j\right]  }\left(  X_{1},X_{2}%
,...,X_{j}\right)
\]
in the same ring. These two equations yield $Q_{j,\left[  m\right]  }\left(
X_{1},X_{2},...,X_{j}\right)  =Q_{j,\left[  j\right]  }\left(  X_{1}%
,X_{2},...,X_{j}\right)  $. Since the elements $X_{1},$ $X_{2},$ $...,$
$X_{j}$ of $\mathbb{Z}\left[  U_{1},U_{2},...,U_{j}\right]  $ are
algebraically independent (by Theorem 4.1 \textbf{(a)}), this yields
$Q_{j,\left[  m\right]  }=Q_{j,\left[  j\right]  }.$ In other words,
$Q_{j,\left[  m\right]  }=N_{j},$ and the 1st Step is proven.

\textit{2nd Step:} For every $m\in\mathbb{N}$, the equation (\ref{Nj1}) is
satisfied in the polynomial ring $\mathbb{Z}\left[  U_{1},U_{2},...,U_{m}%
\right]  $.

\textit{Proof.} Let $m^{\prime}\in\mathbb{N}$ be such that $m^{\prime}\geq m$
and $m^{\prime}\geq j$. Then, the 1st Step (applied to $m^{\prime}$ instead of
$m$) yields that $Q_{j,\left[  m^{\prime}\right]  }=N_{j}.$

The definition of $Q_{j,\left[  m^{\prime}\right]  }$ yields
\[
\sum_{i=1}^{m^{\prime}}U_{i}^{j}=Q_{j,\left[  m^{\prime}\right]  }\left(
X_{1},X_{2},...,X_{j}\right)
\]
in the polynomial ring $\mathbb{Z}\left[  U_{1},U_{2},...,U_{m^{\prime}%
}\right]  $. Applying the canonical ring epimorphism $\mathbb{Z}\left[
U_{1},U_{2},...,U_{m^{\prime}}\right]  \rightarrow\mathbb{Z}\left[
U_{1},U_{2},...,U_{m}\right]  $ (which maps every $U_{i}$ to $\left\{
\begin{array}
[c]{c}%
U_{i},\text{ if }i\leq m;\\
0,\text{ if }i>m
\end{array}
\right.  $) to this equation (and noticing that this epimorphism maps every
$X_{i}$ with $i\geq1$ to the corresponding $X_{i}$ of the image ring), we
obtain%
\[
\sum_{i=1}^{m}U_{i}^{j}=Q_{j,\left[  m^{\prime}\right]  }\left(  X_{1}%
,X_{2},...,X_{j}\right)
\]
in the polynomial ring $\mathbb{Z}\left[  U_{1},U_{2},...,U_{m}\right]  .$
This means that the equation (\ref{Nj1}) is satisfied in the polynomial ring
$\mathbb{Z}\left[  U_{1},U_{2},...,U_{m}\right]  $ (since $Q_{j,\left[
m^{\prime}\right]  }=N_{j}$). This completes the 2nd Step and proves Theorem
9.1 \textbf{(a)}.

\textbf{(b)} We have%
\begin{align*}
T\sum_{i=1}^{m}\dfrac{U_{i}}{1-U_{i}T}  &  =\sum_{i=1}^{m}U_{i}T\underbrace
{\left(  1-U_{i}T\right)  ^{-1}}_{=\sum\limits_{j\in\mathbb{N}}\left(
U_{i}T\right)  ^{j}}=\sum_{i=1}^{m}\sum_{j\in\mathbb{N}}\left(  U_{i}T\right)
^{j+1}=\sum_{i=1}^{m}\sum_{j\in\mathbb{N}\setminus\left\{  0\right\}  }\left(
U_{i}T\right)  ^{j}\\
&  =\sum_{j\in\mathbb{N}\setminus\left\{  0\right\}  }\sum_{i=1}^{m}\left(
U_{i}T\right)  ^{j}=\sum_{j\in\mathbb{N}\setminus\left\{  0\right\}
}\underbrace{\sum_{i=1}^{m}U_{i}^{j}}_{\substack{=N_{j}\left(  X_{1}%
,X_{2},...,X_{j}\right)  \\\text{by (\ref{Nj1})}}}T^{j}=\sum_{j\in
\mathbb{N}\setminus\left\{  0\right\}  }N_{j}\left(  X_{1},X_{2}%
,...,X_{j}\right)  T^{j},
\end{align*}
and Theorem 9.1 \textbf{(b)} is proven.

\textit{Remark:} There is a subtle point here: We have defined, for every
$j\in\mathbb{N}\setminus\left\{  0\right\}  ,$ a polynomial $N_{j}%
\in\mathbb{Z}\left[  \alpha_{1},\alpha_{2},...,\alpha_{j}\right]  $ which
satisfies (\ref{Nj1}) in the polynomial ring $\mathbb{Z}\left[  U_{1}%
,U_{2},...,U_{m}\right]  $ for every $m\in\mathbb{N}$. We \textit{cannot}
define such a polynomial $N_{j}$ for $j=0$. In fact, if we would try to do
this as we did above, then the proof of Theorem 9.1 would fail (in fact, the
canonical ring epimorphism $\mathbb{Z}\left[  U_{1},U_{2},...,U_{m}\right]
\rightarrow\mathbb{Z}\left[  U_{1},U_{2},...,U_{j}\right]  $ would
\textit{not} send $\sum\limits_{i=1}^{m}U_{i}^{j}$ to $\sum\limits_{i=1}%
^{j}U_{i}^{j}$ anymore, because $0^{j}$ is not $0$ for $j=0$). This is why
$N_{j}$ is well-defined only for $j\in\mathbb{N}\setminus\left\{  0\right\}  $
and not for all $j\in\mathbb{N}$.

Now, let us define Adams operations:

\begin{quote}
\textbf{Definition.} Let $\left(  K,\left(  \lambda^{i}\right)  _{i\in
\mathbb{N}}\right)  $ be a special $\lambda$-ring. For every $j\in
\mathbb{N}\setminus\left\{  0\right\}  $, we define a map $\psi^{j}%
:K\rightarrow K$ by%
\begin{equation}
\psi^{j}\left(  x\right)  =N_{j}\left(  \lambda^{1}\left(  x\right)
,\lambda^{2}\left(  x\right)  ,...,\lambda^{j}\left(  x\right)  \right)
\ \ \ \ \ \ \ \ \ \ \text{for every }x\in K. \label{PsiDef}%
\end{equation}
We call $\psi^{j}$ the $j$\textit{-th Adams operation} (or the $j$\textit{-th
Adams character}) of the $\lambda$-ring $\left(  K,\left(  \lambda^{i}\right)
_{i\in\mathbb{N}}\right)  $.
\end{quote}

Before we prove a batch of properties of these Adams characters, let us show
another approach to these Adams characters:

\begin{quote}
\textbf{Theorem 9.2.} Let $\left(  K,\left(  \lambda^{i}\right)
_{i\in\mathbb{N}}\right)  $ be a special $\lambda$-ring.

Define a map $\widetilde{\psi}_{T}:K\rightarrow K\left[  \left[  T\right]
\right]  $ by $\widetilde{\psi}_{T}\left(  x\right)  =\sum\limits_{j\in
\mathbb{N}\setminus\left\{  0\right\}  }\psi^{j}\left(  x\right)  T^{j}$ for
every $x\in K$.\ \ \ \ \footnote{Note that we call this map $\widetilde{\psi
}_{T}$ to distinguish it from the map $\psi_{T}$ in [1] (which is more or less
the same but differs slightly).}

Let $x\in K$.

\textbf{(a)} We have%
\[
\psi^{j}\left(  x\right)  =\left(  -1\right)  ^{j+1}\sum_{i=0}^{j}i\lambda
^{i}\left(  x\right)  \lambda^{j-i}\left(  -x\right)
\ \ \ \ \ \ \ \ \ \ \text{for every }j\in\mathbb{N}\setminus\left\{
0\right\}  \text{.}%
\]


\textbf{(b)} We have $\widetilde{\psi}_{T}\left(  x\right)  =-T\cdot\dfrac
{d}{dT}\log\lambda_{-T}\left(  x\right)  $. Here, for every power series
$u\in1+K\left[  \left[  T\right]  \right]  ^{+}$, the \textit{logarithmic
derivative} $\dfrac{d}{dT}\log u$ of $u$ is defined by $\dfrac{d}{dT}\log
u=\dfrac{\dfrac{d}{dT}u}{u}$ (this definition works even in the cases where
the logarithm doesn't exist, such as rings of positive characteristic), and
$\lambda_{-T}\left(  x\right)  $ denotes $\operatorname*{ev}_{-T}\left(
\lambda_{T}\left(  x\right)  \right)  $.
\end{quote}

\textit{Proof of Theorem 9.2.} \textit{1st step:} For any fixed special
$\lambda$-ring $\left(  K,\left(  \lambda^{i}\right)  _{i\in\mathbb{N}%
}\right)  $ and any fixed $x\in K$, the assertions \textbf{(a)} and
\textbf{(b)} are equivalent.

\textit{Proof.} In $K\left[  \left[  T\right]  \right]  $, we have%
\[
\lambda_{-T}\left(  x\right)  =\sum_{i\in\mathbb{N}}\lambda^{i}\left(
x\right)  \left(  -T\right)  ^{i}=\sum_{i\in\mathbb{N}}\left(  -1\right)
^{i}\lambda^{i}\left(  x\right)  T^{i}.
\]
This yields%
\begin{align*}
\left(  \lambda_{-T}\left(  x\right)  \right)  ^{-1}  &  =\lambda_{-T}\left(
-x\right)  \ \ \ \ \ \ \ \ \ \ \left(  \text{since }\left(  \lambda_{T}\left(
x\right)  \right)  ^{-1}=\lambda_{T}\left(  -x\right)  \text{ by Theorem 2.1
\textbf{(b)}}\right) \\
&  =\sum_{i\in\mathbb{N}}\left(  -1\right)  ^{i}\lambda^{i}\left(  -x\right)
T^{i}%
\end{align*}
and%
\[
\dfrac{d}{dT}\lambda_{-T}\left(  x\right)  =\dfrac{d}{dT}\sum_{i\in\mathbb{N}%
}\left(  -1\right)  ^{i}\lambda^{i}\left(  x\right)  T^{i}=\sum_{i\in
\mathbb{N}}\left(  -1\right)  ^{i}\lambda^{i}\left(  x\right)  iT^{i-1}.
\]
Thus,%
\begin{align*}
-T\cdot\dfrac{d}{dT}\log\lambda_{-T}\left(  x\right)   &  =-T\cdot
\dfrac{\dfrac{d}{dT}\lambda_{-T}\left(  x\right)  }{\lambda_{-T}\left(
x\right)  }=-T\cdot\underbrace{\dfrac{d}{dT}\lambda_{-T}\left(  x\right)
}_{=\sum\limits_{i\in\mathbb{N}}\left(  -1\right)  ^{i}\lambda^{i}\left(
x\right)  iT^{i-1}}\cdot\underbrace{\left(  \lambda_{-T}\left(  x\right)
\right)  ^{-1}}_{=\sum\limits_{i\in\mathbb{N}}\left(  -1\right)  ^{i}%
\lambda^{i}\left(  -x\right)  T^{i}}\\
&  =-T\cdot\sum\limits_{i\in\mathbb{N}}\left(  -1\right)  ^{i}\lambda
^{i}\left(  x\right)  iT^{i-1}\cdot\sum\limits_{i\in\mathbb{N}}\left(
-1\right)  ^{i}\lambda^{i}\left(  -x\right)  T^{i}\\
&  =\sum\limits_{i\in\mathbb{N}}\left(  -1\right)  ^{i+1}\lambda^{i}\left(
x\right)  iT^{i}\cdot\sum\limits_{i\in\mathbb{N}}\left(  -1\right)
^{i}\lambda^{i}\left(  -x\right)  T^{i}\\
&  =\sum_{j\in\mathbb{N}}\sum_{i=0}^{j}\left(  -1\right)  ^{i+1}\lambda
^{i}\left(  x\right)  i\cdot\left(  -1\right)  ^{j-i}\lambda^{j-i}\left(
-x\right)  T^{j}\\
&  =\sum_{j\in\mathbb{N}}\left(  -1\right)  ^{j+1}\sum_{i=0}^{j}i\lambda
^{i}\left(  x\right)  \lambda^{j-i}\left(  -x\right)  \cdot T^{j}\\
&  =\sum_{j\in\mathbb{N}\setminus\left\{  0\right\}  }\left(  -1\right)
^{j+1}\sum_{i=0}^{j}i\lambda^{i}\left(  x\right)  \lambda^{j-i}\left(
-x\right)  \cdot T^{j}+\underbrace{\left(  -1\right)  ^{0+1}\sum_{i=0}%
^{0}i\lambda^{i}\left(  x\right)  \lambda^{0-i}\left(  -x\right)  \cdot T^{0}%
}_{=0}\\
&  =\sum_{j\in\mathbb{N}\setminus\left\{  0\right\}  }\left(  -1\right)
^{j+1}\sum_{i=0}^{j}i\lambda^{i}\left(  x\right)  \lambda^{j-i}\left(
-x\right)  \cdot T^{j}.
\end{align*}
On the other hand, $\widetilde{\psi}_{T}\left(  x\right)  =\sum\limits_{j\in
\mathbb{N}\setminus\left\{  0\right\}  }\psi^{j}\left(  x\right)  T^{j}$.
Hence, $\widetilde{\psi}_{T}\left(  x\right)  =-T\cdot\dfrac{d}{dT}\log
\lambda_{-T}\left(  x\right)  $ holds if and only if%
\[
\psi^{j}\left(  x\right)  =\left(  -1\right)  ^{j+1}\sum_{i=0}^{j}i\lambda
^{i}\left(  x\right)  \lambda^{j-i}\left(  -x\right)
\ \ \ \ \ \ \ \ \ \ \text{for every }j\in\mathbb{N}\setminus\left\{
0\right\}  \text{.}%
\]
This proves that the assertions \textbf{(a)} and \textbf{(b)} are equivalent,
and thus the 1st step is complete.

\textit{2nd step:} We will now show that the assertion \textbf{(b)} holds for
every special $\lambda$-ring $\left(  K,\left(  \lambda^{i}\right)
_{i\in\mathbb{N}}\right)  $ and every $x\in K$ such that $x$ is the sum of
finitely many $1$-dimensional elements of $K$.

\textit{Proof.} Let $\left(  K,\left(  \lambda^{i}\right)  _{i\in\mathbb{N}%
}\right)  $ be a special $\lambda$-ring, and let $x\in K$ be a sum of finitely
many $1$-dimensional elements of $K$. In other words, $x=u_{1}+u_{2}%
+...+u_{m}$ for some $1$-dimensional elements $u_{1},$ $u_{2},$ $...,$ $u_{m}$
of $K$.

Then,%
\[
\lambda_{-T}\left(  x\right)  =\lambda_{-T}\left(  u_{1}+u_{2}+...+u_{m}%
\right)  =\prod_{j=1}^{m}\left(  1-u_{j}T\right)
\]
(since $\lambda_{T}\left(  u_{1}+u_{2}+...+u_{m}\right)  =\prod\limits_{j=1}%
^{m}\left(  1+u_{j}T\right)  $, as shown in the proof of Theorem 8.5), so
that, by the Leibniz formula,%
\begin{align*}
\dfrac{d}{dT}\lambda_{-T}\left(  x\right)   &  =\sum_{k=1}^{m}\left(
\underbrace{\dfrac{d}{dT}\left(  1-u_{k}T\right)  }_{=-u_{k}}\right)
\cdot\underbrace{\prod_{j\in\left\{  1,2,...,m\right\}  \setminus\left\{
k\right\}  }\left(  1-u_{j}T\right)  }_{=\left(  1-u_{k}T\right)  ^{-1}%
\cdot\prod\limits_{j\in\left\{  1,2,...,m\right\}  }\left(  1-u_{j}T\right)
}\\
&  =-\sum_{k=1}^{m}\dfrac{u_{k}}{1-u_{k}T}\cdot\underbrace{\prod
\limits_{j\in\left\{  1,2,...,m\right\}  }\left(  1-u_{j}T\right)  }%
_{=\lambda_{-T}\left(  x\right)  }=-\sum_{k=1}^{m}\dfrac{u_{k}}{1-u_{k}T}%
\cdot\lambda_{-T}\left(  x\right)  .
\end{align*}
Hence,%
\begin{align}
-T\cdot\dfrac{d}{dT}\log\lambda_{-T}\left(  x\right)   &  =-T\cdot
\dfrac{\dfrac{d}{dT}\lambda_{-T}\left(  x\right)  }{\lambda_{-T}\left(
x\right)  }=-T\cdot\dfrac{-\sum\limits_{k=1}^{m}\dfrac{u_{k}}{1-u_{k}T}%
\cdot\lambda_{-T}\left(  x\right)  }{\lambda_{-T}\left(  x\right)
}\nonumber\\
&  =T\cdot\sum\limits_{k=1}^{m}\dfrac{u_{k}}{1-u_{k}T}=T\sum\limits_{i=1}%
^{m}\dfrac{u_{i}}{1-u_{i}T} \label{9.2.2step}%
\end{align}
On the other hand, Theorem 8.5 yields%
\[
\lambda^{i}\left(  x\right)  =\lambda^{i}\left(  u_{1}+u_{2}+...+u_{m}\right)
=\sum\limits_{\substack{S\subseteq\left\{  1,2,...,m\right\}  ;\\\left\vert
S\right\vert =i}}\prod\limits_{k\in S}u_{k}\ \ \ \ \ \ \ \ \ \ \text{for every
}i\in\mathbb{N}.
\]


Consider the polynomial ring $\mathbb{Z}\left[  U_{1},U_{2},...,U_{m}\right]
$. For every $i\in\mathbb{N}$, let $X_{i}=\sum\limits_{\substack{S\subseteq
\left\{  1,2,...,m\right\}  ;\\\left\vert S\right\vert =i}}\prod\limits_{k\in
S}U_{k}$ be the $i$-th elementary symmetric polynomial in the variables
$U_{1},$ $U_{2},$ $...,$ $U_{m}$. There exists a ring homomorphism
$\mathbb{Z}\left[  U_{1},U_{2},...,U_{m}\right]  \rightarrow K$ which maps
$U_{i}$ to $u_{i}$ for every $i$. This homomorphism therefore maps $X_{i}$ to
$\sum\limits_{\substack{S\subseteq\left\{  1,2,...,m\right\}  ;\\\left\vert
S\right\vert =i}}\prod\limits_{k\in S}u_{k}=\lambda^{i}\left(  x\right)  $ for
every $i\in\mathbb{N}$. Hence, applying this homomorphism to (\ref{Nj2}), we
obtain%
\[
T\sum_{i=1}^{m}\dfrac{u_{i}}{1-u_{i}T}=\sum_{j\in\mathbb{N}\setminus\left\{
0\right\}  }\underbrace{N_{j}\left(  \lambda^{1}\left(  x\right)  ,\lambda
^{2}\left(  x\right)  ,...,\lambda^{j}\left(  x\right)  \right)  }_{=\psi
^{j}\left(  x\right)  \text{ by (\ref{PsiDef})}}T^{j}=\sum_{j\in
\mathbb{N}\setminus\left\{  0\right\}  }\psi^{j}\left(  x\right)
T^{j}=\widetilde{\psi}_{T}\left(  x\right)  .
\]
Comparing this with (\ref{9.2.2step}), we obtain%
\[
-T\cdot\dfrac{d}{dT}\log\lambda_{-T}\left(  x\right)  =\widetilde{\psi}%
_{T}\left(  x\right)  .
\]
Hence, the assertion \textbf{(b)} holds for every special $\lambda$-ring
$\left(  K,\left(  \lambda^{i}\right)  _{i\in\mathbb{N}}\right)  $ and every
$x\in K$ such that $x$ is the sum of finitely many $1$-dimensional elements of
$K$. This completes the 2nd step.

\textit{3rd step:} We will now show that the assertion \textbf{(b)} holds for
every special $\lambda$-ring $\left(  K,\left(  \lambda^{i}\right)
_{i\in\mathbb{N}}\right)  $ and every $x\in K$ such that $x$ is the sum of
finitely many $1$-dimensional elements of $K$.

\textit{Proof.} This follows from the 2nd step, since \textbf{(a)} and
\textbf{(b)} are equivalent (by the 1st step).

\textit{4th step:} We will now show that the assertion \textbf{(a)} holds for
every special $\lambda$-ring $\left(  K,\left(  \lambda^{i}\right)
_{i\in\mathbb{N}}\right)  $ and every $x\in K$.

\textit{Proof.} We want to derive this from the 3rd step by applying Theorem 8.4.

Fix some $j\in\mathbb{N}\setminus\left\{  0\right\}  $.

Define a $1$-operation $m$ of special $\lambda$-rings by $m_{\left(  K,\left(
\lambda^{i}\right)  _{i\in\mathbb{N}}\right)  }=\psi^{j}$ for every $\lambda
$-ring $\left(  K,\left(  \lambda^{i}\right)  _{i\in\mathbb{N}}\right)  $.
(This is indeed a $1$-operation, since (\ref{PsiDef}) shows that $\psi^{j}$ is
a polynomial in $\lambda^{1},$ $\lambda^{2},$ $...,$ $\lambda^{j}$ with
integer coefficients.)

Define a $1$-operation $m^{\prime}$ of special $\lambda$-rings by%
\[
m_{\left(  K,\left(  \lambda^{i}\right)  _{i\in\mathbb{N}}\right)  }^{\prime
}\left(  x\right)  =\left(  -1\right)  ^{j+1}\sum_{i=0}^{j}i\lambda^{i}\left(
x\right)  \lambda^{j-i}\left(  -x\right)  \ \ \ \ \ \ \ \ \ \ \text{for every
}x\in K
\]
for every $\lambda$-ring $\left(  K,\left(  \lambda^{i}\right)  _{i\in
\mathbb{N}}\right)  $. (This is, again, a $1$-operation, since it is a
polynomial in $\lambda^{1},$ $\lambda^{2},$ $...,$ $\lambda^{j}$ with integer coefficients.)

These two $1$-operations $m$ and $m^{\prime}$ satisfy both conditions of
Theorem 8.4: The continuity assumption holds (since the operations $m$ and
$m^{\prime}$ are polynomials in $\lambda^{1},$ $\lambda^{2},$ $...,$
$\lambda^{j}$ with integer coefficients, so that the maps $m_{\left(
\Lambda\left(  K\right)  ,\left(  \widehat{\lambda}^{i}\right)  _{i\in
\mathbb{N}}\right)  }$ and $m_{\left(  \Lambda\left(  K\right)  ,\left(
\widehat{\lambda}^{i}\right)  _{i\in\mathbb{N}}\right)  }^{\prime}$ are
polynomials in $\widehat{\lambda}^{1},$ $\widehat{\lambda}^{2},$ $...,$
$\widehat{\lambda}^{j}$ with integer coefficients, and therefore continuous
because of Theorem 5.5 \textbf{(d)}), and the split equality assumption holds
(since it states that for every special $\lambda$-ring $\left(  K,\left(
\lambda^{i}\right)  _{i\in\mathbb{N}}\right)  $ and every $x\in K$ such that
$x$ is the sum of finitely many $1$-dimensional elements of $K$, we have
$m_{\left(  K,\left(  \lambda^{i}\right)  _{i\in\mathbb{N}}\right)  }\left(
x\right)  =m_{\left(  K,\left(  \lambda^{i}\right)  _{i\in\mathbb{N}}\right)
}^{\prime}\left(  x\right)  $; but this simply means that $\psi^{j}\left(
x\right)  =\left(  -1\right)  ^{j+1}\sum\limits_{i=0}^{j}i\lambda^{i}\left(
x\right)  \lambda^{j-i}\left(  -x\right)  ,$ which was proven in the 3rd
step). Hence, by Theorem 8.4, we have $m=m^{\prime}$. Hence, for every special
$\lambda$-ring $\left(  K,\left(  \lambda^{i}\right)  _{i\in\mathbb{N}%
}\right)  $ and every $x\in K$, we have%
\[
\psi^{j}\left(  x\right)  =m_{\left(  K,\left(  \lambda^{i}\right)
_{i\in\mathbb{N}}\right)  }\left(  x\right)  =m_{\left(  K,\left(  \lambda
^{i}\right)  _{i\in\mathbb{N}}\right)  }^{\prime}\left(  x\right)  =\left(
-1\right)  ^{j+1}\sum_{i=0}^{j}i\lambda^{i}\left(  x\right)  \lambda
^{j-i}\left(  -x\right)  .
\]
Thus, the assertion \textbf{(a)} holds for every special $\lambda$-ring
$\left(  K,\left(  \lambda^{i}\right)  _{i\in\mathbb{N}}\right)  $ and every
$x\in K$. This completes the 4th step.

\textit{5th step:} We will now prove that the assertion \textbf{(b)} holds for
every special $\lambda$-ring $\left(  K,\left(  \lambda^{i}\right)
_{i\in\mathbb{N}}\right)  $ and every $x\in K$.

\textit{Proof.} This follows from the 4th step, since \textbf{(a)} and
\textbf{(b)} are equivalent (by the 1st step).

Thus, the proof of Theorem 9.2 is completed.

The Adams operations $\psi^{j}$ have a lot of interesting properties:

\begin{quote}
\textbf{Theorem 9.3.} Let $\left(  K,\left(  \lambda^{i}\right)
_{i\in\mathbb{N}}\right)  $ be a special $\lambda$-ring.

\textbf{(a)} For every $a\in K$, we have $\psi^{1}\left(  a\right)  =a$.

\textbf{(b)} For every $j\in\mathbb{N}\setminus\left\{  0\right\}  $, the map
$\psi^{j}:\left(  K,\left(  \lambda^{i}\right)  _{i\in\mathbb{N}}\right)
\rightarrow\left(  K,\left(  \lambda^{i}\right)  _{i\in\mathbb{N}}\right)  $
is a $\lambda$-ring homomorphism.

\textbf{(c)} For every $i\in\mathbb{N}\setminus\left\{  0\right\}  $ and
$j\in\mathbb{N}\setminus\left\{  0\right\}  $, we have $\psi^{i}\circ\psi
^{j}=\psi^{j}\circ\psi^{i}=\psi^{ij}$.
\end{quote}

Before we come to prove this, let us first show an analogue of Theorem 8.5 for
the $\psi^{i}$:

\begin{quote}
\textbf{Theorem 9.4.} Let $\left(  K,\left(  \lambda^{i}\right)
_{i\in\mathbb{N}}\right)  $ be a $\lambda$-ring. Let $u_{1},$ $u_{2},$ $...,$
$u_{m}$ be $1$-dimensional elements of $K$. Let $j\in\mathbb{N}\setminus
\left\{  0\right\}  $. Then,%
\[
\psi^{j}\left(  u_{1}+u_{2}+...+u_{m}\right)  =u_{1}^{j}+u_{2}^{j}%
+...+u_{m}^{j}.
\]



\end{quote}

\textit{Proof of Theorem 9.4.} Let $x=u_{1}+u_{2}+...+u_{m}$. Just as in the
proof of Theorem 9.2 (in the 2nd step), we can show that%
\[
T\sum\limits_{i=1}^{m}\dfrac{u_{i}}{1-u_{i}T}=\sum\limits_{j\in\mathbb{N}%
\setminus\left\{  0\right\}  }\psi^{j}\left(  x\right)  T^{j}%
\]
(indeed, the proof of this works for any $\lambda$-ring $\left(  K,\left(
\lambda^{i}\right)  _{i\in\mathbb{N}}\right)  $, not only for special ones).

Thus,%
\begin{align*}
\sum\limits_{j\in\mathbb{N}\setminus\left\{  0\right\}  }\psi^{j}\left(
x\right)  T^{j}  &  =T\sum\limits_{i=1}^{m}\dfrac{u_{i}}{1-u_{i}T}%
=\sum\limits_{i=1}^{m}u_{i}T\left(  1-u_{i}T\right)  ^{-1}=\sum\limits_{i=1}%
^{m}u_{i}T\sum_{k\in\mathbb{N}}\left(  u_{i}T\right)  ^{k}=\sum\limits_{i=1}%
^{m}\sum_{k\in\mathbb{N}}\left(  u_{i}T\right)  ^{k+1}\\
&  =\sum\limits_{i=1}^{m}\sum_{j\in\mathbb{N}\setminus\left\{  0\right\}
}\left(  u_{i}T\right)  ^{j}=\sum\limits_{i=1}^{m}\sum_{j\in\mathbb{N}%
\setminus\left\{  0\right\}  }u_{i}^{j}T^{j}=\sum_{j\in\mathbb{N}%
\setminus\left\{  0\right\}  }\sum\limits_{i=1}^{m}u_{i}^{j}T^{j}.
\end{align*}
Comparing coefficients yields $\psi^{j}\left(  x\right)  =\sum\limits_{i=1}%
^{m}u_{i}^{j}$ for every $j\in\mathbb{N}\setminus\left\{  0\right\}  ,$ and
thus Theorem 9.4 is proven.

\textit{Proof of Theorem 9.3.} \textbf{(a)} is trivial (for instance, by
Theorem 9.2 \textbf{(a)}).

\textbf{(b)} Fix some $j\in\mathbb{N}\setminus\left\{  0\right\}  $. First,
let us prove that $\psi^{j}:K\rightarrow K$ is a ring homomorphism.

This means proving that%
\begin{align}
\psi^{j}\left(  0\right)   &  =0;\label{9.3.1}\\
\psi^{j}\left(  x+y\right)   &  =\psi^{j}\left(  x\right)  +\psi^{j}\left(
y\right)  \ \ \ \ \ \ \ \ \ \ \text{for any }x\in K\text{ and }y\in
K;\label{9.3.2}\\
\psi^{j}\left(  1\right)   &  =1;\label{9.3.3}\\
\psi^{j}\left(  xy\right)   &  =\psi^{j}\left(  x\right)  \cdot\psi^{j}\left(
y\right)  \ \ \ \ \ \ \ \ \ \ \text{for any }x\in K\text{ and }y\in K.
\label{9.3.4}%
\end{align}
Out of these four equations, two (namely, (\ref{9.3.1}) and (\ref{9.3.3})) are
trivial (just apply Theorem 9.4, remembering that $1$ is a $1$-dimensional
element), so it remains to prove the other two equations - namely,
(\ref{9.3.2}) and (\ref{9.3.4}).

First, let us prove (\ref{9.3.4}):

Define a $2$-operation $m$ of special $\lambda$-rings as follows: For every
special $\lambda$-ring $K$, let $m_{\left(  K,\left(  \lambda^{i}\right)
_{i\in\mathbb{N}}\right)  }:K^{2}\rightarrow K$ be the map defined by
$m_{\left(  K,\left(  \lambda^{i}\right)  _{i\in\mathbb{N}}\right)  }\left(
x,y\right)  =\psi^{j}\left(  xy\right)  $ for every $x\in K$ and $y\in K$.
(This is indeed a $2$-operation of special $\lambda$-rings, since $\psi^{j}$
is a polynomial in the $\lambda^{1},$ $\lambda^{2},$ $...,$ $\lambda^{j}$ with
integer coefficients.)

Define a $2$-operation $m^{\prime}$ of special $\lambda$-rings as follows: For
every special $\lambda$-ring $K$, let $m_{\left(  K,\left(  \lambda
^{i}\right)  _{i\in\mathbb{N}}\right)  }^{\prime}:K^{2}\rightarrow K$ be the
map defined by $m_{\left(  K,\left(  \lambda^{i}\right)  _{i\in\mathbb{N}%
}\right)  }^{\prime}\left(  x,y\right)  =\psi^{j}\left(  x\right)  \cdot
\psi^{j}\left(  y\right)  $ for every $x\in K$ and $y\in K$. (Again, this is
really a $2$-operation of special $\lambda$-rings.)

We want to prove that $m=m^{\prime}$. According to Theorem 8.4, this will be
done once we have verified the continuity assumption and the split equality
assumption. The continuity assumption is obviously satisfied (since for every
ring $K$, the maps $m_{\left(  \Lambda\left(  K\right)  ,\left(
\widehat{\lambda}^{i}\right)  _{i\in\mathbb{N}}\right)  }:\left(
\Lambda\left(  K\right)  \right)  ^{2}\rightarrow\Lambda\left(  K\right)  $
and $m_{\left(  \Lambda\left(  K\right)  ,\left(  \widehat{\lambda}%
^{i}\right)  _{i\in\mathbb{N}}\right)  }^{\prime}:\left(  \Lambda\left(
K\right)  \right)  ^{2}\rightarrow\Lambda\left(  K\right)  $ are continuous by
Theorem 5.5 \textbf{(d)}, because they are polynomials in $\widehat{\lambda
}^{1},$ $\widehat{\lambda}^{2},$ $...,$ $\widehat{\lambda}^{j}$ with integer
coefficients). Hence, it remains to verify the split equality assumption. This
assumption claims that for every special $\lambda$-ring $\left(  K,\left(
\lambda^{i}\right)  _{i\in\mathbb{N}}\right)  $ and every $\left(  x,y\right)
\in K^{2}$ such that each of $x$ and $y$ is the sum of finitely many
$1$-dimensional elements of $K$, we have $m_{\left(  K,\left(  \lambda
^{i}\right)  _{i\in\mathbb{N}}\right)  }\left(  x,y\right)  =m_{\left(
K,\left(  \lambda^{i}\right)  _{i\in\mathbb{N}}\right)  }^{\prime}\left(
x,y\right)  $.

Since $m_{\left(  K,\left(  \lambda^{i}\right)  _{i\in\mathbb{N}}\right)
}\left(  x,y\right)  =\psi^{j}\left(  xy\right)  $ and $m_{\left(  K,\left(
\lambda^{i}\right)  _{i\in\mathbb{N}}\right)  }^{\prime}\left(  x,y\right)
=\psi^{j}\left(  x\right)  \cdot\psi^{j}\left(  y\right)  $, this is
equivalent to claiming that for every special $\lambda$-ring $\left(
K,\left(  \lambda^{i}\right)  _{i\in\mathbb{N}}\right)  $ and every $\left(
x,y\right)  \in K^{2}$ such that each of $x$ and $y$ is the sum of finitely
many $1$-dimensional elements of $K$, we have $\psi^{j}\left(  xy\right)
=\psi^{j}\left(  x\right)  \cdot\psi^{j}\left(  y\right)  $.

So let us verify this assumption. Let $\left(  K,\left(  \lambda^{i}\right)
_{i\in\mathbb{N}}\right)  $ be a special $\lambda$-ring, and let $\left(
x,y\right)  \in K^{2}$ be such that each of $x$ and $y$ is the sum of finitely
many $1$-dimensional elements of $K$. Thus, there exist $1$-dimensional
elements $u_{1},$ $u_{2},$ $...,$ $u_{m}$ of $K$ such that $x=u_{1}%
+u_{2}+...+u_{m}$, and there exist $1$-dimensional elements $v_{1},$ $v_{2},$
$...,$ $v_{n}$ of $K$ such that $y=v_{1}+v_{2}+...+v_{n}$. Then,%
\begin{align*}
m_{\left(  K,\left(  \lambda^{i}\right)  _{i\in\mathbb{N}}\right)  }\left(
x,y\right)   &  =\psi^{j}\left(  xy\right)  =\psi^{j}\left(  \left(
u_{1}+u_{2}+...+u_{m}\right)  \left(  v_{1}+v_{2}+...+v_{n}\right)  \right) \\
&  =\psi^{j}\left(  \sum_{i=1}^{m}u_{i}\sum_{i^{\prime}=1}^{n}v_{i^{\prime}%
}\right)  =\psi^{j}\left(  \sum_{i=1}^{m}\sum_{i^{\prime}=1}^{n}%
u_{i}v_{i^{\prime}}\right)  =\sum_{i=1}^{m}\sum_{i^{\prime}=1}^{n}\left(
u_{i}v_{i^{\prime}}\right)  ^{j}\\
&  \ \ \ \ \ \ \ \ \ \ \left(
\begin{array}
[c]{c}%
\text{by Theorem 9.4, applied to the }1\text{-dimensional elements }%
u_{i}v_{i^{\prime}}\text{,}\\
\text{which are }1\text{-dimensional because of Theorem 8.3 \textbf{(b)}}%
\end{array}
\right) \\
&  =\sum_{i=1}^{m}\sum_{i^{\prime}=1}^{n}u_{i}^{j}v_{i^{\prime}}^{j}%
=\sum_{i=1}^{m}u_{i}^{j}\sum_{i^{\prime}=1}^{n}v_{i^{\prime}}^{j}%
=\underbrace{\left(  u_{1}^{j}+u_{2}^{j}+...+u_{m}^{j}\right)  }%
_{\substack{=\psi^{j}\left(  u_{1}+u_{2}+...+u_{m}\right)  \\\text{by Theorem
9.4}}}\underbrace{\left(  v_{1}^{j}+v_{2}^{j}+...+v_{n}^{j}\right)
}_{\substack{=\psi^{j}\left(  v_{1}+v_{2}+...+v_{n}\right)  \\\text{by Theorem
9.4}}}\\
&  =\psi^{j}\left(  u_{1}+u_{2}+...+u_{m}\right)  \cdot\psi^{j}\left(
v_{1}+v_{2}+...+v_{n}\right)  =\psi^{j}\left(  x\right)  \cdot\psi^{j}\left(
y\right)  =m_{\left(  K,\left(  \lambda^{i}\right)  _{i\in\mathbb{N}}\right)
}^{\prime}\left(  x,y\right)  ,
\end{align*}
and the proof of the split equality assumption is complete. Thus, using
Theorem 8.4, we obtain that $\psi^{j}\left(  xy\right)  =\psi^{j}\left(
x\right)  \cdot\psi^{j}\left(  y\right)  $ holds for any $x\in K$ and any
$y\in K$.

The main idea of the above proof was that, using Theorem 8.4, we can reduce
our goal - which was to show that $\psi^{j}\left(  xy\right)  =\psi^{j}\left(
x\right)  \cdot\psi^{j}\left(  y\right)  $ for any $x\in K$ and $y\in K$ - to
a simpler goal - namely, to prove the same under the additional condition that
each of $x$ and $y$ is the sum of finitely many $1$-dimensional elements of
$K$, we have $\psi^{j}\left(  xy\right)  =\psi^{j}\left(  x\right)  \cdot
\psi^{j}\left(  y\right)  $. In other words, when proving the equality
$\psi^{j}\left(  xy\right)  =\psi^{j}\left(  x\right)  \cdot\psi^{j}\left(
y\right)  $, we could WLOG assume that each of $x$ and $y$ is the sum of
finitely many $1$-dimensional elements of $K$. Under this assumption, the
equality $\psi^{j}\left(  xy\right)  =\psi^{j}\left(  x\right)  \cdot\psi
^{j}\left(  y\right)  $ was an easy consequence of Theorem 9.4. This way, we
have proven (\ref{9.3.4}). Similarly, we can show (\ref{9.3.2}).

Again, for every $i\in\mathbb{N}$, we can use the same tactic to show that
$\left(  \psi^{j}\circ\lambda^{i}\right)  \left(  x\right)  =\left(
\lambda^{i}\circ\psi^{j}\right)  \left(  x\right)  $ for every $x\in K$
(namely, we use Theorem 8.4 to reduce the proof to the case when $x$ is the
sum of finitely many $1$-dimensional elements of $K$, and we apply Theorems
9.4, 8.5 and 8.3 \textbf{(b)} to verify it in this case). Hence, $\psi
^{j}\circ\lambda^{i}=\lambda^{i}\circ\psi^{j}$ for every $i\in\mathbb{N}$, and
thus $\psi^{j}$ is a $\lambda$-ring homomorphism. Theorem 9.3 \textbf{(b)} is proven.

\textbf{(c)} Fix $n\in\mathbb{N}$ and $m\in\mathbb{N}.$ We have to prove that
$\left(  \psi^{n}\circ\psi^{m}\right)  \left(  x\right)  =\left(  \psi
^{m}\circ\psi^{n}\right)  \left(  x\right)  =\psi^{nm}\left(  x\right)  $ for
every $x\in K$. This can be done by the same method as in the proof of part
\textbf{(b)}: First, reduce the proof to the case when $x$ is the sum of
finitely many $1$-dimensional elements of $K$ (by an application of Theorem
8.4); then, verify $\left(  \psi^{n}\circ\psi^{m}\right)  \left(  x\right)
=\left(  \psi^{m}\circ\psi^{n}\right)  \left(  x\right)  =\psi^{nm}\left(
x\right)  $ in this case by applying Theorems 9.4 and 8.3 \textbf{(b)}. Thus,
Theorem 9.3 \textbf{(c)} is proven.

\begin{quotation}
\textit{Exercise 9.1.} Let $K$ be a ring. Let $u\in1+K\left[  T\right]  ^{+}$.
For every $j\in\mathbb{N}$, let us denote by $\widehat{\psi}^{j}$ the $j$-th
Adams operation of the $\lambda$-ring $\left(  \Lambda\left(  K\right)
,\left(  \widehat{\lambda}^{i}\right)  _{i\in\mathbb{N}}\right)  $.

Assume that $u=\Pi\left(  \widetilde{K}_{u},\left[  u_{1},u_{2},...,u_{m}%
\right]  \right)  $ for some $\left(  \widetilde{K}_{u},\left[  u_{1}%
,u_{2},...,u_{m}\right]  \right)  \in K^{\operatorname*{int}}$. Let
$j\in\mathbb{N}$. Then, $\widehat{\psi}^{j}\left(  u\right)  =\Pi\left(
\widetilde{K}_{u},\left[  u_{1}^{j},u_{2}^{j},...,u_{m}^{j}\right]  \right)  $.

[This gives a formula for $\widehat{\psi}^{j}$ similar to the formula for
$\widehat{\lambda}^{j}$ given in Theorem 5.3 \textbf{(d)}.]

\textit{Exercise 9.2.} Let $K$ be a ring. Let $i\in\mathbb{N}\setminus\left\{
0\right\}  $. Define a mapping $\operatorname*{coeff}\nolimits_{i}%
:\Lambda\left(  K\right)  \rightarrow K$ by $\operatorname*{coeff}%
\nolimits_{i}\left(  \sum\limits_{j\in\mathbb{N}}a_{j}T^{j}\right)  =a_{i}$
for every $\sum\limits_{j\in\mathbb{N}}a_{j}T^{j}\in\Lambda\left(  K\right)  $
(with $a_{j}\in K$ for every $j\in\mathbb{N}$). (In other words,
$\operatorname*{coeff}\nolimits_{i}$ is the mapping that takes a power series
and returns its coefficient before $T^{i}.$)

\textbf{(a)} Prove that the map%
\begin{align*}
\Lambda\left(  K\right)   &  \rightarrow K,\\
u  &  \mapsto\left(  -1\right)  ^{i}\operatorname*{coeff}\nolimits_{i}\left(
-T\dfrac{d}{dT}\log u\right)
\end{align*}
is a ring homomorphism.

\textbf{(b)} This fact, combined with Theorem 9.2 \textbf{(b)}, can be used to
give a new proof of a part of Theorem 9.3 \textbf{(b)}. Which part, and how?

\textit{Exercise 9.3.} Let $\left(  K,\left(  \lambda^{i}\right)
_{i\in\mathbb{N}}\right)  $ be a special $\lambda$-ring.

\textbf{(a)} Prove that%
\[
n\lambda^{n}\left(  x\right)  =\sum_{i=1}^{n}\left(  -1\right)  ^{i-1}%
\lambda^{n-i}\left(  x\right)  \psi^{i}\left(  x\right)
\ \ \ \ \ \ \ \ \ \ \text{for every }x\in K\text{ and }n\in\mathbb{N}\text{.}%
\]


\textbf{(b)} Let $x\in K$ and $n\in\mathbb{N}$. Let $A_{n}=\left(
a_{i,j}\right)  _{1\leq i\leq n,\ 1\leq j\leq n}\in K^{n\times n}$ be the
matrix defined by%
\[
a_{i,j}=\left\{
\begin{array}
[c]{c}%
\psi^{i-j+1}\left(  x\right)  ,\text{ if }i\geq j;\\
i,\text{ if }i=j-1;\\
0,\text{ if }i<j-1
\end{array}
\right.  .
\]
Prove that $n!\lambda^{n}\left(  x\right)  =\det A_{n}$.

[The matrix $A_{n}$ has the following form:%
\[
A_{n}=\left(
\begin{array}
[c]{cccccc}%
\psi^{1}\left(  x\right)  & 1 & 0 & \cdots & 0 & 0\\
\psi^{2}\left(  x\right)  & \psi^{1}\left(  x\right)  & 2 & \cdots & 0 & 0\\
\psi^{3}\left(  x\right)  & \psi^{2}\left(  x\right)  & \psi^{1}\left(
x\right)  & \cdots & 0 & 0\\
\vdots & \vdots & \vdots & \ddots & \vdots & \vdots\\
\psi^{n-1}\left(  x\right)  & \psi^{n-2}\left(  x\right)  & \psi^{n-3}\left(
x\right)  & \cdots & \psi^{1}\left(  x\right)  & n-1\\
\psi^{n}\left(  x\right)  & \psi^{n-1}\left(  x\right)  & \psi^{n-2}\left(
x\right)  & \cdots & \psi^{2}\left(  x\right)  & \psi^{1}\left(  x\right)
\end{array}
\right)  .
\]
]

\bigskip
\end{quotation}

\fbox{\textbf{WARNING:} The following is incomplete and no proofs have been
added yet.}

\begin{center}
\fbox{\textbf{10. Todd homomorphisms of polynomials}}
\end{center}

\textbf{WARNING:} The following may be wrong or differ from the standard
notations. I am trying to generalize [1], I \S 6 (mostly because it is
slightly flawed\footnote{[1], I \S 6, p. 24 states that $\operatorname*{td}%
_{\varphi}\left(  e\right)  $ is a universal polynomial in $\lambda^{1}\left(
e\right)  ,$ $...,$ $\lambda^{r}\left(  e\right)  $, determined by $\varphi$
alone. I think it isn't; instead, it is just a power series. On the other
hand, my generalization $\operatorname*{td}\nolimits_{\varphi,T}$ is a
universal polynomial.} and the generalization looks more natural to me), but I
cannot guarantee that this is the right generalization.

\begin{quote}
\textbf{Definition \& Theorem 10.1.} Let $\varphi\in\mathbb{Z}\left[
t\right]  $ be a polynomial with constant term equal to $1$. There exists one
and only one polynomial $\operatorname*{Td}_{\varphi,j}\in\mathbb{Z}\left[
\alpha_{1},\alpha_{2},...,\alpha_{j}\right]  $ for every $j\in\mathbb{N}$ such
that%
\[
\prod_{i=1}^{m}\varphi\left(  U_{i}T\right)  =\sum_{j\in\mathbb{N}%
}\operatorname*{Td}\nolimits_{\varphi,j}\left(  X_{1},X_{2},...,X_{j}\right)
T^{j}%
\]
in the ring $\left(  \mathbb{Z}\left[  U_{1},U_{2},...,U_{m}\right]  \right)
\left[  \left[  T\right]  \right]  $ for every $m\in\mathbb{N}$, where
$X_{i}=\sum\limits_{\substack{S\subseteq\left\{  1,2,...,m\right\}
;\\\left\vert S\right\vert =i}}\prod\limits_{k\in S}U_{k}$ as usual.

\textbf{Definition.} Let $\left(  K,\left(  \lambda^{i}\right)  _{i\in
\mathbb{N}}\right)  $ be a $\lambda$-ring. Let $\varphi\in\mathbb{Z}\left[
T\right]  $ be a polynomial with constant term equal to $1$. We define a map
$\operatorname*{td}_{\varphi,T}:K\rightarrow K\left[  \left[  T\right]
\right]  $ by%
\[
\operatorname*{td}\nolimits_{\varphi,T}\left(  x\right)  =\sum_{j\in
\mathbb{N}}\operatorname*{Td}\nolimits_{\varphi,j}\left(  \lambda^{1}\left(
x\right)  ,\lambda^{2}\left(  x\right)  ,...,\lambda^{j}\left(  x\right)
\right)  T^{j}\ \ \ \ \ \ \ \ \ \ \text{for every }x\in K.
\]



\end{quote}

[What [1] considers is the so-called \textit{Todd homomorphism}
$\operatorname*{td}_{\varphi}:=\operatorname*{td}\nolimits_{\varphi,1}$ (where
$\operatorname*{td}\nolimits_{\varphi,1}$ means "take the formal power series
$\operatorname*{td}\nolimits_{\varphi,T}$ and replace every $T$ by $1$), which
is only defined on $x$ if $x$ is finite-dimensional, i. e. $\lambda^{i}\left(
x\right)  =0$ for all sufficiently large $i.$ But I prefer the power series
$\operatorname*{td}\nolimits_{\varphi,T}\left(  x\right)  $ since it is
defined on \textit{every }$x.$]

\begin{quote}
\textbf{Theorem 10.2.} Let $\left(  K,\left(  \lambda^{i}\right)
_{i\in\mathbb{N}}\right)  $ be a special $\lambda$-ring. Let $u_{1},$ $u_{2},$
$...,$ $u_{m}$ be $1$-dimensional elements of $K$. Let $j\in\mathbb{N}$. Then,%
\[
\operatorname*{td}\nolimits_{\varphi,T}\left(  u_{1}+u_{2}+...+u_{m}\right)
=\prod_{i=1}^{m}\varphi\left(  u_{i}T\right)  .
\]


\textbf{Theorem 10.3.} Let $\left(  K,\left(  \lambda^{i}\right)
_{i\in\mathbb{N}}\right)  $ be a special $\lambda$-ring. Then,
$\operatorname*{td}_{\varphi,T}\left(  K\right)  \subseteq\Lambda\left(
K\right)  $, and $\operatorname*{td}_{\varphi,T}\left(  K\right)
:K\rightarrow\Lambda\left(  K\right)  $ is a homomorphism of $\mathbb{Z}$-modules.

\textbf{Definition.} Let $j\in\mathbb{N}\setminus\left\{  0\right\}  $. Let
$\left(  K,\left(  \lambda^{i}\right)  _{i\in\mathbb{N}}\right)  $ be a
$\lambda$-ring. Define a homomorphism $\theta_{T}^{j}:K\rightarrow K\left[
\left[  T\right]  \right]  $ by $\theta_{T}^{j}=\operatorname*{td}%
_{\varphi_{j},T},$ where $\varphi_{j}\in\mathbb{Z}\left[  t\right]  $ is the
polynomial $1+t+t^{2}+...+t^{j-1}=\dfrac{1-t^{j}}{1-t}$.
\end{quote}

[Again, [1] considers only $\theta^{j}:=\theta_{1}^{j},$ which again is
defined on $x$ only if $x$ is finite-dimensional. These $\theta^{j}$ (or
$\theta^{j}\left(  x\right)  $ ?) are called \textit{Bott's cannibalistic
classes}, for whatever reason.]

\begin{quote}
\textbf{Theorem 10.4.} Let $\left(  K,\left(  \lambda^{i}\right)
_{i\in\mathbb{N}}\right)  $ be a special $\lambda$-ring. Let $x\in K$. Let
$j\in\mathbb{N}$. Then, $\left(  \psi^{j}\left[  \left[  T\right]  \right]
\right)  \left(  \lambda_{-T}\left(  x\right)  \right)  =\lambda_{-T}\left(
x\right)  \theta_{T}^{j}\left(  x\right)  $ (where $\psi^{j}\left[  \left[
T\right]  \right]  $ means the homomorphism $K\left[  \left[  T\right]
\right]  \rightarrow K\left[  \left[  T\right]  \right]  $ defined by $\left(
\psi^{j}\left[  \left[  T\right]  \right]  \right)  \left(  \sum
\limits_{i\in\mathbb{N}}a_{i}T^{i}\right)  =\sum\limits_{i\in\mathbb{N}}%
\psi^{j}\left(  a_{i}\right)  T^{i}$ for every power series $\sum
\limits_{i\in\mathbb{N}}a_{i}T^{i}\in K\left[  \left[  T\right]  \right]  $).
\end{quote}

\begin{center}
\fbox{\textbf{X. Positive structure on }$\lambda$\textbf{-rings}}
\end{center}

Almost all $\lambda$-rings in Fulton/Lang [1] and many $\lambda$-rings in
nature carry an additional structure called a \textit{positive structure}:

\begin{quote}
\textbf{Definition.} \textbf{1)} Let $\left(  K,\left(  \lambda^{i}\right)
_{i\in\mathbb{N}}\right)  $ be a $\lambda$-ring. Let $\varepsilon
:K\rightarrow\mathbb{Z}$ be a surjective\footnote{I am quoting this from [1].
Personally, I have never have met a non-surjective ring homomorphism in my
life.} ring homomorphism. Let $\mathbf{E}$ be a subset of $K$ such that
$\mathbf{E}$ is closed under addition and multiplication and contains the
subset $\mathbb{Z}^{+}$ of $K$ (that is, the image of $\mathbb{Z}^{+}$ under
the canonical ring homomorphism $\mathbb{Z}^{+}\rightarrow K$). Also assume
that $K=\mathbf{E}-\mathbf{E}$ (that is, every element of $K$ can be written
as difference of two elements of $\mathbf{E}$). Furthermore, assume that every
$e\in\mathbf{E}$ satisfies%
\[
\varepsilon\left(  e\right)  >0;\ \ \ \ \ \ \ \ \ \ \lambda^{i}\left(
e\right)  =0\ \ \ \ \ \ \ \ \ \ \text{for any }i>\varepsilon\left(  e\right)
,\ \ \ \ \ \ \ \ \ \ \text{and that }\lambda^{\varepsilon\left(  e\right)
}\left(  e\right)  \text{ is a unit in the ring }K.
\]


Then, $\left(  \varepsilon,\mathbf{E}\right)  $ is called a \textit{positive
structure} on the $\lambda$-ring. The homomorphism $\varepsilon:K\rightarrow
\mathbb{Z}$ is called an \textit{augmentation} for the $\lambda$-ring $\left(
K,\left(  \lambda^{i}\right)  _{i\in\mathbb{N}}\right)  $ with its positive
structure $\left(  \varepsilon,\mathbf{E}\right)  $. The elements of the set
$\mathbf{E}$ are called the \textit{positive elements} of the $\lambda$-ring
$\left(  K,\left(  \lambda^{i}\right)  _{i\in\mathbb{N}}\right)  $ with its
positive structure $\left(  \varepsilon,\mathbf{E}\right)  $.

[Some axioms may be missing here. For example, every element $u\in\mathbf{E}$
satisfying $\varepsilon\left(  u\right)  =1$ is invertible, which is easy to
prove; but [1] goes on claiming that the inverse must lie in $\mathbf{E}$ as
well, what I cannot prove. Maybe it should be an additional axiom. Besides,
maybe $\lambda^{i}\left(  \mathbf{E}\right)  \subseteq\mathbf{E}$ should also
be a requirement.]

\textbf{2)} Let $\left(  K,\left(  \lambda^{i}\right)  _{i\in\mathbb{N}%
}\right)  $ be a $\lambda$-ring with a positive structure $\left(
\varepsilon,\mathbf{E}\right)  .$ The subset $\left\{  u\in\mathbf{E}%
\ \mid\ \varepsilon\left(  u\right)  =1\right\}  $ of $\mathbf{E}$ is usually
denoted as $\mathbf{L}$. The elements of $\mathbf{L}$ are called the
\textit{line elements} of the $\lambda$-ring $\left(  K,\left(  \lambda
^{i}\right)  _{i\in\mathbb{N}}\right)  $ with its positive structure $\left(
\varepsilon,\mathbf{E}\right)  $.

\textbf{Theorem X.1.} Let $\left(  K,\left(  \lambda^{i}\right)
_{i\in\mathbb{N}}\right)  $ be a $\lambda$-ring with a positive structure
$\left(  \varepsilon,\mathbf{E}\right)  .$

\textbf{(a)} Then, $\mathbf{L}=\left\{  u\in\mathbf{E}\ \mid\ \varepsilon
\left(  u\right)  =1\right\}  $ is a subgroup of the (multiplicative) unit
group $K^{\times}$ of $K$. [As I said, this may require additional axioms.]

\textbf{(b)} We have $\mathbf{L}=\left\{  u\in\mathbf{E}\ \mid\ \lambda
_{T}\left(  u\right)  =1+uT\right\}  =\left\{  u\in\mathbf{E}\ \mid\ u\text{
is }1\text{-dimensional}\right\}  $.
\end{quote}

\textit{Proof of Theorem X.1.} \textbf{(b)} Every $u\in\mathbf{L}$ satisfies%
\[
\lambda_{T}\left(  u\right)  =\sum\limits_{i\in\mathbb{N}}\lambda^{i}\left(
u\right)  T^{i}=\underbrace{\lambda^{0}\left(  u\right)  }_{=1}+\underbrace
{\lambda^{1}\left(  u\right)  }_{=u}T+\sum\limits_{i\geq2}\underbrace
{\lambda^{i}\left(  u\right)  }_{\substack{=0,\text{ since}\\i>1=\varepsilon
\left(  u\right)  \\\text{and }u\in\mathbf{L}\subseteq\mathbf{E}}}T^{i}=1+uT.
\]
Conversely, if some $u\in\mathbf{E}$ satisfies $\lambda_{T}\left(  u\right)
=1+uT$, then%
\[
\sum\limits_{i\in\mathbb{N}}\lambda^{i}\left(  u\right)  T^{i}=\lambda
_{T}\left(  u\right)  =1+uT
\]
shows that $\lambda^{i}\left(  u\right)  =0$ for every $i>1$, and thus
$\varepsilon\left(  u\right)  \leq1$ (because $\lambda^{\varepsilon\left(
u\right)  }\left(  u\right)  $ is a unit in the ring $K$ (since $u\in
\mathbf{E}$), so that $\lambda^{\varepsilon\left(  u\right)  }\left(
u\right)  \neq0$ and thus $\varepsilon\left(  u\right)  \leq1$). Together with
$\varepsilon\left(  u\right)  >0$, this yields $\varepsilon\left(  u\right)
=1$ and thus $u\in\mathbf{L}$. This proves \textbf{(b)}.

\textbf{(a)} First we need to show that every $u\in\mathbf{L}$ is invertible
in $K$. In fact, let $u\in\mathbf{L}$. Then, $\lambda^{\varepsilon\left(
u\right)  }\left(  u\right)  $ is a unit in the ring $K$ (since $u\in
\mathbf{E}$). But $\varepsilon\left(  u\right)  =1$ and thus $\lambda
^{\varepsilon\left(  u\right)  }\left(  u\right)  =\lambda^{1}\left(
u\right)  =u$. Thus, $u$ is a unit in $K$; that is, $u$ is invertible. Its
inverse $u^{-1}$ must lie in $\mathbf{E}$ as well

[\#2] [I'm not sure anymore that it must. Maybe this should be added as an
additional axiom for positive structure?]

\begin{quote}
\textbf{Theorem X.2.} If $\left(  K,\left(  \lambda^{i}\right)  _{i\in
\mathbb{N}}\right)  $ is a $\lambda$-ring with a positive structure $\left(
\varepsilon,\mathbf{E}\right)  $, then $\left(  K,\left(  \lambda^{i}\right)
_{i\in\mathbb{N}}\right)  $ is a special $\lambda$-ring.

\textbf{Theorem X.3.} Let $\left(  K,\left(  \lambda^{i}\right)
_{i\in\mathbb{N}}\right)  $ be a $\lambda$-ring with a positive structure
$\left(  \varepsilon,\mathbf{E}\right)  $. Let $e\in\mathbf{E}$, and let
$r=\varepsilon\left(  e\right)  -1$. Define a polynomial $p_{e}\in K\left[
T\right]  $ by $p_{e}\left(  T\right)  =\sum\limits_{i=0}^{r+1}\left(
-1\right)  ^{i}\lambda^{i}\left(  e\right)  T^{r+1-i}$. Set $K_{e}=K\left[
T\right]  \diagup\left(  p_{e}\left(  T\right)  \right)  =K\left[
\ell\right]  $, where $\ell$ denotes the equivalence class of $T$ modulo
$p_{e}\left(  T\right)  $. Then, $K_{e}$ is a finite-free extension ring of
$K$. There exists a map $\widetilde{\lambda}^{i}:K_{e}\rightarrow K_{e}$ for
every $i\in\mathbb{N}$ such that $\left(  K_{e},\left(  \widetilde{\lambda
}^{i}\right)  _{i\in\mathbb{N}}\right)  $ is a $\lambda$-ring such that the
inclusion $K\rightarrow K_{e}$ is a $\lambda$-ring homomorphism and such that
$\ell\in K_{e}$ is a $1$-dimensional element. Moreover, there exists a
positive structure $\left(  \varepsilon_{e},\mathbf{E}_{e}\right)  $ on
$K_{e}$ defined by $\varepsilon_{e}\left(  \ell\right)  =1$ and $\mathbf{E}%
_{e}=\left\{  \sum\limits_{i\in\mathbb{N},\ j\in\mathbb{N}}a_{i,j}\ell
^{i}\left(  e-\ell\right)  ^{j}\ \mid\text{ }a_{i,j}\in\mathbb{E}\text{ for
all }i\in\mathbb{N}\text{ and }j\in\mathbb{N}\right\}  .$
\end{quote}

By iterating the construction in Theorem X.3, we can find, for any
$e\in\mathbf{E}$, an extension ring of $K$ with a $\lambda$-ring structure in
which $e$ is the sum of $r$ $1$-dimensional elements. This is called the
\textit{splitting principle}, and is what Fulton/Lang [1] use instead of
Theorem 8.4 above when they want to prove an identity just by verifying it for
sums of $1$-dimensional elements. However, this way they can only show it for
positive elements, while Theorem 8.4 yields it for arbitrary elements.

\begin{quote}
[add noncont. splitting principle generalizing 8.4][...]
\end{quote}

\begin{center}
\fbox{\textbf{Hints and solutions to exercises}}
\end{center}

\textit{Exercise 2.1: Solution:} \textbf{(a)} Theorem 2.1 \textbf{(c)} says
that $f$ is a homomorphism of $\lambda$-rings if and only if $\mu_{T}\circ
f=f\left[  \left[  T\right]  \right]  \circ\lambda_{T}$. Thus, it remains to
show that $\mu_{T}\circ f=f\left[  \left[  T\right]  \right]  \circ\lambda
_{T}$ holds if and only if every $e\in E$ satisfies $\mu_{T}\circ f=f\left[
\left[  T\right]  \right]  \circ\lambda_{T}$. Since $E$ is a generating set of
the $\mathbb{Z}$-module $K$, this comes down to proving the following three facts:

\begin{itemize}
\item We have $\left(  \mu_{T}\circ f\right)  \left(  0\right)  =\left(
f\left[  \left[  T\right]  \right]  \circ\lambda_{T}\right)  \left(  0\right)
$.

\item We have $\left(  \mu_{T}\circ f\right)  \left(  -x\right)  =\left(
f\left[  \left[  T\right]  \right]  \circ\lambda_{T}\right)  \left(
-x\right)  $ for every $x\in K$ which satisfies $\left(  \mu_{T}\circ
f\right)  \left(  x\right)  =\left(  f\left[  \left[  T\right]  \right]
\circ\lambda_{T}\right)  \left(  x\right)  $.

\item We have $\left(  \mu_{T}\circ f\right)  \left(  x+y\right)  =\left(
f\left[  \left[  T\right]  \right]  \circ\lambda_{T}\right)  \left(
x+y\right)  $ for any $x\in K$ and $y\in K$ which satisfy $\left(  \mu
_{T}\circ f\right)  \left(  x\right)  =\left(  f\left[  \left[  T\right]
\right]  \circ\lambda_{T}\right)  \left(  x\right)  $ and $\left(  \mu
_{T}\circ f\right)  \left(  y\right)  =\left(  f\left[  \left[  T\right]
\right]  \circ\lambda_{T}\right)  \left(  y\right)  $.
\end{itemize}

We will only prove the last of these three assertions (the other two are
similar): If $\left(  \mu_{T}\circ f\right)  \left(  x\right)  =\left(
f\left[  \left[  T\right]  \right]  \circ\lambda_{T}\right)  \left(  x\right)
$ and $\left(  \mu_{T}\circ f\right)  \left(  y\right)  =\left(  f\left[
\left[  T\right]  \right]  \circ\lambda_{T}\right)  \left(  y\right)  $, then%
\begin{align*}
\left(  \mu_{T}\circ f\right)  \left(  x+y\right)   &  =\mu_{T}\left(
f\left(  x+y\right)  \right)  =\mu_{T}\left(  f\left(  x\right)  +f\left(
y\right)  \right)  \ \ \ \ \ \ \ \ \ \ \left(  \text{since }f\text{ is a ring
homomorphism}\right) \\
&  =\mu_{T}\left(  f\left(  x\right)  \right)  \cdot\mu_{T}\left(  f\left(
y\right)  \right)  \ \ \ \ \ \ \ \ \ \ \left(  \text{by Theorem 2.1
\textbf{(a)}, applied to the }\lambda\text{-ring }\left(  L,\left(  \mu
^{i}\right)  _{i\in\mathbb{N}}\right)  \right) \\
&  =\left(  \mu_{T}\circ f\right)  \left(  x\right)  \cdot\left(  \mu_{T}\circ
f\right)  \left(  y\right)  =\left(  f\left[  \left[  T\right]  \right]
\circ\lambda_{T}\right)  \left(  x\right)  \cdot\left(  f\left[  \left[
T\right]  \right]  \circ\lambda_{T}\right)  \left(  y\right) \\
&  =\left(  f\left[  \left[  T\right]  \right]  \right)  \left(
\underbrace{\lambda_{T}\left(  x\right)  \cdot\lambda_{T}\left(  y\right)
}_{=\lambda_{T}\left(  x+y\right)  \text{ by Theorem 2.1 \textbf{(a)}}%
}\right)  =\left(  f\left[  \left[  T\right]  \right]  \circ\lambda
_{T}\right)  \left(  x+y\right)  ,
\end{align*}
qed.

\textbf{(b)} This follows from \textbf{(a)} in the same way as Theorem 2.1
\textbf{(c)} was proven.

\textit{Exercise 2.2: Solution:} It is an exercise in basic algebra to see
that $L-L$ is a subring of $K$. Thus, it only remains to show that
$\lambda^{i}\left(  L-L\right)  \subseteq L-L$ for every $i\in\mathbb{N}$. In
other words, we have to prove that $\lambda^{i}\left(  \ell-\ell^{\prime
}\right)  \in L-L$ for every $\ell\in L$ and $\ell^{\prime}\in L$.

We are going to prove this by induction, so we assume that $\lambda^{j}\left(
\ell-\ell^{\prime}\right)  \in L-L$ for all $j<i$. Then,%
\begin{align*}
\lambda^{i}\left(  \ell\right)   &  =\lambda^{i}\left(  \left(  \ell
-\ell^{\prime}\right)  +\ell^{\prime}\right)  =\sum_{j=0}^{i}\lambda
^{j}\left(  \ell-\ell^{\prime}\right)  \lambda^{i-j}\left(  \ell^{\prime
}\right)  \ \ \ \ \ \ \ \ \ \ \left(  \text{by the definition of }%
\lambda\text{-rings}\right) \\
&  =\sum_{j=0}^{i-1}\underbrace{\lambda^{j}\left(  \ell-\ell^{\prime}\right)
}_{\substack{\in L-L,\text{ since}\\j<n}}\underbrace{\lambda^{i-j}\left(
\ell^{\prime}\right)  }_{\substack{\in L\text{, since}\\\ell^{\prime}\in
L}}+\lambda^{i}\left(  \ell-\ell^{\prime}\right)  \underbrace{\lambda
^{0}\left(  \ell^{\prime}\right)  }_{=1}\in\underbrace{\left(  L-L\right)
L}_{=L-L}+\lambda^{i}\left(  \ell-\ell^{\prime}\right)  \subseteq\left(
L-L\right)  +\lambda^{i}\left(  \ell-\ell^{\prime}\right)  .
\end{align*}
But $\lambda^{i}\left(  \ell\right)  $ itself lies in $L-L$ (since it lies in
$L$, because $\ell\in L$), so this yields $\lambda^{i}\left(  \ell
-\ell^{\prime}\right)  \in L-L,$ and this completes our induction.

\textit{Exercise 3.1: Hints to solution:} Count how often a prime $q$ appears
in $n!$ and $x\cdot\left(  x-1\right)  \cdot...\cdot\left(  x-i+1\right)  $. I
would prefer a better proof, but I know of none.

\textit{Exercise 3.2:} \textit{Hints to solution:} Let $\mathbb{N}_{K}^{+}$ be
the subset $\left\{  1,2,3,...\right\}  $ of $K$. Obviously, this subset is
multiplicatively closed and contains no zero-divisors. Hence, the localization
$\left(  \mathbb{N}_{K}^{+}\right)  ^{-1}K$ can be considered as an extension
ring of $K$. We now can define $\dbinom{x}{i}\in\left(  \mathbb{N}_{K}%
^{+}\right)  ^{-1}K$ for every $x\in\left(  \mathbb{N}_{K}^{+}\right)  ^{-1}K$
and $i\in\mathbb{N}$. It remains to show that $\dbinom{x}{i}\in K$ for every
$x\in K$ and $i\in\mathbb{N}$, given that $\dbinom{x}{i}\in K$ for every $x\in
E$ and $i\in\mathbb{N}$.

This will follow once we have seen that for $i\in\mathbb{N}$ fixed,

\begin{itemize}
\item the polynomial $\dbinom{X+Y}{i}\in\mathbb{Q}\left[  X,Y\right]  $ is a
polynomial in $\dbinom{X}{0},$ $\dbinom{X}{1},$ $...,$ $\dbinom{X}{i},$
$\dbinom{Y}{0},$ $\dbinom{Y}{1},$ $...,$ $\dbinom{Y}{i}$ with integer coefficients;

\item the polynomial $\dbinom{-X}{i}\in\mathbb{Q}\left[  X\right]  $ is a
polynomial in $\dbinom{X}{0},$ $\dbinom{X}{1},$ $...,$ $\dbinom{X}{i}$ with
integer coefficients;

\item the polynomial $\dbinom{XY}{i}\in\mathbb{Q}\left[  X,Y\right]  $ is a
polynomial in $\dbinom{X}{0},$ $\dbinom{X}{1},$ $...,$ $\dbinom{X}{i},$
$\dbinom{Y}{0},$ $\dbinom{Y}{1},$ $...,$ $\dbinom{Y}{i}$ with integer coefficients.
\end{itemize}

The first of these three properties follows from (\ref{vandermonde}) (applied
to $x=X$ and $y=Y$, which is not tragic since it holds in every binomial ring,
in particular $\mathbb{Q}\left[  X,Y\right]  $). The second one follows from
the first one and the identity $\dbinom{-x}{i}=\left(  -1\right)  ^{i}%
\dbinom{x+i-1}{i}$ for every $x\in\mathbb{Z}$ and $i\in\mathbb{N}$ (this is
the so-called \textit{upper negation identity}). The third one, alas, doesn't
seem to be that easy. One way to prove it is to use the proof of Theorem 7.1
below. Another, more elementary way (leading to a different polynomial!!) is
by recalling the fact ([4], Proposition I.7.3) that the subset%
\[
\left\{  p\in\mathbb{Q}\left[  X\right]  \text{ }\mid\ p\left(  n\right)
\in\mathbb{Z}\text{ for every }n\in\mathbb{Z}\right\}
\]
of the polynomial ring $\mathbb{Q}\left[  X\right]  $ is the $\mathbb{Z}%
$-linear span of the polynomials $\dbinom{X}{0},$ $\dbinom{X}{1},$ $\dbinom
{X}{2},$ $...$. This generalizes to two variables: The subset
\[
\left\{  p\in\mathbb{Q}\left[  X,Y\right]  \text{ }\mid\ p\left(  n,m\right)
\in\mathbb{Z}\text{ for every }n\in\mathbb{Z}\text{ and every }m\in
\mathbb{Z}\right\}
\]
of the polynomial ring $\mathbb{Q}\left[  X,Y\right]  $ is the $\mathbb{Z}%
$-linear span of the polynomials $\dbinom{X}{i}\dbinom{Y}{j}$ for
$i\in\mathbb{N}$ and $j\in\mathbb{N}$. Of course, the polynomial $\dbinom
{XY}{i}$ belongs to this subset, so all we need is proven.

\textit{Exercise 3.3: Hints to solution:} \textbf{(a)} Use Theorem 2.1
\textbf{(a)} and $\left(  1+pT\right)  ^{x}\left(  1+pT\right)  ^{y}=\left(
1+pT\right)  ^{x+y}$.

\textbf{(b)} Use the binomial formula.

\textit{Exercise 3.4: Hints to solution:} It is clear from the very definition
of $\lambda^{i}$ that Theorem 2.1 \textbf{(a)} is to be applied here.

\textit{Exercise 3.5: Hints to solution:} Same idea as for Exercise 3.4.

\textit{Exercise 5.1: Hints to solution:} This can be proven by induction: The
ring $K\left[  T\right]  \diagup\left(  p\right)  $ is a finite-free extension
of $K$ containing a root of the polynomial $P$ (namely, the equivalence class
$\overline{T}$ of $T\in K\left[  T\right]  $ modulo $\left(  p\right)  $).
Now, the polynomial $\dfrac{p\left(  S\right)  }{S-\overline{T}}\in\left(
K\left[  T\right]  \diagup\left(  p\right)  \right)  \left[  S\right]  $ is
monic and has degree $n-1$, so by induction there exists a finite-free
extension ring $K_{p}$ of the ring $K\left[  T\right]  \diagup\left(
p\right)  $ and $n-1$ elements $p_{2},$ $...,$ $p_{n}$ of this extension ring
$K_{p}$ such that $\dfrac{p\left(  S\right)  }{S-\overline{T}}=\prod
\limits_{i=2}^{n}\left(  T-p_{i}\right)  $ in $K_{p}\left[  S\right]  $. Thus,
$p\left(  S\right)  =\left(  S-\overline{T}\right)  \prod\limits_{i=2}%
^{n}\left(  T-p_{i}\right)  $ in $K_{p}\left[  S\right]  $. If we denote
$\overline{T}$ by $p_{1},$ this takes the form $p\left(  S\right)
=\prod\limits_{i=1}^{n}\left(  T-p_{i}\right)  ,$ which shows that we have
just completed the induction step.

Notice the similarity between this solution and the proof of the existence of
splitting fields in Galois theory.

\textit{Exercise 6.1: Hints to solution:} We have $\left(  1+K\left[
T\right]  ^{+}\right)  ^{-1}K\left[  T\right]  \cap\Lambda\left(  K\right)
=\left(  1+K\left[  T\right]  ^{+}\right)  \widehat{-}\left(  1+K\left[
T\right]  ^{+}\right)  $, where $U\widehat{-}U$ denotes the set $\left\{
u\widehat{-}u^{\prime}\mid u\in U,\ u^{\prime}\in U\right\}  $ for every
subset $U$ of $\Lambda\left(  K\right)  $ (where $\widehat{-}$ means
subtraction with respect to the addtion $\widehat{+}$ in $\Lambda\left(
K\right)  $). The subset $1+K\left[  T\right]  ^{+}$ of $\Lambda\left(
K\right)  $ is closed under the addition $\widehat{+}$, the multiplication
$\widehat{\cdot}$ and the maps $\widehat{\lambda}^{i}$ (according to Theorem
5.3, since $1+K\left[  T\right]  ^{+}=\Pi\left(  K^{\operatorname*{int}%
}\right)  $) and contains the zero $1$ and the unity $1+T$. Thus, by Exercise
2.2 (applied to $\Lambda\left(  K\right)  $ and $1+K\left[  T\right]  ^{+}$
instead of $K$ and $L$), the conclusion follows.

\textit{Exercise 6.2: Solution:} Assume, for the sake of contradiction, that
$m=0$ in $K$ for some positive integer $m$. Then, Theorem 2.1 \textbf{(a)}
yields%
\begin{align*}
\lambda_{T}\left(  m\right)   &  =\lambda_{T}\left(  \underbrace
{1+1+...+1}_{m\text{ times}}\right)  =\underbrace{\lambda_{T}\left(  1\right)
\cdot\lambda_{T}\left(  1\right)  \cdot...\cdot\lambda_{T}\left(  1\right)
}_{m\text{ times}}=\left(  \lambda_{T}\left(  1\right)  \right)  ^{m}\\
&  =\left(  \lambda_{T}\left(  1\right)  \right)  ^{m}=\left(  1+T\right)
^{m}=1+\sum_{i=1}^{m-1}\dbinom{m}{i}T^{i}+T^{m}\ \ \ \ \ \ \ \ \ \ \left(
\text{by the binomial formula}\right)
\end{align*}
in $K\left[  \left[  T\right]  \right]  $. On the other hand, $\lambda
_{T}\left(  m\right)  =\lambda_{T}\left(  0\right)  =1.$ Contradiction.

\textit{Exercise 6.3: Hints to solution:} Use Exercise 6.4 or the very
definition of special $\lambda$-rings together with Exercise 2.1. Do not
forget to check that the map $\lambda_{T}$ is well-defined.

\textit{Exercise 6.4: Hints to solution:} Repeat the proof of Theorem 6.1,
replacing every appearance of "$x\in K$" by "$x\in E$" and every appearance of
"$y\in K$" by "$y\in E$". You need the fact that $\lambda_{T}$ is a $\lambda
$-ring homomorphism if and only if it satisfies the three conditions%
\begin{align*}
\lambda_{T}\left(  xy\right)   &  =\lambda_{T}\left(  x\right)  \widehat
{\cdot}\lambda_{T}\left(  y\right)  \ \ \ \ \ \ \ \ \ \ \text{for every }x\in
E\text{ and }y\in E,\\
\lambda_{T}\left(  1\right)   &  =1+T,\ \ \ \ \ \ \ \ \ \ \text{and}\\
\lambda_{T}\left(  \lambda^{j}\left(  x\right)  \right)   &  =\widehat
{\lambda}^{j}\left(  \lambda_{T}\left(  x\right)  \right)
\ \ \ \ \ \ \ \ \ \ \text{for every }j\in\mathbb{N}\text{ and }x\in E.
\end{align*}
This is because the first two of these conditions, together with the
preassumptions that $E$ is a generating set of $K$ as a $\mathbb{Z}$-module
and that $\lambda_{T}$ is an additive group homomorphism, are equivalent to
claiming that $\lambda_{T}$ is a ring homomorphism; and the third condition
then makes $\lambda_{T}$ a $\lambda$-ring homomorphism (according to Exercise
2.1 \textbf{(b)}).

\textit{Exercise 6.5: Hints to solution:} First, the mapping
$\operatorname*{coeff}\nolimits_{i}:\Lambda\left(  K\right)  \rightarrow K$ is
continuous (with respect to the $\left(  T\right)  $-topology on
$\Lambda\left(  K\right)  $ and \textit{any arbitrary topology} on $K$), and
the operations $\widehat{+}$ and $\widehat{\cdot}$ are continuous as well (by
Theorem 5.5 \textbf{(d)}); besides, the subset $1+K\left[  T\right]  ^{+}$ of
$1+K\left[  \left[  T\right]  \right]  ^{+}=\Lambda\left(  K\right)  $ is
dense (by Theorem 5.5 \textbf{(a)}). Hence, in order to prove that
$\operatorname*{coeff}\nolimits_{i}\left(  u\right)  =\operatorname*{coeff}%
\nolimits_{1}\left(  \widehat{\lambda}^{i}\left(  u\right)  \right)  $ for
every $u\in\Lambda\left(  K\right)  ,$ it is enough to verify that
$\operatorname*{coeff}\nolimits_{i}\left(  u\right)  =\operatorname*{coeff}%
\nolimits_{1}\left(  \widehat{\lambda}^{i}\left(  u\right)  \right)  $ for
every $u\in1+K\left[  T\right]  ^{+}.$ So let us assume that $u\in1+K\left[
T\right]  ^{+}$. Then, there exist some $\left(  \widetilde{K}_{u},\left[
u_{1},u_{2},...,u_{m}\right]  \right)  \in K^{\operatorname*{int}}$ such that
$u=\Pi\left(  \widetilde{K}_{u},\left[  u_{1},u_{2},...,u_{m}\right]  \right)
.$%
\begin{align*}
u  &  =\Pi\left(  \widetilde{K},\left[  u_{1},u_{2},...,u_{m}\right]  \right)
=\prod_{i=1}^{m}\left(  1+u_{i}T\right)  =\sum_{i\in\mathbb{N}}\sum
_{K\in\mathcal{P}_{i}\left(  \left\{  1,2,...,m\right\}  \right)  }%
\prod\limits_{k\in K}\left(  u_{k}T\right) \\
&  =\sum_{i\in\mathbb{N}}\left(  \sum_{K\in\mathcal{P}_{i}\left(  \left\{
1,2,...,m\right\}  \right)  }\prod\limits_{k\in K}u_{k}\right)  \cdot T^{i}%
\end{align*}
yields $\operatorname*{coeff}\nolimits_{i}u=\sum\limits_{K\in\mathcal{P}%
_{i}\left(  \left\{  1,2,...,m\right\}  \right)  }\prod\limits_{k\in K}u_{k}.$
On the other hand, Theorem 5.3 \textbf{(d)} yields%
\begin{align*}
\widehat{\lambda}^{i}\left(  u\right)   &  =\Pi\left(  \widetilde{K}%
_{u},\left[  \prod\limits_{k\in K}u_{k}\ \mid\ K\in\mathcal{P}_{i}\left(
\left\{  1,2,...,m\right\}  \right)  \right]  \right) \\
&  =\prod_{K\in\mathcal{P}_{i}\left(  \left\{  1,2,...,m\right\}  \right)
}\left(  1+\prod\limits_{k\in K}u_{k}T\right)  =1+\sum_{K\in\mathcal{P}%
_{i}\left(  \left\{  1,2,...,m\right\}  \right)  }\prod\limits_{k\in K}%
u_{k}\cdot T+\left(  \text{higher powers of }T\right)  ,
\end{align*}
so that%
\[
\operatorname*{coeff}\nolimits_{1}\left(  \widehat{\lambda}^{i}\left(
u\right)  \right)  =\sum_{K\in\mathcal{P}_{i}\left(  \left\{
1,2,...,m\right\}  \right)  }\prod\limits_{k\in K}u_{k}.
\]
Comparing with $\operatorname*{coeff}\nolimits_{i}u=\sum\limits_{K\in
\mathcal{P}_{i}\left(  \left\{  1,2,...,m\right\}  \right)  }\prod
\limits_{k\in K}u_{k},$ we get $\operatorname*{coeff}\nolimits_{i}\left(
u\right)  =\operatorname*{coeff}\nolimits_{1}\left(  \widehat{\lambda}%
^{i}\left(  u\right)  \right)  ,$ qed.

\textit{Exercise 6.6: Hints to solution:} Consider the maps $\widehat{\lambda
}^{i}:\Lambda\left(  K\right)  \rightarrow\Lambda\left(  K\right)  $ that we
have defined in Section 5. Theorem 5.1 \textbf{(b)} yields that $\left(
\Lambda\left(  K\right)  ,\left(  \widehat{\lambda}^{i}\right)  _{i\in
\mathbb{N}}\right)  $ is a $\lambda$-ring. Since $\left(  K,\left(
\lambda^{i}\right)  _{i\in\mathbb{N}}\right)  $ is a special $\lambda$-ring,
the map $\lambda_{T}:K\rightarrow\Lambda\left(  K\right)  $ defined in Theorem
5.6 is a $\lambda$-ring homomorphism. Also, we know that the ring homomorphism
$\varphi:K\rightarrow A$ induces a $\lambda$-ring homomorphism $\Lambda\left(
\varphi\right)  :\Lambda\left(  K\right)  \rightarrow\Lambda\left(  A\right)
$. Now, consider the composed $\lambda$-ring homomorphism $\Lambda\left(
\varphi\right)  \circ\lambda_{T}:K\rightarrow\Lambda\left(  A\right)  $.

\textit{1st Step:} We claim that $\operatorname*{coeff}\nolimits_{1}^{A}%
\circ\Lambda\left(  \varphi\right)  \circ\lambda_{T}=\varphi.$

\textit{Proof.} Define a mapping $\operatorname*{coeff}\nolimits_{i}%
:\Lambda\left(  K\right)  \rightarrow K$ for every $i\in\mathbb{N}$ as in
Exercise 6.5. Then, $\operatorname*{coeff}\nolimits_{1}^{A}\circ\Lambda\left(
\varphi\right)  =\varphi\circ\operatorname*{coeff}\nolimits_{1}$ (by the
definition of $\Lambda\left(  \varphi\right)  $) and $\operatorname*{coeff}%
\nolimits_{1}\circ\lambda_{T}=\operatorname*{id}_{K}$ (by Theorem 8.2). Thus,
$\underbrace{\operatorname*{coeff}\nolimits_{1}^{A}\circ\Lambda\left(
\varphi\right)  }_{=\varphi\circ\operatorname*{coeff}\nolimits_{1}}%
\circ\lambda_{T}=\varphi\circ\underbrace{\operatorname*{coeff}\nolimits_{1}%
\circ\lambda_{T}}_{=\operatorname*{id}_{K}}=\varphi,$ and the 1st Step is proven.

\textit{2nd Step:} We claim that if $\widetilde{\varphi}:K\rightarrow
\Lambda\left(  A\right)  $ is a $\lambda$-ring homomorphism such that
$\operatorname*{coeff}\nolimits_{1}^{A}\circ\widetilde{\varphi}=\varphi,$ then
$\widetilde{\varphi}=\Lambda\left(  \varphi\right)  \circ\lambda_{T}.$

\textit{Proof.} For every $i\in\mathbb{N},$ define a mapping
$\operatorname*{coeff}\nolimits_{i}^{A}:\Lambda\left(  A\right)  \rightarrow
A$ by $\operatorname*{coeff}\nolimits_{i}^{A}\left(  \sum\limits_{j\in
\mathbb{N}}a_{j}T^{j}\right)  =a_{i}$ for every $\sum\limits_{j\in\mathbb{N}%
}a_{j}T^{j}\in\Lambda\left(  A\right)  $ (with $a_{j}\in A$ for every
$j\in\mathbb{N}$). (In other words, $\operatorname*{coeff}\nolimits_{i}^{A}$
is the mapping that takes a power series and returns its coefficient before
$T^{i}.$) Then, Exercise 6.5 (applied to the ring $A$ instead of $K$) yields
$\operatorname*{coeff}\nolimits_{i}^{A}=\operatorname*{coeff}\nolimits_{1}%
^{A}\circ\widehat{\lambda}_{A}^{i}.$ Hence,%
\[
\operatorname*{coeff}\nolimits_{i}^{A}\circ\widetilde{\varphi}%
=\operatorname*{coeff}\nolimits_{1}^{A}\circ\underbrace{\widehat{\lambda}%
_{A}^{i}\circ\widetilde{\varphi}}_{\substack{=\widetilde{\varphi}\circ
\lambda^{i},\\\text{since }\widetilde{\varphi}\text{ is a}\\\lambda
\text{-ring}\\\text{homomorphism}}}=\underbrace{\operatorname*{coeff}%
\nolimits_{1}^{A}\circ\widetilde{\varphi}}_{=\varphi}\circ\lambda^{i}%
=\varphi\circ\lambda^{i}.
\]
But on the other hand,%
\[
\underbrace{\operatorname*{coeff}\nolimits_{i}^{A}\circ\Lambda\left(
\varphi\right)  }_{\substack{=\varphi\circ\operatorname*{coeff}\nolimits_{i}%
\text{ by the}\\\text{definition of }\Lambda\left(  \varphi\right)  }%
}\circ\lambda_{T}=\varphi\circ\underbrace{\operatorname*{coeff}\nolimits_{i}%
\circ\lambda_{T}}_{\substack{=\lambda^{i}\text{, by the}\\\text{definition of
}\lambda_{T}}}=\varphi\circ\lambda^{i}.
\]
Therefore, $\operatorname*{coeff}\nolimits_{i}^{A}\circ\widetilde{\varphi
}=\operatorname*{coeff}\nolimits_{i}^{A}\circ\Lambda\left(  \varphi\right)
\circ\lambda_{T}$ for every $i\in\mathbb{N}$. Thus, $\left(
\operatorname*{coeff}\nolimits_{i}^{A}\circ\widetilde{\varphi}\right)  \left(
u\right)  =\left(  \operatorname*{coeff}\nolimits_{i}^{A}\circ\Lambda\left(
\varphi\right)  \circ\lambda_{T}\right)  \left(  u\right)  $ for every
$i\in\mathbb{N}$ for every $u\in K$. In other words, for every $u\in K$ and
for every $i\in\mathbb{N}$, the power series $\widetilde{\varphi}\left(
u\right)  \in\Lambda\left(  A\right)  $ and $\left(  \Lambda\left(
\varphi\right)  \circ\lambda_{T}\right)  \left(  u\right)  $ have the same
coefficient before $T^{i}$. Since this holds for all $i\in\mathbb{N}$ at the
same time, this simply means that for every $u\in K$, the power series
$\widetilde{\varphi}\left(  u\right)  \in\Lambda\left(  A\right)  $ and
$\left(  \Lambda\left(  \varphi\right)  \circ\lambda_{T}\right)  \left(
u\right)  $ are equal. In other words, $\widetilde{\varphi}=\Lambda\left(
\varphi\right)  \circ\lambda_{T},$ and thus the 2nd Step is proven.

Together, the 1st and the 2nd Steps yield the assertion of Exercise 6.6 (in
fact, the 1st Step yields the existence of a $\lambda$-ring homomorphism
$\widetilde{\varphi}:K\rightarrow\Lambda\left(  A\right)  $ such that
$\operatorname*{coeff}\nolimits_{1}^{A}\circ\widetilde{\varphi}=\varphi,$
namely the homomorphism $\Lambda\left(  \varphi\right)  \circ\lambda_{T},$ and
the 2nd Step proves that this is the only such homomorphism).

\textit{Exercise 7.1: Hints to solution:} See the more general Exercise 7.2.

\textit{Exercise 7.2: Hints to solution:} Repeat the argument used in the
proof of Theorem 7.3.

\textit{Exercise 8.1: Solution:}

\textit{First solution:} In order to prove this, we must verify that%
\begin{align}
\operatorname*{coeff}\nolimits_{1}\left(  1\right)   &  =0;\label{8.2.1}\\
\operatorname*{coeff}\nolimits_{1}\left(  u\widehat{+}v\right)   &
=\operatorname*{coeff}\nolimits_{1}u+\operatorname*{coeff}\nolimits_{1}%
v\ \ \ \ \ \ \ \ \ \ \text{for every }u\in\Lambda\left(  K\right)  \text{ and
}v\in\Lambda\left(  K\right)  ;\label{8.2.2}\\
\operatorname*{coeff}\nolimits_{1}\left(  1+T\right)   &  =1;\label{8.2.3}\\
\operatorname*{coeff}\nolimits_{1}\left(  u\widehat{\cdot}v\right)   &
=\operatorname*{coeff}\nolimits_{1}u\cdot\operatorname*{coeff}\nolimits_{1}%
v\ \ \ \ \ \ \ \ \ \ \text{for every }u\in\Lambda\left(  K\right)  \text{ and
}v\in\Lambda\left(  K\right)  . \label{8.2.4}%
\end{align}
The equations (\ref{8.2.1}) and (\ref{8.2.3}) are immediately obvious. In
order to verify the equations (\ref{8.2.2}) and (\ref{8.2.4}), we notice that
$\operatorname*{coeff}\nolimits_{1}:\Lambda\left(  K\right)  \rightarrow K$ is
a continuous mapping (with respect to the $\left(  T\right)  $-topology on
$\Lambda\left(  K\right)  $ and \textit{any arbitrary topology} on $K$) and
the operations $\widehat{+}$ and $\widehat{\cdot}$ are continuous as well (by
Theorem 5.5 \textbf{(d)}), and the subset $1+K\left[  T\right]  ^{+}$ of
$1+K\left[  \left[  T\right]  \right]  ^{+}=\Lambda\left(  K\right)  $ is
dense (by Theorem 5.5 \textbf{(a)}), so it suffices to verify the equations
(\ref{8.2.2}) and (\ref{8.2.4}) for $u\in1+K\left[  T\right]  ^{+}$ and
$v\in1+K\left[  T\right]  ^{+}$ only. So let $u\in1+K\left[  T\right]  ^{+}$
and $v\in1+K\left[  T\right]  ^{+}$.

Then, there exist some $\left(  \widetilde{K}_{u},\left[  u_{1},u_{2}%
,...,u_{m}\right]  \right)  \in K^{\operatorname*{int}}$ such that
$u=\Pi\left(  \widetilde{K}_{u},\left[  u_{1},u_{2},...,u_{m}\right]  \right)
,$ and some $\left(  \widetilde{K}_{v},\left[  v_{1},v_{2},...,v_{n}\right]
\right)  \in K^{\operatorname*{int}}$ such that $v=\Pi\left(  \widetilde
{K}_{v},\left[  v_{1},v_{2},...,v_{n}\right]  \right)  .$ Obviously,%
\[
u=\Pi\left(  \widetilde{K},\left[  u_{1},u_{2},...,u_{m}\right]  \right)
=\prod_{i=1}^{m}\left(  1+u_{i}T\right)  =1+\sum_{i=1}^{m}u_{i}\cdot T+\left(
\text{higher powers of }T\right)
\]
yields $\operatorname*{coeff}\nolimits_{1}u=\sum\limits_{i=1}^{m}u_{i}.$
Similarly, $\operatorname*{coeff}\nolimits_{1}v=\sum\limits_{j=1}^{n}v_{j}$.

By Theorem 5.3 \textbf{(a)}, there exists a finite-free extension ring
$\widetilde{K}_{u,v}$ of $K$ which contains both $\widetilde{K}_{u}$ and
$\widetilde{K}_{v}$ as subrings. Theorem 5.3 \textbf{(c)} yields%
\begin{align*}
u\widehat{\cdot}v  &  =\Pi\left(  \widetilde{K}_{u,v},\left[  u_{i}v_{j}%
\mid\left(  i,j\right)  \in\left\{  1,2,...,m\right\}  \times\left\{
1,2,...,n\right\}  \right]  \right)  =\prod_{\left(  i,j\right)  \in\left\{
1,2,...,m\right\}  \times\left\{  1,2,...,n\right\}  }\left(  1+u_{i}%
v_{j}T\right) \\
&  =1+\sum_{\left(  i,j\right)  \in\left\{  1,2,...,m\right\}  \times\left\{
1,2,...,n\right\}  }u_{i}v_{j}\cdot T+\left(  \text{higher powers of
}T\right)  ,
\end{align*}
and thus%
\[
\operatorname*{coeff}\nolimits_{1}\left(  u\widehat{\cdot}v\right)
=\sum_{\left(  i,j\right)  \in\left\{  1,2,...,m\right\}  \times\left\{
1,2,...,n\right\}  }u_{i}v_{j}=\sum_{i=1}^{m}\sum_{j=1}^{n}u_{i}v_{j}%
=\sum_{i=1}^{m}u_{i}\sum_{j=1}^{n}v_{j}=\operatorname*{coeff}\nolimits_{1}%
u\cdot\operatorname*{coeff}\nolimits_{1}v,
\]
so that (\ref{8.2.4}) is proven.

Besides,%
\begin{align*}
u\widehat{+}v  &  =uv=\Pi\left(  \widetilde{K}_{u},\left[  u_{1}%
,u_{2},...,u_{m}\right]  \right)  \cdot\Pi\left(  \widetilde{K}_{v},\left[
v_{1},v_{2},...,v_{n}\right]  \right) \\
&  =\prod_{i=1}^{m}\left(  1+u_{i}T\right)  \cdot\prod_{j=1}^{n}\left(
1+v_{j}T\right)  =1+\left(  \sum_{i=1}^{m}u_{i}+\sum_{j=1}^{n}v_{j}\right)
\cdot T+\left(  \text{higher powers of }T\right)  ,
\end{align*}
and consequently%
\[
\operatorname*{coeff}\nolimits_{1}\left(  u\widehat{+}v\right)  =\sum
_{i=1}^{m}u_{i}+\sum_{j=1}^{n}v_{j}=\operatorname*{coeff}\nolimits_{1}%
u+\operatorname*{coeff}\nolimits_{1}v,
\]
and (\ref{8.2.2}) is proven. Thus, $\operatorname*{coeff}\nolimits_{1}%
:\Lambda\left(  K\right)  \rightarrow K$ is a ring homomorphism, and Theorem
8.2 \textbf{(a)} is proven.

\textit{Second solution (sketched):} In fact, (\ref{8.2.2}) is trivial using
$u\widehat{+}v=uv$, and (\ref{8.2.4}) follows from the definition of
$\widehat{\cdot}$ and the fact that $P_{1}\left(  a_{1},b_{1}\right)
=a_{1}\cdot b_{1}$.

\textit{Exercise 8.2: Hints to solution:} Let $y=x^{-1}$. We proceed as in the
proof of Theorem 8.3 \textbf{(b)}, except that we don't know that $\lambda
_{T}\left(  y\right)  =\Pi\left(  K,\left[  y\right]  \right)  $ and thus
cannot conclude anything from this. Instead, $xy=xx^{-1}=1$ yields%
\begin{align*}
\lambda_{T}\left(  xy\right)   &  =\lambda_{T}\left(  1\right)
=1+T\ \ \ \ \ \ \ \ \ \ \left(
\begin{array}
[c]{c}%
\text{since }\lambda_{T}:K\rightarrow\Lambda\left(  K\right)  \text{ is a ring
homomorphism,}\\
\text{and }1+T\text{ is the multiplicative unity of }\Lambda\left(  K\right)
\end{array}
\right) \\
&  =1+xyT=\Pi\left(  K,\left[  xy\right]  \right)  =\Pi\left(  K,\left[
x\right]  \right)  \widehat{\cdot}\Pi\left(  K,\left[  y\right]  \right)
\ \ \ \ \ \ \ \ \ \ \left(  \text{by Theorem 5.3 \textbf{(c)}}\right) \\
&  =\lambda_{T}\left(  x\right)  \widehat{\cdot}\left(  1+yT\right)  .
\end{align*}
Together with $\lambda_{T}\left(  xy\right)  =\lambda_{T}\left(  x\right)
\widehat{\cdot}\lambda_{T}\left(  y\right)  $, this yields $\lambda_{T}\left(
x\right)  \widehat{\cdot}\lambda_{T}\left(  y\right)  =\lambda_{T}\left(
x\right)  \widehat{\cdot}\left(  1+yT\right)  $, so that $\lambda_{T}\left(
y\right)  =1+yT$ (since $\lambda_{T}\left(  x\right)  \in\Lambda\left(
K\right)  $ is invertible, because $x\in K$ is invertible and $\lambda
_{T}:K\rightarrow\Lambda\left(  K\right)  $ is a ring homomorphism), and
Theorem 8.1 yields that $y$ is $1$-dimensional, qed.

\textit{Exercise 8.3: Hints to solution:} According to Exercise 6.4, we only
have to prove that (\ref{LkxyE}) and (\ref{LkLjxE}) hold. This is equivalent
to showing that%
\begin{align*}
\lambda_{T}\left(  xy\right)   &  =\lambda_{T}\left(  x\right)  \widehat
{\cdot}\lambda_{T}\left(  y\right)  \ \ \ \ \ \ \ \ \ \ \text{for every }x\in
E\text{ and }y\in E,\ \ \ \ \ \ \ \ \ \ \text{and}\\
\lambda_{T}\left(  \lambda^{j}\left(  x\right)  \right)   &  =\widehat
{\lambda}^{j}\left(  \lambda_{T}\left(  x\right)  \right)
\ \ \ \ \ \ \ \ \ \ \text{for every }j\in\mathbb{N}\text{ and }x\in E
\end{align*}
(because of the definitions of $\widehat{\cdot}$ and $\widehat{\lambda}^{j}$
and since two formal power series are equal if and only if their respective
coefficients are equal). But this is true, since%
\begin{align*}
\lambda_{T}\left(  xy\right)   &  =1+xyT\ \ \ \ \ \ \ \ \ \ \left(
\text{since }xy\text{ is }1\text{-dimensional by Theorem 8.3 \textbf{(b)}%
}\right) \\
&  =\Pi\left(  K,\left[  xy\right]  \right)  =\Pi\left(  K,\left[  x\right]
\right)  \widehat{\cdot}\Pi\left(  K,\left[  y\right]  \right)
\ \ \ \ \ \ \ \ \ \ \left(  \text{by Theorem 5.3 \textbf{(c)}}\right) \\
&  =\left(  1+xT\right)  \cdot\left(  1+yT\right)  =\lambda_{T}\left(
x\right)  \widehat{\cdot}\lambda_{T}\left(  y\right)
\ \ \ \ \ \ \ \ \ \ \left(  \text{since }x\text{ and }y\text{ are
}1\text{-dimensional}\right)
\end{align*}
for every $x\in E$ and $y\in E,$ and%
\begin{align*}
\lambda_{T}\left(  \lambda^{j}\left(  x\right)  \right)   &  =\lambda
_{T}\left(  \left\{
\begin{array}
[c]{c}%
1,\text{ if }j=0;\\
x,\text{ if }j=1;\\
0,\text{ if }j>1
\end{array}
\right.  \right)  \ \ \ \ \ \ \ \ \ \ \left(  \text{since }x\text{ is
}1\text{-dimensional, so that }\lambda^{j}\left(  x\right)  =\left\{
\begin{array}
[c]{c}%
1,\text{ if }j=0;\\
x,\text{ if }j=1;\\
0,\text{ if }j>1
\end{array}
\right.  \right) \\
&  =\left\{
\begin{array}
[c]{c}%
\lambda_{T}\left(  1\right)  ,\text{ if }j=0;\\
\lambda_{T}\left(  x\right)  ,\text{ if }j=1;\\
\lambda_{T}\left(  0\right)  ,\text{ if }j>1
\end{array}
\right.  =\left\{
\begin{array}
[c]{c}%
1+T,\text{ if }j=0;\\
\lambda_{T}\left(  x\right)  ,\text{ if }j=1;\\
1,\text{ if }j>1
\end{array}
\right. \\
&  =\widehat{\lambda}^{j}\left(  \lambda_{T}\left(  x\right)  \right)
\ \ \ \ \ \ \ \ \ \ \left(
\begin{array}
[c]{c}%
\text{since }\lambda_{T}\left(  x\right)  \text{ is a }1\text{-dimensional
element of }\Lambda\left(  K\right)  \text{ by Theorem 8.3,}\\
\text{so that }\widehat{\lambda}^{j}\left(  \lambda_{T}\left(  x\right)
\right)  =\left\{
\begin{array}
[c]{c}%
1+T,\text{ if }j=0;\\
\lambda_{T}\left(  x\right)  ,\text{ if }j=1;\\
1,\text{ if }j>1
\end{array}
\right.
\end{array}
\right)
\end{align*}
for every $j\in\mathbb{N}$ and $x\in E.$

\textit{Exercise 9.1: Hints to solution:} As before, we use the $\widehat
{\sum}$ sign for summation inside the ring $\Lambda\left(  K\right)  $. We
remember that the addition inside the ring $\Lambda\left(  K\right)  $ was
defined by $u\widehat{+}v=uv$ for any $u\in\Lambda\left(  K\right)  $ and
$v\in\Lambda\left(  K\right)  $ (in other words, addition in $\Lambda\left(
K\right)  $ is the multiplication inherited from $K\left[  \left[  T\right]
\right]  $), so that $\widehat{\sum}=\prod.$ Now,
\[
u=\Pi\left(  \widetilde{K}_{u},\left[  u_{1},u_{2},...,u_{m}\right]  \right)
=\prod_{i=1}^{m}\left(  1+u_{i}T\right)  =\widehat{\sum_{i=1}^{m}}\left(
1+u_{i}T\right)  .
\]
But since $1+u_{i}T$ is a $1$-dimensional element of $\Lambda\left(  K\right)
$ for every $i\in\left\{  1,2,...,m\right\}  $ (by Theorem 8.3 \textbf{(c)}),
Theorem 9.4 yields%
\[
\widehat{\psi}^{j}\left(  \widehat{\sum_{i=1}^{m}}\left(  1+u_{i}T\right)
\right)  =\widehat{\sum_{i=1}^{m}}\left(  1+u_{i}T\right)  ^{\widehat{j}},
\]
where $\left(  1+u_{i}T\right)  ^{\widehat{j}}$ means the $j$-th power of
$1+u_{i}T$ \textit{in the ring }$\Lambda\left(  K\right)  $ (in other words,
$\left(  1+u_{i}T\right)  ^{\widehat{j}}=\underbrace{\left(  1+u_{i}T\right)
\widehat{\cdot}\left(  1+u_{i}T\right)  \widehat{\cdot}...\widehat{\cdot
}\left(  1+u_{i}T\right)  }_{j\text{ times}},$ as opposed to $\left(
1+u_{i}T\right)  ^{j}=\underbrace{\left(  1+u_{i}T\right)  \cdot\left(
1+u_{i}T\right)  \cdot...\cdot\left(  1+u_{i}T\right)  }_{j\text{ times}}$
which is the $j$-th power of $1+u_{i}T$ \textit{in the ring }$K\left[  \left[
T\right]  \right]  $).

Hence,%
\begin{align*}
\widehat{\psi}^{j}\left(  u\right)   &  =\widehat{\psi}^{j}\left(
\widehat{\sum_{i=1}^{m}}\left(  1+u_{i}T\right)  \right)  =\widehat{\sum
_{i=1}^{m}}\left(  1+u_{i}T\right)  ^{\widehat{j}}=\widehat{\sum_{i=1}^{m}%
}\underbrace{\left(  \Pi\left(  \widetilde{K}_{u},\left[  u_{i}\right]
\right)  \right)  ^{\widehat{j}}}_{\substack{=\Pi\left(  \widetilde{K}%
_{u},\left[  u_{i}^{j}\right]  \right)  \text{ by}\\\text{Corollary 5.4
\textbf{(b)}}}}=\widehat{\sum_{i=1}^{m}}\Pi\left(  \widetilde{K}_{u},\left[
u_{i}^{j}\right]  \right) \\
&  =\Pi\left(  \widetilde{K}_{u},\left[  u_{1}^{j},u_{2}^{j},...,u_{m}%
^{j}\right]  \right)  \ \ \ \ \ \ \ \ \ \ \left(  \text{by Corollary 5.4
\textbf{(a)}}\right)  .
\end{align*}


\textit{Exercise 9.2: Hints to solution:} \textbf{(a)} It is easy to see that
the map sends the zero $1$ of $\Lambda\left(  K\right)  $ to the zero $0$ of
$K$ and the multiplicative unity $1+T$ of $\Lambda\left(  K\right)  $ to the
multiplicative unity $1$ of $K$. So it only remains to prove that any two
power series $u\in\Lambda\left(  K\right)  $ and $v\in\Lambda\left(  K\right)
$ satisfy%
\begin{equation}
\left(  -1\right)  ^{i}\operatorname*{coeff}\nolimits_{i}\left(  -T\dfrac
{d}{dT}\log\left(  u\widehat{+}v\right)  \right)  =\left(  -1\right)
^{i}\operatorname*{coeff}\nolimits_{i}\left(  -T\dfrac{d}{dT}\log u\right)
+\left(  -1\right)  ^{i}\operatorname*{coeff}\nolimits_{i}\left(  -T\dfrac
{d}{dT}\log v\right)  \label{9.2.plus}%
\end{equation}
and%
\begin{equation}
\left(  -1\right)  ^{i}\operatorname*{coeff}\nolimits_{i}\left(  -T\dfrac
{d}{dT}\log\left(  u\widehat{\cdot}v\right)  \right)  =\left(  -1\right)
^{i}\operatorname*{coeff}\nolimits_{i}\left(  -T\dfrac{d}{dT}\log u\right)
\cdot\left(  -1\right)  ^{i}\operatorname*{coeff}\nolimits_{i}\left(
-T\dfrac{d}{dT}\log v\right)  . \label{9.2.times}%
\end{equation}
This needs to be verified for $u\in1+K\left[  T\right]  ^{+}$ and
$v\in1+K\left[  T\right]  ^{+}$ only (since the operations $\widehat{+}$ and
$\widehat{\cdot}$ and the mapping%
\begin{align*}
\Lambda\left(  K\right)   &  \rightarrow K,\\
u  &  \mapsto\left(  -1\right)  ^{i}\operatorname*{coeff}\nolimits_{i}\left(
-T\dfrac{d}{dT}\log u\right)
\end{align*}
are continuous (where the topology on $K$ can be chosen arbitrarily), and
$1+K\left[  T\right]  ^{+}$ is a dense subset of $1+K\left[  \left[  T\right]
\right]  ^{+}=\Lambda\left(  K\right)  $). So let us assume that
$u\in1+K\left[  T\right]  ^{+}$ and $v\in1+K\left[  T\right]  ^{+}$. Then,
there exists some $\left(  \widetilde{K}_{u},\left[  u_{1},u_{2}%
,...,u_{m}\right]  \right)  \in K^{\operatorname*{int}}$ such that
$u=\Pi\left(  \widetilde{K}_{u},\left[  u_{1},u_{2},...,u_{m}\right]  \right)
$ and some $\left(  \widetilde{K}_{v},\left[  v_{1},v_{2},...,v_{n}\right]
\right)  \in K^{\operatorname*{int}}$ such that $v=\Pi\left(  \widetilde
{K}_{v},\left[  v_{1},v_{2},...,v_{n}\right]  \right)  $. Then, Theorem 5.3
\textbf{(c)} says that%
\[
u\widehat{\cdot}v=\Pi\left(  \widetilde{K}_{u,v},\left[  u_{i}v_{j}\mid\left(
i,j\right)  \in\left\{  1,2,...,m\right\}  \times\left\{  1,2,...,n\right\}
\right]  \right)  .
\]


Now,%
\[
u=\Pi\left(  \widetilde{K}_{u},\left[  u_{1},u_{2},...,u_{m}\right]  \right)
=\prod_{k=1}^{m}\left(  1+u_{k}T\right)
\]
entails%
\begin{align*}
\dfrac{d}{dT}u  &  =\dfrac{d}{dT}\prod_{k=1}^{m}\left(  1+u_{k}T\right)
=\sum_{\tau=1}^{m}\underbrace{\prod_{k\in\left\{  1,2,...,m\right\}
\setminus\left\{  \tau\right\}  }\left(  1+u_{k}T\right)  }_{=\dfrac
{\prod\limits_{k=1}^{m}\left(  1+u_{k}T\right)  }{1+u_{\tau}T}}\cdot
\underbrace{\dfrac{d}{d\tau}\left(  1+u_{\tau}T\right)  }_{=u_{\tau}%
}\ \ \ \ \ \ \ \ \ \ \left(  \text{by the Leibniz rule}\right) \\
&  =\sum_{\tau=1}^{m}\left(  \underbrace{\prod\limits_{k=1}^{m}\left(
1+u_{k}T\right)  }_{=u}\right)  \cdot\dfrac{u_{\tau}}{1+u_{\tau}T}=u\sum
_{\tau=1}^{m}\dfrac{u_{\tau}}{1+u_{\tau}T}%
\end{align*}
and thus%
\begin{align*}
-T\dfrac{d}{dT}\log u  &  =-T\dfrac{\dfrac{d}{dT}u}{u}=-T\dfrac{u\sum
\limits_{\tau=1}^{m}\dfrac{u_{\tau}}{1+u_{\tau}T}}{u}=-T\sum\limits_{\tau
=1}^{m}\dfrac{u_{\tau}}{1+u_{\tau}T}=\sum\limits_{\tau=1}^{m}\left(  -u_{\tau
}T\right)  \left(  1+u_{\tau}T\right)  ^{-1}\\
&  =\sum\limits_{\tau=1}^{m}\left(  -u_{\tau}T\right)  \sum_{\rho\in
\mathbb{N}}\left(  -1\right)  ^{\rho}\left(  u_{\tau}T\right)  ^{\rho}%
=\sum\limits_{\tau=1}^{m}\sum_{\rho\in\mathbb{N}}\left(  -1\right)  ^{\rho
+1}\left(  u_{\tau}T\right)  ^{\rho+1}=\sum\limits_{\tau=1}^{m}\sum
_{i\in\mathbb{N}\setminus\left\{  0\right\}  }\left(  -1\right)  ^{i}\left(
u_{\tau}T\right)  ^{i}\\
&  =\sum\limits_{\tau=1}^{m}\sum_{i\in\mathbb{N}\setminus\left\{  0\right\}
}\left(  -1\right)  ^{i}u_{\tau}^{i}T^{i}=\sum_{i\in\mathbb{N}\setminus
\left\{  0\right\}  }\left(  -1\right)  ^{i}\sum\limits_{\tau=1}^{m}u_{\tau
}^{i}T^{i},
\end{align*}
so that $\operatorname*{coeff}\nolimits_{i}\left(  -T\dfrac{d}{dT}\log
u\right)  =\left(  -1\right)  ^{i}\sum\limits_{\tau=1}^{m}u_{\tau}^{i}$. In
other words, $\left(  -1\right)  ^{i}\operatorname*{coeff}\nolimits_{i}\left(
-T\dfrac{d}{dT}\log u\right)  =u_{\tau}^{i}$. Similarly, $v=\Pi\left(
\widetilde{K}_{v},\left[  v_{1},v_{2},...,v_{n}\right]  \right)  $ yields
$\left(  -1\right)  ^{i}\operatorname*{coeff}\nolimits_{i}\left(  -T\dfrac
{d}{dT}\log v\right)  =\sum\limits_{\sigma=1}^{n}v_{\sigma}^{i}$, and
$u\widehat{\cdot}v=\Pi\left(  \widetilde{K}_{u,v},\left[  u_{i}v_{j}%
\mid\left(  i,j\right)  \in\left\{  1,2,...,m\right\}  \times\left\{
1,2,...,n\right\}  \right]  \right)  $ yields $\left(  -1\right)
^{i}\operatorname*{coeff}\nolimits_{i}\left(  -T\dfrac{d}{dT}\log\left(
u\widehat{\cdot}v\right)  \right)  =\sum\limits_{\tau=1}^{m}\sum
\limits_{\sigma=1}^{n}\left(  u_{\tau}v_{\sigma}\right)  ^{i}$. Thus,%
\begin{align*}
\left(  -1\right)  ^{i}\operatorname*{coeff}\nolimits_{i}\left(  -T\dfrac
{d}{dT}\log\left(  u\widehat{\cdot}v\right)  \right)   &  =\sum\limits_{\tau
=1}^{m}\sum\limits_{\sigma=1}^{n}\left(  u_{\tau}v_{\sigma}\right)  ^{i}%
=\sum\limits_{\tau=1}^{m}\sum\limits_{\sigma=1}^{n}u_{\tau}^{i}v_{\sigma}%
^{i}=\sum\limits_{\tau=1}^{m}u_{\tau}^{i}\cdot\sum\limits_{\sigma=1}%
^{n}v_{\sigma}^{i}\\
&  =\left(  -1\right)  ^{i}\operatorname*{coeff}\nolimits_{i}\left(
-T\dfrac{d}{dT}\log u\right)  \cdot\left(  -1\right)  ^{i}%
\operatorname*{coeff}\nolimits_{i}\left(  -T\dfrac{d}{dT}\log v\right)  ,
\end{align*}
and (\ref{9.2.times}) is thus proven. Similarly we can show (\ref{9.2.plus}).
This completes the proof.

\textbf{(b)} Using Exercise 9.2 \textbf{(a)}, we can easily prove the
following fact:

\textit{Assertion }$\mathcal{F}$\textit{:} If $\left(  K,\left(  \lambda
^{i}\right)  _{i\in\mathbb{N}}\right)  $ is a special $\lambda$-ring, then,
for any $i\in\mathbb{N}\setminus\left\{  0\right\}  $, the $i$-th Adams
operation $\psi^{i}:K\rightarrow K$ is a ring homomorphism.

This assertion is a part of Theorem 9.3 \textbf{(b)}.

In order to prove Assertion $\mathcal{F}$ using Exercise 9.2 \textbf{(a)}, we
notice that the composition of the map $\lambda_{T}:K\rightarrow\Lambda\left(
K\right)  $ (which is a ring homomorphism, since the $\lambda$-ring $\left(
K,\left(  \lambda^{i}\right)  _{i\in\mathbb{N}}\right)  $ is special) with the
map%
\begin{align*}
\Lambda\left(  K\right)   &  \rightarrow K,\\
u  &  \mapsto\left(  -1\right)  ^{i}\operatorname*{coeff}\nolimits_{i}\left(
-T\dfrac{d}{dT}\log u\right)
\end{align*}
(which is a ring homomorphism according to Exercise 9.2 \textbf{(a)}) is
$\psi^{i}$ (because every $x\in K$ satisfies
\[
\sum\limits_{i\in\mathbb{N}\setminus\left\{  0\right\}  }\psi^{i}\left(
x\right)  T^{i}=\widetilde{\psi}_{T}\left(  x\right)  =-T\dfrac{d}{dT}%
\log\lambda_{-T}\left(  x\right)  \ \ \ \ \ \ \ \ \ \ \left(  \text{by Theorem
9.2 \textbf{(b)}}\right)  ,
\]
what (upon the substitution of $-T$ for $T$) becomes%
\[
\sum\limits_{i\in\mathbb{N}\setminus\left\{  0\right\}  }\psi^{i}\left(
x\right)  \left(  -T\right)  ^{i}=-\left(  -T\right)  \dfrac{d}{d\left(
-T\right)  }\log\lambda_{-\left(  -T\right)  }\left(  x\right)  =-T\dfrac
{d}{dT}\log\lambda_{T}\left(  x\right)  ,
\]
what rewrites as $\sum\limits_{i\in\mathbb{N}\setminus\left\{  0\right\}
}\left(  -1\right)  ^{i}\psi^{i}\left(  x\right)  T^{i}=-T\dfrac{d}{dT}%
\log\lambda_{T}\left(  x\right)  $, and thus we have $\left(  -1\right)
^{i}\psi^{i}\left(  x\right)  =\operatorname*{coeff}\nolimits_{i}\left(
-T\dfrac{d}{dT}\log\lambda_{T}\left(  x\right)  \right)  $ and therefore
$\psi^{i}\left(  x\right)  =\left(  -1\right)  ^{i}\operatorname*{coeff}%
\nolimits_{i}\left(  -T\dfrac{d}{dT}\log\lambda_{T}\left(  x\right)  \right)
$). Thus, $\psi^{i}$ is a ring homomorphism (since the composition of two ring
homomorphisms is a ring homomorphism). This proves Assertion $\mathcal{F}$.

\textit{Exercise 9.3: Hints to solution:} \textbf{(a)} According to Theorem
9.2 \textbf{(b)}, we have $\widetilde{\psi}_{T}\left(  x\right)
=-T\cdot\dfrac{d}{dT}\log\lambda_{-T}\left(  x\right)  =-T\cdot\dfrac
{\dfrac{d}{dT}\lambda_{-T}\left(  x\right)  }{\lambda_{-T}\left(  x\right)  }$
for every $x\in K$, where the map $\widetilde{\psi}_{T}:K\rightarrow K\left[
\left[  T\right]  \right]  $ is defined by $\widetilde{\psi}_{T}\left(
x\right)  =\sum\limits_{j\in\mathbb{N}\setminus\left\{  0\right\}  }\psi
^{j}\left(  x\right)  T^{j}$ for every $x\in K$. Here, $\lambda_{-T}\left(
x\right)  $ denotes $\operatorname*{ev}_{-T}\left(  \lambda_{T}\left(
x\right)  \right)  $. Therefore, if we denote $\operatorname*{ev}_{-T}\left(
\widetilde{\psi}_{T}\left(  x\right)  \right)  $ by $\widetilde{\psi}%
_{-T}\left(  x\right)  ,$ then we have%
\[
\widetilde{\psi}_{-T}\left(  x\right)  =\operatorname*{ev}\nolimits_{-T}%
\left(  \widetilde{\psi}_{T}\left(  x\right)  \right)  =\operatorname*{ev}%
\nolimits_{-T}\left(  -T\cdot\dfrac{\dfrac{d}{dT}\lambda_{-T}\left(  x\right)
}{\lambda_{-T}\left(  x\right)  }\right)  =-\left(  -T\right)  \cdot
\dfrac{\dfrac{d}{d\left(  -T\right)  }\lambda_{-\left(  -T\right)  }\left(
x\right)  }{\lambda_{-\left(  -T\right)  }\left(  x\right)  }=-T\cdot
\dfrac{\dfrac{d}{dT}\lambda_{T}\left(  x\right)  }{\lambda_{T}\left(
x\right)  }.
\]
Thus,%
\begin{equation}
\lambda_{T}\left(  x\right)  \cdot\widetilde{\psi}_{-T}\left(  x\right)
=-T\cdot\dfrac{d}{dT}\lambda_{T}\left(  x\right)  . \label{9.ex3.s1}%
\end{equation}


Now, $\widetilde{\psi}_{T}\left(  x\right)  =\sum\limits_{j\in\mathbb{N}%
\setminus\left\{  0\right\}  }\psi^{j}\left(  x\right)  T^{j}$ yields
$\widetilde{\psi}_{-T}\left(  x\right)  =\sum\limits_{j\in\mathbb{N}%
\setminus\left\{  0\right\}  }\psi^{j}\left(  x\right)  \left(  -T\right)
^{j}=\sum\limits_{j\in\mathbb{N}\setminus\left\{  0\right\}  }\left(
-1\right)  ^{j}\psi^{j}\left(  x\right)  T^{j}.$ Together with $\lambda
_{T}\left(  x\right)  =\sum\limits_{i\in\mathbb{N}}\lambda^{i}\left(
x\right)  T^{i}$, this yields%
\[
\lambda_{T}\left(  x\right)  \cdot\widetilde{\psi}_{-T}\left(  x\right)
=\sum_{n\in\mathbb{N}}\sum_{j=1}^{n}\left(  -1\right)  ^{j}\psi^{j}\left(
x\right)  \lambda^{n-j}\left(  x\right)  T^{n}.
\]
On the other hand, $\lambda_{T}\left(  x\right)  =\sum\limits_{i\in\mathbb{N}%
}\lambda^{i}\left(  x\right)  T^{i}=\sum\limits_{n\in\mathbb{N}}\lambda
^{n}\left(  x\right)  T^{n}$ yields $\dfrac{d}{dT}\lambda_{T}\left(  x\right)
=\sum\limits_{n\in\mathbb{N}}n\lambda^{n}\left(  x\right)  T^{n-1}$ and thus
$-T\cdot\dfrac{d}{dT}\lambda_{T}\left(  x\right)  =-\sum\limits_{n\in
\mathbb{N}}n\lambda^{n}\left(  x\right)  T^{n}$. Hence, (\ref{9.ex3.s1})
becomes%
\[
\sum_{n\in\mathbb{N}}\sum_{j=1}^{n}\left(  -1\right)  ^{j}\psi^{j}\left(
x\right)  \lambda^{n-j}\left(  x\right)  T^{n}=-\sum\limits_{n\in\mathbb{N}%
}n\lambda^{n}\left(  x\right)  T^{n}.
\]
Thus, for every $n\in\mathbb{N}$, we have%
\[
\sum_{j=1}^{n}\left(  -1\right)  ^{j}\psi^{j}\left(  x\right)  \lambda
^{n-j}\left(  x\right)  =-n\lambda^{n}\left(  x\right)  .
\]
Dividing this by $-1,$ this becomes%
\[
\sum_{j=1}^{n}\left(  -1\right)  ^{j-1}\psi^{j}\left(  x\right)  \lambda
^{n-j}\left(  x\right)  =n\lambda^{n}\left(  x\right)  ,
\]
which rewrites as%
\[
n\lambda^{n}\left(  x\right)  =\sum_{j=1}^{n}\left(  -1\right)  ^{j-1}\psi
^{j}\left(  x\right)  \lambda^{n-j}\left(  x\right)  =\sum_{i=1}^{n}\left(
-1\right)  ^{i-1}\psi^{i}\left(  x\right)  \lambda^{n-i}\left(  x\right)  ,
\]
which is exactly what Exercise 9.3 \textbf{(a)} claimed.

\textbf{(b)} We will prove the equation $n!\lambda^{n}\left(  x\right)  =\det
A_{n}$ by induction over $n$.

The base case, $n=0$, is trivial (for $0!=1,$ $\lambda^{0}\left(  x\right)
=1$, and the determinant of a $0\times0$ matrix is $1$ by definition). If you
do not believe in $0\times0$ matrices, the $n=1$ case is trivial as well
(since $\lambda^{1}\left(  x\right)  =x$ and $\psi^{1}\left(  x\right)  =x$ by
Theorem 9.3 \textbf{(a)}) and can equally serve as a base case. The
interesting part is the induction step.

For this step, we develop the determinant of the matrix $A_{n}$ along the
$n$-th row. We obtain%
\begin{equation}
\det A_{n}=\sum_{k=1}^{n}\left(  -1\right)  ^{n-k}\psi^{n-k+1}\left(
x\right)  \cdot\det\left(  A_{n}\left[  \dfrac{\sim k}{\sim n}\right]
\right)  , \label{9.ex3.s2}%
\end{equation}
where $A_{n}\left[  \dfrac{\sim k}{\sim n}\right]  $ is the matrix obtained
from $A_{n}$ by removing the $n$-th row and the $k$-th column.

Now, the matrix $A_{n}\left[  \dfrac{\sim k}{\sim n}\right]  $ turns out to be
a block-triangular matrix, with the left upper block (of size $\left(
k-1\right)  \times\left(  k-1\right)  $) being equal to $A_{k-1}$ and the
right lower block (of size $\left(  n-k\right)  \times\left(  n-k\right)  $)
being a lower triangular matrix with the numbers $k,$ $k+1,$ $...,$ $n-1$ on
its diagonal. Hence, $\det\left(  A_{n}\left[  \dfrac{\sim k}{\sim n}\right]
\right)  =\det A_{k-1}\cdot\left(  k\left(  k+1\right)  ...\left(  n-1\right)
\right)  .$ Since we are proceeding by induction, we can take $\det
A_{k-1}=\left(  k-1\right)  !\lambda^{k-1}\left(  x\right)  $ for granted
(since $k-1<n$), and thus obtain%
\begin{align*}
\det\left(  A_{n}\left[  \dfrac{\sim k}{\sim n}\right]  \right)   &  =\det
A_{k-1}\cdot\left(  k\left(  k+1\right)  ...\left(  n-1\right)  \right)
=\left(  k-1\right)  !\lambda^{k-1}\left(  x\right)  \cdot\left(  k\left(
k+1\right)  ...\left(  n-1\right)  \right) \\
&  =\underbrace{\left(  k-1\right)  !\cdot\left(  k\left(  k+1\right)
...\left(  n-1\right)  \right)  }_{=\left(  n-1\right)  !}\cdot\lambda
^{k-1}\left(  x\right)  =\left(  n-1\right)  !\cdot\lambda^{k-1}\left(
x\right)  .
\end{align*}


Thus, (\ref{9.ex3.s2}) becomes%
\begin{align*}
\det A_{n}  &  =\sum_{k=1}^{n}\left(  -1\right)  ^{n-k}\psi^{n-k+1}\left(
x\right)  \cdot\left(  n-1\right)  !\cdot\lambda^{k-1}\left(  x\right)
=\left(  n-1\right)  !\cdot\sum_{k=1}^{n}\left(  -1\right)  ^{n-k}\psi
^{n-k+1}\left(  x\right)  \lambda^{k-1}\left(  x\right) \\
&  =\left(  n-1\right)  !\cdot\underbrace{\sum_{i=1}^{n}\left(  -1\right)
^{i-1}\psi^{i}\left(  x\right)  \lambda^{n-i}\left(  x\right)  }%
_{=n\lambda^{n}\left(  x\right)  \text{ by \textbf{(a)}}}%
\ \ \ \ \ \ \ \ \ \ \left(  \text{here we substituted }i\text{ for
}n-k+1\text{ in the sum}\right) \\
&  =\underbrace{\left(  n-1\right)  !\cdot n}_{=n!}\lambda^{n}\left(
x\right)  =n!\lambda^{n}\left(  x\right)  ,
\end{align*}
completing the induction step, qed.

\begin{center}
\fbox{\textbf{References}}
\end{center}

[1] William Fulton, Serge Lang, \textit{Riemann-Roch algebra}, New York 1985.

[2] Donald Knutson, $\lambda$\textit{-Rings and the Representation Theory of
the Symmetric Group}, New York 1973.

[3] M. F. Atiyah, I. G. Macdonald, \textit{Introduction to Commutative
Algebra}, Addison-Wesley 1969.

[4] Robin Hartshorne, \textit{Algebraic Geometry}, Springer 1977.

[5] Michiel Hazewinkel, \textit{Niceness theorems}, arXiv:0810.5691v1
[math.HO], 2008.\newline\texttt{http://arxiv.org/abs/0810.5691}

[6] Michiel Hazewinkel, \textit{Witt vectors. Part 1}, arXiv:0804.3888v1
[math.RA], 2008.\newline\texttt{http://arxiv.org/abs/0804.3888}


\end{document}