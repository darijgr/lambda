\documentclass[12pt,final,notitlepage,onecolumn,german]{article}%
\usepackage[all,cmtip]{xy}
\usepackage{lscape}
\usepackage{amsfonts}
\usepackage{amssymb}
\usepackage{graphicx}
\usepackage{amsmath}%
\setcounter{MaxMatrixCols}{30}
%TCIDATA{OutputFilter=latex2.dll}
%TCIDATA{Version=5.00.0.2570}
%TCIDATA{LastRevised=Saturday, September 26, 2009 00:47:57}
%TCIDATA{<META NAME="GraphicsSave" CONTENT="32">}
%TCIDATA{<META NAME="SaveForMode" CONTENT="1">}
\voffset=-2.5cm
\hoffset=-2.5cm
\setlength\textheight{24cm}
\setlength\textwidth{15.5cm}
\begin{document}

\begin{center}
\textbf{Lambda rings}

Darij Grinberg, after Fulton/Lang and other sources

\textit{Version 0.0.1}
\end{center}

This is a \textbf{BETA VERSION} and has never been proofread. \textbf{Please
notify me of any mistakes, typos and hard-to-understand arguments you
find!\footnote{my email address is \texttt{A@B.com}, where
\texttt{A=darijgrinberg} and \texttt{B=gmail}}}

\bigskip

\fbox{\textbf{What is this?}}

This is not really original content. It is my attempt to make some parts of
Chapter 1 of Fulton/Lang [1] better accessible by adding some omitted proofs
and changing the order of exposition.

\begin{center}
\fbox{\textbf{0. Notation}}
\end{center}

In the following, $\mathbb{N}$ will denote the set $\left\{
0,1,2,...\right\}  $.

When we say "ring", we will always mean "commutative ring with unity".

\begin{center}
\fbox{\textbf{1. Motivation}}
\end{center}

What is the point of $\lambda$-rings?

Fulton/Lang [1] motivate $\lambda$-rings through vector bundles. Here we are
going for a more elementary motivation, namely through representation rings in
group representation theory:

Consider a finite group $G$ and a field $K$. In representation theory, one
define the so-called \textit{representation ring} of the group $G$ over the
field $K$. This ring can be constructed as follows:\footnote{I hope I got this
right. At least in the case $\left\vert G\right\vert \nmid\operatorname*{char}%
K$ it should be correct.}

Let $\operatorname*{Rep}_{K}G$ be the set of all representations of the group
$G$ over the field $K$. (We disregard the set-theoretic problematics stemming
from the notion of such a big set. If you wish, you can call it a class or a
SET instead of a set, or restrict yourself to a smaller subset containing
every representation up to isomorphism.)

Let $\operatorname*{FRep}_{K}G$ be the free abelian group over
$\operatorname*{Rep}_{K}G$. Denote by $1$ the trivial representation of $G$ on
$K$ (with every element of $G$ acting as identity). We define a ring structure
on $\operatorname*{FRep}_{K}G$ by letting $1$ be the one of this ring, and
defining the product of two representations of $G$ as their tensor product
(over $K$).

Now let $I$ be the two-sided ideal%
\[
I=\left\langle U-V\ \mid\ U\text{ and }V\text{ are two isomorphic
representations of }G\right\rangle
\]
of $\operatorname*{FRep}_{K}G$. The \textit{representation ring of the group
}$G$\textit{ over the field }$K$ is then defined as the ring
$\operatorname*{FRep}_{K}G\diagup I$. It is a commutative ring, as $V\otimes
W\cong W\otimes V$ for any two representations $V$ and $W$.

[...]

\begin{center}
\fbox{\textbf{2. }$\lambda$\textbf{-rings}}
\end{center}

\begin{quote}
\textbf{Definition:} \textbf{1)} Let $K$ be a ring. Let $\lambda
^{i}:K\rightarrow K$ be a mapping\footnote{Here, "mapping" actually means
"mapping" and not "group homomorphism" of "ring homomorphism".} for every
$i\in\mathbb{N}$ such that $\lambda^{0}\left(  x\right)  =1$ and $\lambda
^{1}\left(  x\right)  =x$ for every $x\in K$. Assume that%
\begin{equation}
\lambda^{k}\left(  x+y\right)  =\sum_{i=0}^{k}\lambda^{i}\left(  x\right)
\lambda^{k-i}\left(  y\right)  \ \ \ \ \ \ \ \ \ \ \text{for every }%
k\in\mathbb{N},\text{ }x\in K\text{ and }y\in K. \label{lambda1}%
\end{equation}
Then, we call $\left(  K,\left(  \lambda_{i}\right)  _{i\in\mathbb{N}}\right)
$ a $\lambda$\textit{-ring}. We will also call $K$ itself a $\lambda$-ring if
there is an obvious (from the context) choice of the sequence of mappings
$\left(  \lambda_{i}\right)  _{i\in\mathbb{N}}$ which makes $\left(  K,\left(
\lambda_{i}\right)  _{i\in\mathbb{N}}\right)  $ a $\lambda$-ring.

\textbf{2)} Let $\left(  K,\left(  \lambda_{i}\right)  _{i\in\mathbb{N}%
}\right)  $ be a $\lambda$-ring. Let $\varepsilon:K\rightarrow\mathbb{Z}$ be a
surjective\footnote{I am quoting this from [1]. Personally, I have never have
met a non-surjective ring homomorphism in my life.} ring homomorphism. Let
$\mathbf{E}$ be a subset of $K$ such that $\mathbf{E}$ is closed under
addition and multiplication and contains the subset $\mathbb{Z}^{+}$ of $K$
(that is, the image of $\mathbb{Z}^{+}$ under the canonical ring homomorphism
$\mathbb{Z}^{+}\rightarrow K$). Also assume that $K=\mathbf{E}-\mathbf{E}$
(that is, every element of $K$ can be written as difference of two elements of
$\mathbf{E}$). Furthermore, assume that every $e\in\mathbf{E}$ satisfies%
\begin{align*}
\varepsilon\left(  e\right)   &  >0;\ \ \ \ \ \ \ \ \ \ \lambda^{i}\left(
e\right)  =0\ \ \ \ \ \ \ \ \ \ \text{for any }i>\varepsilon\left(  e\right)
,\\
&  \ \ \ \ \ \ \ \ \ \ \text{and that }\lambda^{\varepsilon\left(  e\right)
}\left(  e\right)  \text{ is a unit in the ring }K.
\end{align*}


Then, $\left(  \varepsilon,\mathbf{E}\right)  $ is called a \textit{positive
structure} on the $\lambda$-ring. The homomorphism $\varepsilon:K\rightarrow
\mathbb{Z}$ is called an \textit{augmentation} for the $\lambda$-ring $\left(
K,\left(  \lambda_{i}\right)  _{i\in\mathbb{N}}\right)  $ with its positive
structure $\left(  \varepsilon,\mathbf{E}\right)  $. The elements of the set
$\mathbf{E}$ are called the \textit{positive elements} of the $\lambda$-ring
$\left(  K,\left(  \lambda_{i}\right)  _{i\in\mathbb{N}}\right)  $ with its
positive structure $\left(  \varepsilon,\mathbf{E}\right)  $.

\textbf{3)} Let $\left(  K,\left(  \lambda_{i}\right)  _{i\in\mathbb{N}%
}\right)  $ be a $\lambda$-ring with a positive structure $\left(
\varepsilon,\mathbf{E}\right)  .$ The subset $\left\{  u\in\mathbf{E}%
\ \mid\ \varepsilon\left(  u\right)  =1\right\}  $ of $\mathbf{E}$ is usually
denoted as $\mathbf{L}$. The elements of $\mathbf{L}$ are called the
\textit{line elements} of the $\lambda$-ring $\left(  K,\left(  \lambda
_{i}\right)  _{i\in\mathbb{N}}\right)  $ with its positive structure $\left(
\varepsilon,\mathbf{E}\right)  $.
\end{quote}

First, we give an alternative characterization of $\lambda$-rings:

\begin{quote}
\textbf{Theorem 2.1.} Let $K$ be a ring. Let $\lambda^{i}:K\rightarrow K$ be a
mapping\footnote{Here, "mapping" actually means "mapping" and not "group
homomorphism" of "ring homomorphism".} for every $i\in\mathbb{N}$ such that
$\lambda^{0}\left(  x\right)  =1$ and $\lambda^{1}\left(  x\right)  =x$ for
every $x\in K$. Consider the ring $K\left[  \left[  t\right]  \right]  $ of
formal series in the indeterminate $t$ over the ring $K.$ Define a map
$\lambda_{t}:K\rightarrow K\left[  \left[  t\right]  \right]  $ by
$\lambda_{t}\left(  x\right)  =\sum\limits_{i\in\mathbb{N}}\lambda^{i}\left(
x\right)  t^{i}$. Note that the power series $\lambda_{t}\left(  x\right)
=\sum\limits_{i\in\mathbb{N}}\lambda^{i}\left(  x\right)  t^{i}$ has the
coefficient $\lambda^{0}\left(  x\right)  =1$ before $t^{0};$ thus, it is
invertible in $K\left[  \left[  t\right]  \right]  $.

\textbf{a)} Then, $\lambda_{t}\left(  x\right)  \cdot\lambda_{t}\left(
y\right)  =\lambda_{t}\left(  x+y\right)  $ for every $x\in K$ and every $y\in
K$ if and only if $\left(  K,\left(  \lambda_{i}\right)  _{i\in\mathbb{N}%
}\right)  $ is a $\lambda$-ring.

\textbf{b)} Let $\left(  K,\left(  \lambda_{i}\right)  _{i\in\mathbb{N}%
}\right)  $ be a $\lambda$-ring. Then, $\lambda_{t}\left(  0\right)  =1$,
$\lambda_{t}\left(  -x\right)  =\left(  \lambda_{t}\left(  x\right)  \right)
^{-1}$ (in particular, this means that $\lambda_{t}\left(  x\right)  $ is
invertible) and $\lambda_{t}\left(  x\right)  \cdot\lambda_{t}\left(
y\right)  =\lambda_{t}\left(  x+y\right)  $ for every $x\in K$ and every $y\in
K$.
\end{quote}

\textit{Proof of Theorem 2.1:} \textbf{a)} Every $x\in K$ and every $y\in K$
satisfy%
\[
\lambda_{t}\left(  x\right)  \cdot\lambda_{t}\left(  y\right)  =\sum
\limits_{i\in\mathbb{N}}\lambda^{i}\left(  x\right)  t^{i}\cdot\sum
\limits_{i\in\mathbb{N}}\lambda^{i}\left(  y\right)  t^{i}=\sum\limits_{i\in
\mathbb{N}}\sum\limits_{j\in\mathbb{N}}\lambda^{i}\left(  x\right)
\lambda^{j}\left(  y\right)  t^{i+j}=\sum_{k\in\mathbb{N}}\sum_{i=0}%
^{k}\lambda^{i}\left(  x\right)  \lambda^{k-i}\left(  y\right)  \cdot t^{k}%
\]
and%
\[
\lambda_{t}\left(  x+y\right)  =\sum_{i\in\mathbb{N}}\lambda^{i}\left(
x+y\right)  t^{i}=\sum_{k\in\mathbb{N}}\lambda^{k}\left(  x+y\right)  t^{k}.
\]
Hence, $\lambda_{t}\left(  x\right)  \cdot\lambda_{t}\left(  y\right)
=\lambda_{t}\left(  x+y\right)  $ is equivalent to $\sum\limits_{k\in
\mathbb{N}}\sum\limits_{i=0}^{k}\lambda^{i}\left(  x\right)  \lambda
^{k-i}\left(  y\right)  \cdot t^{k}=\sum\limits_{k\in\mathbb{N}}\lambda
^{k}\left(  x+y\right)  t^{k}$, which, in turn, means that $\sum
\limits_{i=0}^{k}\lambda^{i}\left(  x\right)  \lambda^{k-i}\left(  y\right)
=\lambda^{k}\left(  x+y\right)  $. Thus, $\lambda_{t}\left(  x\right)
\cdot\lambda_{t}\left(  y\right)  =\lambda_{t}\left(  x+y\right)  $ for every
$x\in K$ and every $y\in K$ if and only if $\left(  K,\left(  \lambda
_{i}\right)  _{i\in\mathbb{N}}\right)  $ is a $\lambda$-ring. This proves
Theorem 2.1 \textbf{a)}.

\textbf{b)} Part \textbf{a)} tells us that $\lambda_{t}\left(  x\right)
\cdot\lambda_{t}\left(  y\right)  =\lambda_{t}\left(  x+y\right)  $ for every
$x\in K$ and every $y\in K$. Applied to $x=y=0,$ this takes the form
$\lambda_{t}\left(  0\right)  \cdot\lambda_{t}\left(  0\right)  =\lambda
_{t}\left(  0\right)  ,$ what yields $\lambda_{t}\left(  0\right)  =1$ (since
$\lambda_{t}\left(  0\right)  $ is invertible in $K\left[  \left[  t\right]
\right]  $). Now, $\lambda_{t}\left(  x\right)  \cdot\lambda_{t}\left(
y\right)  =\lambda_{t}\left(  x+y\right)  $, applied to $y=-x$, yields
$\lambda_{t}\left(  x\right)  \cdot\lambda_{t}\left(  -x\right)
=\underbrace{\lambda_{t}\left(  0\right)  }_{=1}$, hence $\lambda_{t}\left(
-x\right)  =\left(  \lambda_{t}\left(  x\right)  \right)  ^{-1}$, and Theorem
2.1 \textbf{b)} is proven.

\begin{quote}
\textbf{Theorem 2.2.} Let $\left(  K,\left(  \lambda_{i}\right)
_{i\in\mathbb{N}}\right)  $ be a $\lambda$-ring with a positive structure
$\left(  \varepsilon,\mathbf{E}\right)  .$

\textbf{a)} Then, $\mathbf{L}=\left\{  u\in\mathbf{E}\ \mid\ \varepsilon
\left(  u\right)  =1\right\}  $ is a subgroup of the (multiplicative) unit
group $K^{\times}$ of $K$. [I don't believe this.]

\textbf{b)} We have $\mathbf{L}=\left\{  u\in\mathbf{E}\ \mid\ \lambda
_{t}\left(  u\right)  =1+ut\right\}  $.
\end{quote}

\textit{Proof of Theorem 2.2.} \textbf{b)} Every $u\in\mathbf{L}$ satisfies%
\[
\lambda_{t}\left(  u\right)  =\sum\limits_{i\in\mathbb{N}}\lambda^{i}\left(
u\right)  t^{i}=\underbrace{\lambda^{0}\left(  u\right)  }_{=1}+\underbrace
{\lambda^{1}\left(  u\right)  }_{=u}t+\sum\limits_{i\geq2}\underbrace
{\lambda^{i}\left(  u\right)  }_{\substack{=0,\text{ since}\\i>1=\varepsilon
\left(  u\right)  \\\text{and }u\in\mathbf{L}\subseteq\mathbf{E}}}t^{i}=1+ut.
\]
Conversely, if some $u\in\mathbf{E}$ satisfies $\lambda_{t}\left(  u\right)
=1+ut$, then%
\[
\sum\limits_{i\in\mathbb{N}}\lambda^{i}\left(  u\right)  t^{i}=\lambda
_{t}\left(  u\right)  =1+ut
\]
shows that $\lambda^{i}\left(  u\right)  =0$ for every $i>1$, and thus
$\varepsilon\left(  u\right)  \leq1$ (because $\lambda^{\varepsilon\left(
u\right)  }\left(  u\right)  $ is a unit in the ring $K$ (since $u\in
\mathbf{E}$), so that $\lambda^{\varepsilon\left(  u\right)  }\left(
u\right)  \neq0$ and thus $\varepsilon\left(  u\right)  \leq1$). Together with
$\varepsilon\left(  u\right)  >0$, this yields $\varepsilon\left(  u\right)
=1$ and thus $u\in\mathbf{L}$. This proves \textbf{b)}.

\textbf{a)} First we need to show that every $u\in\mathbf{L}$ is invertible in
$K$. In fact, let $u\in\mathbf{L}$. Then, $\lambda^{\varepsilon\left(
u\right)  }\left(  u\right)  $ is a unit in the ring $K$ (since $u\in
\mathbf{E}$). But $\varepsilon\left(  u\right)  =1$ and thus $\lambda
^{\varepsilon\left(  u\right)  }\left(  u\right)  =\lambda^{1}\left(
u\right)  =u$. Thus, $u$ is a unit in $K$; that is, $u$ is invertible. Its
inverse $u^{-1}$ must lie in $\mathbf{E}$ as well

[1] [Why the fuck does it?]

\begin{center}
\fbox{\textbf{3. Basic examples}}
\end{center}

[2] [...] [Examples]

\begin{center}
\fbox{\textbf{4. Intermezzo: Symmetric polynomials}}
\end{center}

Our next plan is to introduce a rather general example of $\lambda$-rings that
we will use as a prototype to the notion of \textit{special }$\lambda
$\textit{-rings}. Before we do this, we need some rather clumsy theory of
symmetric polynomials. It's not a bad idea to skip this paragraph until its
results are actually needed in paragraph 5.

\begin{quote}
\textbf{Theorem 4.1 (characterization of symmetric polynomials).} Let $K$ be a
ring. Let $m\in\mathbb{N}$. Consider the ring $K\left[  U_{1},U_{2}%
,...,U_{m}\right]  $ (the polynomial ring in $m$ indeterminates $U_{1},$
$U_{2},$ $...,$ $U_{m}$ over the ring $K$). For every $i\in\mathbb{N}$, let
$X_{i}=\sum\limits_{\substack{S\subseteq\left\{  1,2,...,m\right\}
;\\\left\vert S\right\vert =i}}\prod\limits_{k\in S}U_{i}$ be the so-called
$i$\textit{-th elementary symmetric polynomial} in the variables $U_{1},$
$U_{2},$ $...,$ $U_{m}$. (In particular, $I_{0}=1$ and $I_{i}=0$ for every
$i>m$.)

A polynomial $P\in K\left[  U_{1},U_{2},...,U_{m}\right]  $ is called
\textit{symmetric} if it satisfies $P\left(  U_{1},U_{2},...,U_{m}\right)
=P\left(  U_{\pi\left(  1\right)  },U_{\pi\left(  2\right)  },...,U_{\pi
\left(  m\right)  }\right)  $ for every permutation $\pi$ of the set $\left\{
1,2,...,m\right\}  $.

\textbf{(a)} Let $P\in K\left[  U_{1},U_{2},...,U_{m}\right]  $ be a symmetric
polynomial. Then, there exists one and only one polynomial $Q\in
\underbrace{K\left[  \alpha_{1},\alpha_{2},...,\alpha_{m}\right]
}_{\text{polynomial ring}}$ such that $P\left(  U_{1},U_{2},...,U_{m}\right)
=Q\left(  X_{1},X_{2},...,X_{m}\right)  $.\ \ \ \ \footnote{In other words,
the subring%
\[
\left\{  P\in K\left[  U_{1},U_{2},...,U_{m}\right]  \ \mid\ P\text{ is
symmetric}\right\}
\]
of the polynomial ring $K\left[  U_{1},U_{2},...,U_{m}\right]  $ is generated
by the elements $X_{1},$ $X_{2},$ $...,$ $X_{m}$. Moreover, these elements
$X_{1},$ $X_{2},$ $...,$ $X_{m}$ are algebraically independent, so that the
map%
\[
\underbrace{K\left[  \alpha_{1},\alpha_{2},...,\alpha_{m}\right]
}_{\text{polynomial ring}}\rightarrow\left\{  P\in K\left[  U_{1}%
,U_{2},...,U_{m}\right]  \ \mid\ P\text{ is symmetric}\right\}
\]
which maps every $\alpha_{i}$ to $U_{i}$ is a bijection.}

\textbf{(b)} Let $\ell\in\mathbb{N}$. Assume, moreover, that $P\in K\left[
U_{1},U_{2},...,U_{m}\right]  $ is a symmetric polynomial of total degree
$\leq\ell$ in the variables $U_{1},$ $U_{2},$ $...,$ $U_{m}$. Then, the
variables $\alpha_{i}$ for $i>\ell$ do not appear in the polynomial $Q$.

There is a canonical homomorphism $K\left[  \alpha_{1},\alpha_{2}%
,...,\alpha_{m}\right]  \rightarrow K\left[  \alpha_{1},\alpha_{2}%
,...,\alpha_{\ell}\right]  $ (which is a surjection if $\ell\leq m$ and an
injection if $\ell\geq m$). If we denote by $Q_{\ell}$ the image of $Q\in
K\left[  \alpha_{1},\alpha_{2},...,\alpha_{m}\right]  $ under this
homomorphism, then, $P\left(  U_{1},U_{2},...,U_{m}\right)  =Q\left(
X_{1},X_{2},...,X_{m}\right)  =Q_{\ell}\left(  X_{1},X_{2},...,X_{\ell
}\right)  $.
\end{quote}

We are not going to prove Theorem 4.1 here, since pretty much every good
algebra book does it.\footnote{Only one remark about Theorem 4.1 \textbf{(b)}:
We have $Q\left(  X_{1},X_{2},...,X_{m}\right)  =Q_{\ell}\left(  X_{1}%
,X_{2},...,X_{\ell}\right)  $, because the variables $\alpha_{i}$ for $i>\ell$
do not appear in the polynomial $Q$.} But we are going to extend it to two
sets of indeterminates:

\begin{quote}
\textbf{Theorem 4.2 (characterization of UV-symmetric polynomials).} Let $K$
be a ring. Let $m\in\mathbb{N}$ and $n\in\mathbb{N}$. Consider the ring
$K\left[  U_{1},U_{2},...,U_{m},V_{1},V_{2},...,V_{n}\right]  $ (the
polynomial ring in $m+n$ indeterminates $U_{1},$ $U_{2},$ $...,$ $U_{m},$
$V_{1},$ $V_{2},$ $...,$ $V_{n}$ over the ring $K$). For every $i\in
\mathbb{N}$, let $X_{i}=\sum\limits_{\substack{S\subseteq\left\{
1,2,...,m\right\}  ;\\\left\vert S\right\vert =i}}\prod\limits_{k\in S}U_{i}$
be the so-called $i$\textit{-th elementary symmetric polynomial} in the
variables $U_{1},$ $U_{2},$ $...,$ $U_{m}$. For every $j\in\mathbb{N}$, let
$Y_{j}=\sum\limits_{\substack{S\subseteq\left\{  1,2,...,n\right\}
;\\\left\vert S\right\vert =j}}\prod\limits_{k\in S}V_{j}$ be the $j$-th
elementary symmetric polynomial in the variables $V_{1},$ $V_{2},$ $...,$
$V_{n}$.

A polynomial $P\in K\left[  U_{1},U_{2},...,U_{m},V_{1},V_{2},...,V_{n}%
\right]  $ is called \textit{UV-symmetric} if it satisfies%
\[
P\left(  U_{1},U_{2},...,U_{m},V_{1},V_{2},...,V_{n}\right)  =P\left(
U_{\pi\left(  1\right)  },U_{\pi\left(  2\right)  },...,U_{\pi\left(
m\right)  },V_{\sigma\left(  1\right)  },V_{\sigma\left(  2\right)
},...,V_{\sigma\left(  n\right)  }\right)
\]
for every permutation $\pi$ of the set $\left\{  1,2,...,m\right\}  $ and
every permutation $\sigma$ of the set $\left\{  1,2,...,n\right\}  $.

\textbf{(a)} Let $P\in K\left[  U_{1},U_{2},...,U_{m},V_{1},V_{2}%
,...,V_{n}\right]  $ be a UV-symmetric polynomial. Then, there exists one and
only one polynomial $Q\in K\left[  \alpha_{1},\alpha_{2},...,\alpha_{m}%
,\beta_{1},\beta_{2},...,\beta_{n}\right]  $ such that $P\left(  U_{1}%
,U_{2},...,U_{m},V_{1},V_{2},...,V_{n}\right)  =Q\left(  X_{1},X_{2}%
,...,X_{m},Y_{1},Y_{2},...,Y_{m}\right)  $.\ \ \ \ \footnote{In other words,
the subring%
\[
\left\{  P\in K\left[  U_{1},U_{2},...,U_{m},V_{1},V_{2},...,V_{n}\right]
\ \mid\ P\text{ is UV-symmetric}\right\}
\]
of the polynomial ring $K\left[  U_{1},U_{2},...,U_{m},V_{1},V_{2}%
,...,V_{n}\right]  $ is generated by the elements $X_{1},$ $X_{2},$ $...,$
$X_{m},$ $Y_{1},$ $Y_{2},$ $...,$ $Y_{n}$. Moreover, these elements $X_{1},$
$X_{2},$ $...,$ $X_{m},$ $Y_{1},$ $Y_{2},$ $...,$ $Y_{n}$ are algebraically
independent, so that the map%
\[
\underbrace{K\left[  \alpha_{1},\alpha_{2},...,\alpha_{m},\beta_{1},\beta
_{2},...,\beta_{n}\right]  }_{\text{polynomial ring}}\rightarrow\left\{  P\in
K\left[  U_{1},U_{2},...,U_{m},V_{1},V_{2},...,V_{n}\right]  \ \mid\ P\text{
is UV-symmetric}\right\}
\]
which maps every $\alpha_{i}$ to $X_{i}$ and every $\beta_{j}$ to $Y_{j}$ is a
bijection.}

\textbf{(b)} Let $\ell\in\mathbb{N}$ and $k\in\mathbb{N}$. Assume, moreover,
that $P\in K\left[  U_{1},U_{2},...,U_{m},V_{1},V_{2},...,V_{n}\right]  $ is a
UV-symmetric polynomial of total degree $\leq\ell$ in the variables $U_{1},$
$U_{2},$ $...,$ $U_{m}$ and of total degree $\leq k$ in the variables $V_{1},$
$V_{2},$ $...,$ $V_{n}.$ Then, neither the variables $\alpha_{i}$ for $i>\ell$
nor the variables $\beta_{j}$ for $j>k$ ever appear in the polynomial $Q$.

There is a canonical homomorphism $K\left[  \alpha_{1},\alpha_{2}%
,...,\alpha_{m},\beta_{1},\beta_{2},...,\beta_{n}\right]  \rightarrow K\left[
\alpha_{1},\alpha_{2},...,\alpha_{\ell},\beta_{1},\beta_{2},...,\beta
_{k}\right]  $ (which is composed of the canonical homomorphisms $K\left[
\alpha_{1},\alpha_{2},...,\alpha_{m}\right]  \rightarrow K\left[  \alpha
_{1},\alpha_{2},...,\alpha_{\ell}\right]  $ and $K\left[  \beta_{1},\beta
_{2},...,\beta_{n}\right]  \rightarrow K\left[  \beta_{1},\beta_{2}%
,...,\beta_{k}\right]  $).

If we denote by $Q_{\ell,k}$ the image of $Q\in K\left[  \alpha_{1},\alpha
_{2},...,\alpha_{m},\beta_{1},\beta_{2},...,\beta_{n}\right]  $ under this
homomorphism, then, $P\left(  U_{1},U_{2},...,U_{m},V_{1},V_{2},...,V_{n}%
\right)  =Q_{\ell,k}\left(  X_{1},X_{2},...,X_{\ell},Y_{1},Y_{2}%
,...,Y_{k}\right)  $.
\end{quote}

\textit{Proof of Theorem 4.2 (\textbf{very} roughly sketched):} \textbf{(a)}
Consider $P$ as a polynomial in the indeterminates $V_{1},$ $V_{2},$ $...,$
$V_{n}$ over the ring $K\left[  U_{1},U_{2},...,U_{m}\right]  $. Then, $P$ is
a symmetric polynomial in these indeterminates $V_{1},$ $V_{2},$ $...,$
$V_{n}$, so Theorem 4.1 \textbf{(a)} yields the existence of one and only one
polynomial $\widehat{Q}\in\underbrace{\left(  K\left[  U_{1},U_{2}%
,...,U_{m}\right]  \right)  \left[  \beta_{1},\beta_{2},...,\beta_{n}\right]
}_{\text{polynomial ring}}$ such that $P\left(  U_{1},U_{2},...,U_{m}%
,V_{1},V_{2},...,V_{n}\right)  =\widehat{Q}\left(  Y_{1},Y_{2},...,Y_{n}%
\right)  $. Now, for every $n$-tuple $\left(  \lambda_{1},\lambda
_{2},...,\lambda_{n}\right)  \in\mathbb{N}^{n}$, the coefficient of this
polynomial $\widehat{Q}$ before $\beta_{1}^{\lambda_{1}}\beta_{2}^{\lambda
_{2}}...\beta_{n}^{\lambda_{n}}$ is a symmetric polynomial in the variables
$U_{1},$ $U_{2},$ $...,$ $U_{m}\ \ \ \ $\footnote{Okay, the "symmetric" part
may need an explanation.
\par
{}In fact, let $\sigma$ be a permutation of the set $\left\{
1,2,...,m\right\}  .$
\par
Let $\rho$ be the canonical $K$-algebra isomorphism $\left(  K\left[
U_{1},U_{2},...,U_{m}\right]  \right)  \left[  \beta_{1},\beta_{2}%
,...,\beta_{n}\right]  \rightarrow\left(  K\left[  \beta_{1},\beta
_{2},...,\beta_{n}\right]  \right)  \left[  U_{1},U_{2},...,U_{m}\right]  $.
\par
Let $\widehat{Q}_{\sigma}\in\left(  K\left[  U_{1},U_{2},...,U_{m}\right]
\right)  \left[  \beta_{1},\beta_{2},...,\beta_{n}\right]  $ be the polynomial
defined by $\widehat{Q}_{\sigma}=\rho^{-1}\left(  \left(  \rho\left(
\widehat{Q}\right)  \right)  \left(  U_{\sigma\left(  1\right)  }%
,U_{\sigma\left(  2\right)  },...,U_{\sigma\left(  m\right)  }\right)
\right)  .$ Then, $P\left(  U_{1},U_{2},...,U_{m},V_{1},V_{2},...,V_{n}%
\right)  =\widehat{Q}\left(  Y_{1},Y_{2},...,Y_{n}\right)  $ yields $P\left(
U_{\sigma\left(  1\right)  },U_{\sigma\left(  2\right)  },...,U_{\sigma\left(
m\right)  },V_{1},V_{2},...,V_{n}\right)  =\widehat{Q}_{\sigma}\left(
Y_{1},Y_{2},...,Y_{n}\right)  $. But $P\left(  U_{1},U_{2},...,U_{m}%
,V_{1},V_{2},...,V_{n}\right)  =P\left(  U_{\sigma\left(  1\right)
},U_{\sigma\left(  2\right)  },...,U_{\sigma\left(  m\right)  },V_{1}%
,V_{2},...,V_{n}\right)  $ (since $P$ is UV-symmetric). Thus, $\widehat
{Q}\left(  Y_{1},Y_{2},...,Y_{n}\right)  =\widehat{Q}_{\sigma}\left(
Y_{1},Y_{2},...,Y_{n}\right)  $ for every permutation $\sigma$ of $\left\{
1,2,...,m\right\}  $. This yields $\widehat{Q}=\widehat{Q}_{\sigma}$ for every
permutation $\sigma$ of $\left\{  1,2,...,m\right\}  $ (since $Y_{1},$
$Y_{2},$ $...,$ $Y_{n}$ are algebraically independent over $K\left[
U_{1},U_{2},...,U_{m}\right]  $, as we can see by applying Theorem 4.1 to
$K\left[  U_{1},U_{2},...,U_{m}\right]  $ instead of $K$ and to $V_{1},$
$V_{2},$ $...,$ $V_{n}$ instead of $U_{1},$ $U_{2},$ $...,$ $U_{m}$). This
means that the coefficient of this polynomial $\widehat{Q}$ before $\beta
_{1}^{\lambda_{1}}\beta_{2}^{\lambda_{2}}...\beta_{n}^{\lambda_{n}}$ is a
symmetric polynomial in the variables $U_{1},$ $U_{2},$ $...,$ $U_{m}$ for
every $n$-tuple $\left(  \lambda_{1},\lambda_{2},...,\lambda_{n}\right)
\in\mathbb{N}^{n}$.}. Hence, by Theorem 4.1 \textbf{(a)}, there exists a
polynomial $R_{\left(  \lambda_{1},\lambda_{2},...,\lambda_{n}\right)  }%
\in\underbrace{K\left[  \alpha_{1},\alpha_{2},...,\alpha_{m}\right]
}_{\text{polynomial ring}}$ such that the coefficient of the polynomial
$\widehat{Q}$ before $\beta_{1}^{\lambda_{1}}\beta_{2}^{\lambda_{2}}%
...\beta_{n}^{\lambda_{n}}$ is $R_{\left(  \lambda_{1},\lambda_{2}%
,...,\lambda_{n}\right)  }\left(  X_{1},X_{2},...,X_{m}\right)  $. Thus,%
\begin{align*}
&  P\left(  U_{1},U_{2},...,U_{m},V_{1},V_{2},...,V_{n}\right)  =\widehat
{Q}\left(  Y_{1},Y_{2},...,Y_{n}\right) \\
&  =\sum_{\left(  \lambda_{1},\lambda_{2},...,\lambda_{n}\right)
\in\mathbb{N}^{n}}\left(  \text{the coefficient of the polynomial }\widehat
{Q}\text{ before }\beta_{1}^{\lambda_{1}}\beta_{2}^{\lambda_{2}}...\beta
_{n}^{\lambda_{n}}\right)  Y_{1}^{\lambda_{1}}Y_{2}^{\lambda_{2}}%
...Y_{n}^{\lambda_{n}}\\
&  =\sum_{\left(  \lambda_{1},\lambda_{2},...,\lambda_{n}\right)
\in\mathbb{N}^{n}}R_{\left(  \lambda_{1},\lambda_{2},...,\lambda_{n}\right)
}\left(  X_{1},X_{2},...,X_{m}\right)  Y_{1}^{\lambda_{1}}Y_{2}^{\lambda_{2}%
}...Y_{n}^{\lambda_{n}}.
\end{align*}
Thus, the polynomial $Q\in K\left[  \alpha_{1},\alpha_{2},...,\alpha_{m}%
,\beta_{1},\beta_{2},...,\beta_{n}\right]  $ defined by%
\[
Q=\sum_{\left(  \lambda_{1},\lambda_{2},...,\lambda_{n}\right)  \in
\mathbb{N}^{n}}R_{\left(  \lambda_{1},\lambda_{2},...,\lambda_{n}\right)
}\left(  \alpha_{1},\alpha_{2},...,\alpha_{m}\right)  \beta_{1}^{\lambda_{1}%
}\beta_{2}^{\lambda_{2}}...\beta_{n}^{\lambda_{n}}%
\]
satisfies $P\left(  U_{1},U_{2},...,U_{m},V_{1},V_{2},...,V_{n}\right)
=Q\left(  X_{1},X_{2},...,X_{m},Y_{1},Y_{2},...,Y_{m}\right)  $. It only
remains to prove that this is the only such polynomial. This amounts to
showing that $X_{1},$ $X_{2},$ $...,$ $X_{m},$ $Y_{1},$ $Y_{2},$ $...,$
$Y_{n}$ are algebraically independent over $K$. But this is clear since the
variables $U_{1},$ $U_{2},$ $...,$ $U_{m}$ are algebraically independent over
$K$ (by Theorem 4.1) and the variables $Y_{1},$ $Y_{2},$ $...,$ $Y_{n}$ are
algebraically independent over $K\left[  X_{1},X_{2},...,X_{m}\right]  $ (by
Theorem 4.1, applied to $K\left[  X_{1},X_{2},...,X_{m}\right]  $ instead of
$K$ and to $V_{1},$ $V_{2},$ $...,$ $V_{n}$ instead of $U_{1},$ $U_{2},$
$...,$ $U_{m}$). [How does this implication work? Just write it down and see.]

For part \textbf{(b)}, we argue the same way as in \textbf{(a)}, but applying
Theorem 4.1 \textbf{(b)} along with Theorem 4.1 \textbf{(a)}.

Theorem 4.2 allows us to make the following definition:

\begin{quote}
\textbf{Definition.} Let $K$ be a ring.

Let $k\in\mathbb{N}$. Our goal now is to define a polynomial $P_{k}\in
K\left[  \alpha_{1},\alpha_{2},...,\alpha_{k},\beta_{1},\beta_{2}%
,...,\beta_{k}\right]  $ such that%
\begin{equation}
\sum_{\substack{S\subseteq\left\{  1,2,...,m\right\}  \times\left\{
1,2,...,n\right\}  ;\\\left\vert S\right\vert =k}}\prod_{\left(  i,j\right)
\in S}U_{i}V_{j}=P_{k}\left(  X_{1},X_{2},...,X_{k},Y_{1},Y_{2},...,Y_{k}%
\right)  \label{Pk1}%
\end{equation}
in the polynomial ring $K\left[  U_{1},U_{2},...,U_{m},V_{1},V_{2}%
,...,V_{n}\right]  $ for every $n\in\mathbb{N}$ and $m\in\mathbb{N}$, where
$X_{i}=\sum\limits_{\substack{S\subseteq\left\{  1,2,...,m\right\}
;\\\left\vert S\right\vert =i}}\prod\limits_{k\in S}U_{i}$ is the $i$-th
elementary symmetric polynomial in the variables $U_{1},$ $U_{2},$ $...,$
$U_{m}$ for every $i\in\mathbb{N}$, and $Y_{j}=\sum
\limits_{\substack{S\subseteq\left\{  1,2,...,n\right\}  ;\\\left\vert
S\right\vert =j}}\prod\limits_{k\in S}V_{j}$ is the $j$-th elementary
symmetric polynomial in the variables $V_{1},$ $V_{2},$ $...,$ $V_{n}$ for
every $j\in\mathbb{N}$.

In order to do this, we first fix some $n\in\mathbb{N}$ and $m\in\mathbb{N}$.
The polynomial%
\[
\sum_{\substack{S\subseteq\left\{  1,2,...,m\right\}  \times\left\{
1,2,...,n\right\}  ;\\\left\vert S\right\vert =k}}\prod_{\left(  i,j\right)
\in S}U_{i}V_{j}\in K\left[  U_{1},U_{2},...,U_{m},V_{1},V_{2},...,V_{n}%
\right]
\]
is UV-symmetric. Thus, Theorem 4.2 \textbf{(a)} yields that there exists one
and only one polynomial $Q\in K\left[  \alpha_{1},\alpha_{2},...,\alpha
_{m},\beta_{1},\beta_{2},...,\beta_{n}\right]  $ such that%
\[
\sum_{\substack{S\subseteq\left\{  1,2,...,m\right\}  \times\left\{
1,2,...,n\right\}  ;\\\left\vert S\right\vert =k}}\prod_{\left(  i,j\right)
\in S}U_{i}V_{j}=Q\left(  X_{1},X_{2},...,X_{m},Y_{1},Y_{2},...,Y_{m}\right)
.
\]
Since the polynomial $\sum\limits_{\substack{S\subseteq\left\{
1,2,...,m\right\}  \times\left\{  1,2,...,n\right\}  ;\\\left\vert
S\right\vert =k}}\prod_{\left(  i,j\right)  \in S}U_{i}V_{j}$ has total degree
$\leq k$ in the variables $U_{1},$ $U_{2},$ $...,$ $U_{m}$ and of total degree
$\leq k$ in the variables $V_{1},$ $V_{2},$ $...,$ $V_{n}$, Theorem 4.2
\textbf{(b)} yields that%
\[
\sum_{\substack{S\subseteq\left\{  1,2,...,m\right\}  \times\left\{
1,2,...,n\right\}  ;\\\left\vert S\right\vert =k}}\prod_{\left(  i,j\right)
\in S}U_{i}V_{j}=Q_{k,k}\left(  X_{1},X_{2},...,X_{k},Y_{1},Y_{2}%
,...,Y_{k}\right)  ,
\]
where $Q_{k,k}$ is the image of the polynomial $Q$ under the canonical
homomorphism $K\left[  \alpha_{1},\alpha_{2},...,\alpha_{m},\beta_{1}%
,\beta_{2},...,\beta_{n}\right]  \rightarrow K\left[  \alpha_{1},\alpha
_{2},...,\alpha_{k},\beta_{1},\beta_{2},...,\beta_{k}\right]  $. However, this
polynomial $Q_{k,k}$ is not independent of $n$ and $m$ yet (as the polynomial
$P_{k}$ that we intend to construct should be), so we call it $Q_{k,k,\left[
n,m\right]  }$ rather than just $Q_{k,k}$.

Now we forget that we fixed $n\in\mathbb{N}$ and $m\in\mathbb{N}$. We have
learnt that%
\[
\sum_{\substack{S\subseteq\left\{  1,2,...,m\right\}  \times\left\{
1,2,...,n\right\}  ;\\\left\vert S\right\vert =k}}\prod_{\left(  i,j\right)
\in S}U_{i}V_{j}=Q_{k,k,\left[  n,m\right]  }\left(  X_{1},X_{2}%
,...,X_{k},Y_{1},Y_{2},...,Y_{k}\right)
\]
in the polynomial ring $K\left[  U_{1},U_{2},...,U_{m},V_{1},V_{2}%
,...,V_{n}\right]  $ for every $n\in\mathbb{N}$ and $m\in\mathbb{N}$. Now,
define a polynomial $P_{k}\in K\left[  \alpha_{1},\alpha_{2},...,\alpha
_{k},\beta_{1},\beta_{2},...,\beta_{k}\right]  $ by $P_{k}=Q_{k,k,\left[
k,k\right]  }.$

\textbf{Theorem 4.3.} \textbf{(a)} The polynomial $P_{k}$ just defined
satisfies the equation (\ref{Pk1}) in the polynomial ring $K\left[
U_{1},U_{2},...,U_{m},V_{1},V_{2},...,V_{n}\right]  $ for every $n\in
\mathbb{N}$ and $m\in\mathbb{N}$. (Hence, the goal mentioned above in the
definition is actually achieved.)

\textbf{(b)} For every $n\in\mathbb{N}$ and $m\in\mathbb{N}$, we have%
\begin{equation}
\prod_{\left(  i,j\right)  \in\left\{  1,2,...,m\right\}  \times\left\{
1,2,...,n\right\}  }\left(  1+U_{i}V_{j}T\right)  =\sum_{k\in\mathbb{N}}%
P_{k}\left(  X_{1},X_{2},...,X_{k},Y_{1},Y_{2},...,Y_{k}\right)  T^{k}
\label{Pk2}%
\end{equation}
in the ring $\left(  K\left[  U_{1},U_{2},...,U_{m},V_{1},V_{2},...,V_{n}%
\right]  \right)  \left[  \left[  T\right]  \right]  $. (Note that the right
hand side of this equation is a power series with coefficient $1$ before
$T^{0}$, since $P_{0}=1$.)
\end{quote}

\textit{Proof of Theorem 4.3:} \textbf{(a)} \textit{1st Step:} Fix
$n\in\mathbb{N}$ and $m\in\mathbb{N}$ such that $n\geq k$ and $m\geq k$. Then,
we claim that $Q_{k,k,\left[  n,m\right]  }=P_{k}$.

\textit{Proof.} The definition of $Q_{k,k,\left[  n,m\right]  }$ yields
\[
\sum_{\substack{S\subseteq\left\{  1,2,...,m\right\}  \times\left\{
1,2,...,n\right\}  ;\\\left\vert S\right\vert =k}}\prod_{\left(  i,j\right)
\in S}U_{i}V_{j}=Q_{k,k,\left[  n,m\right]  }\left(  X_{1},X_{2}%
,...,X_{k},Y_{1},Y_{2},...,Y_{k}\right)
\]
in the polynomial ring $K\left[  U_{1},U_{2},...,U_{m},V_{1},V_{2}%
,...,V_{n}\right]  $. Applying the canonical ring epimorphism $K\left[
U_{1},U_{2},...,U_{m},V_{1},V_{2},...,V_{n}\right]  \rightarrow K\left[
U_{1},U_{2},...,U_{k},V_{1},V_{2},...,V_{k}\right]  $ (which maps every
$U_{i}$ to $\left\{
\begin{array}
[c]{c}%
U_{i},\text{ if }i\leq k;\\
0,\text{ if }i>k
\end{array}
\right.  $ and every $V_{j}$ to $\left\{
\begin{array}
[c]{c}%
V_{j},\text{ if }j\leq k;\\
0,\text{ if }j>k
\end{array}
\right.  $) to this equation (and noticing that this epimorphism maps every
$X_{i}$ to the corresponding $X_{i}$ of the image ring and every $Y_{j}$ to
the corresponding $Y_{j}$ of the image ring!), we obtain%
\[
\sum_{\substack{S\subseteq\left\{  1,2,...,k\right\}  \times\left\{
1,2,...,k\right\}  ;\\\left\vert S\right\vert =k}}\prod_{\left(  i,j\right)
\in S}U_{i}V_{j}=Q_{k,k,\left[  n,m\right]  }\left(  X_{1},X_{2}%
,...,X_{k},Y_{1},Y_{2},...,Y_{k}\right)
\]
in the polynomial ring $K\left[  U_{1},U_{2},...,U_{k},V_{1},V_{2}%
,...,V_{k}\right]  $. On the other hand, the definition of $Q_{k,k,\left[
k,k\right]  }$ yields%
\[
\sum_{\substack{S\subseteq\left\{  1,2,...,k\right\}  \times\left\{
1,2,...,k\right\}  ;\\\left\vert S\right\vert =k}}\prod_{\left(  i,j\right)
\in S}U_{i}V_{j}=Q_{k,k,\left[  k,k\right]  }\left(  X_{1},X_{2}%
,...,X_{k},Y_{1},Y_{2},...,Y_{k}\right)
\]
in the same ring. These two equations yield%
\[
Q_{k,k,\left[  n,m\right]  }\left(  X_{1},X_{2},...,X_{k},Y_{1},Y_{2}%
,...,Y_{k}\right)  =Q_{k,k,\left[  k,k\right]  }\left(  X_{1},X_{2}%
,...,X_{k},Y_{1},Y_{2},...,Y_{k}\right)  .
\]
Since the elements $X_{1},$ $X_{2},$ $...,$ $X_{k},$ $Y_{1},$ $Y_{2},$ $...,$
$Y_{k}$ of $K\left[  U_{1},U_{2},...,U_{k},V_{1},V_{2},...,V_{k}\right]  $ are
algebraically independent (by Theorem 4.2 \textbf{(a)}), this yields
$Q_{k,k,\left[  n,m\right]  }=Q_{k,k,\left[  k,k\right]  }.$ In other words,
$Q_{k,k,\left[  n,m\right]  }=P_{k},$ and the 1st Step is proven.

\textit{2nd Step:} For every $n\in\mathbb{N}$ and $m\in\mathbb{N}$, the
equation (\ref{Pk1}) is satisfied in the polynomial ring $K\left[  U_{1}%
,U_{2},...,U_{m},V_{1},V_{2},...,V_{n}\right]  $.

\textit{Proof.} Let $n^{\prime}\in\mathbb{N}$ be such that $n^{\prime}\geq n$
and $n^{\prime}\geq k$. Let $m^{\prime}\in\mathbb{N}$ be such that $m^{\prime
}\geq m$ and $m^{\prime}\geq k$. Then, the 1st Step (applied to $n^{\prime}$
and $m^{\prime}$ instead of $n$ and $m$) yields that $Q_{k,k,\left[
n^{\prime},m^{\prime}\right]  }=P_{k}.$

The definition of $Q_{k,k,\left[  n^{\prime},m^{\prime}\right]  }$ yields
\[
\sum_{\substack{S\subseteq\left\{  1,2,...,m^{\prime}\right\}  \times\left\{
1,2,...,n^{\prime}\right\}  ;\\\left\vert S\right\vert =k}}\prod_{\left(
i,j\right)  \in S}U_{i}V_{j}=Q_{k,k,\left[  n^{\prime},m^{\prime}\right]
}\left(  X_{1},X_{2},...,X_{k},Y_{1},Y_{2},...,Y_{k}\right)
\]
in the polynomial ring $K\left[  U_{1},U_{2},...,U_{m^{\prime}},V_{1}%
,V_{2},...,V_{n^{\prime}}\right]  $. Applying the canonical ring epimorphism
$K\left[  U_{1},U_{2},...,U_{m^{\prime}},V_{1},V_{2},...,V_{n^{\prime}%
}\right]  \rightarrow K\left[  U_{1},U_{2},...,U_{m},V_{1},V_{2}%
,...,V_{n}\right]  $ (which maps every $U_{i}$ to $\left\{
\begin{array}
[c]{c}%
U_{i},\text{ if }i\leq m;\\
0,\text{ if }i>m
\end{array}
\right.  $ and every $V_{j}$ to $\left\{
\begin{array}
[c]{c}%
V_{j},\text{ if }j\leq n;\\
0,\text{ if }j>n
\end{array}
\right.  $) to this equation (and noticing that this epimorphism maps every
$X_{i}$ to the corresponding $X_{i}$ of the image ring and every $Y_{j}$ to
the corresponding $Y_{j}$ of the image ring!), we obtain%
\[
\sum_{\substack{S\subseteq\left\{  1,2,...,m\right\}  \times\left\{
1,2,...,n\right\}  ;\\\left\vert S\right\vert =k}}\prod_{\left(  i,j\right)
\in S}U_{i}V_{j}=\underbrace{Q_{k,k,\left[  n^{\prime},m^{\prime}\right]  }%
}_{=P_{k}}\left(  X_{1},X_{2},...,X_{k},Y_{1},Y_{2},...,Y_{k}\right)
\]
in the polynomial ring $K\left[  U_{1},U_{2},...,U_{m},V_{1},V_{2}%
,...,V_{n}\right]  $. Hence, the equation (\ref{Pk1}) is satisfied in the
polynomial ring $K\left[  U_{1},U_{2},...,U_{m},V_{1},V_{2},...,V_{n}\right]
.$ This completes the 2nd Step and proves Theorem 4.3 \textbf{(a)}.

\textbf{(b)} Immediately follows from part \textbf{(a)}.

Just as our above definition of the polynomials $P_{k}$ and Theorem 4.3 based
upon Theorem 4.2, we can make another definition basing upon Theorem 4.1:

\begin{quote}
\textbf{Definition.} Let $K$ be a ring.

For every set $H$ and every $j\in\mathbb{N}$, we denote by $\mathcal{P}%
_{j}\left(  H\right)  $ the set of all $j$-element subsets of $H.$ (This is
also often denoted as $\dbinom{H}{j}$.)

Let $j\in\mathbb{N}$. Let $k\in\mathbb{N}$. Our goal now is to define a
polynomial $P_{k,j}\in K\left[  \alpha_{1},\alpha_{2},...,\alpha_{kj}\right]
$ such that%
\begin{equation}
\sum_{\substack{S\subseteq\mathcal{P}_{j}\left(  \left\{  1,2,...,m\right\}
\right)  ;\\\left\vert S\right\vert =k}}\prod_{I\in S}\prod_{i\in I}%
U_{i}=P_{k,j}\left(  X_{1},X_{2},...,X_{kj}\right)  \label{Pkj1}%
\end{equation}
in the polynomial ring $K\left[  U_{1},U_{2},...,U_{m}\right]  $ for every
$m\in\mathbb{N}$, where $X_{i}=\sum\limits_{\substack{S\subseteq\left\{
1,2,...,m\right\}  ;\\\left\vert S\right\vert =i}}\prod\limits_{k\in S}U_{i}$
is the $i$-th elementary symmetric polynomial in the variables $U_{1},$
$U_{2},$ $...,$ $U_{m}$ for every $i\in\mathbb{N}$.

In order to do this, we first fix some $m\in\mathbb{N}$. The polynomial%
\[
\sum_{\substack{S\subseteq\mathcal{P}_{j}\left(  \left\{  1,2,...,m\right\}
\right)  ;\\\left\vert S\right\vert =k}}\prod_{I\in S}\prod_{i\in I}U_{i}\in
K\left[  U_{1},U_{2},...,U_{m}\right]
\]
is symmetric. Thus, Theorem 4.1 \textbf{(a)} yields that there exists one and
only one polynomial $Q\in K\left[  \alpha_{1},\alpha_{2},...,\alpha
_{m}\right]  $ such that%
\[
\sum_{\substack{S\subseteq\mathcal{P}_{j}\left(  \left\{  1,2,...,m\right\}
\right)  ;\\\left\vert S\right\vert =k}}\prod_{I\in S}\prod_{i\in I}%
U_{i}=Q\left(  X_{1},X_{2},...,X_{m}\right)  .
\]
Since the polynomial $\sum\limits_{\substack{S\subseteq\mathcal{P}_{j}\left(
\left\{  1,2,...,m\right\}  \right)  ;\\\left\vert S\right\vert =k}%
}\prod\limits_{I\in S}\prod\limits_{i\in I}U_{i}$ has total degree $\leq kj$
in the variables $U_{1},$ $U_{2},$ $...,$ $U_{m}$, Theorem 4.1 \textbf{(b)}
yields that%
\[
\sum_{\substack{S\subseteq\mathcal{P}_{j}\left(  \left\{  1,2,...,m\right\}
\right)  ;\\\left\vert S\right\vert =k}}\prod_{I\in S}\prod_{i\in I}%
U_{i}=Q_{k,j}\left(  X_{1},X_{2},...,X_{kj}\right)  ,
\]
where $Q_{k,j}$ is the image of the polynomial $Q$ under the canonical
homomorphism $K\left[  \alpha_{1},\alpha_{2},...,\alpha_{m}\right]
\rightarrow K\left[  \alpha_{1},\alpha_{2},...,\alpha_{kj}\right]  $. However,
this polynomial $Q_{k,j}$ is not independent of $m$ yet (as the polynomial
$P_{k,j}$ that we intend to construct should be), so we call it
$Q_{k,j,\left[  m\right]  }$ rather than just $Q_{k,j}$.

Now we forget that we fixed $m\in\mathbb{N}$. We have learnt that%
\[
\sum_{\substack{S\subseteq\mathcal{P}_{j}\left(  \left\{  1,2,...,m\right\}
\right)  ;\\\left\vert S\right\vert =k}}\prod_{I\in S}\prod_{i\in I}%
U_{i}=Q_{k,j,\left[  m\right]  }\left(  X_{1},X_{2},...,X_{kj}\right)  ,
\]
in the polynomial ring $K\left[  U_{1},U_{2},...,U_{m}\right]  $ for every
$m\in\mathbb{N}$. Now, define a polynomial $P_{k,j}\in K\left[  \alpha
_{1},\alpha_{2},...,\alpha_{kj}\right]  $ by $P_{k,j}=Q_{k,j,\left[
kj\right]  }.$

\textbf{Theorem 4.4.} \textbf{(a)} The polynomial $P_{k,j}$ just defined
satisfies the equation (\ref{Pkj1}) in the polynomial ring $K\left[
U_{1},U_{2},...,U_{m}\right]  $ for every $m\in\mathbb{N}$. (Hence, the goal
mentioned above in the definition is actually achieved.)

\textbf{(b)} For every $m\in\mathbb{N}$ and $j\in\mathbb{N}$, we have%
\begin{equation}
\prod_{I\in\mathcal{P}_{j}\left(  \left\{  1,2,...,m\right\}  \right)
}\left(  1+\prod_{i\in I}U_{i}\cdot T\right)  =\sum_{k\in\mathbb{N}}%
P_{k,j}\left(  X_{1},X_{2},...,X_{kj}\right)  T^{k} \label{Pkj2}%
\end{equation}
in the ring $\left(  K\left[  U_{1},U_{2},...,U_{m}\right]  \right)  \left[
\left[  T\right]  \right]  $. (Note that the right hand side of this equation
is a power series with coefficient $1$ before $T^{0}$, since $P_{0,j}=1$.)
\end{quote}

\textit{Proof of Theorem 4.4:} \textbf{(a)} \textit{1st Step:} Fix
$m\in\mathbb{N}$ such that $m\geq kj$. Then, we claim that $Q_{k,j,\left[
m\right]  }=P_{k,j}$.

\textit{Proof.} The definition of $Q_{k,j,\left[  m\right]  }$ yields
\[
\sum_{\substack{S\subseteq\mathcal{P}_{j}\left(  \left\{  1,2,...,m\right\}
\right)  ;\\\left\vert S\right\vert =k}}\prod_{I\in S}\prod_{i\in I}%
U_{i}=Q_{k,j,\left[  m\right]  }\left(  X_{1},X_{2},...,X_{kj}\right)
\]
in the polynomial ring $K\left[  U_{1},U_{2},...,U_{m}\right]  $. Applying the
canonical ring epimorphism $K\left[  U_{1},U_{2},...,U_{m}\right]  \rightarrow
K\left[  U_{1},U_{2},...,U_{kj}\right]  $ (which maps every $U_{i}$ to
$\left\{
\begin{array}
[c]{c}%
U_{i},\text{ if }i\leq kj;\\
0,\text{ if }i>kj
\end{array}
\right.  $) to this equation (and noticing that this epimorphism maps every
$X_{i}$ to the corresponding $X_{i}$ of the image ring!), we obtain%
\[
\sum_{\substack{S\subseteq\mathcal{P}_{j}\left(  \left\{  1,2,...,kj\right\}
\right)  ;\\\left\vert S\right\vert =k}}\prod_{I\in S}\prod_{i\in I}%
U_{i}=Q_{k,j,\left[  m\right]  }\left(  X_{1},X_{2},...,X_{kj}\right)
\]
in the polynomial ring $K\left[  U_{1},U_{2},...,U_{kj}\right]  $. On the
other hand, the definition of $Q_{k,j,\left[  kj\right]  }$ yields%
\[
\sum_{\substack{S\subseteq\mathcal{P}_{j}\left(  \left\{  1,2,...,kj\right\}
\right)  ;\\\left\vert S\right\vert =k}}\prod_{I\in S}\prod_{i\in I}%
U_{i}=Q_{k,j,\left[  kj\right]  }\left(  X_{1},X_{2},...,X_{kj}\right)
\]
in the same ring. These two equations yield%
\[
Q_{k,j,\left[  m\right]  }\left(  X_{1},X_{2},...,X_{kj}\right)
=Q_{k,j,\left[  kj\right]  }\left(  X_{1},X_{2},...,X_{kj}\right)  .
\]
Since the elements $X_{1},$ $X_{2},$ $...,$ $X_{kj}$ of $K\left[  U_{1}%
,U_{2},...,U_{kj}\right]  $ are algebraically independent (by Theorem 4.1
\textbf{(a)}), this yields $Q_{k,j,\left[  m\right]  }=Q_{k,j,\left[
kj\right]  }.$ In other words, $Q_{k,j,\left[  m\right]  }=P_{k,j},$ and the
1st Step is proven.

\textit{2nd Step:} For every $m\in\mathbb{N}$, the equation (\ref{Pkj1}) is
satisfied in the polynomial ring $K\left[  U_{1},U_{2},...,U_{m}\right]  $.

\textit{Proof.} Let $m^{\prime}\in\mathbb{N}$ be such that $m^{\prime}\geq m$
and $m^{\prime}\geq kj$. Then, the 1st Step (applied to $m^{\prime}$ instead
of $m$) yields that $Q_{k,j,\left[  m^{\prime}\right]  }=P_{k,j}.$

The definition of $Q_{k,j,\left[  m^{\prime}\right]  }$ yields
\[
\sum_{\substack{S\subseteq\mathcal{P}_{j}\left(  \left\{  1,2,...,m^{\prime
}\right\}  \right)  ;\\\left\vert S\right\vert =k}}\prod_{I\in S}\prod_{i\in
I}U_{i}=Q_{k,j,\left[  m^{\prime}\right]  }\left(  X_{1},X_{2},...,X_{kj}%
\right)
\]
in the polynomial ring $K\left[  U_{1},U_{2},...,U_{m^{\prime}}\right]  $.
Applying the canonical ring epimorphism $K\left[  U_{1},U_{2},...,U_{m^{\prime
}}\right]  \rightarrow K\left[  U_{1},U_{2},...,U_{m}\right]  $ (which maps
every $U_{i}$ to $\left\{
\begin{array}
[c]{c}%
U_{i},\text{ if }i\leq m;\\
0,\text{ if }i>m
\end{array}
\right.  $) to this equation (and noticing that this epimorphism maps every
$X_{i}$ to the corresponding $X_{i}$ of the image ring!), we obtain%
\[
\sum_{\substack{S\subseteq\mathcal{P}_{j}\left(  \left\{  1,2,...,m\right\}
\right)  ;\\\left\vert S\right\vert =k}}\prod_{I\in S}\prod_{i\in I}%
U_{i}=\underbrace{Q_{k,j,\left[  m^{\prime}\right]  }}_{=P_{k,j}}\left(
X_{1},X_{2},...,X_{kj}\right)
\]
in the polynomial ring $K\left[  U_{1},U_{2},...,U_{m}\right]  .$ This means
that the equation (\ref{Pkj1}) is satisfied in the polynomial ring $K\left[
U_{1},U_{2},...,U_{m}\right]  .$ This completes the 2nd Step and proves
Theorem 4.4 \textbf{(a)}.

\textbf{(b)} Immediately follows from part \textbf{(a)}.

\begin{center}
\fbox{\textbf{5. A }$\lambda$\textbf{-ring structure on }$\Lambda\left(
K\right)  =1+K\left[  \left[  t\right]  \right]  ^{+}$}
\end{center}

Now we are going to introduce a $\lambda$-ring structure on a particular set
defined for any given ring $K.$

\begin{quote}
\textbf{Definition:} Let $K$ be a ring. Let $K\left[  \left[  t\right]
\right]  ^{+}$ denote the subset%
\[
tK\left[  \left[  t\right]  \right]  =\left\{  \sum_{i\in\mathbb{N}}a_{i}%
t^{i}\in K\left[  \left[  t\right]  \right]  \ \mid\ a_{i}\in K\text{ for all
}i,\text{ and }a_{0}=0\right\}  .
\]
We will later define a ring structure on the set $1+K\left[  \left[  t\right]
\right]  ^{+}=\left\{  1+u\mid u\in K\left[  \left[  t\right]  \right]
\right\}  $. First, we define an Abelian group structure on this set:

Define an addition $+_{\wedge}$ on the set $1+K\left[  \left[  t\right]
\right]  ^{+}$ by $u+_{\wedge}v=uv$ for every $u\in1+K\left[  \left[
t\right]  \right]  ^{+}$ and $v\in1+K\left[  \left[  t\right]  \right]  ^{+}$.
In other words, addition on $1+K\left[  \left[  t\right]  \right]  ^{+}$ is
defined as multiplication of power series. The zero of $1+K\left[  \left[
t\right]  \right]  ^{+}$ will be $1$.

Then, clearly, $\left(  1+K\left[  \left[  t\right]  \right]  ^{+},+_{\wedge
}\right)  $ is an Abelian group with zero $1$.

Now, define a multiplication $\cdot_{\Lambda}$ on the set $1+K\left[  \left[
t\right]  \right]  ^{+}$ by%
\[
\sum_{i\in\mathbb{N}}a_{i}t^{i}\cdot\sum_{i\in\mathbb{N}}b_{i}t^{i}=\sum
_{k\in\mathbb{N}}P_{k}\left(  a_{1},a_{2},...,a_{k},b_{1},b_{2},...,b_{k}%
\right)  t^{k}.
\]
The multiplicative unity of the ring $1+K\left[  \left[  t\right]  \right]
^{+}$ will be $1+t$.

Also, for every $j\in\mathbb{N}$, define a mapping $\lambda^{j}:1+K\left[
\left[  t\right]  \right]  ^{+}\rightarrow1+K\left[  \left[  t\right]
\right]  ^{+}$ by%
\[
\lambda^{j}\left(  \sum_{i\in\mathbb{N}}a_{i}t^{i}\right)  =\sum
_{k\in\mathbb{N}}P_{kj}\left(  a_{1},a_{2},...,a_{kj}\right)  t^{k}.
\]


\textbf{Theorem 5.1.} \textbf{(a)} The multiplication $\cdot_{\Lambda}$ just
defined is associative, and $1+t$ is actually its unity. Thus, $\left(
1+K\left[  \left[  t\right]  \right]  ^{+},+_{\wedge},\cdot_{\wedge}\right)  $
is a ring. We will call this ring $\Lambda\left(  K\right)  $.

\textbf{(b)} The above defined maps $\lambda^{j}$ make $\left(  \Lambda\left(
K\right)  ,\left(  \lambda_{i}\right)  _{i\in\mathbb{N}}\right)  $ a $\lambda$-ring.
\end{quote}

Before we prove this Theorem 5.1, we try to motivate the above definition of
$\Lambda\left(  K\right)  $:

The set $1+K\left[  t\right]  ^{+}$ (where%
\[
K\left[  t\right]  ^{+}=tK\left[  t\right]  =\left\{  \sum_{i\in\mathbb{N}%
}a_{i}t^{i}\in K\left[  t\right]  \ \mid\ a_{i}\in K\text{ for all }i,\text{
and }a_{0}=0\right\}
\]
) is a subset of $1+K\left[  \left[  t\right]  \right]  ^{+}.$ The elements of
$1+K\left[  t\right]  ^{+}$ are polynomials. We notice that:

\begin{quote}
\textbf{Theorem 5.2.} For every element $p\in1+K\left[  t\right]  ^{+}$, there
exists an integer $n$ (the degree of the polynomial $p$), a finite extension
$K_{p}$ of the ring $K$ and $n$ elements $p_{1},$ $p_{2},$ $...,$ $p_{n}$ of
this extension $K_{p}$ such that $p=\prod\limits_{i=1}^{n}\left(
1+p_{i}t\right)  $ in $1+K\left[  t\right]  ^{+}.$
\end{quote}

\textit{Proof of Theorem 5.2.} Let $p=\sum\limits_{i=0}^{n}a_{i}t^{i}$, where
$n=\deg p$. Define a new polynomial $\widetilde{p}=\sum\limits_{i=0}%
^{n}a_{n-i}t^{i}\in K\left[  t\right]  $. Then, the polynomial $\widetilde{p}$
is monic (since $p\in1+K\left[  t\right]  ^{+}$), and the equation
$p=\prod\limits_{i=1}^{n}\left(  1+p_{i}t\right)  $ becomes equivalent to
$\widetilde{p}=\prod\limits_{i=1}^{n}\left(  p_{i}+t\right)  $, so that
Theorem 5.2 simply claims that for every monic polynomial over a ring, we can
find a finite extension of the ring over which the polynomial splits into
linear polynomials. But this is an easy fact (proven in the same way as the
existence of splitting fields in Galois theory). Thus, Theorem 5.2 is proven.

Let us introduce some notation again:

\begin{quote}
\textbf{Definition.} For every set $H$, let $\mathcal{P}_{\operatorname*{fin}%
}^{\ast}\left(  H\right)  $ denote the set of all finite multisets of $H$. We
denote the multiset formed by the elements $u_{1},$ $u_{2},$ $...,$ $u_{n}$
(with multiplicity) by $\left[  u_{1},u_{2},...,u_{n}\right]  $.

For our ring $K$, let $\operatorname*{Exten}K$ be the set of all finite ring
extensions of $K$. (Again, this is not a set. Again, we don't care. Basically
it is enough to consider all ring extensions of the form $K\left[  X_{1}%
,X_{2},...,X_{n}\right]  \diagup I$ with $I$ being an ideal of $K\left[
X_{1},X_{2},...,X_{n}\right]  $, and \textit{these} extensions do form a set.)

Let $K^{\operatorname*{int}}$ be the subset%
\[
\left\{  \left(  \widetilde{K},\left[  u_{1},u_{2},...,u_{n}\right]  \right)
\in\bigcup_{K^{\prime}\subseteq\operatorname*{Exten}K}\mathcal{P}%
_{\operatorname*{fin}}^{\ast}\left(  K^{\prime}\right)  \ \ \mid
\ \ \prod\limits_{i=1}^{n}\left(  1+u_{i}t\right)  \in K\left[  t\right]
\right\}
\]
of $\bigcup\limits_{K^{\prime}\subseteq\operatorname*{Exten}K}\mathcal{P}%
_{\operatorname*{fin}}^{\ast}\left(  K^{\prime}\right)  $. We define an
equivalence relation $\sim$ on $K^{\operatorname*{int}}$ by%
\begin{align*}
&  \ \ \left(  \widetilde{K},\left[  u_{1},u_{2},...,u_{n}\right]  \right)
\sim\left(  \overline{K},\left[  v_{1},v_{2},...,v_{n}\right]  \right) \\
&  \Longleftrightarrow\ \ \left(
\begin{array}
[c]{c}%
\text{there exists a }K\text{-algebra homomorphism }\phi:K\left[  u_{1}%
,u_{2},...,u_{n}\right]  \rightarrow K\left[  v_{1},v_{2},...,v_{n}\right] \\
\text{such that }\phi\left(  u_{i}\right)  =v_{i}\text{ for every }%
i\in\left\{  1,2,...,n\right\}
\end{array}
\right)  .
\end{align*}


We define a map%
\[
r:1+K\left[  \left[  t\right]  \right]  ^{+}\rightarrow K^{\operatorname*{int}%
}\diagup\sim
\]
(the $r$ stands for "roots" here) through
\[
r\left(  p\right)  =\overline{\left(  K_{p},\left[  p_{1},p_{2},...,p_{n}%
\right]  \right)  },
\]
where $K_{p}$ and $\left[  p_{1},p_{2},...,p_{n}\right]  $ are defined as in
Theorem 5.2, and $\overline{}$ means "equivalence class modulo $\sim$".
\end{quote}

++++++

This way, we have a one-to-one correspondence between elements of $1+K\left[
t\right]  ^{+}$ and multisets of elements of an extension of $K$ (modulo
homomorphisms of extensions of $K$).

Now, for any two polynomials $u\in1+K\left[  t\right]  ^{+}$ and
$v\in1+K\left[  t\right]  ^{+}$, there exist integers $n$ and $m$, a finite
extension $\left(  K_{u}\right)  _{v}$ of the ring $K$ and $n+m$ elements
$u_{1},$ $u_{2},$ $...,$ $u_{n},$ $v_{1},$ $v_{2},$ $...,$ $v_{m}$

[...]

\begin{quote}
Define a multiplication $\cdot_{\wedge}$ on the set $1+K\left[  \left[
t\right]  \right]  ^{+}$ by%
\[
\prod_{i=1}^{n}\left(  1+a_{i}t\right)  \cdot_{\wedge}\prod_{j=1}^{m}\left(
1+b_{j}t\right)  =\prod_{i=1}^{n}\prod_{j=1}^{m}\left(  1+a_{i}b_{j}t\right)
\]
[this is not a def]
\end{quote}

\begin{center}
\fbox{\textbf{References}}
\end{center}

[1] William Fulton, Serge Lang, \textit{Riemann-Roch algebra}, New York 1985.

[2] Donald Knutson, $\lambda$\textit{-Rings and the Representation Theory of
the Symmetric Group}, New York 1973.
\end{document}